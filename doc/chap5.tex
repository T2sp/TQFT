\documentclass[TQFT_main]{subfiles}

\begin{document}

\setcounter{chapter}{4}

\chapter{Higher condensation theory}

Review of ~\cite{gaiotto2025condensationshighercategories}.

\section{Definition of condensation}

In this section, we refer to an $\bm{(n,\, n)}$\textbf{-category} as an $(\infty,\, n)$-category which is ``truncated" at morphism degree $n$.
Intuitively, if $\Cat{C}$ is an $(n,\, n)$-category, then, for arbitrary $n-1$-morphisms $f,\, g \in \Cat{C}_{n-1}$, the space of $n$-morphisms between $f$ and $g$ has discrete topology. 
Therefore, we can compose $n$-morphisms strictly (not up to homotopy), associatively, and unitally.

Note that a $\bm{(0,\, 0)}$\textbf{-category} $\Cat{C}$ is merely a set\footnote{That is, $\Cat{C}$ consists of a set $\Cat{C}_0$ of objects (0-morphisms) only.}.

\begin{mydef}[label=def:0-cond]{0-condensation}
    Let $\Cat{C}$ be a $(0,\, 0)$-category.
    A $\bm{0}$\textbf{-condensation} on $\Cat{C}$ is an equality between elements of a set $\Cat{C}_0$.
\end{mydef}

$n$-condensation is defined by induction.

\begin{mydef}[label=def:n-cond]{{$n$}-condensation}
    Fix $n \ge 0$.
    Let $\Cat{C}$ be an $(n,\, n)$-category, and let $x,\, y \in \Cat{C}_0$ be objects of $\Cat{C}$.
    We define $n$-condensation by induction on $n$.

    $\bm{n}$\textbf{-condensation} of $x$ onto $y$ in $\Cat{C}$ consists of three data:
    \begin{itemize}
        \item A 1-morphism $r \in \inftyMap{\Cat{C}} (x,\, y)_0$
        \item A 1-morphism $i \in \inftyMap{\Cat{C}} (y,\, x)_0$
        \item An $(n-1)$-condensation of $r \circ i$ onto $\Id_y$ in $\inftyMap{\Cat{C}} (y,\, y)$\footnote{Roughly speaking, a mapping space $\inftyMap{\Cat{C}} (y,\, y)$ itself is an $(n-1,\, n-1)$-category. Strickly speaking, we need enriched $\infty$-category theory.} 
        % $\bigl( \varphi \in \inftyMap{\inftyMap{\Cat{C}} (y,\, y)} (f \circ g,\, \Id_{y}),\, \psi \in \inftyMap{\inftyMap{\Cat{C}} (y,\, y)} (\Id_y,\, f \circ g);\, \alpha \bigr)$
    \end{itemize}
\end{mydef}

\begin{myexample}[label=def:1-cond]{1-condensation}
    Let $\Cat{C}$ be a $(1,\, 1)$-category (i.e. an ordinary category).
    \textbf{1-condensation} of $x$ onto $y$ in $\Cat{C}$ consists of these data:
    \begin{itemize}
        \item A 1-morphism $r \in \Hom{\Cat{C}} (x,\, y)$
        \item A 1-morphism $i \in \Hom{\Cat{C}} (y,\, x)$
        \item A 0-condensation of $r \circ i$ onto $\Id_y$ in $\Hom{\Cat{C}} (y,\, y)$. 
        i.e. $r \circ i = \Id_y$.
    \end{itemize}
    In the context of $(1,\, 1)$-categories, such $y$ is called a \textbf{retract of $\bm{x}$}.
\end{myexample}

\begin{myexample}[label=def:2-cond]{2-condensation}
    Let $\Cat{C}$ be a $(2,\, 2)$-category (i.e. a bicategory).
    \textbf{2-condensation} of $x$ onto $y$ in $\Cat{C}$ consists of these data:
    \begin{itemize}
        \item A 1-morphism $r \in \inftyMap{\Cat{C}} (x,\, y)_0$
        \item A 1-morphism $i \in \inftyMap{\Cat{C}} (y,\, x)_0$
        \item A 1-condensation of $r \circ i$ onto $\Id_y$ in $(1,\, 1)$-category $\inftyMap{\Cat{C}} (y,\, y)$.
    \end{itemize}
    By \exref{def:1-cond}, 
    \begin{itemize}
        \item A 1-morphism $r \in \inftyMap{\Cat{C}} (x,\, y)_0$
        \item A 1-morphism $i \in \inftyMap{\Cat{C}} (y,\, x)_0$
        \item A 2-morphism (in $\Cat{C}$) $(r \circ i \xrightarrow{\rho} \Id_y) \in \inftyMap{\Cat{C}} (y,\, y)_1$
        \item A 2-morphism (in $\Cat{C}$) $(\Id_y \xrightarrow{\iota} r \circ i) \in \inftyMap{\Cat{C}} (y,\, y)_1$
        \item An equality $\rho \circ \iota = \Id_{\Id_y}$
    \end{itemize}
\end{myexample}

\begin{myexample}[label=def:3-cond]{3-condensation}
    Let $\Cat{C}$ be a $(3,\, 3)$-category.
    By \exref{def:2-cond},
    \textbf{3-condensation} of $x$ onto $y$ in $\Cat{C}$ consists of these data:
    \begin{itemize}
        \item A 1-morphism $r \in \inftyMap{\Cat{C}} (x,\, y)_0$
        \item A 1-morphism $i \in \inftyMap{\Cat{C}} (y,\, x)_0$
        \item A 2-morphism $(r \circ i \xrightarrow{\rho} \Id_y) \in \inftyMap{\Cat{C}} (y,\, y)_1$
        \item A 2-morphism $(\Id_y \xrightarrow{\iota} r \circ i) \in \inftyMap{\Cat{C}} (y,\, y)_1$
        \item A 3-morphism
        \begin{align}
            \left(
            \begin{tikzcd}[
                baseline={([yshift=-.5ex]current bounding box.center)},
                ampersand replacement=\&
            ]
                {\Id_y} \arrow[r, "\rho \circ \iota", bend left, shift left, ""{name=U, below}] \arrow[r, "{\Id_{\Id_y}}"', bend right, shift right,""{name=D, above}] \& {\Id_y}
                \arrow[Rightarrow, "\alpha", from=U, to=D]
            \end{tikzcd}
            \right)
            \in \inftyMap{\Cat{C}} (y,\, y)_2
        \end{align}
        \item A 3-morphism
        \begin{align}
            \left(
            \begin{tikzcd}[
                baseline={([yshift=-.5ex]current bounding box.center)},
                ampersand replacement=\&
            ]
                {\Id_y} \arrow[r, "{\Id_{\Id_y}}", bend left, shift left, ""{name=U, below}] \arrow[r, "\rho \circ \iota"', bend right, shift right,""{name=D, above}] \& {\Id_y}
                \arrow[Rightarrow, "\beta", from=U, to=D]
            \end{tikzcd}
            \right)
            \in \inftyMap{\Cat{C}} (y,\, y)_2
        \end{align}
        \item An equiality 
        \begin{align}
            % https://tikzcd.yichuanshen.de/#N4Igdg9gJgpgziAXAbVABwnAlgFyxMJZABgBpiBdUkANwEMAbAVxiRAB12BJKAfQE8QAX1LpMufIRQAmclVqMWbTjwHD5MKAHN4RUADMAThAC2SMiBwQkARmoAjGGChIALAE5qcABZZ9OJABaWQVmVkQObj5gFT5+IWFRECNTc2orWwcnF0QLHz8A3OoGOkcGAAVxPAI2LDBsWBBqejDldnpDNF8AAk4AYyxDPt72NGxEg2MzIstrRDsQR2cgjy9ffyQQlqUI2N4YqIEE4tKYCqrJNkMsLW8AoQohIA
            \begin{tikzcd}[
                baseline={([yshift=-.5ex]current bounding box.center)},
                ampersand replacement=\&
            ]
                \Id_y \arrow[rr, "\Id_{\Id_y}", bend left=49, shift left=2, ""{name=U,below}] \arrow[rr, "\rho \circ \iota" description,""{name=MU,above},""{name=MD,below}] \arrow[rr, "\Id_{\Id_y}"', bend right=49, shift right=2,""{name=D,above}] \&  \& \Id_y
                \arrow[Rightarrow,"\beta",from=U,to=MU]
                \arrow[Rightarrow,"\alpha",from=MD,to=D]
            \end{tikzcd}
            =
            \begin{tikzcd}[
                baseline={([yshift=-.5ex]current bounding box.center)},
                ampersand replacement=\&
            ]
                \Id_y \arrow[rr, "\Id_{\Id_y}", bend left=49, shift left=2, ""{name=U,below}] \arrow[rr, "\Id_{\Id_y}"', bend right=49, shift right=2,""{name=D,above}] \&  \& \Id_y
                \arrow[Rightarrow,"\Id_{\Id_{\Id_y}}",from=U,to=D]
            \end{tikzcd}
        \end{align}
        % $\alpha \circ \beta = \Id_{\Id_{\Id_y}}$
    \end{itemize}
\end{myexample}

\begin{myexample}[label=def:4-cond]{4-condensation}
    Let $\Cat{C}$ be a $(4,\, 4)$-category.
    By \exref{def:2-cond},
    \textbf{3-condensation} of $x$ onto $y$ in $\Cat{C}$ consists of these data:
    \begin{itemize}
        \item A 1-morphism $r \in \inftyMap{\Cat{C}} (x,\, y)_0$
        \item A 1-morphism $i \in \inftyMap{\Cat{C}} (y,\, x)_0$
        \item A 2-morphism $(r \circ i \xrightarrow{\rho} \Id_y) \in \inftyMap{\Cat{C}} (y,\, y)_1$
        \item A 2-morphism $(\Id_y \xrightarrow{\iota} r \circ i) \in \inftyMap{\Cat{C}} (y,\, y)_1$
        \item A 3-morphism
        \begin{align}
            \left(
            \begin{tikzcd}[
                baseline={([yshift=-.5ex]current bounding box.center)},
                ampersand replacement=\&
            ]
                {\Id_y} \arrow[r, "\rho \circ \iota", bend left, shift left, ""{name=U, below}] \arrow[r, "{\Id_{\Id_y}}"', bend right, shift right,""{name=D, above}] \& {\Id_y}
                \arrow[Rightarrow, "\alpha", from=U, to=D]
            \end{tikzcd}
            \right)
            \in \inftyMap{\Cat{C}} (y,\, y)_2
        \end{align}
        \item A 3-morphism
        \begin{align}
            \left(
                \begin{tikzcd}[
                    baseline={([yshift=-.5ex]current bounding box.center)},
                    ampersand replacement=\&
                ]
                    {\Id_y} \arrow[r, "{\Id_{\Id_y}}", bend left, shift left, ""{name=U, below}] \arrow[r, "\rho \circ \iota"', bend right, shift right,""{name=D, above}] \& {\Id_y}
                    \arrow[Rightarrow, "\beta", from=U, to=D]
                \end{tikzcd}
            \right)
            \in \inftyMap{\Cat{C}} (y,\, y)_2
        \end{align}
        \item A 4-morphism
        \begin{align}
            \left( 
                \begin{tikzpicture}[
                    nat/.style={
                        -{>[length=1pt 1 0,inset=5pt]}, double, double distance=2pt,line cap=butt
                    },
                    mor2/.style={
                        gray
                    },
                    mor3/.style={
                        blue
                    },
                    mor4/.style={
                        red
                    },baseline={([yshift=-.5ex]current bounding box.center)}
                ]
                    \pgfmathsetmacro{\dx}{4}
                    \pgfmathsetmacro{\dy}{1}
                    \pgfmathsetmacro{\Margin}{.3}
                    \path 
                        coordinate (P)
                    +(0:\dx) coordinate (Q)
                    ;
                    \coordinate (O) at ($(P)!.5!(Q)$);
                    % 1-mor
                    \node (id_1) at (P) {$\Id_y$};
                    \node (id_2) at (Q) {$\Id_y$};
                    % 2-mor
                    \draw[->,mor2] 
                        (id_1) to[out=60,in=60] 
                        coordinate[pos=.6] (Ini) 
                        node[pos=.5,above] {$\Id_{\Id_y}$}
                        (id_2)
                    ;
                    \draw[->,mor2,spath/save=Front]
                        (id_1) to[out=-120,in=-120] 
                        coordinate[pos=.4] (End) 
                        node[pos=.5,above] {$\Id_{\Id_y}$}
                        (id_2)
                    ;
                    \draw[->,mor2,spath/save=U] 
                        (id_1) to[out=90,in=90,looseness=1.7] 
                        coordinate[pos=.5] (Mid) 
                        node[pos=.4,above left] {$\rho \circ \iota$}
                        (id_2)
                    ;
                    % 3-mor
                    \path[spath/save=DArr_1] 
                        ($(Ini) + (90:\Margin)$) to[out=90,in=30,looseness=1] 
                        node[pos=.5,above right,mor3] {$\beta$} 
                        ($(Mid) + (30:\Margin)$)
                    ;
                    \path[spath/save=DArr_2] 
                        ($(Mid) + (-150:\Margin)$) to[out=-150,in=90,looseness=1] 
                        node[pos=.5,above left,mor3] {$\alpha$} 
                        ($(End) + (90:\Margin)$)
                    ;
                    \draw[white,line width=5pt,spath/use=DArr_2]; %%% crossing
                    \path[spath/save=DArr_3] 
                        ($(Ini) + (-90:\Margin)$) to[out=-90,in=-90,looseness=1.5] 
                        node[pos=.5,below right,mor3] {$\Id$} 
                        coordinate[pos=.5] (FEnd) 
                        ($(End) + (-90:\Margin)$)
                    ;
                    \DoubleArrow{DArr_1}{mor3}
                    \DoubleArrow{DArr_2}{mor3}
                    \DoubleArrow{DArr_3}{mor3}
                    % 4-mor
                    \path[spath/save=F] 
                        ($(Mid) + (-90:\Margin)$) -- 
                        node[pos=.5,right,mor4] {$R$}
                        ($(FEnd) + (90:\Margin)$)
                    ;
                    \TriArrow{F}{mor4}
                    % crossings
                    \Over{U}{.5}{.9}{gray}{.4pt}
                    \Over{Front}{.5}{.9}{gray}{.4pt}
                \end{tikzpicture}
            \right)
            \in \inftyMap{\Cat{C}} (y,\, y)_3
        \end{align}
        \item A 4-morphism
        \begin{align}
            \left( 
                \begin{tikzpicture}[
                    nat/.style={
                        -{>[length=1pt 1 0,inset=5pt]}, double, double distance=2pt,line cap=butt
                    },
                    mor2/.style={
                        gray
                    },
                    mor3/.style={
                        blue
                    },
                    mor4/.style={
                        red
                    },baseline={([yshift=-.5ex]current bounding box.center)}
                ]
                    \pgfmathsetmacro{\dx}{4}
                    \pgfmathsetmacro{\dy}{1}
                    \pgfmathsetmacro{\Margin}{.3}
                    \path 
                        coordinate (P)
                    +(0:\dx) coordinate (Q)
                    ;
                    \coordinate (O) at ($(P)!.5!(Q)$);
                    % 1-mor
                    \node (id_1) at (P) {$\Id_y$};
                    \node (id_2) at (Q) {$\Id_y$};
                    % 2-mor
                    \draw[->,mor2] 
                        (id_1) to[out=60,in=60] 
                        coordinate[pos=.6] (Ini) 
                        node[pos=.5,above] {$\Id_{\Id_y}$}
                        (id_2)
                    ;
                    \draw[->,mor2,spath/save=Front]
                        (id_1) to[out=-120,in=-120] 
                        coordinate[pos=.4] (End) 
                        node[pos=.5,above] {$\Id_{\Id_y}$}
                        (id_2)
                    ;
                    \draw[->,mor2,spath/save=U] 
                        (id_1) to[out=-90,in=-90,looseness=1.7] 
                        coordinate[pos=.5] (Mid)
                        node[pos=.6,below right] {$\rho \circ \iota$}
                        (id_2)
                    ;
                    % 3-mor
                    \path[spath/save=DArr_1] 
                        ($(Ini) + (-90:\Margin)$) to[out=-90,in=30,looseness=1] 
                        node[pos=.5,below right,mor3] {$\beta$} 
                        ($(Mid) + (30:\Margin)$)
                    ;
                    \path[spath/save=DArr_2] 
                        ($(Mid) + (-150:\Margin)$) to[out=-150,in=-90,looseness=1] 
                        node[pos=.5,below left,mor3] {$\alpha$} 
                        ($(End) + (-90:\Margin)$)
                    ;
                    \draw[white,line width=5pt,spath/use=DArr_2]; %%% crossing
                    \path[spath/save=DArr_3] 
                        ($(Ini) + (90:\Margin)$) to[out=90,in=90,looseness=1.5] 
                        node[pos=.5,above left,mor3] {$\Id$} 
                        coordinate[pos=.5] (FEnd) 
                        ($(End) + (90:\Margin)$)
                    ;
                    \draw[white,line width=5pt,spath/use=DArr_3]; %%% crossing
                    \DoubleArrow{DArr_1}{mor3}
                    \DoubleArrow{DArr_2}{mor3}
                    \DoubleArrow{DArr_3}{mor3}
                    % 4-mor
                    \path[spath/save=F] 
                        ($(FEnd) + (-90:\Margin)$) --
                        node[pos=.5,right,mor4] {$I$}
                        ($(Mid) + (90:\Margin)$)
                    ;
                    \TriArrow{F}{mor4}
                    % crossings
                    \Over{U}{.5}{.9}{gray}{.4pt}
                    \Over{Front}{.1}{.9}{gray}{.4pt}
                \end{tikzpicture}
            \right)
            \in \inftyMap{\Cat{C}} (y,\, y)_3
        \end{align}
        % $\alpha \circ \beta = \Id_{\Id_{\Id_y}}$
        \item An equality
        \begin{align}
                \begin{tikzpicture}[
                    nat/.style={
                        -{>[length=1pt 1 0,inset=5pt]}, double, double distance=2pt,line cap=butt
                    },
                    mor2/.style={
                        gray
                    },
                    mor3/.style={
                        blue
                    },
                    mor4/.style={
                        red
                    },baseline={([yshift=-.5ex]current bounding box.center)}
                ]
                    \pgfmathsetmacro{\dx}{4}
                    \pgfmathsetmacro{\dy}{1}
                    \pgfmathsetmacro{\Margin}{.3}
                    \path 
                        coordinate (P)
                    +(0:\dx) coordinate (Q)
                    ;
                    \coordinate (O) at ($(P)!.5!(Q)$);
                    % 1-mor
                    \node (id_1) at (P) {$\Id_y$};
                    \node (id_2) at (Q) {$\Id_y$};
                    % 2-mor
                    \draw[->,mor2] 
                        (id_1) to[out=60,in=60] 
                        coordinate[pos=.6] (Ini) 
                        node[pos=.5,above] {$\Id_{\Id_y}$}
                        (id_2)
                    ;
                    \draw[->,mor2,spath/save=Front]
                        (id_1) to[out=-120,in=-120] 
                        coordinate[pos=.4] (End) 
                        node[pos=.3,below] {$\Id_{\Id_y}$}
                        (id_2)
                    ;
                    \draw[->,mor2,spath/save=Middle] 
                        (id_1) --
                        coordinate[pos=.5] (Mid)
                        node[pos=.5,below right] {$\rho \circ \iota$}
                        (id_2)
                    ;
                    % 3-mor
                    \path[spath/save=DArr_1] 
                        ($(Ini) + (-150:\Margin)$) --
                        node[pos=.5,right,mor3] {$\beta$} 
                        ($(Mid) + (30:\Margin)$)
                    ;
                    \path[spath/save=DArr_2] 
                        ($(Mid) + (-150:\Margin)$) --
                        node[pos=.5,right,mor3] {$\alpha$} 
                        ($(End) + (30:\Margin)$)
                    ;
                    \path[spath/save=DArr_3] 
                        ($(Ini) + (90:\Margin)$) to[out=90,in=90,looseness=1.5] 
                        node[pos=.5,above left,mor3] {$\Id$} 
                        coordinate[pos=.5] (FIni) 
                        ($(End) + (90:\Margin)$)
                    ;
                    \draw[white,line width=5pt,spath/use=DArr_3]; %%% crossing
                    \path[spath/save=DArr_4]
                        ($(Ini) + (-90:\Margin)$) to[out=-90,in=-90,looseness=1.5] 
                        node[pos=.5,below right,mor3] {$\Id$} 
                        coordinate[pos=.5] (FEnd) 
                        ($(End) + (-90:\Margin)$)
                    ;
                    \DoubleArrow{DArr_1}{mor3}
                    \DoubleArrow{DArr_2}{mor3}
                    \DoubleArrow{DArr_3}{mor3}
                    \DoubleArrow{DArr_4}{mor3}
                    % 4-mor
                    \path[spath/save=F_1] 
                        ($(FIni) + (-90:\Margin)$) --
                        node[pos=.5,right,mor4] {$I$}
                        ($(Mid) + (90:\Margin)$)
                    ;
                    \path[spath/save=F_2] 
                        ($(Mid) + (-90:\Margin)$) --
                        node[pos=.5,right,mor4] {$R$}
                        ($(FEnd) + (90:\Margin)$)
                    ;
                    \TriArrow{F_1}{mor4}
                    \TriArrow{F_2}{mor4}
                    % crossings
                    \Over{Middle}{.5}{.9}{gray}{.4pt}
                    \Over{Front}{.1}{.9}{gray}{.4pt}
                \end{tikzpicture}
                =
                \begin{tikzpicture}[
                    nat/.style={
                        -{>[length=1pt 1 0,inset=5pt]}, double, double distance=2pt,line cap=butt
                    },
                    mor2/.style={
                        gray
                    },
                    mor3/.style={
                        blue
                    },
                    mor4/.style={
                        red
                    },baseline={([yshift=-.5ex]current bounding box.center)}
                ]
                    \pgfmathsetmacro{\dx}{4}
                    \pgfmathsetmacro{\dy}{1}
                    \pgfmathsetmacro{\Margin}{.3}
                    \path 
                        coordinate (P)
                    +(0:\dx) coordinate (Q)
                    ;
                    \coordinate (O) at ($(P)!.5!(Q)$);
                    % 1-mor
                    \node (id_1) at (P) {$\Id_y$};
                    \node (id_2) at (Q) {$\Id_y$};
                    % 2-mor
                    \draw[->,mor2] 
                        (id_1) to[out=60,in=60] 
                        coordinate[pos=.6] (Ini) 
                        node[pos=.5,above] {$\Id_{\Id_y}$}
                        (id_2)
                    ;
                    \draw[->,mor2,spath/save=Front]
                        (id_1) to[out=-120,in=-120] 
                        coordinate[pos=.4] (End) 
                        node[pos=.3,below] {$\Id_{\Id_y}$}
                        (id_2)
                    ;
                    % 3-mor
                    \path[spath/save=DArr_3] 
                        ($(Ini) + (90:\Margin)$) to[out=90,in=90,looseness=1.5] 
                        node[pos=.5,above left,mor3] {$\Id$} 
                        coordinate[pos=.5] (FIni) 
                        ($(End) + (90:\Margin)$)
                    ;
                    \draw[white,line width=5pt,spath/use=DArr_3]; %%% crossing
                    \path[spath/save=DArr_4]
                        ($(Ini) + (-90:\Margin)$) to[out=-90,in=-90,looseness=1.5] 
                        node[pos=.5,below right,mor3] {$\Id$} 
                        coordinate[pos=.5] (FEnd) 
                        ($(End) + (-90:\Margin)$)
                    ;
                    \DoubleArrow{DArr_3}{mor3}
                    \DoubleArrow{DArr_4}{mor3}
                    % 4-mor
                    \path[spath/save=F_1] 
                        ($(FIni) + (-90:\Margin)$) --
                        node[pos=.5,right,mor4] {$\Id$}
                        ($(FEnd) + (90:\Margin)$)
                    ;
                    \TriArrow{F_1}{mor4}
                    % crossings
                    \Over{Front}{.1}{.9}{gray}{.4pt}
                \end{tikzpicture}
        \end{align}
        
    \end{itemize}
\end{myexample}

\subsection{Walking condensation}

Let $\Cat{C}$ be an $(n,\, n)$-category.
Roughly speaking, a \textbf{walking} $\bm{n}$\textbf{-condensation} is a \Caution{(strict?)} $n$-category $\spadesuit_n$ which ``generates" \hyperref[def:n-cond]{$n$-condensation} in $\Cat{C}$.
i.e. the functor category $\FUN(\spadesuit_n,\, \Cat{C})$ is equivalent to the category of $n$-condensations in $\Cat{C}$.

\subsection{Condensation monad}

Fix an $(n,\, n)$-category $\Cat{C}$. 
Now we will outline the definition of \textbf{condensation monad in $\Cat{C}$}. 

Let $\clubsuit_n \subset \spadesuit_n$ be a \Caution{subcategory}\footnote{(18 February, 2026) Full-subness of $\clubsuit_n$ is still conjecture~\cite{gaiotto2025condensationshighercategories}.} which consists of walking $n$-condensations with a single object.
A \textbf{condensation monad} in $\Cat{C}$ is a functor $A \colon \clubsuit_n \lto \Cat{C}$.
An $(n,\, n)$-category $\Cat{C}$ is said to \textbf{have all condensates} 
if each condensation monad $\clubsuit_n \xrightarrow{A} \Cat{C}$ extends to an \hyperref[def:n-cond]{$n$-condensation} $\spadesuit_n \xrightarrow{\bar{A}} \Cat{C}$:

\begin{center}
    % https://tikzcd.yichuanshen.de/#N4Igdg9gJgpgziAXAbVABwnAlgFyxMJZABgBpiBdUkANwEMAbAVxiRAB12BjZgIzia4A+oQC+pdJlz5CKAIzkqtRizacAwnRzB1okOMnY8BImTlL6zVog7s4aOrAHCxSmFADm8IqABmAJwgAWyQyEBwIJAVlKzYAQX0JEADg0OoIpAAmdLosBjYACwgIAGtEv0CQxGzwyMRohiwwaxAoOjgC9xBqBjpeGAYABSljWRB-LA8CnG6Y1RtOXjp-YDi9UQpRIA
    \begin{tikzcd}
    \clubsuit_n \arrow[r, "A"] \arrow[d, hook]  & \Cat{C} \\
    \spadesuit_n \arrow[ru, "\bar{A}"', dashed] &        
    \end{tikzcd}
\end{center}

However, \Caution{there are some technical issues in this definition},
so we should work on more concrete definition.

\begin{mydef}[label=def:cond-monad]{condensation monad}
    Let $\Cat{C}$ be an $(n,\, n)$-category. 
    A \textbf{condensation monad} in $\Cat{C}$ is a sequence of ``commuting condensation squares":
    \begin{align}
        \spadesuit^{\times 0} &\xrightarrow{e} \Cat{C}, \\
        \spadesuit^{\times 1} &\lto \Cat{C}_{/e}, \\
        \spadesuit^{\times 2} &\lto \Cat{C}_{/e}, \\
        &\vdots
    \end{align}
    More explicitly, condensation monad consists of the following diagrams:
    \begin{align}
        e &\in \Cat{C}_0, \\
        e \circ e &\lcond e,
    \end{align}
    
\end{mydef}


\section{Physical interpretation}

\section{Tannaka-Krein reconstruction}

Fix \Caution{an algebraically closed} field $\mathbb{K}$.
From now on, we denote the $(1,\, 1)$-category of \underline{finite dimensional} $\mathbb{K}$-vector spaces as $\finVEC{\mathbb{K}}$\footnote{This is the fully-dualizable part of $\VEC{\mathbb{K}}$, which is a $(1,\, 1)$-category of $\mathbb{K}$-vector spaces.}.

\subsection{A bicategory of 2-vactor spaces {$\nVec{2}_{\mathbb{K}}$}}

Let $\Presentable^{\mathrm{L}}$ be the symmetric monoidal $(\infty,\, 2)$-category of presentable $(\infty,\, 1)$-categories, colimit-preserving functors, and natural transformations~\cite[Definition 5.5.3.1.]{lurie2008higher}.
Note that $\finVEC{\mathbb{C}}$ is an $\mathbb{E}_2$-algebra in $\Presentable^{\mathrm{L}}$.
We define $\PrC \coloneqq \RMod_{\finVEC{\mathbb{K}}} (\Presentable^{\mathrm{L}})$~\cite[Definition 4.2.1.13.]{Lurie2017HA}.
More concretely, $\PrC$ is the symmetric monoidal $(\infty,\, 2)$-category of presentable $\VEC{\mathbb{C}}$-enriched categories, colimit-preserving $\finVEC{\mathbb{C}}$-enriched functors, and natural transformations.

After ~\cite{Green2023TannakaKrein}, we define $\nVec{2}_{\mathbb{K}}$ as a $(2,\, 2)$-category (bicategory) as follows:
\begin{itemize}
    \item Objects are \Caution{finite semisimple} $\finVEC{\mathbb{K}}$-enriched categories
    \item 1-morphisms are $\finVEC{\mathbb{K}}$-enriched functors.
    \item 2-morphisms are natural transformations.
\end{itemize}
Note that $\nVec{2}_{\mathbb{K}}$ is fully-dualizable. 
In fact, $\nVec{2}_{\mathbb{K}} \subset \PrC$ is the full-subcategory of $\PrC$ which consists of fully dualizable objects of $\PrC$.
For more details, see~\cite[APPENDIX A.]{BDSV2015modular}.

\subsection{Morita bicategory of algebras}

Note that $\VEC{\mathbb{K}}$ \Caution{($\finVEC{\mathbb{K}}$?)} is a symmetric monoidal ($(\infty,\, 1)$-)category.
Then, we obtain a notion of $\mathbb{E}_1$-algebra objects in $\VEC{\mathbb{K}}$, according to~\cite{Lurie2017HA}.
More concretely, an $\mathbb{E}_1$-algebra object in $\VEC{\mathbb{K}}$ is none other than an ordinary associative algebra (i.e. $\mathbb{K}$-vector spaces with associative multiplication)\footnote{\Caution{With respect to the open embedding $\emptyset \hookrightarrow \mathbb{R}$, $\mathbb{E}_1$-algebra in $\VEC{\mathbb{K}}$ has units. However, there appears no physical motivation to introduce unit. Therefore it's better to work on nonunital associative algebras.}}.

\textbf{Morita category of $\mathbb{E}_1$-algebras} in $\VEC{\mathbb{K}}$ is a bicategory $\MOR{\mathbb{E}_1}{\VEC{\mathbb{K}}}$ consists of the following data:
\begin{itemize}
    \item Objects are $\mathbb{E}_1$-algebras in $\VEC{\mathbb{K}}$.
    \item 1-morphisms between two $\mathbb{E}_1$-algebras $A,\, B \in \Obj{\MOR{\mathbb{E}_1}{\VEC{\mathbb{K}}}}$ are $(A,\, B)$-bimodules in $\VEC{\mathbb{K}}$.
    \item 2-morphisms between two $(A,\, B)$-bimodules $M,\, N \in \Obj{\BIMODC{\VEC{\mathbb{K}}}{A}{B}}$ are $(A,\, B)$-bimodule homomorphisms.
\end{itemize}
Composition of 1-morphisms (or horizontal composition) in $\MOR{\mathbb{E}_1}{\VEC{\mathbb{K}}}$ is the relative tensor product functor:
\begin{align}
    \otimes_B \colon \BIMODC{\VEC{\mathbb{K}}}{B}{C} \boxtimes \BIMODC{\VEC{\mathbb{K}}}{A}{B} \lto \BIMODC{\VEC{\mathbb{K}}}{A}{C}
\end{align}

\subsection{Tannaka-Krein reconstruction}



% \section{SPT相}

% SPT相とは,大雑把に言うと\hyperref[def:quantum-phase]{gappedな量子相}であって,同値関係の定義においてある種の対称性を考慮しており,かつ命題\ref{prop:monoidalTO}のモノイド構造が可逆であるようなもののことである.
% 対称性を考慮した最も簡単なクラスのgappedな量子相ということである.
% なお,対称性と言ったときに,\hyperref[def:p-form-sym]{0-form symmetry}のみならずnon-invertible symmetryなどの一般化対称性や,subsystem symmetry, modulated symmetryなどを考えても良い.


% \subsection{SRE状態とLRE状態}

% まず,~\cite[p.3]{ChenGuWen2010}に倣って\textbf{SRE状態} (Short Range Entangled states) と\textbf{LRE状態} (Long Range Entangled states) を定義する.
% ~\cite[p.4]{ChenGuWen2010}
% \begin{myconjph}[label=conj:CGW]{Chen-Gu-Wenの仮説}
%     \hyperref[def:bosonic-lattice-model]{bosonic}かつ\hyperref[def:gapped]{gapped}な2つの基底状態 $\ket{\Phi_0},\, \ket{\Phi_1}$ が同じ\hyperref[def:quantum-phase]{量子相}にあるならば,以下の条件を充たす\hyperref[def:bosonic-lattice-model]{bosonicな格子模型}の族 $\Familyset[\big]{\hat{H}(t)}{t \in [0,\, 1]}$ が存在する:
%     \begin{itemize}
%         \item $\forall t \in [0,\, 1]$ に対して,$\hat{H}(t)$ の基底状態は熱力学極限を取った際に\hyperref[def:gapped]{gapped}である.
%         \item $\ket{\Phi_0},\, \ket{\Phi_1}$ はそれぞれ $\hat{H}(0),\, \hat{H}(1)$ の基底状態である.
%     \end{itemize}
% \end{myconjph}

% \begin{mypropph}[label=prop:LU]{LU transformation}
%     以下の2つは同値である:
%     \begin{enumerate}
%         \item \hyperref[def:bosonic-lattice-model]{bosonic}かつ\hyperref[def:gapped]{gapped}な2つの基底状態 $\ket{\Phi_0},\, \ket{\Phi_1}$ が同じ\hyperref[def:quantum-phase]{量子相}にある
%         \item \hyperref[def:bosonic-lattice-model]{bosonicな格子模型}の族 $\Familyset[\big]{\hat{\tilde{H}}(t)}{t \in [0,\, 1]}$ が存在して
%         \begin{align}
%             \label{eq:LU}
%             \ket{\Phi_1} = \mathcal{T} \bigl[ e^{-\iunit \int_0^1 \dd{t} \hat{\tilde{H}}(t)} \bigr] \ket{\Phi_0}
%         \end{align}
%         を充たす.ただし $\mathcal{T}$ は経路順序積である.
%     \end{enumerate}
% \end{mypropph}

% \begin{proof}
%     \begin{description}
%         \item[$\bm{(\Longrightarrow)}$] 
        
%         仮説\ref{conj:CGW}による.

%         \item[$\bm{(\Longleftarrow)}$] 
        

%     \end{description}
    
% \end{proof}

% \eqref{eq:LU}を\textbf{局所ユニタリ発展} (local unitary evolution) と呼ぶ.

% \begin{mydefph}[label=def:SRE]{SRE状態}
%     \hyperref[def:bosonic-lattice-model]{bosonic}かつ\hyperref[def:gapped]{gapped}な基底状態 $\ket{\Phi} \in \Gnd{\mfd{\Sigma}{D}}$ が\textbf{SRE状態} (short range entangled state) であるとは,
%     あるseparableな状態
%     \begin{align}
%         \bigotimes_{\bm{x} \in \Lambda} \ket{\psi_{\bm{x}}} \WHERE \forall \bm{x} \in \Lambda,\, \ket{\psi_{\bm{x}}} \in \mathcal{H}_{\bm{x}}
%     \end{align}
%     と $\ket{\Psi}$ との間に\hyperref[prop:LU]{局所ユニタリ発展}が存在すること.
    
%     \tcblower

%     SRE状態でない基底状態のことを\textbf{LRE状態} (long range entangled state) と呼ぶ.
% \end{mydefph}

% 定義から明らかに,任意のSRE状態は同一の\hyperref[def:quantum-phase]{量子相}に属する.

% \subsection{Bosonic SPT相}

% \hyperref[def:SRE]{SRE状態の定義}はそのままだと面白くないが,対称性を考慮すると話は変わってくる.
% 位相群 $G$ を与える.また,格子 $\Lambda$ は空間群 $S$ の対称性を持つ\footnote{$S$ の $\Lambda$ への左作用を $\btr \colon S \times \Lambda \lto \Lambda$ と書く.}とする.

% \begin{mydef}[label=def:semiprod-group-inner]{外部半直積}
% 	$N,\, H$ を群とし,$\phi \colon H \to \Aut N,\; h \mapsto \phi_h$ を準同型写像とする\footnote{$\Aut N$ は,$N$ から $N$ 自身への同型写像全体の集合に,写像の合成を群の演算として群構造を入れたもので,\textbf{自己同型群} (automorphism group) と呼ばれる.}.
% 	このとき,集合 $N \times H$ は次の二項演算 $\cdot \mathrel{} \colon N\times H \to N\times H$ に関して群を成す:
% 	\begin{align}
% 		(n_1,\, h_1) \cdot (n_2,\, h_2) \coloneqq \bigl( n_1 \phi_{\textcolor{red}{h_1}}(n_2),\, h_1h_2 \bigr) 
% 	\end{align}
% 	この群 $\bigl( N \times H,\, \cdot \mathrel{},\, (1_N,\, 1_H) \bigr)$ のことを $N,\, H$ の(外部)\textbf{半直積} (semidirect product) と呼び,$\bm{H \ltimes_\phi N}$ または $\bm{N \rtimes_\phi H}$ と書く.
% \end{mydef}


% \begin{mydef}[label=def:blat-G-equiv]{$G$-対称な格子模型}
%     \hyperref[def:bosonic-lattice-model]{bosonicな格子模型} $\hat{H}$ が\textbf{$\bm{G}$-対称}~\cite[p.12]{Xiong2019SPT}であるとは,
%     \begin{itemize}
%         \item 群準同型 $\irm{\rho}{spa} \colon G \lto S$
%         \item 群準同型 $\irm{\phi}{int} \colon G \lto \{\pm 1\}$
%         \item 群のユニタリ or 反ユニタリ表現 $\irm{\rho}{int} \colon G \ltimes_{\btr \circ \irm{\rho}{spa}} S \lto \bigl\{\, \text{unitary or antiunitary operator}\; \irm{\mathcal{H}}{tot} \to \irm{\mathcal{H}}{tot} \,\bigr\} $
%     \end{itemize}
%     が存在して以下を充たすことを言う:
%     \begin{description}
%         \item[\textbf{(Gsym-1)}] 
        
%         $\forall g \in G$ に対し,
%         \begin{align}
%             \irm{\phi}{int} (g)
%             = 
%             \begin{cases}
%                 +1, &\irm{\rho}{int}(g)\; \text{is unitary} \\
%                 -1, &\irm{\rho}{int}(g)\; \text{is antiunitary}
%             \end{cases}
%         \end{align}
        
%         \item[\textbf{(Gsym-2)}] 
        
%         群 $G$ の $\irm{\mathcal{H}}{tot}$ への作用
%         \begin{align}
%             \rho \colon G &\lto \LGL (\irm{\mathcal{H}}{tot}), \\
%             g &\lmto \left( \bigotimes_{\bm{x} \in \Lambda} \ket{\psi_{\bm{x}}} \lmto \bigotimes_{\bm{x} \in \Lambda} \irm{\rho}{int} \bigl( g,\, \irm{\rho}{spa}(g)^{-1} \btr \bm{x} \bigr) \ket{\psi_{\irm{\rho}{spa}(g)^{-1} \btr \bm{x}}}\right) 
%         \end{align}
%         に関して,
%         \begin{align}
%             \forall g \in G,\; \comm{\hat{H}}{\rho(g)} = 0
%         \end{align}
%         が成り立つ.
%     \end{description}
    
% \end{mydef}

% \hyperref[def:blat-G-equiv]{$G$-対称な格子模型}全体の集合を $\bm{\LatG{G}{\mfd{\Sigma}{D}}} \subset \Lat{\mfd{\Sigma}{D}}$ と書き\footnote{$\Lat{\mfd{\Sigma}{D}}$ からのsubspace topologyを入れる.}.
% その基底状態全体の集合を $\bm{\GndG{G}{\mfd{\Sigma}{D}}} \subset \Gnd{\mfd{\Sigma}{D}}$ と書く.
% % 簡単のため,以下では暫くの間,群準同型 $\irm{\rho}{spa} \colon G \lto S,\; g \lmto 1_S$ の場合のみを考える.

% \begin{mydefph}[label=def:Gequiv-bqp]{$G$-同変な量子相}
%     2つの\hyperref[def:bosonic-lattice-model]{bosonicかつ $G$-対称な格子模型} $\hat{H}_0,\, \hat{H}_1 \in \mathrm{Lat}_G(\mfd{\Sigma}{D})$ を与える.
%     \begin{itemize}
%         \item $\hat{H}_0$ の基底状態 $\ket{\Phi_0} \in \GndG{G}{\mfd{\Sigma}{D}}$ と $\hat{H}_1$ の基底状態 $\ket{\Phi_1} \in \GndG{G}{\mfd{\Sigma}{D}}$ が同じ\textbf{$\bm{G}$-同変な量子相} ($G$-equivalent quantum phase) にあるとは,$C^\infty$ 曲線 $\hat{H} \colon [0,\, 1] \lto \mathrm{Lat}_G(\mfd{\Sigma}{D})$ が存在して
%             $\hat{H}(0) = \hat{H}_0 \AND \hat{H}(1) = \hat{H}_1$ を充たすこと.これは $\GndG{G}{\mfd{\Sigma}{D}}$ 上の同値関係を成す.
%         \item 商集合 $\GndG{G}{\mfd{\Sigma}{D}}/{\sim}$ の元のことを\textbf{$\bm{G}$-同変な量子相}と呼ぶ.
%     \end{itemize}
% \end{mydefph}

% \begin{mydefph}[label=def:SPT-traditional]{SPT相 (Chen-Gu-Wenによる)}
%     \hyperref[def:bosonic-lattice-model]{bosonic}かつ\hyperref[def:gapped]{gapped}かつ\hyperref[def:blat-G-equiv]{$G$-同変な量子相} $[\ket{\Phi}] \in \GndG{G}{\mfd{\Sigma}{D}}$ が\textbf{SPT相} (symmetry protected topological phase\footnote{\textbf{symmetry protected trivial phase}と呼ぶこともある~\cite{Wen2014SPT}.}) であるとは,
%     $\forall \ket{\Psi} \in [\ket{\Phi}]$ が\hyperref[def:SRE]{SRE状態}であることを言う.
% \end{mydefph}
% つまり,任意の代表元が $G$-対称性を破ればseparableな状態に滑らかにつながるような\hyperref[def:Gequiv-bqp]{量子相}のことをSPT相と呼ぶ.SPT相の名前はこのことに由来する.


% \subsection{Fermionic SPT相}

% \section{Bosonic SPT相の分類:群コホモロジーによる方法}

% ~\cite[p.16, VIII]{ChenGuLiuWen2013}は,\hyperref[def:Dijkgraaf-Witten]{Dijkgraaf-Witten理論}を用いて
% \footnote{彼女らの論文においてはDijkgraaf-Witten理論との関連は明示的に書かれていない.~\cite{Wen2014SPT}には顕に書かれている.}
% かなり多くの\hyperref[def:SPT-traditional]{SPT相}を書き下す系統的な方法を発明した.
% この節ではその方法を紹介する.

% 簡単のため,\hyperref[def:Gequiv-bqp]{$G$-対称な格子模型}のうち群準同型 $\irm{\rho}{spa}$ が自明なもの\footnote{i.e. $\forall g \in G$ に対して $\irm{\rho}{spa}(g) = 1_S$}のみ考える.
% また,$G$ は局所コンパクト\footnote{任意の点がコンパクト近傍を持つ}であるとする.このとき $G$ はHaar測度を持つのでそれを $\int_G \dd{g}$ とおく.
% このとき,非ゼロな $\ket{\psi} \in \mathcal{H}_{\bm{x}}$ を1つ固定して $\forall g \in G$ に対して
% \begin{align}
%     \ket{g} \coloneqq \irm{\rho}{int}(g) \ket{\psi}
% \end{align}
% とおくと,Haar測度の左右不変性から族 $\Familyset[\big]{\ket{g}}{g \in G}$ は\href{https://en.wikipedia.org/wiki/Coherent_states_in_mathematical_physics}{一般化コヒーレント状態}を成す.
% ここで $\forall \Familyset[\big]{g_{\bm{x}}}{\bm{x} \in \Lambda} \in \prod_{\bm{x} \in \Lambda} G$ に対して
% \begin{align}
%     \ket{\Familyset[\big]{g_{\bm{x}}}{\bm{x} \in \Lambda}} \coloneqq \bigotimes_{\bm{x} \in \Lambda} \ket{g_{\bm{x}}} \in \irm{\mathcal{H}}{tot}
% \end{align}
% とおこう.さらに以下では $G$ は\underline{離散群}であるとする.

% \begin{myprop}[label=prop:SPT-CGLW]{SPT相の構成}
%     \begin{itemize}
%         \item 空間多様体 $\Sigma$ を境界にもつ $D+1$ 次元多様体 $\mfdcal{N}{D+1}$ 
%         \item $\mfdcal{N}{D+1}$ の三角形分割 $\abs{K} \xrightarrow{\approx} \mfdcal{N}{D+1}$ であって,その\hyperref[def:SimpSet]{0-単体}(頂点)$K_0$ が $\partial \mfdcal{N}{D+1}$ において格子 $\Lambda$ を再現するもの
%         \item $\omega \in \coGrp{D+1}{G,\, \LU(1)}$
%     \end{itemize}
%     をとる.このとき
%     \begin{align}
%         \ket{\Psi}_\omega \coloneqq \frac{1}{\abs{G}^{\abs{\Lambda}}} \sum_{\{g_{j}\}_{j \in K_0}} \prod_{\{j_0,\, \dots,\, j_{D+1}\} \in K_{D+1}} \omega (g_{j_0},\, \dots,\, g_{j_{D+1}})^{\epsilon_{\{j_0,\, \dots,\, j_{D+1}\}}} \ket{\Familyset[\big]{g_{\bm{x}}}{\bm{x} \in \Lambda}}
%     \end{align}
%     は\hyperref[def:SPT-traditional]{SPT相}の代表元である.ただし $\epsilon_{\{j_0,\, \dots,\, j_{D+1}\}}$ は $D+1$-単体 $\{j_0,\, \dots,\, j_{D+1}\} \in K_{D+1}$ の向きである.
% \end{myprop}

% \begin{proof}
%     まず,$\ket{\Psi}_\omega$ が\hyperref[def:blat-G-equiv]{$G$-対称}であることを示す.
%     実際 $\forall \Familyset[\big]{g_{\bm{x}}}{\bm{x} \in \Lambda} \in \prod_{\bm{x} \in \Lambda} G$ および $\forall g \in G$ に対して
%     \begin{align}
%         \mel{\Familyset[\big]{g_{\bm{x}}}{\bm{x} \in \Lambda}}{\rho(g)}{\Psi}_\omega
%         &= \mel{\Familyset[\big]{g_{\bm{x}}}{\bm{x} \in \Lambda}}{\bigotimes_{\bm{x} \in \Lambda} \irm{\rho}{int} (g)}{\Psi}_\omega \\
%         &= \braket{\Familyset[\big]{g^{-1}g_{\bm{x}}}{\bm{x} \in \Lambda}}{\Psi}_\omega \\
%         &= \frac{1}{\abs{G}^{\abs{\Lambda}}} \sum_{\{g_{j}\}_{j \in K_0 \setminus \Lambda}} \prod_{\{j_0,\, \dots,\, j_{D+1}\} \in K_{D+1}} \omega (g_{j_0},\, \dots,\, g_{j_{D+1}})^{\epsilon_{\{j_0,\, \dots,\, j_{D+1}\}}} \\
%         &= \braket{\Familyset[\big]{g_{\bm{x}}}{\bm{x} \in \Lambda}}{\Psi}_\omega
%     \end{align}
%     が成り立つ.ただし3つ目の等号でコサイクルの左不変性を使った.

%     次に,$\ket{\Psi}_\omega$ が\hyperref[def:SRE]{SRE状態}であることを示す.簡単のため $\Sigma = S^D,\, \mfdcal{N}{D+1} = D^{D+1}$ の場合を考える\footnote{一般の場合でも,$C^\infty$ 多様体はCW複体の構造を持つので問題ないと思われる.}.
%     このとき $K_{0} \setminus \Lambda = \{*\}$ となるような三角形分割をとることができて,
%     \begin{align}
%         \ket{\Psi}_\omega 
%         &= \left( \sum_{\{g_{\bm{x}}\}_{\bm{x} \in \Lambda}} \prod_{\{j_0,\, \dots ,\, j_{D},\, *\} \in K_{D+1}} \omega (g_{j_0},\, \dots,\, g_{j_D},\, g_{*}) \ketbra{\Familyset[\big]{g_{\bm{x}}}{\bm{x} \in \Lambda}}{\Familyset[\big]{g_{\bm{x}}}{\bm{x} \in \Lambda}} \right) \bigotimes_{\bm{x} \in \Lambda} \left( \frac{1}{\abs{G}} \sum_{g_{\bm{x}} \in G} \ket{g_{\bm{x}}}\right) \\
%         &= \left( \prod_{\{j_0,\, \dots ,\, j_{D},\, *\} \in K_{D+1}} \sum_{\{g_{j_n}\}_{n=0}^D \in G^{D+1}} \omega (g_{j_0},\, \dots,\, g_{j_D},\, g_{*}) \ketbra{\{g_{j_n}\}_{n=0}^D}{\{g_{j_n}\}_{n=0}^D} \right) \bigotimes_{\bm{x} \in \Lambda} \left( \frac{1}{\abs{G}} \sum_{g_{\bm{x}} \in G} \ket{g_{\bm{x}}}\right) 
%     \end{align}
%     と書けるので\hyperref[def:SRE]{SRE状態}である.

%     最後に,$\ket{\Psi}$ が属する\hyperref[def:SPT-traditional]{SPT相}が $\omega \in \coGrp{D+1}{G,\, \LU(1)}$ の代表元の取り方によらないことを示す.実際,$D$-コチェイン $\eta \in C^D_{\mathrm{Grp}}(G,\, \LU(1))$ に対して
%     $\omega \lmto \omega \cdot \delta \eta$ と取り替えると
%     \begin{align}
%         \ket{\Psi}_{\omega \cdot \delta \eta}
%         &= \left(\prod_{\{j_0,\, \dots,\, j_{D},\, *\} \in K_{D+1}} \sum_{\{g_{j_0},\, \dots,\, g_{j_D}\} \in G^{D+1}} \eta (g_{j_0},\, \dots,\, g_{j_D}) \ketbra{\{g_{j_n}\}_{n=0}^D}{\{g_{j_n}\}_{n=0}^D} \right) \ket{\Psi}_\omega
%     \end{align}
%     となるが,この変換は明らかに $\rho(g)$ と可換なので $G$-同変な\hyperref[prop:LU]{局所ユニタリ発展}である.
% \end{proof}

% \hyperref[def:Dijkgraaf-Witten]{Dijkgraaf-Witten理論}との関係は,大域的 $G$-対称性をゲージ化することにより明らかになる~\cite[APPENDIX E]{Wen2014SPT}.
% ゲージ化によって,$\mfdcal{N}{D+1}$ の三角形分割の $1$-単体(辺)$e \in K_1$ 上に $G$ の元 $h_e$ が指定される.ただし,$G$ が離散群なので $h_{e}$ は\hyperref[def:flat-connection-homotopy]{平坦接続}でなくてはならない.よってもし3つの $1$-単体 $e_1,\, e_2,\, e_3$ がある $2$-単体 $d$ について $\partial_i^2 (d) = e_i$ を充たすならば
% \begin{align}
%     h_{e_1} h_{e_2} h_{e_3} = 1_G
% \end{align}
% が成り立たねばならない.また,「物質場」$g_{x}$ とゲージ場 $h_{e}$ のゲージ変換は,$\Familyset[\big]{k_x}{x \in K_0} \in G^{\abs{K_0}}$ を用いてそれぞれ
% \begin{align}
%     g_x &\lmto k_x g_x, \\
%     h_e &\lmto k_{\partial_1^1(e)}^{-1} h_e k_{\partial_0^1(e)}
% \end{align}
% のようになる.故に,命題\ref{prop:SPT-CGLW}の構成で用いた「物質場」の分配関数
% \begin{align}
%     Z(\Familyset[\big]{g_{i}}{i \in K_0};\, \mfdcal{N}{D+1}) \coloneqq \prod_{\{j_0,\, \dots,\, j_{D+1}\} \in K_{D+1}} \omega (g_{j_0},\, \dots,\, g_{j_{D+1}})^{\epsilon_{\{j_0,\, \dots,\, j_{D+1}\}}}
% \end{align}
% は $G$-対称性のゲージ化によって
% \begin{align}
%     &\irm{Z}{gauged}(\Familyset[\big]{g_{i}}{i \in K_0},\, \Familyset[\big]{h_{e}}{e \in K_1};\,\mfdcal{N}{D+1}) \\   
%     &= \frac{1}{\abs{G}^{\abs{\Lambda}}} \prod_{\{j_0,\, \dots,\, j_{D+1}\} \in K_{D+1}} \omega (g_{j_0},\, h_{j_0j_1} g_{j_1},\, h_{j_0j_1}h_{j_1j_2} g_{j_2} \dots,\, h_{j_0j_1} \cdots h_{j_D j_{D+1}}g_{j_{D+1}})^{\epsilon_{\{j_0,\, \dots,\, j_{D+1}\}}} \\
%     &= \frac{1}{\abs{G}^{\abs{\Lambda}}} \prod_{\{j_0,\, \dots,\, j_{D+1}\} \in K_{D+1}} \alpha (g_{j_0}^{-1}h_{j_0j_1} g_{j_1},\, g_{j_1}^{-1} h_{j_1j_2} g_{j_2},\, \dots,\,g_{j_D}^{-1} h_{j_D j_{D+1}}g_{j_{D+1}})^{\epsilon_{\{j_0,\, \dots,\, j_{D+1}\}}}
% \end{align}
% になる\footnote{このゲージ化の方法は,理論のゲージ不変性を要請することによって得られる.}.ゲージ場を外場と見做すことにより\hyperref[def:Dijkgraaf-Witten]{Dijkgraaf-Witten理論}の作用が得られる:
% \begin{align}
%     &\sum_{\{g_{j}\}_{j \in K_1}} \irm{Z}{gauged}(\Familyset[\big]{g_{i}}{i \in K_0},\, \Familyset[\big]{h_{e}}{e \in K_1};\,\mfdcal{N}{D+1}) \\
%     &= \prod_{\{j_0,\, \dots,\, j_{D+1}\} \in K_{D+1}} \alpha (h_{j_0j_1},\, h_{j_1j_2},\, \dots,\, h_{j_D j_{D+1}})^{\epsilon_{\{j_0,\, \dots,\, j_{D+1}\}}} \\
%     &= e^{2\pi \iunit \expval{\gamma^* \alpha,\, []}}
% \end{align}
% 以上の議論により,$D$ 次元のbosonicな\footnote{より正確には $G$ が離散群でかつon site symmetryのとき}\hyperref[def:SPT-traditional]{SPT相}の分類は $\coGrp{D+1}{G,\, \LU(1)}$ によって成される,などと言う~\cite{ChenGuLiuWen2013}.

% \section{Bosonic SPT相の分類:$\Omega$-スペクトラムによる方法}

% 上述の群コホモロジーによるbosonicなSPT相の分類は低次元においては十分有効だが,高次元だと不十分になったり,逆に細かくなり過ぎることが知られている~\cite{Kapustin2014SPT}.現代的には一般コホモロジー理論によって分類することが多い~\cite{Xiong2019SPT}.
% その際には,そもそも\hyperref[prop:LU]{局所ユニタリ発展}は使わずにSPT相を定義する.





\end{document}