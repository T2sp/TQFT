\documentclass[TQFT_main]{subfiles}

\begin{document}

\setcounter{chapter}{1}

\newcommand{\btl}{\blacktriangleleft}
\newcommand{\btr}{\blacktriangleright}
\newcommand{\Lie}[1]{#1^{\mathrm{L}}}
\newcommand{\evalunit}{\mathrm{ev}_{1_G}}
\mathchardef\mhyphen="2D

\chapter{Chern-Simons理論の導入}

この章は~\cite[Chapter4, 5]{Simon2021}に相当する.

\section{Charge-Flux composite}

\subsection{Aharonov-Bohm効果}

空間を表す多様体を $\Sigma$ と書く.電荷 $q$ を持つ1つの粒子からなる系を考えよう.この系に静磁場をかけたとき,粒子の古典的作用は自由粒子の項 $S_0$ と,粒子と場の結合を表す項とに分かれる:
\begin{align}
    S[l] = S_0[l] + q \int_{\irm{t}{i}}^{\irm{t}{f}} \dd{t} \dot{\bm{x}} \vdot \bm{A} = S_0[l] + q \int_l \dd{\bm{x}} \vdot \bm{A}
\end{align}
ただし $l \colon [\irm{t}{i},\, \irm{t}{f}] \lto \Sigma$ は粒子の軌跡を表す.

ここで,いつもの2重スリットを導入する.粒子が $\irm{\bm{x}}{i} = \bm{x}(\irm{t}{i})$ から出発して $\irm{\bm{x}}{f} = \bm{x}(\irm{t}{f})$ に到達するとき,これらの2点を結ぶ経路全体の集合 $\mathcal{C}(\irm{\bm{x}}{i},\, \irm{\bm{x}}{t})$ のホモトピー類は,スリット $1,\, 2$ を通る経路それぞれでちょうど $2$ つある.i.e.
プロパゲーターは経路積分によって
\begin{align}
    \sum_{l \in \mathcal{C}(\irm{\bm{x}}{i},\, \irm{\bm{x}}{t})\; \mathrm{s.t.}\;\mathrm{slit}\; 1} e^{\iunit S_0 [l] / \hbar + \iunit (q / \hbar) \int_l \dd{\bm{x}} \vdot \bm{A}} + \sum_{l \in \mathcal{C}(\irm{\bm{x}}{i},\, \irm{\bm{x}}{t})\; \mathrm{s.t.}\;\mathrm{slit}\; 2} e^{\iunit S_0 [l] / \hbar + \iunit (q / \hbar) \int_l \dd{\bm{x}} \vdot \bm{A}}
\end{align}
と計算される.第1項と第2項の位相差は,片方の経路の逆をもう片方に足すことでできる閉曲線 $\partial S$ について
\begin{align}
    \exp \left[ \frac{\iunit q}{\hbar} \oint_{\partial S} \dd{\bm{x}} \vdot \bm{A} \right] = \exp \left[ \frac{\iunit q}{\hbar} \int_S \dd{\bm{S}} \vdot (\curl{\bm{A}}) \right] = \exp \left[ \frac{\iunit q}{\hbar} \Phi_{S}\right] 
\end{align}
となる\footnote{粒子が侵入できない領域にのみ磁場がかかっているとする.なお,粒子の配位空間が単連結でないことが本質的に重要である.このとき,領域 $S$ をホモトピーで1点に収縮することで,無限に細い管状の磁束 (flux tube) の概念に到達する.}.

\begin{enumerate}
    \item 磁束が $\Phi_0 = 2\pi \hbar / q $ の整数倍の時は,位相シフトがない場合と物理的に区別がつかない.
    \item 実は,静止した電荷の周りに磁束を動かしても全く同じ位相シフトが引き起こされる~\cite{Aharonov1984}.
\end{enumerate}

\subsection{Charge-Flux compositeとしてのエニオン}

荷電粒子と無限に細い磁束管 (flux tube) が互いに束縛し合って近接しているものを考える.この対を2次元系における,$(q,\, \Phi)$ なるチャージを持つ1つの粒子と見做してみよう.

さて,粒子 $i (=1,\, 2)$ がチャージ $(q,\, \Phi)$ を持つとしよう.この2つの同種粒子の配位空間の基本群は前章の議論から $\mathbb{Z}_2$ であり,
\begin{enumerate}
    \item 粒子 $1$ を $2$ の周りに1周させる操作
    \item 粒子の交換を\underline{2回}行う操作
\end{enumerate}
の2つが同じホモトピー類に属すことがわかる.故に,これら2つの操作で得られる位相シフトは等しい.
操作 (1) による位相シフトはAB効果によるもので,$e^{2\iunit q \Phi /\hbar}$ である\footnote{2がつくのは,粒子1の $q$ が粒子2の $\Phi$ の周りを1周するAB効果だけでなく,粒子1の $\Phi$ が粒子2の $q$ の周りを1周するAB効果の寄与があるからである.一般に,粒子 $i$ のチャージが $(q_i,\, \Phi_i)$ ならば $e^{\iunit (q_1 \Phi_2 + q_2 \Phi_1) / \hbar}$ の位相シフトが起こる.}.
故に,この粒子が\underline{1回}交換することによって得られる位相シフトは $e^{\iunit q \Phi /\hbar}$ であるが,これは $\theta = q \Phi / \hbar$ なる可換エニオンの統計性である.


次に,エニオンの\textbf{フュージョン} (fusion) を経験的に導入する.これは,エニオン $(q_1,\, \Phi_1),\, (q_2,\, \Phi_2)$ が「融合」してエニオン $(q_1 + q_2,\, \Phi_1 + \Phi_2)$ になる,と言うものであり,今回の場合だと電荷,磁束の保存則に由来すると考えることができる.
エニオン $(q,\, \Phi)$ と $(-q,\, -\Phi)$ がフュージョンすると $I \coloneqq (0,\, 0)$ になるだろう.この $I$ をエニオンの真空とみなし\footnote{しかし,$I$ のことは粒子として捉える.},$(-q, -\Phi)$ のことを $(q,\, \Phi)$ の反エニオン (anti-anyon) と見做す.
反エニオンをエニオンの周りに一周させたときの位相シフトが $e^{-2\iunit \theta}$ になることには注意すべきである.

\subsection{トーラス上のエニオンの真空}

トーラス $T^1 \coloneqq S^1 \times S^1$ の上のエニオン系の基底状態(真空)を考える.

トーラスには非自明なサイクルがちょうど2つあるので,それらを $C_1,\, C_2$ とおく.そして系の時間発展演算子のうち,次のようなものを考える:
\begin{description}
    \item[\textbf{$\hat{T}_1$}] ある時刻に $C_1$ の1点において粒子-反粒子対を生成し,それらを $C_1$ 上お互いに反対向きに動かし,有限時間経過後に $C_1$ の対蹠点で対消滅させる.
    \item[\textbf{$\hat{T}_2$}] ある時刻に $C_2$ の1点において粒子-反粒子対を生成し,それらを $C_2$ 上お互いに反対向きに動かし,有限時間経過後に $C_2$ の対蹠点で対消滅させる.
\end{description}
$\hat{T}_1,\, \hat{T}_2$ は非可換であり,基底状態への作用を考える限り,フュージョンダイアグラムとbraidingの等式から
\begin{align}
    \label{eq:2-T-comm}
    \hat{T}_2 \hat{T}_1 = e^{-\iunit 2 \theta} \hat{T}_1 \hat{T}_2
\end{align}
が成り立つことが分かる.然るに,基底状態が張る部分空間に制限すると $\comm{T_1}{H} = \comm{T_2}{H} = 0$ なので\footnote{基底状態 $\ket{0}$ と $\hat{T}_1\ket{0}$ は同じエネルギーである.},基底状態が縮退していることがわかる.

さて,$T_i$ はユニタリなので,$T_1 \ket{\alpha} = e^{\iunit \alpha} \ket{\alpha}$ とおける.この時\eqref{eq:2-T-comm}より
\begin{align}
    T_1 (T_2 \ket{\alpha}) = e^{\iunit (\alpha + 2\theta)} T_2 \ket{\alpha}
\end{align}
である.つまり,$\ket{\alpha}$ が基底状態ならば $\ket{\alpha + 2\theta} = T_2 \ket{\alpha}$ もまた基底状態である.この操作を続けて,基底状態 $\ket{\alpha + 2n \theta} = (T_2)^n \ket{\alpha}\; (n \in \mathbb{Z}_{\ge 0})$ を得る.
特に $\theta = \pi p / m\quad (p,\, m\; \text{は互いに素})$ である場合を考えると,基底状態は $m$ 重縮退を示している.




\section{可換Chern-Simons理論の経験的導入}

ゲージ場\footnote{一般相対論に倣い,時空を表す多様体 $\mathcal{M}$ の座標のうち時間成分を $x^0$,空間成分を $x^1,\, x^2$ とする.} $A_\alpha = (a_0,\, a_1,\, a_2)$ が印加された $N$ 粒子2次元系であって,ラグランジアンが
\begin{align}
    \label{eq:lagrangian-SC}
    L = L_0 + \int_\Sigma \dd[2]{x} \left( \frac{\mu}{2} \epsilon^{\alpha\beta\gamma} A_{\alpha} \partial_\beta A_\gamma -  j^\alpha A_\alpha \right) \eqqcolon L_0 + \int_{\Sigma} \dd[2]{x} \mathcal{L}
\end{align}
と書かれるものを考える.ただし,$L_0$ は場と粒子の結合を無視したときの粒子のラグランジアンであり,空間を表す多様体を $\Sigma$ で書いた.
粒子 $n$ はチャージ $q_n$ を持つものとし,$j^\alpha = (j^0,\, \bm{j})$ は
\begin{align}
    j^0(\bm{x}) &\coloneqq \sum_{n = 1}^N q_n \delta (\bm{x} - \bm{x}_n), \\
    \bm{j}(\bm{x}) &\coloneqq \sum_{n = 1}^N q_n \dot{\bm{x}}_n\delta (\bm{x} - \bm{x}_n)
\end{align}
と定義される粒子のカレントである.
ラグランジアン密度 $\mathcal{L}$ の第1項は場自身を記述し,第2項は場と粒子の結合を記述する.
% 時空を表す多様体を $\mathcal{M}$ と書く.

\subsection{ゲージ不変性}

ラグランジアン\eqref{eq:lagrangian-SC}のゲージ不変性は次のようにしてわかる:
ゲージ変換
\begin{align}
    A_\alpha \lto A_\alpha + \partial_\alpha \chi
\end{align}
による $\mathcal{L}$ の変化は
\begin{align}
    \frac{\mu}{2} \epsilon^{\alpha\beta\gamma} \partial_{\alpha}\chi \partial_\beta A_\gamma + \cancel{\frac{\mu}{2} \epsilon^{\alpha\beta\gamma} A_\alpha\partial_\beta \partial_\gamma \chi + \frac{\mu}{2} \epsilon^{\alpha\beta\gamma} \partial_{\alpha}\chi \partial_\beta \partial_\gamma \chi} 
    -  j^\alpha \partial_\alpha \chi
\end{align}
であるから,空間積分を実行すると
\begin{align}
    &\int_{\Sigma} \dd[2]{x} \frac{\mu}{2} \partial_\alpha \bigl( \epsilon^{\alpha\beta\gamma} \chi \partial_\beta A_\gamma \bigr) - \int_{\Sigma} \dd[2]{x} \frac{\mu}{2} \cancel{\epsilon^{\alpha\beta\gamma} \chi \partial_\alpha\partial_\beta A_\gamma }
    - \int_{\Sigma} \dd[2]{x} \partial_\alpha \bigl( j^\alpha \chi \bigr) + \int_{\Sigma} \dd[2]{x} \cancel{\partial_\alpha} j^\alpha \chi \\
    &= \int_{\partial \Sigma} \dd S_\alpha \left(\frac{\mu}{2} \epsilon^{\alpha\beta\gamma} \chi \partial_\beta A_\gamma - j^\alpha \chi \right)
\end{align}
となる.ただしチャージの保存則 $\partial_\alpha j^\alpha = 0$ を使った.このことから,もし空間を表す多様体 $\Sigma$ の境界が $\partial \Sigma = \emptyset$ ならば\footnote{このような多様体の中で重要なのが\textbf{閉多様体} (closed manifold) である.}ラグランジアンはゲージ不変である.

\subsection{運動方程式}

ラグランジアン密度 $\mathcal{L}$ から導かれるEuler-Lagrange方程式は
\begin{align}
    \pdv{\mathcal{L}}{A_\alpha} = \partial_\beta \left( \pdv{\mathcal{L}}{(\partial_\beta A_\alpha)} \right) 
\end{align}
である.
\begin{align}
    \pdv{\mathcal{L}}{A_\alpha} 
    &= \frac{\mu}{2} \epsilon^{\alpha \beta\gamma }\partial_\beta A_\gamma - j^\alpha, \\
    \partial_\beta \left( \pdv{\mathcal{L}}{(\partial_\beta A_\alpha)} \right) 
    &= \partial_\beta \left( \frac{\mu}{2} \epsilon^{\alpha \beta \gamma} A_\alpha \right) = -\frac{\mu}{2} \epsilon^{\alpha\beta\gamma} \partial_{\beta} A_\gamma
\end{align}
なのでこれは
\begin{align}
    j^\alpha = \mu \epsilon^{\alpha\beta\gamma} \partial_\beta A_\gamma
\end{align}
となる.特に第0成分は,「磁場」$\bm{b} \coloneqq \curl{\bm{A}}$ を導入することで
\begin{align}
    \sum_{n=1}^N \frac{q_n}{\mu} \delta (\bm{x} - \bm{x}_n) = b^0
\end{align}
となる.つまり,位置 $\bm{x}_n$ に強さ $q_n / \mu$ の磁束管が点在している,という描像になり,charge-flux compositeを説明できている.

\subsection{プロパゲーター}

簡単のため,全ての粒子のチャージが等しく $q$ であるとする.$N$ 粒子の配位空間 $\mathcal{C}$ における初期配位と終了時の配位をそれぞれ $\{\irm{\bm{x}}{i}\},\, \{\irm{\bm{x}}{f}\}$ とし,それらを繋ぐ経路全体の集合を $\mathcal{C}(\irm{\bm{x}}{i},\, \irm{\bm{x}}{f})$ と書くと,プロパゲーターは経路積分によって
\begin{align}
    \sum_{l \in \mathcal{C}(\irm{\bm{x}}{i},\, \irm{\bm{x}}{f})} e^{\iunit S_0[l]/\hbar } \int_{\mathcal{M}} \mathcal{D}\bigl( A_\mu (x)\bigr)\; e^{\iunit \irm{S}{CS}[A_\mu (x)]/\hbar} e^{\iunit (q/\hbar) \int_l \dd{x^\alpha} A_\alpha (x) }
\end{align}
と計算される.ここに $\mathcal{D}\bigl(A_\mu (x)\bigr)$ は汎函数積分の測度を表す.詳細は後述するが,場に関する汎函数積分を先に実行してしまうと,実は
\begin{align}
    \sum_{l \in \mathcal{C}(\irm{\bm{x}}{i},\, \irm{\bm{x}}{f})} e^{\iunit S_0 [l] /\hbar + \iunit \theta W(l)}
\end{align}
の形になることが知られている.ここに $W(l)$ は,経路 $l$ の巻きつき数である.経路に依存する位相因子 $e^{\iunit \theta W(l)}$ は前章で議論した $\pi_1 \mathcal{C}$ の1次元ユニタリ表現そのものであり,エニオンの統計性が発現する機構がChern-Simons項により説明できることを示唆している.

\subsection{真空中の可換Chern-Simons理論}

粒子が存在しないとき,経路積分は
\begin{align}
    Z(\mathcal{M}) \coloneqq \int_{\mathcal{M}} \mathcal{D} A_\mu (x)\, e^{\iunit \irm{S}{CS}[A_\mu (x)]/\hbar}
\end{align}
の形をする.$Z(\mathcal{M})$ は $\mathcal{M}$ についてホモトピー不変であり,\textbf{分配関数} (partition function) と呼ばれる.$Z(\mathcal{M})$ がTQFTにおいて重要な役割を果たすことを後の章で見る.

\subsection{正準量子化}

$A_0 = 0$ なるゲージをとると,ラグランジアン密度におけるChern-Simons項は $-A_1 \partial_0 A_2 + A_2 \partial_0 A_1$ の形になる.これは $A_1$ (resp. $A_2$)が $A_2$ (resp. $A_1$)の共役運動量であることを意味するので,正準量子化を行うならば
\begin{align}
    \comm{A_1(\bm{x})}{A_2(\bm{y})} = \frac{\iunit \hbar}{\mu} \delta^2 (\bm{x} - \bm{y})
\end{align}
を要請する\footnote{しかし,トーラス上の座標をどのように取るかと言うことは問題である.}.

さて,このときトーラス $T^2$ 上の2つのサイクル $C_1,\, C_2$ に対してWilsonループ
\begin{align}
    W_j = \exp \left( \frac{\iunit q}{\hbar} \oint_{C_j} \dd{\bm{x}} \vdot \bm{A} \right) 
\end{align}
を考える.
$\comm{A}{B}$ がc数である場合のBCH公式から
\begin{align}
    W_1 W_2 &= e^{\iunit q^2 / (\mu \hbar)} W_2 W_1
\end{align}
を得るが,これは\eqref{eq:2-T-comm}を説明している.つまり,演算子 $T_1,\, T_2$ とはWilson loopのことだったのである\footnote{疑問:座標の時間成分はどこへ行ったのか?}.

\section{非可換Chern-Simons理論の経験的導入}

この節では自然単位系を使う.前節を一般化して,ゲージ場 $A_\mu (x)$ があるLie代数 $\mathfrak{g}$ に値をとるものとしよう.つまり,Lie代数 $\mathfrak{g}$ の基底を $\sigma_a / (2\iunit)$ とすると\footnote{因子 $1/(2\iunit)$ は物理学における慣習である.ややこしいことに,文献によってこの因子が異なる場合がある.}
\begin{align}
    A_\mu (x) = A_\mu^a (x) \frac{\sigma_a}{2\iunit}
\end{align}
と書かれるような状況を考える\footnote{ゲージ接続がLie代数に値をとる1-形式である,ということ.}.$\sigma_a \in \mathfrak{g}$ が一般に非可換であることから,このような理論は非可換Chern-Simons理論と呼ばれる.

時空多様体 $\mathcal{M}$ 上の閉曲線 $\gamma$ に沿った\textbf{Wilson loop}は,\textbf{経路順序積} (path ordering) $\mathcal{P}$ を用いて
\begin{align}
    W_\gamma \coloneqq \Tr \left[ \mathcal{P} \exp \left( \oint_\gamma \dd{x}^\mu A_\mu (x) \right)  \right] 
\end{align}
と定義される.Aharonov-Bohm位相の一般化という気持ちであるが,経路 $\gamma$ の異なる2点 $x,\, y$ を取ってきたときに $A_\mu (x)$ と $A_\mu (y)$ が一般に非可換であることが話をややこしくする.
% 経路順序積 $\mathcal{P}$ は
% \begin{align}
%     \mathcal{P} \exp \left( \oint_l \dd{x}^\mu A_\mu (x) \right) 
%     \coloneqq 
% \end{align}
\subsection{ゲージ不変性}

非可換Chern-Simons理論におけるゲージ変換は,$U \colon \mathcal{M} \lto G$ を用いて
\begin{align}
    \label{eq:gauge-transform}
    A_\mu(x) \lto U(x) \bigl(A_\mu(x) + \partial_\mu\bigr) U(x)^{-1}
\end{align}
の形をする.このゲージ変換がWilson loopを不変に保つことを確認しておこう.

$\mathcal{M}$ の任意の2点 $\irm{x}{i},\, \irm{x}{f} \in \mathcal{M}$ を結ぶ曲線 $\gamma \colon [\irm{t}{i},\, \irm{t}{f}] \lto \mathcal{M}$ をとり,\textbf{Wilson line}を
\begin{align}
    \tilde{W}_\gamma (x,\, y) \coloneqq \mathcal{P} \exp \left( \int_\gamma \dd{x^\mu} A_\mu (x) \right)
\end{align}
で定義する.$[\irm{t}{i},\, \irm{t}{f}]$ の分割 $\irm{t}{i} \eqqcolon t_0 < t_1 < \dots < t_N \coloneqq \irm{t}{f}$ を与えて $x_i \coloneqq \gamma(t_i),\; \dd{x_i} \coloneqq x_{i+1} - x_{i}$ とおく\footnote{$\dd{x_i}$ は,厳密には2点 $x_i,\, x_{i+1}$ を含むある $\mathcal{M}$ のチャート $\bigl( U,\, (x^\mu) \bigr)$ をとってきた時の座標関数の値の差 $\dd{x_i}^\mu \coloneqq x^\mu(x_{i+1}) - x^\mu(x_{i})$ として理解する.}と,
\begin{align}
    \tilde{W}_\gamma (\irm{x}{i},\, \irm{x}{f}) 
    &= \mathcal{P} \exp \left( \int_{\irm{x}{i}}^{x_1} \dd{x}^\mu A_\mu (x) + \int_{x_1}^{x_2} \dd{x}^\mu A_\mu (x) + \cdots + \int_{x_{N-1}}^{\irm{x}{f}} \dd{x}^\mu A_\mu (x) \right) \\
    &\coloneqq \lim_{N \to \infty} \exp \left(\int_{\irm{x}{i}}^{x_1} \dd{x}^\mu A_\mu (x)\right) \exp \left( \int_{x_1}^{x_2} \dd{x}^\mu A_\mu (x) \right) \cdots \exp \left( \int_{x_{N-1}}^{\irm{x}{f}} \dd{x}^\mu A_\mu (x) \right) \\
    &= \lim_{N \to \infty} \tilde{W}_{\gamma|_{[\irm{t}{i},\, t_1]}} (\irm{x}{i},\, x_1) \tilde{W}_{\gamma|_{[t_1,\, t_2]}} (x_1,\, x_2) \cdots \tilde{W}_{\gamma|_{[t_{N-1},\, \irm{t}{f}]}} (x_{N-1},\, \irm{x}{f}) 
\end{align}
と書ける.
% 無限小だけ離れた2点 $x,\, x + \dd{x} \in \mathcal{M}$ を取ってくると
% \begin{align}
%     \tilde{W}_C (x,\, x + \dd{x}) = 1 + A_\mu (x)\dd{x^\mu}
% \end{align}
% と書けるので,
$N$ が十分大きい時は $0 \le \forall i \le N-1$ に対して $\abs{\dd{x_i}}$ が十分小さく,
\begin{align}
    \tilde{W}_{\gamma|_{[t_{i},\, t_{i+1}]}} (x_{i},\, x_{i+1})
    &\approx \exp \left(\int_{x_i}^{x_{i+1}} \dd{x}^\mu A_\mu (x)\right) \\ 
    &\approx 1 + \int_{x_{i}}^{x_{i+1}} \dd{x}^\mu A_\mu (x) \\
    &\approx 1 + A_\mu (x_i)\, \dd{x_i}^\mu
\end{align}
と書ける\footnote{$N \to \infty$ の極限で等式になる.}.このときゲージ変換\eqref{eq:gauge-transform}に伴って
\begin{align}
    \tilde{W}_{\gamma|_{[t_{i},\, t_{i+1}]}} (x_{i},\, x_{i+1})
    &\lto 1 + U(x_{i}) \bigl(A_\mu (x_{i}) + \partial_\mu\bigr) U(x_{i})^{-1}\, \dd{x_{i}}^\mu \\
    &\approx U(x_i)\bigl( 1 + A_\mu (x_i) \dd{x_i}^\mu \bigr) \bigl(U(x_{i})^{-1} + \partial_\mu (U(x_{i})^{-1})\, \dd{x_i}^\mu \bigr) \\
    &\approx U(x_i) \tilde{W}_{\gamma|_{[t_{i},\, t_{i+1}]}} (x_{i},\, x_{i+1}) U(x_{i+1})^{-1}
\end{align}
と変換するので,結局 $\irm{x}{i},\, \irm{x}{f} \in \mathcal{M}$ を繋ぐWlison lineがゲージ変換\eqref{eq:gauge-transform}に伴って
\begin{align}
    \tilde{W}_\gamma (\irm{x}{i},\, \irm{x}{f}) \lto U(\irm{x}{i}) \tilde{W}_\gamma (\irm{x}{i},\, \irm{x}{f}) U(\irm{x}{f})^{-1}
\end{align}
と変換することがわかった.Wlison loopの場合は $\irm{x}{i} = \irm{x}{f}$ でかつトレースをとるので,ゲージ不変になる.

\subsection{Chern-Simons作用}

いささか天下り的だが,\textbf{Chern-Simons action}を
\begin{align}
    \irm{S}{CS}[A_\mu] \coloneqq \frac{k}{4\pi} \int_{\mathcal{M}} \dd[3]{x} \epsilon^{\alpha\beta\gamma} \Tr \left[ A_\alpha \partial_\beta A_\gamma + \frac{2}{3} A_\alpha A_\beta A_\gamma \right] 
\end{align}
により定義する.第2項は可換な場合には必ず零になるので前節では登場しなかった.
$\irm{S}{CS}[A]$ が時空 $\mathcal{M}$ の計量によらない\footnote{\textbf{計量不変} (metric invariant) であると言う.}ことは,ゲージ場をLie代数値1-形式 $A \in \Omega^1(\mathcal{M}) \otimes \mathfrak{g}$ として書き表したときに
\begin{align}
    \irm{S}{CS}[A] = \frac{k}{4\pi} \int_{\mathcal{M}} \Tr \left( A \wedge \dd{A} + \frac{2}{3} A \wedge A \wedge A \right) 
\end{align}
と書けることからわかる\footnote{...と言うのは微妙に的を外している.より正確には $2+1$ 次元多様体 $\mathcal{M}$ を境界に持つような4次元多様体 $\mathcal{N}$ を用意し,$\mathcal{N}$ の作用 $\mathcal{S}[A] \coloneqq k/(4\pi) \int_{\mathcal{N}} \Tr (F \wedge F)$ を部分積分することで $\irm{S}{CS}[A]$ を定義する.}.
% を,成分計算により確認しておこう.$\mathcal{M}$ の一般座標変換 $(x^\mu) \lto (x'{}^\mu)$ を考えたとき,

$\irm{S}{CS}[A]$ にゲージ変換\eqref{eq:gauge-transform}を施した結果は
\begin{align}
    &\irm{S}{CS}[A_\mu] \lto \irm{S}{CS}[A_\mu] + 2\pi \nu k, \label{eq:CS-gauge} \\
    \WHERE &\nu \coloneqq \frac{1}{24 \pi^2} \int_{\mathcal{M}} \dd[3]{x} \epsilon^{\alpha\beta\gamma} \Tr \bigl[ (U^{-1} \partial_\alpha U)(U^{-1} \partial_\beta U)(U^{-1} \partial_\gamma U) \bigr] 
\end{align}
となる.$\nu$ は写像 $U \colon \mathcal{M} \lto G$ の\textbf{巻きつき数} (winding number),もしくは\textbf{Pontryagin index}と呼ばれ,常に整数値をとる.この極めて非自明な結果についても後述する.
\eqref{eq:CS-gauge}から,$\irm{S}{CS}[X]$ は厳密にはゲージ不変ではない.然るに,もし $k \in \mathbb{Z}$ ならば(このとき $k$ の値は\textbf{level}と呼ばれる),分配関数 $Z (\mathcal{M})$ がゲージ不変な形になってくれるので問題ない,と考える.
$2+1$ 次元においては,1つのゲージ場からなる作用であって
\begin{itemize}
    \item トポロジカル不変性(i.e. 計量不変性)
    \item 上述の意味のゲージ不変性
\end{itemize}
の2つを充たすものは他にない.

% \subsection{Wilson loopと結び目不変量}

% Wilson loopの真空期待値が結び目不変量になることが知られている.

% \begin{align}
%     &\irm{S}{CS} = \frac{k}{4\pi} \int_{\mathcal{M}} \dd[3]{x} \epsilon^{\alpha\beta\gamma} \Tr \left[ A_\alpha \partial_\beta A_\gamma + \frac{2}{3} A_\alpha A_\beta A_\gamma \right] \\
%     &\lto \frac{k}{4\pi} \int_{\mathcal{M}} \dd[3]{x} \epsilon^{\alpha\beta\gamma} \Tr \left[ (U^{-1} A_\alpha U + U^{-1} \partial_\alpha U) \partial_\beta (U^{-1} A_\gamma U + U^{-1} \partial_\gamma U) \right. \\
%     &\quad \left. + \frac{2}{3} (U^{-1} A_\alpha U + U^{-1} \partial_\alpha U) (U^{-1} A_\beta U + U^{-1} \partial_\beta U) (U^{-1} A_\gamma U + U^{-1} \partial_\gamma U) \right] \\
%     &= \frac{k}{4\pi} \int_{\mathcal{M}} \dd[3]{x} \epsilon^{\alpha\beta\gamma} \Tr \left[ U^{-1} A_\alpha U (\partial_\beta U^{-1}) A_\gamma U + U^{-1} A_\alpha (\partial_\beta A_\gamma) U + U^{-1} A_\alpha A_\gamma (\partial_\beta U)  \right. \\
%     &\quad + U^{-1} (\partial_\alpha U) (\partial_\beta U^{-1}) A_\gamma U + U^{-1} (\partial_\alpha U)(\partial_\beta A_\gamma) U + U^{-1} (\partial_\alpha U) A_\gamma (\partial_\beta U) \\
%     &\quad + U^{-1} A_\alpha U (\partial_\beta U^{-1}) (\partial_\gamma U) + U^{-1} A_\alpha (\partial_\beta\partial_\gamma U) \\
%     &\quad + U^{-1} (\partial_\alpha U) U (\partial_\beta U^{-1}) (\partial_\gamma U) + U^{-1} (\partial_\alpha U) (\partial_\beta\partial_\gamma U) \\
%     &\quad + \frac{2}{3} U^{-1} A_\alpha A_\beta A_\gamma U \\
%     &\quad + \frac{2}{3} U^{-1} \bigl( (\partial_\alpha U) U^{-1} A_\beta A_\gamma + A_\alpha (\partial_\beta U) U^{-1} A_\gamma + A_\alpha A_\beta (\partial_\gamma U) U^{-1}\bigr) U \\
%     &\quad + \frac{2}{3} U^{-1} \bigl( (\partial_\alpha U) U^{-1} (\partial_\beta U) A_\gamma + A_\alpha (\partial_\beta U) U^{-1} (\partial_\gamma U) U^{-1} + (\partial_\alpha U) U^{-1} A_\beta (\partial_\gamma U) U^{-1}\bigr) U \\
%     &\quad \left.+ \frac{2}{3} U^{-1} (\partial_\alpha U) U^{-1} (\partial_\beta U) (\partial_\gamma U)  \right] \\
%     &= \irm{S}{CS} + \frac{k}{4\pi} \int_{\mathcal{M}} \dd[3]{x} \epsilon^{\alpha\beta\gamma} \Tr \left[ A_\alpha U (\partial_\beta U^{-1}) A_\gamma + U^{-1} A_\alpha A_\gamma (\partial_\beta U)  \right. \\
%     &\quad + (\partial_\alpha U) (\partial_\beta U^{-1}) A_\gamma + (\partial_\alpha U)(\partial_\beta A_\gamma) + U^{-1} (\partial_\alpha U) A_\gamma (\partial_\beta U) \\
%     &\quad + U^{-1} A_\alpha U (\partial_\beta U^{-1}) (\partial_\gamma U) + U^{-1} A_\alpha (\partial_\beta\partial_\gamma U) \\
%     &\quad + U^{-1} (\partial_\alpha U) U (\partial_\beta U^{-1}) (\partial_\gamma U) + U^{-1} (\partial_\alpha U) (\partial_\beta\partial_\gamma U) \\
%     &\quad + \frac{2}{3} \bigl( (\partial_\alpha U) U^{-1} A_\beta A_\gamma + A_\alpha (\partial_\beta U) U^{-1} A_\gamma + A_\alpha A_\beta (\partial_\gamma U) U^{-1}\bigr) \\
%     &\quad + \frac{2}{3} \bigl( (\partial_\alpha U) U^{-1} (\partial_\beta U) A_\gamma + A_\alpha (\partial_\beta U) U^{-1} (\partial_\gamma U) U^{-1} + (\partial_\alpha U) U^{-1} A_\beta (\partial_\gamma U) U^{-1}\bigr) \\
%     &\quad \left.+ \frac{2}{3} U^{-1} (\partial_\alpha U) U^{-1} (\partial_\beta U) U^{-1} (\partial_\gamma U)  \right] \\
% \end{align}

\section{古典的ゲージ理論の数学}

時空の多様体を $\mathcal{M}$ と書く.

% \subsection{ゲージ場の定義}

場\footnote{この段階では,\textbf{場}とはその配位を記述する空間 $F$ (これは \cinfty 多様体だったりベクトル空間だったりする)と \cinfty 写像 $\varphi \colon \mathcal{M} \lto F$ の組のことと考える.この描像は後にファイバー束の \cinfty 切断として定式化される.} $\varphi \colon \mathcal{M} \lto \mathbb{K}^N,\; x \lmto \bigl(\varphi_1(x),\, \dots,\, \varphi_N(x)\bigr)$ が線型Lie群 $G \subset \gGL{N}{\mathbb{K}}$ で記述される\footnote{ここでは $\mathbb{K} = \mathbb{R},\, \mathbb{C}$ としておく.}内部対称性を持っているような系を考える.
つまり,ゲージ原理を要請し,任意の \cinfty 写像 $U \colon \mathcal{M} \lto G$ に対して\footnote{\underline{内部}対称性という言葉を使うのは,$U$ が定数写像とは限らないことを意味する.},系のラグランジアン密度の場に関する項 $\mathcal{L}[\varphi_\mu (x)]$ が $\mathcal{L}\bigl[\, [U(x)]^j_i \varphi_j (x) \bigr] = \mathcal{L}[\varphi_i (x)]$ を充たすとする.
もしくは,場 $\varphi \colon \mathcal{M} \lto \mathbb{K}^N$ であって,時空の各点 $x \in \mathcal{M}$ および任意の \cinfty 写像 $U \colon \mathcal{M} \lto G$ に対して $\varphi (x) \lto U(x) \varphi (x)$ と変換する
\footnote{
    一般相対論の数学的定式化におけるテンソル場の変換性は,時空の多様体 $\mathcal{M}$ 上の一般座標変換(i.e. チャートの取り替え)に由来するものであった.
    同じように,ここで考えている場の変換性はどのような数学的定式化に由来するのかということを考えると,時空 $\mathcal{M}$ を底空間とする主束 $G \hookrightarrow P \xrightarrow{\pi} \mathcal{M}$ の同伴ベクトル束 $\mathbb{K}^N \hookrightarrow P \times_\rho \mathbb{K}^N \xrightarrow{q} \mathcal{M}$ における,$\mathcal{M}$ のチャートの取り替えに伴う局所自明化の取り替え(i.e. 変換関数のファイバーへの作用)の概念に行き着くのである.詳細は次の小節で議論する.
}
ものを考えると言っても良い.


この系を経路積分により量子化することを見据えて,このような変換性を充たす全ての場がなす空間の幾何学を考察すると見通しが良いだろう.
そのため,まず時空上の無限小だけ離れた2点 $\irm{x}{i},\, \irm{x}{f} \in \mathcal{M}$ における場の配位 $\varphi(\irm{x}{i}),\, \varphi(\irm{x}{f})$ を比較しよう.
内部自由度による変換性を議論したいので,$\varphi (\irm{x}{f}) - \varphi(\irm{x}{i})$ なる量を調べても意味がない.$\irm{x}{i},\, \irm{x}{f}$ を結ぶ \cinfty 曲線 $\gamma \colon [\irm{t}{i},\, \irm{t}{f}] \lto \mathcal{M}$ を持ってきて,$\gamma$ に沿って $\varphi(\irm{x}{i})$ を $\irm{x}{f}$ まで流してやるのが良い.つまり,場の配位を記述する空間 $\mathbb{K}^N$ 上の \cinfty 曲線 $\varphi^{(\gamma)} \coloneqq \varphi \circ \gamma \colon [\irm{t}{i},\, \irm{t}{f}] \lto \mathbb{K}^N$ を考えれば,
量 $\varphi (\irm{x}{f}) - \varphi^{(\gamma)}(\irm{t}{f})$ は $U(\irm{x}{f}) \in G$ による変換を受けるはずである.
$\irm{x}{i},\, \irm{x}{f}$ の両方を含む $\mathcal{M}$ のチャート $\bigl( V,\, (x^\mu) \bigr)$ を持ってきて成分計算すると,$\dd{x} \coloneqq \irm{x}{f} - \irm{x}{i}$ が\footnote{厳密にはこれは座標関数の差 $\dd{x}^\mu \coloneqq x^\mu(\irm{x}{f}) - x^\mu(\irm{x}{i})$ の絶対値が小さいことを主張している.}微小なのでTaylor展開において $\dd{x}$ の1次の項まで残すことで
\begin{align}
    \varphi^{(\gamma)}_i(\irm{t}{f}) &\eqqcolon \varphi_i (\irm{x}{i}) - [\, \textcolor{red}{A_\mu(\irm{x}{i})} \,]^j_i \varphi_j (\irm{x}{i}) \dd{x}^\mu \label{eq:2-gauge1} \\
    \varphi(\irm{x}{f}) &= \varphi (\irm{x}{i}) + \partial_\mu \varphi (x) \dd{x}^\mu
\end{align}
と書けるはずである
% \footnote{この定義の係数は,物理で使うベクトルポテンシャル $\mathcal{A}_\mu$ と $A_\mu = \iunit \mathcal{A}_\mu$ の関係にある.}
.ただし,式\eqref{eq:2-gauge1}の右辺によって $\dim \mathcal{M}$ 個の成分を持つ\underline{新しい場} $A_\mu \colon \mathcal{M} \lto \gGL{N}{\mathbb{K}}$ を定義した.この場は\textbf{ゲージ場}と呼ばれる.

ゲージ場 $A_\mu$ を時空の各点 $x \in \mathcal{M}$ における変換性によって特徴付けよう.
そのためには,量
\begin{align}
    \varphi (\irm{x}{f}) - \varphi^{(\gamma)}(\irm{t}{f}) = \bigl( \partial_\mu \varphi (\irm{x}{i}) + A_\mu(\irm{x}{i}) \varphi (\irm{x}{i}) \bigr) \dd{x}^\mu
\end{align}
が $U(\irm{x}{f}) \in G$ による変換を受けることに注目すれば良い.つまり,\textbf{共変微分}と呼ばれる線型写像を $\mathcal{D}_\mu (x) \coloneqq \partial_\mu + A_\mu (x)$ で定義すると,$\forall x \in \mathcal{M}$ における,内部対称性による変換
\begin{align}
    \label{eq:2-inner1}
    \varphi(x) \lto \tilde{\varphi}(x) \coloneqq U(x) \varphi (x)
\end{align}
に伴って $\mathcal{D}_\mu (x) \varphi (x)$ は
\begin{align}
    \mathcal{D}_\mu(x) \varphi(x) \lto \tilde{\mathcal{D}_\mu} (x) \tilde{\varphi}(x) \coloneqq U(x) \mathcal{D}_\mu (x) \varphi(x)
\end{align}
の変換を受ける.このことから,場 $\varphi$ の変換\eqref{eq:2-inner1}に伴う共変微分自身の変換則は
\begin{align}
    \label{eq:def-codv-transform}
    \mathcal{D}_\mu (x) \lto \tilde{D_\mu} (x) = U(x) \mathcal{D}_\mu (x) U(x)^{-1}
\end{align}
となる.従って場 $A_\mu \colon \mathcal{M} \lto \gGL{N}{\mathbb{K}}$ の,場 $\varphi$ の変換\eqref{eq:2-inner1}に伴う変換則が
\begin{align}
    \label{eq:def-gauge-transform}
    A_\mu (x) \lto U(x)\bigl(\partial_\mu + A_\mu(x) \bigr) U(x)^{-1}
\end{align}
だと分かった.このような場の変換則を\textbf{ゲージ変換} (gauge transformation) と呼ぶ.
% \begin{align}
    
% \end{align}

% が $t \in [0,\, 1]$ を変化させた時にどのように変化するかを考察しよう.
% いつもの如くまずは $\irm{x}{i},\, \irm{x}{f}$ が無限小だけ離れている場合を考える.無駄に煩雑になるので $\irm{x}{i} = x$,$\irm{x}{f} = x + \dd{x}$ と書いておく.
% Lie群 $G$ のLie代数 $\mathfrak{g}$ およびその基底 $\sigma_a / (2\iunit)$ を考えると,$\forall g \in G$ はある $\theta^a \sigma_a / (2\iunit)$ によって $g = \exp (\iunit \theta^a \sigma_a)$ 
\subsection{主束と内部対称性の定式化}

% 前小節では場 $\varphi$ が $N$ 次元ベクトル空間に値を持つ場合を考えた.つまり,場の配位空間と呼ぶべきものはランク $N$ の\textbf{ベクトル束} (vector bundle) であった:

% \begin{mydef}[label=def:vect]{ベクトル束}
    
% \end{mydef}
ゲージ場は,主束の接続として定式化できる.特に,主束の同伴ベクトル束が重要である.

まずファイバー束と主束を定義し,内部対称性を持つ場の記述には主束の同伴ベクトル束が適していることを見る\footnote{従って,この小節で行うのはゲージ場が登場する舞台の定式化であって,ゲージ場自身の定式化は次の小節で行う.}.
\cinfty 多様体 $M$ の\textbf{微分同相群} (diffeomorphism group) $\bm{\Diff M}$ とは,
\begin{itemize}
    \item 台集合 $\Diff M \coloneqq \bigl\{\, f \colon M \lto M \bigm| \text{微分同相写像} \,\bigr\}$
    \item 単位元を恒等写像
    \item 積を写像の合成
\end{itemize}
として構成される群のことを言う.

\begin{mydef}[label=def:Lie-action]{Lie群の作用}
    \begin{itemize}
        \item Lie群 $G$ の \cinfty 多様体 $M$ への\textbf{左作用}とは,
        群準同型 $\rho \colon G \lto \Diff M$ であって写像
        \begin{align}
            \blacktriangleright \colon G \times M \lto M,\; (g,\, x) \lmto \rho(g)(x)
        \end{align}
        が \cinfty 写像となるようなもののこと.
        $\bm{g \blacktriangleright x} \coloneqq \blacktriangleright(g,\, x)$ と略記する.
        \item Lie群 $G$ の \cinfty 多様体 $M$ への\textbf{右作用}とは,
        群準同型 $\rho \colon G^{\mathrm{op}} \lto \Diff M$ であって写像
        \begin{align}
            \btl \colon M \times G \lto M,\; (x,\, g) \lmto \rho(g)(x)
        \end{align}
        が \cinfty 写像となるようなもののこと.
        $\bm{x \blacktriangleleft g} \coloneqq \blacktriangleleft(g,\, x)$ と略記する.
        \item Lie群の左(resp. 右)作用が\textbf{自由} (free) であるとは,$\forall x \in X,\; \forall g  \in G \setminus \{1_G\},\; g \btr x \neq x\quad (\text{resp.}\quad x \btl g \neq x)$ を充たすことを言う.
        \item Lie群の左(resp. 右)作用が\textbf{効果的} (effective) であるとは,$\rho \colon G \lto \Diff M\quad (\text{resp.}\quad \rho \colon G^{\text{op}} \lto \Diff M)$ が単射であることを言う.
    \end{itemize}
\end{mydef}

\begin{mydef}[label=def.fiber-1,breakable]{\cinfty ファイバー束}
    Lie群 $G$ が \cinfty 多様体 $F$ に\hyperref[def:Lie-action]{効果的に作用}しているとする.
    $\bm{C^\infty}$ \textbf{ファイバー束} (fiber bundle) とは,
    \begin{itemize}
        \item \cinfty 多様体 $E,\, B,\, F$
        \item \cinfty の全射 $\pi \colon E \lto B$
        \item Lie群 $G$ と,$G$ の $F$ への\hyperref[def:Lie-action]{左作用} $\btr \colon G \times F \lto F$
        \item $B$ の開被覆 $\bigl\{\, U_\lambda  \,\bigr\}_{\lambda \in \Lambda}$
        \item 
        微分同相写像の族
        \begin{align}
            \bigl\{\, \varphi_\lambda \colon \pi^{-1}(U_\lambda) \lto U_\lambda \times F  \,\bigr\}_{\lambda \in \Lambda}
        \end{align}
        であって,$\forall \lambda \in \Lambda$ に対して図\ref{fig.fiber1}を可換にするもの.
        \begin{figure}[H]
            \centering
            \begin{tikzcd}
                \pi^{-1}(U_\lambda) \arrow[d, "\pi"'] \arrow{r}{\varphi} & 
                    U_\lambda \times F \arrow{dl}{\mathrm{proj}_1} \\
                    U_\lambda &
            \end{tikzcd}
            \caption{局所自明性}
            \label{fig.fiber1}
        \end{figure}%
        \item \cinfty 写像の族
        \begin{align}
            \bigl\{\, t_{\alpha\beta} \colon B \lto G \bigm| \forall (p,\, f) \in (U_\alpha \cap U_\beta) \times F,\; \varphi_\beta^{-1} (p,\, f) = \varphi_\alpha^{-1} \bigl(p,\, t_{\alpha\beta} (p) \btr f\bigr)  \,\bigr\}_{\alpha,\, \beta \in \Lambda}
        \end{align}
    \end{itemize}
    の6つのデータの組みのこと.記号としては $\bm{(E,\, \pi,\, B,\, F)}$ や $\bm{F \hookrightarrow E \xrightarrow{\pi} B}$ と書く.
	% \cinfty 多様体 $F,\, E,\, B$ と \cinfty 写像 $\pi \colon E \lto B$ を与える.
    % $\pi$ が以下の条件を充たすとき,組 $(E,\, \pi ,\, B,\, F)$ を $F$ をファイバーとする $\bm{C^\infty}$ \textbf{ファイバー束} (smooth fiber bundle) と呼ぶ:
	% \begin{description}
	% 	\item[\textbf{(局所自明性)}] 
		
	% 	$\forall b \in B$ に対して,$b$ のある開近傍 $U$ と微分同相写像 $\varphi \colon \pi^{-1}(U) \xrightarrow{\simeq} U \times F$ が存在して
	% 	\begin{align}
	% 		\forall u \in \pi^{-1} (U),\; \pi(u) = \mathrm{proj}_1 \circ \varphi(u)
	% 	\end{align}
	% 	となる
    %     \footnote{ただし,$\mathrm{proj}_1$ は第一成分への射影である:
	% 	\begin{align}
	% 		\mathrm{proj}_1 \colon U \times F \to U ,\; (p,\, f) \mapsto p
	% 	\end{align}}
    %     ,i.e. 図\ref{fig.fiber1}が可換図式となる.
    %     \begin{figure}[H]
    %         \centering
    %         \begin{tikzcd}
    %             \pi^{-1}(U) \arrow[d, "\pi"'] \arrow{r}{\varphi} & 
    %                 U \times F \arrow{dl}{\mathrm{proj}_1} \\
    %             U &
    %         \end{tikzcd}
    %         \caption{局所自明性}
    %         \label{fig.fiber1}
    %     \end{figure}%
	% \end{description}
    % 微分同相写像 $\varphi$ のことを\textbf{局所自明化} (local trivialization) と呼ぶ.
\end{mydef}

以下ではファイバー束と言ったら \hyperref[def.fiber-1]{\cinfty ファイバー束}のことを指すようにする.
% \begin{marker}
% 	定義\ref{def.fiber-1}において,$\pi$ を連続写像に,$\varphi$ を位相同型写像に置き換えると一般のファイバー束の定義が得られる.
% 	しかし,以降では微分可能ファイバー束しか考えないので定義\ref{def.fiber-1}の条件を充たす $(E,\, \pi ,\, B,\, F)$ のことを\textbf{ファイバー束}と呼ぶことにする.
% \end{marker}
ファイバー束 $(E,\, \pi ,\, B,\, F)$ に関して,
\begin{itemize}
	\item $E$ を\textbf{全空間} (total space)
	\item $B$ を\textbf{底空間} (base space)
	\item $F$ を\textbf{ファイバー} (fiber)
	\item $\pi$ を\textbf{射影} (projection)
	\item $\varphi_\lambda$ を\textbf{局所自明化} (local trivialization)
	\item $t_{\alpha\beta}$ を\textbf{変換関数} (transition map)
\end{itemize}
と呼ぶ\footnote{紛らわしくないとき,ファイバー束 $(E,\, \pi,\, B,\, F)$ のことを $\pi \colon E \to B$ ,または単に $E$ と略記することがある.}.また,射影 $\pi$ による1点集合 $\{b\}$ の逆像 $\pi^{-1}(\{b\}) \subset E$ のことを\textbf{点} $\bm{b}$ \textbf{のファイバー} (fiber) と呼び,$E_b$ と書く.\label{def:point-fiber}

\begin{mydef}[label=def:vect]{ベクトル束}
    ファイバーを $n$ 次元 $\mathbb{K}$-ベクトル空間 $V$ とし,構造群を $\gGL{n}{\mathbb{K}}$ とするような\hyperref[def.fiber-1]{ファイバー束} $V \hookrightarrow E \xrightarrow{\pi} M$ であって,
    その局所自明化 $\Familyset[\big]{\varphi_\lambda \colon \pi^{-1}(U_\lambda) \lto U_\lambda \times V}{\lambda \in \Lambda}$ が以下の条件を充たすもののことを\textbf{階数 $\bm{n}$ のベクトル束} (vector bundle of rank $n$) と呼ぶ:
    \begin{description}
        \item[\textbf{(vect-1)}] 
        
        $\forall \lambda \in \Lambda$ および $\forall x \in U_\lambda$ に対して,$\mathrm{proj}_2 \circ \varphi_\alpha|_{\pi^{-1}(\{x\})} \colon \pi^{-1}(\{x\}) \lto V$ は $\mathbb{K}$-ベクトル空間の同型写像である.
    \end{description}
    
\end{mydef}

\begin{myexample}[label=ex:tangentbundle]{接束}
    $n$ 次元\cinfty 多様体 $M$ の\hyperref[def:tangentbundle]{接束}は,構造群を $\LGL(n,\, \mathbb{R})$ とするベクトル束
    $\mathbb{R}^n \hookrightarrow TM \xrightarrow{\pi} M$ である.
    実際,$M$ のチャート $\bigl( U_\lambda,\, (x^\mu) \bigr)$ に対して局所自明化は
    \begin{align}
        \varphi_\lambda \colon \pi^{-1} (U_\lambda) \lto U_\lambda \times \mathbb{R}^n,\; \left(p,\, v^\mu \eval{\pdv{}{x^\mu}}_p \right) \lmto \bigl(p,\, \mqty[v^1 \\ \vdots \\ v^n] \bigr)
    \end{align}
    となり,チャート $\bigl( U_\alpha,\, (x^\mu) \bigr),\, \bigl( U_\beta,\, (y^\mu) \bigr)$ に対して
    \begin{align}
        \varphi_\beta^{-1} \bigl( p,\, (v^1,\, \dots,\, v^n) \bigr) = \varphi_\alpha^{-1} \bigl( p,\, \mqty[\pdv{x^1}{y^1}()(p) &\cdots &\pdv{x^1}{y^n}()(p) \\ \vdots &\ddots &\vdots \\ \pdv{x^n}{y^1}()(p) &\cdots &\pdv{x^n}{y^n}()(p)]\mqty[v^1 \\ \vdots \\ v^n] \bigr) 
    \end{align}
    となる.故に変換関数は
    \begin{align}
        t_{\alpha\beta} (p) \coloneqq \mqty[\pdv{x^1}{y^1}()(p) &\cdots &\pdv{x^1}{y^n}()(p) \\ \vdots &\ddots &\vdots \\ \pdv{x^n}{y^1}()(p) &\cdots &\pdv{x^n}{y^n}()(p)] \in \gGL{n}{\mathbb{R}}
    \end{align}
    で,ファイバーへの構造群の左作用とはただ単に $n$ 次元の数ベクトルに行列を左から掛けることである.
\end{myexample}


% \subsection{束写像}

% \cinfty 多様体 $F$ を共通のファイバーに持つ二つのファイバー束 $(E_i,\, \pi_i,\, B_i,\, F)$ を考える.このとき,二つの底空間 $B_i$ の間の\cinfty 写像と同様に,全空間 $E_i$ の間の微分同相写像を考えることができる.これら二つの\cinfty 写像は\textbf{束写像} (bundle map) と呼ばれる.

\begin{mydef}[label=def.bundlemap, breakable]{束写像}
	ファイバー $F$ と構造群 $G$ を共有する二つのファイバー束 $\xi_i = (E_i,\, \pi_i,\, B_i,\, F)$ を与える.
    \begin{itemize}
        \item $\xi_1$ から $\xi_2$ への\textbf{束写像} (bundle map) とは,二つの\cinfty 写像 $\textcolor{blue}{f} \colon B_1 \to B_2,\; \textcolor{red}{\tilde{f}} \colon E_1 \to E_2$ の組であって図\ref{fig.bundlemap}
        \begin{figure}[H]
            \centering
            \begin{tikzcd}
                    E_1 \arrow[d, "\pi_1"'] \arrow[red]{r}[red]{\tilde{f}}
                        & E_2 \arrow{d}{\pi_2} \\
                    B_1 \arrow[blue]{r}[blue]{f}
                    &B_2
            \end{tikzcd}
            \caption{束写像}
            \label{fig.bundlemap}
        \end{figure}%
        を可換にし,かつ底空間 $B_1$ の各点 $b$ において,\textbf{点} $\bm{b}$ \textbf{のファイバー} $\bm{\pi_1^{-1}(\{b\})} \subset E_1$ への $\textcolor{red}{\tilde{f}}$ の制限 
        \begin{align}
            \tilde{f}|_{\pi_1^{-1}(\{b\})} \colon \pi_1^{-1}(\{b\}) \to \tilde{f} \bigl( \pi_1^{-1}(\{b\}) \bigr) \subset E_2
        \end{align}
        が微分同相写像になっているもののことを言う.
        \item ファイバー束 $\xi_1$ と $\xi_2$ が\textbf{同型} (isomorphic) であるとは,$B_1 = B_2 = B$ であってかつ
        $f \colon B \to B$ が恒等写像となるような束写像 $\tilde{f} \colon E_1 \to E_2$ が存在することを言う.記号としては $\xi_1 \bm{\simeq} \xi_2$ とかく.
        \begin{figure}[H]
            \centering
            \begin{tikzcd}[column sep=small]
                    E_1 \arrow[dr, "\pi_1"'] \arrow[red]{rr}[red]{\tilde{f}} &	& E_2 \arrow{dl}{\pi_2} \\
                    & B &
            \end{tikzcd}
            \caption{ファイバー束の同型}
            \label{fig.bundle_homo}
        \end{figure}%
        \item 積束 $(B \times F,\, \mathrm{proj}_1,\, B,\, F)$ と同型なファイバー束を\textbf{自明束} (trivial bundle) と呼ぶ.
    \end{itemize}
\end{mydef}

% \begin{mydef}[label=def.bundle_isomorphism]{ファイバー束の同型}
% 	ファイバー $F$ と底空間 $B$ を共有する二つのファイバー束 $\xi_i = (E_i,\, \pi_i,\, B,\, F)$ を与える.
% 	このとき,ファイバー束 $\xi_1$ と $\xi_2$ が\textbf{同型} (isomorphic) であるとは,
% 	$f \colon B \to B$ が恒等写像となるような束写像 $\tilde{f} \colon E_1 \to E_2$ が存在することを言う.記号として $\xi_1 \simeq \xi_2$ とかく.
% \end{mydef}

% \subsection{切断}

ファイバー束 $(E,\, \pi,\, B,\, F)$ は,射影 $\pi$ によってファイバー $F$ の情報を失う.$F$ を復元するためにも,$s \colon B \to E$ なる写像の存在が必要であろう.

\begin{mydef}[label=def.section]{\cinfty 切断}
	ファイバー束 $\xi = (E,\, \pi,\, B,\, F)$ の \cinfty \textbf{切断} (cross section) とは,\cinfty 写像 $s \colon B \to E$ であって $ \pi \circ s = \mathrm{id}_B$ となるもののことを言う.
    \tcblower 
    $\xi$ の切断全体の集合を $\bm{\Gamma (B,\, E)}$ あるいは $\bm{\Gamma(E)}$ と書く.
\end{mydef}
% 各点 $b \in B$ に対して,明らかに $s(b) \in \pi^{-1}(\{b\})$ である.

% 切断は\textbf{大域的な}対象であり,与えられたファイバー束が切断を持つとは限らない.一方,各点 $b \in B$ の開近傍 $U$ 上であれば,図\ref{fig.fiber1}の示す局所自明性から\textbf{局所切断} $s \colon U \to \pi^{-1}(U)$ が必ず存在する.
% $\mathrm{proj}_1^{-1}(\{	b \}) = \{b\} \times F$ であることを考慮すると $\pi^{-1}(\{b\}) \simeq F$ とわかるので,局所切断 $s \colon U \to \pi^{-1}(U)$ は \cinfty 写像 $\tilde{s} \colon U \to F$ と一対一に対応する.

% $B$ 上の切断全体の集合を $\Gamma(B,\, E)$ と書くことにする.
$\xi = (E,\, \pi,\, B,\, F)$ を\hyperref[def.fiber-1]{ファイバー束}とする.底空間 $B$ の開被覆 $\{ U_\lambda\}_{\lambda \in \Lambda}$ をとると,定義\ref{def.fiber-1}から,どの $\alpha \in \Lambda$ に対しても局所自明性(図\ref{subfig.fiber-l})
が成り立つ.ここでもう一つの $\beta \in \Lambda$ をとり,$U_\alpha \cap U_\beta$ に関して局所自明性の図式を横に並べることで,自明束 $\mathrm{proj}_1 \colon (U_\alpha \cap U_\beta) \times F \to U_\alpha \cap U_\beta$ の\hyperref[def.bundlemap]{束の自己同型}(図\ref{subfig.fiber-lm})が得られる.
\begin{figure}[H]
	\centering
	\begin{subfigure}{0.4\columnwidth}
		\centering
		\begin{tikzcd}
			U_\alpha \times F \arrow[dr, "\mathrm{proj}_1"']  & 
				\pi^{-1}(U_\alpha) \arrow[l, "\varphi_\alpha"'] \arrow[d, "\pi"] \\
			& U_\alpha
		\end{tikzcd}
		\caption{$U_\alpha$ に関する局所自明性}
		\label{subfig.fiber-l}
	\end{subfigure}
	\hspace{5mm}
	\begin{subfigure}{0.4\columnwidth}
		\centering
		\begin{tikzcd}
			\pi^{-1}(U_\beta) \arrow[d, "\pi"'] \arrow{r}{\varphi_\beta} & 
			U_\beta \times F \arrow{dl}{\mathrm{proj}_1} \\
			U_\beta &
		\end{tikzcd}
		\caption{$U_\beta$ に関する局所自明性}
		\label{subfig.fiber-m}
	\end{subfigure}
	\vspace{5mm}
	\begin{subfigure}{0.4\columnwidth}
		\centering
		\begin{tikzcd}[column sep=small]
			(U_\alpha \cap U_\beta) \times F \arrow[dr, "\mathrm{proj}_1"'] \arrow[red]{rr}[red]{\varphi_\beta \circ \varphi_\alpha^{-1}} & &
				(U_\alpha \cap U_\beta) \times F \arrow{dl}{\mathrm{proj}_1}  \\
				& U_\alpha \cap U_\beta &
		\end{tikzcd}
		\caption{自明束 $(U_\alpha \cap U_\beta) \times F$ の自己同型}
		\label{subfig.fiber-lm}
	\end{subfigure}
	\caption{局所自明性の結合}
	\label{fig.fiber2}
\end{figure}
全ての $U_\alpha \cap U_\beta$ に関する変換関数の族 $\{t_{\alpha\beta}\}$ が $\forall b \in U_\alpha \cap U_\beta \cap U_\gamma$ に対して条件
\begin{align}
	\label{eq.cocycle}
	t_{\alpha\beta}(b) t_{\beta\gamma}(b) = t_{\alpha\gamma}(b)
\end{align}
を充たすことは図式\ref{fig.fiber2}より明かである.
次の命題は,ファイバー束 $(E,\, \pi,\, B,\, F)$ を構成する「素材」には
\begin{itemize}
	\item 底空間となる\cinfty 多様体 $B$
	\item ファイバーとなる\cinfty 多様体 $F$
	\item Lie群 $G$ と,その $F$ への\hyperref[def:Lie-action]{左作用} $\btr \colon G \times F \lto F$
	\item $B$ の開被覆 $\{ U_\lambda \}$
	\item \eqref{eq.cocycle}を充たす\cinfty 写像の族 $\{t_{\alpha\beta} \colon U_\beta \cap U_\alpha \to G\}_{\alpha,\, \beta \in \Lambda}$
\end{itemize}
があれば十分であることを主張する:

\begin{myprop}[label=prop.cocycle]{ファイバー束の構成}
	\begin{itemize}
        \item \cinfty 多様体 $B,\, F$
        \item Lie群 $G$ と,$G$ の $F$ への\hyperref[def:Lie-action]{左作用} $\btr \colon G \times F \lto G$
        \item $B$ の開被覆 $\{ U_\lambda \}_{\lambda \in \Lambda}$
        \item \textbf{コサイクル条件}\eqref{eq.cocycle}を充たす\cinfty 関数の族 $\{t_{\alpha\beta} \colon U_\beta \cap U_\alpha \to G\}$
    \end{itemize}
    を与える.
    このとき,構造群 $G$ と変換関数 $\{t_{\alpha\beta}\}_{\alpha,\, \beta \in \Lambda}$ を持つファイバー束 $\xi = (E,\, \pi,\, B,\, F)$ が存在する.
\end{myprop}
\begin{proof}
	まず手始めに,cocycle条件\eqref{eq.cocycle}より
	\begin{align}
		t_{\alpha\alpha}(b) t_{\alpha\alpha} (b) = t_{\alpha\alpha}(b),\quad \forall b \in U_\alpha
	\end{align}
	だから $t_{\alpha\alpha}(b) = 1_G$ であり,また
	\begin{align}
		t_{\alpha\beta}(b) t_{\beta\alpha} (b) = t_{\alpha\alpha}(b) = 1_G,\quad \forall b \in U_\alpha \cap U_\beta
	\end{align}
	だから $t_{\beta\alpha}(b) = t_{\alpha\beta}(b)^{-1}$ である.

	開被覆 $\{U_\lambda\}$ の添字集合を $\Lambda$ とする.このとき $\forall \lambda \in \Lambda$ に対して,$U_\lambda \subset B$ には底空間 $B$ からの\hyperref[def.reltopo]{相対位相}を入れ,$U_\lambda \times F$ にはそれと $F$ の位相との\hyperref[def.prodtopo]{積位相}を入れることで,\hyperref[def.disjoint_topo]{直和位相空間}
	\begin{align}
	\mathcal{E} \coloneqq \coprod_{\lambda \in \Lambda} U_\lambda \times F
	\end{align}
	を作ることができる\footnote{$\mathcal{E}$ はいわば,「貼り合わせる前の互いにバラバラな素材(局所自明束 $U_\alpha \times F$)」である.証明の以降の部分では,これらの「素材」を $U_\alpha \cap U_\beta \neq \emptyset$ の部分に関して「良い性質\eqref{eq.cocycle}を持った接着剤 $\{ t_{\alpha\beta} \}$」を用いて「貼り合わせる」操作を,位相を気にしながら行う.}.
	$\mathcal{E}$ の任意の元は $(\textcolor{red}{\lambda},\, b,\, f) \in  \textcolor{red}{\Lambda} \times  U_\lambda \times F$ と書かれる.

	さて,$\mathcal{E}$ 上の二項関係 $\sim$ を以下のように定める:
	\begin{align}
		\label{eq.prop9-1_equiv}
		(\alpha,\, b,\, f) \bm{\sim} \bigl(\beta,\, b,\, t_{\alpha\beta}(b) \btr f\bigr) \quad \forall b \in U_\alpha \cap U_\beta,\; \forall f \in F
	\end{align}
	$\sim$ が同値関係の公理を充たすことを確認する:
	\begin{description}
		\item[\textbf{反射律}] 冒頭の議論から $t_{\alpha\alpha}(b) = 1_G$ なので良い.
		\item[\textbf{対称律}] 冒頭の議論から $t_{\beta\alpha}(b) = t_{\alpha\beta}(b)^{-1}$  なので,
		\begin{align}
			&(\alpha,\, b,\, f) \sim (\beta,\, c,\, h) \\
            &\Longrightarrow \quad b=c \in U_\alpha \cap U_\beta \AND f = t_{\alpha\beta}(b) \btr h \\
			&\Longrightarrow \quad c=b \in U_\alpha \cap U_\beta \AND h = t_{\alpha\beta}(b)^{-1} \btr f = t_{\beta\alpha}(b) \btr f \\
			&\Longrightarrow \quad (\beta,\, c,\, h) \sim (\alpha,\, b,\, f).
		\end{align}
		\item[\textbf{推移律}] cocycle条件\eqref{eq.cocycle}より
		\begin{align}
			&(\alpha,\, b,\, f) \sim (\beta,\, c,\, h) \AND (\beta,\, c,\, h) \sim (\gamma,\, d,\, k) \\
			&\Longrightarrow \quad b=c \in U_\alpha \cap U_\beta \AND c=d \in U_\beta \cap U_\gamma\AND f = t_{\alpha\beta}(b) \btr h,\, h = t_{\beta\gamma}(c) \btr k \\
			&\Longrightarrow \quad b=d \in U_\alpha \cap U_\beta \cap U_\gamma \AND f = \bigl(t_{\alpha\beta}(b) t_{\beta\gamma}(b)\bigr) \btr k = t_{\alpha\gamma}(b)\btr k  \\
			&\Longrightarrow \quad (\alpha,\, b,\, f) \sim (\gamma,\, d,\, k).
		\end{align}
	\end{description}
	したがって $\sim$ は同値関係である.
	$\sim$ による $\mathcal{E}$ の商集合を $E$ と書き,商写像を $\mathrm{pr} \colon \mathcal{E} \to E,\; (\alpha,\, b,\, f)  \mapsto [ (\alpha,\, b,\, f)]$ と書くことにする.

	集合 $E$ に商位相を入れて $E$ を位相空間にする.このとき商位相の定義から開集合 $\{\alpha\} \times U_\alpha \times F \subset \mathcal{E}$ は $\mathrm{pr}$ によって $E$ の開集合 $\mathrm{pr}(\{\alpha\} \times U_\alpha \times F) \subset{E}$ に移される.ゆえに $E$ は $\bigl\{\, \mathrm{pr}(\{\alpha\} \times U_\alpha \times V_\beta)\, \bigr\}$ を座標近傍にもつ\cinfty 多様体である(ここに $\{ V_\beta \}$ は,\cinfty 多様体 $F$ の座標近傍である).
	
	次に\cinfty の全射 $\pi \colon E \lto B$ を
	\begin{align}
		\pi \bigl(\, [(\alpha,\, b,\, f)]\, \bigr) \coloneqq b
	\end{align}
	と定義すると,これは $\forall \alpha \in \Lambda$ に対して微分同相写像
    \footnote{
        逆写像は $\varphi_\alpha^{-1} \colon U_\alpha \times F \lto \pi^{-1}(U_\alpha),\; (b,\, f) \lmto [(\alpha,\, b,\, f)]$ である.
        $\varphi_\alpha$ も $\varphi_\alpha^{-1}$ も \cinfty 写像の合成で書けるので \cinfty 写像である.
    }
	\begin{align}
		\varphi_\alpha \colon \pi^{-1}(U_\alpha) \lto U_\alpha \times F,\; [(\alpha,\, b,\, f)] \lmto (b,\, f)
	\end{align}
	による\hyperref[fig.bundle_homo]{局所自明性}を持つ.
	従って組 $\xi \coloneqq (E,\, \pi,\, B,\, F)$ は構造群 $G$,局所自明化 $\{\varphi_\alpha\}_{\alpha \in \Lambda}$,変換関数 $\{t_{\alpha\beta}\}_{\alpha,\, \beta \in \Lambda}$ を持つ\hyperref[def.fiber-1]{ファイバー束}になり,証明が終わる.
\end{proof}

\begin{mydef}[label=def.PFD]{主束}
    構造群を $G$ に持つ\hyperref[def.fiber-1]{ファイバー束} $\xi = (P,\, \pi,\, M,\, G)$ が\textbf{主束} (principal bundle) であるとは,
    $G$ の $G$ 自身への左作用が自然な\hyperref[def:Lie-action]{左作用}\footnote{つまり,$g \btr x \coloneqq gx$(Lie群の積)である.}であることを言う.
\end{mydef}

% 主束 $(P,\, \pi,\, M,\, G)$ は $(P,\, \pi,\, M,\, G)$ とか $P(M,\, G)$ と書かれることもある.
次の命題は証明の構成が極めて重要である:
\begin{myprop}[label=prop.PFD_right]{主束の全空間への右作用}
	$\xi = (P,\, \pi,\, M,\, G)$ を\hyperref[def.PFD]{主束}とする.
    このとき,$G$ の全空間 $P$ への\hyperref[def:Lie-action]{自由な右作用}が自然に定義され,
	その軌道空間 (orbit space) $P/G$ が $M$ になる.
\end{myprop}

\begin{proof}
    $\xi$ の\hyperref[def.fiber-1]{局所自明化}を $\{\, \varphi_\lambda \,\}_{\lambda \in \Lambda}$,変換関数を $\{t_{\alpha\beta} \colon U_\alpha \cap U_\beta \to G\}_{\alpha,\, \beta \in \Lambda}$ と書く.
	$\forall u \in P,\, \forall g \in G$ をとる.$\pi(u) \in U_\alpha$ となる $\alpha \in \Lambda$ を選び,対応する\hyperref[def.fiber-1]{局所自明化} $\varphi_\alpha$ による $u$ の像を $\varphi_\alpha (u) \eqqcolon (p,\, h) \in U_\alpha \times G$ とおく\footnote{つまり,$p \coloneqq \pi(u),\; h \coloneqq \mathrm{proj}_2 \circ \varphi_\alpha (u)$ と言うことである.}.
    このとき $G$ の $P$ への右作用 $\btl \colon P \times G \lto P$ を次のように定義する\footnote{右辺の $h \textcolor{red}{g}$ はLie群の乗法である.}:
    % \footnote{$G$ の $G$ 自身への右作用は,$G$ の右からの積演算を選ぶ.この作用は推移的かつ効果的である.}:
	\begin{align}
		\label{def:ractionP}
		u \btl \textcolor{red}{g} \coloneqq \varphi_\alpha^{-1}(p,\, h \textcolor{red}{g})
	\end{align}
	
	\begin{description}
		\item[\textbf{$\btl$ のwell-definedness}] 

			$\beta \neq \alpha$ に対しても $\pi(u) \in U_\beta$ であるとする.このとき $\varphi_\beta(u) = (p,\, h') \in (U_\alpha \cap U_\beta) \times G$ と書けて,また変換関数の定義から
			\begin{align}
				h' = t_{\alpha\beta}(p) h \quad \bigl(\, t_{\alpha\beta}(p) \in G\, \bigr)
			\end{align}
			である.したがって
			\begin{align}
				\varphi_\beta^{-1}(p,\, h'g) = \varphi_\beta^{-1}\Bigl(p,\, \bigl(t_{\alpha\beta}(p) h\bigr)g\Bigr) = \varphi_\beta^{-1}\bigl(p,\, t_{\alpha\beta}(p)  hg \bigr) = \varphi_\beta^{-1} \circ (\varphi_\beta \circ \varphi_\alpha^{-1})(p,\, h g) = \varphi_\alpha^{-1}(p,\, h g)
			\end{align}
			が分かり,式\eqref{def:ractionP}の右辺は局所自明化の取り方によらない.
		\item[\textbf{$\btl$ は右作用}] 
			写像 $\rho \colon G^{\text{op}} \lto \Diff P,\; g \lmto (u \lmto u \btl g)$ が群準同型であることを示す.
            \begin{enumerate}
				\item $u \btl 1_G = \varphi_\alpha^{-1}(p,\, h1_G) = \varphi_\alpha^{-1}(p,\, h) = u$
				\item $\forall g_1,\, g_2 \in G$ をとる.
				\begin{align}
					u \btl (g_1g_2) = \varphi_\alpha^{-1}\bigl(p,\, (hg_1)g_2 \bigr) = \varphi_\alpha^{-1}(p,\, hg_1) \btl g_2 = (u \btl g_1) \btl g_2
				\end{align} 
			\end{enumerate}
		\item[\textbf{$\btl$ は自由}] 
		
            $\forall \alpha \in \Lambda$ に対して
            $\forall u = (p,\, g) \in \pi^{-1}(U_\alpha)$ をとる.$u \btl g' = u$ ならば
            \begin{align}
                u \btl g' = \varphi_\alpha^{-1} (p,\, gg') = u = \varphi_\alpha^{-1}(p,\, g 1_G)
            \end{align}
            が成り立つが,局所自明化は全単射なので $gg' = g$ が言える.$g$ は任意なので $g' = 1_G$ が分かった.
        \item[\textbf{軌道空間が $\bm{M}$}]  
        
            $\forall \alpha \in \Lambda$ に対して,$G$ の右作用\eqref{def:ractionP}による $U \times G$ の軌道空間は $(U \times G) / G = U \times \{1_G\} = U$ となる.故に $P$ 全域に対しては $P/G = B$ となる.
	\end{description}
\end{proof}

\begin{mytheo}[label=thm:principal]{}
    コンパクトHausdorff空間 $P$ と,$P$ に\hyperref[def:Lie-action]{自由に作用}しているコンパクトLie群 $G$ を与える.この時,軌道空間への商写像
    \begin{align}
        \pi \colon P \lto P/G
    \end{align}
    は\hyperref[def.PFD]{主束}である.
\end{mytheo}

\begin{proof}
    
\end{proof}

構造群を $G$ とする\hyperref[def.fiber-1]{ファイバー束} $F \hookrightarrow E \xrightarrow{\pi} M$ が与えられたとき,命題\ref{prop.cocycle}を使うと,変換関数が共通の\hyperref[def.PFD]{主束} $G \hookrightarrow P \xrightarrow{p} M$ が存在することがわかる.
このようにして得られる主束をファイバー束 $F \hookrightarrow E \xrightarrow{\pi} M$ に\textbf{同伴する} (associated) 主束と呼ぶ.

\begin{myexample}[label=def:framebundle]{フレーム束}
    変換関数 $\{t_{\alpha\beta}\colon M \lto \gGL{N}{\mathbb{K}}\}$ を持つ\hyperref[def:vect]{階数 $N$ のベクトル束} $\mathbb{K}^N \hookrightarrow E \xrightarrow{\pi} M$ に同伴する主束は,例えば次のようにして構成できる:
    $\forall x \in M$ に対して
    \begin{align}
        P_x \coloneqq \bigl\{\, f \in \Hom{}(\mathbb{K}^N,\, E_x) \bigm| \text{同型写像} \,\bigr\} 
    \end{align}
    とし,
    \begin{align}
        P \coloneqq \coprod_{x \in M} P_x, \quad
        \varpi \colon P \lto M,\; (x,\, f) \lmto x
    \end{align}
    と定める.$\gGL{N}{\mathbb{K}} \hookrightarrow P \xrightarrow{\varpi} M$ に適切な局所自明化を入れて,変換関数が $\{t_{\alpha\beta}\colon M \lto \gGL{N}{\mathbb{K}}\}$ となるような主束を構成する.

     $\forall (x,\, f) \in P_x$ をとる. 
    このとき $\mathbb{K}^N$ の標準基底を $e_1,\, \dots,\, e_N$ とすると,$f \in \Hom{}(\mathbb{K}^n,\, E_x)$ は $E_x$ の基底 $f(e_1),\, \dots,\, f(e_N)$ と同一視される
    \footnote{実際 $\forall v = v^\mu e_\mu \in \mathbb{K}^n$ に対して $f(v) = v^\mu f(e_\mu)$ が成り立つので,$f(e_1),\, \dots,\, f(e_N) \in E_x$ が指定されれば $f$ が一意的に決まる.}
    ことに注意しよう.
    このことに由来して,$f_\mu \coloneqq f(e_\mu)$ とおいて $f = (f_1,\, \dots,\, f_N) \in P_x$ と表すことにする.
    % $\gGL{n}{\mathbb{K}}$ の右作用は基底の取り替えに相当する.

     $E$ の局所自明化 $\{\varphi_\alpha \colon \pi^{-1}(U_\alpha) \lto U_\alpha \times \mathbb{K}^n \}$ を与える.このとき,$n$ 個の \underline{$E$ の}\hyperref[def.section]{局所切断} $s_\alpha{}_1,\, \dots s_\alpha{}_N \in \Gamma(E|_{U_\alpha})$ を
    \begin{align}
        s_\alpha{}_\mu(x) \coloneqq \varphi_\alpha^{-1} (x,\, e_\mu)
    \end{align}
    と定義すると,$\forall x \in U_\alpha$ に対して $s_\alpha{}_1(x),\, \dots ,\, s_{\alpha}{}_N(x)$ が $E_x$ の基底となる
    \footnote{\hyperref[def:vect]{ベクトル束の定義}から $\mathrm{proj}_2 \circ \varphi_\alpha|_{E_x} \colon E_x \lto \mathbb{K}^N$ が $\mathbb{K}$-ベクトル空間の同型写像であるため.}.
    故に,\underline{$P$ の}局所切断 $p_\alpha \in \Gamma(P|_{U_\alpha})$ を
    \begin{align}
        p_\alpha (x) \coloneqq \Bigl(x,\, \bigl( s_\alpha{}_1(x),\, \dots,\, s_\alpha{}_N(x) \bigr)\Bigr) \in P_x
    \end{align}
    により定義できる.
    このとき,$\forall (x,\, f) = \bigl(x,\, (f_1,\, \dots,\, f_N)\bigr) \in \varpi^{-1}(U_\alpha)$ に対してある $g \in \gGL{N}{\mathbb{K}}$ が存在して $f = p_\alpha(x) g$ と書ける.ただし $g$ は基底の取り替え行列で,ただ単に行列の積として右から作用している.
    
    ここで,目当ての \underline{$P$ の}局所自明化を
    \begin{align}
        \psi_\alpha \colon \varpi^{-1}(U_\alpha) \lto U_\alpha \times \gGL{n}{\mathbb{K}},\; (x,\, f) = \bigl(x,\, p_\alpha (x) g\bigr)\lmto (x,\, g)
    \end{align}
    と定義する.変換関数を計算すると
    \begin{align}
        \psi_\beta^{-1}(x,\, g) 
        &= \bigl(x,\, p_\beta (x) g\bigr) \\
        &= \Bigl(x,\, \bigl(s_\beta{}_1(x),\, \dots,\, s_\beta{}_N(x)\bigr) g\Bigr) \\
        &= \Bigl(x,\, \bigl(\varphi_\beta^{-1}(x,\, e_1),\, \dots,\, \varphi_\beta^{-1}(x,\, e_N)\bigr) g\Bigr) \\
        &= \biggl(x,\, \Bigl(\varphi_\alpha^{-1}\bigl(x,\, t_{\alpha\beta}(x) e_1\bigr),\, \dots,\, \varphi_\alpha^{-1}\bigl(x,\, t_{\alpha\beta}(x) e_N\bigr)\Bigr) g\biggr) 
        % \\
        % &= \biggl(x,\, \Bigl(\varphi_\alpha^{-1}\bigl(x,\, e_1\bigr),\, \dots,\, \varphi_\alpha^{-1}\bigl(x,\, e_N\bigr)\Bigr) t_{\alpha\beta}(x) g\biggr) \\
        % &= \biggl(x,\, \bigl(s_\alpha{}_{1}(x),\, \dots,\, s_\alpha{}_N(x)\bigr) t_{\alpha\beta}(x)g\biggr) \\
        % &= \bigl(x,\, p_{\alpha} (x) t_{\alpha\beta}(x) g\bigr) \\
        % &= \psi_\alpha^{-1} (x,\, t_{\alpha\beta}(x) g)
    \end{align}
    となるが,$e_\mu$ が標準基底なので
    \begin{align}
        t_{\alpha\beta}(x) e_\mu = \mqty[t_{\alpha\beta}(x)^1{}_\mu \\ t_{\alpha\beta}(x)^2{}_\mu \\ \vdots \\ t_{\alpha\beta}(x)^n{}_\mu] = e_\nu t_{\alpha\beta}(x)^\nu{}_\mu
    \end{align}
    が成り立つこと,および\hyperref[def:vect]{ベクトル束の定義}から $\mathrm{proj}_2 \circ \varphi_\alpha|_{E_x} \colon E_x \lto \mathbb{K}^N$ が $\mathbb{K}$-ベクトル空間の同型写像であることに注意すると
    \begin{align}
        &\biggl(x,\, \Bigl(\varphi_\alpha^{-1}\bigl(x,\, t_{\alpha\beta}(x) e_1\bigr),\, \dots,\, \varphi_\alpha^{-1}\bigl(x,\, t_{\alpha\beta}(x) e_N\bigr)\Bigr) g\biggr) \\
        &= \biggl(x,\, \Bigl(\varphi_\alpha^{-1}\bigl(x,\, e_\nu\bigr) t_{\alpha\beta}(x)^\nu{}_1,\, \dots,\, \varphi_\alpha^{-1}\bigl(x,\, e_\nu\bigr)t_{\alpha\beta}(x)^\nu{}_N\Bigr) g\biggr) \\
        &= \biggl(x,\, \Bigl(\varphi_\alpha^{-1}\bigl(x,\, e_1\bigr) ,\, \dots,\, \varphi_\alpha^{-1}\bigl(x,\, e_N\bigr)\Bigr) t_{\alpha\beta}(x)g\biggr) \\
        &= \biggl(x,\, \bigl(s_\alpha{}_{1}(x),\, \dots,\, s_\alpha{}_N(x)\bigr) t_{\alpha\beta}(x)g\biggr) \\
        &= \bigl(x,\, p_{\alpha} (x) t_{\alpha\beta}(x) g\bigr) \\
        &= \psi_\alpha^{-1} (x,\, t_{\alpha\beta}(x) g)
    \end{align}
    だとわかり,目標が達成された.この $\gGL{N}{\mathbb{K}} \hookrightarrow P \xrightarrow{\pi} M$ のことを\textbf{フレーム束}と呼ぶ.
\end{myexample}

逆に,与えられた主束を素材にして,変換関数を共有するファイバー束を構成することができる.

\begin{myprop}[label=prop:Borelconst,breakable]{Borel構成}
    $G \hookrightarrow P \xrightarrow{\pi} M$ を\hyperref[def.PFD]{主束}とし,Lie群 $G$ の \cinfty 多様体への\hyperref[def:Lie-action]{左作用} $\btr \colon G \times F \lto F$ を与える.
    \eqref{def:ractionP}で定義された $G$ の \underline{$P$ への右作用}を $\btl \colon P \times G \lto P$ と書く.
    \begin{itemize}
        \item 積多様体 $P \times F$ への $G$ の新しい右作用 $\textcolor{red}{\btl} \colon (P \times F) \times G \lto P \times F$ を
        \begin{align}
            (u,\, f) \textcolor{red}{\btl}\, g \coloneqq (u \btl g,\, g^{-1} \btr f)
        \end{align}
        と定義し,この右作用による $P \times F$ の軌道空間を $\bm{P \times_G F} \coloneqq (P \times F) / G$ と書く.
        \item 商写像 $\varpi\colon P \times F \lto P \times _G F,\; (u,\, f) \lmto (u,\, f) \textcolor{red}{\btl}\, G$ による $(u,\, f) \in P \times F$ の像を $u \times_G f \in P \times_G F$ と書く.このとき写像
        \begin{align}
            q \colon P \times_G F \lto M,\; u \times_G f \lmto \pi (u)
        \end{align}
        がwell-definedになる.
    \end{itemize}
    このとき,$F \hookrightarrow P \times_G F \xrightarrow{q} M$ は構造群 $G$ をもち,変換関数が $G \hookrightarrow P \xrightarrow{\pi} M$ のそれと同じであるような\hyperref[def.fiber-1]{ファイバー束}である.
\end{myprop}

\begin{proof}
    $q$ のwell-definednessは,\eqref{def:ractionP}で定義した右作用 $\btl$ が $\pi(u)$ を不変に保つので明らか.

    主束 $G \hookrightarrow P \xrightarrow{\pi} M$ の開被覆,局所自明化,変換関数をそれぞれ $\{U_\lambda\}_{\lambda \in \Lambda},\, \{\varphi_\lambda \colon \pi^{-1}(U_\lambda) \lto U_\lambda \times G\}_{\lambda \in \Lambda},\, \bigl\{ t_{\alpha\beta} \colon M \lto G \bigr\}_{\alpha,\, \beta \in \Lambda}$ と書く.
    また,$\forall \lambda \in \Lambda$ に対して\hyperref[def.section]{局所切断} $s_\lambda \in \Gamma(P|_{U_\alpha})$ を
    \begin{align}
        s_\lambda \colon M \lto \pi^{-1}(U_\alpha),\; x \lmto \varphi_\lambda^{-1}(x,\, 1_G)
    \end{align}
    と定義する.

    このとき,$\forall \lambda \in \Lambda$ に対して \cinfty 写像
    \begin{align}
        \label{eq:loctriv-Borel}
        \psi_\lambda \colon q^{-1}(U_\lambda) \lto U_\lambda \times F,\; s_\lambda(x) \times_G f \lmto \bigl( x,\, f \bigr) 
    \end{align}
    がwell-definedな
    \footnote{$\forall u \times_G f \in q^{-1}(U_\lambda)$ をとる.このとき $q(u \times_G f) = \pi(u) \in U_\lambda$ なので $u \in P$ に\underline{主束 $G \hookrightarrow P \xrightarrow{\pi} M$ の}局所自明化 $\varphi_\lambda \colon \pi^{-1}(U_\lambda) \lto U_\lambda \times F$ を作用させることができる.
        従って $g(u) \coloneqq  \mathrm{proj}_2 \circ \varphi_\lambda (u) \in G$ とおけば,
        $G$ の $P$ への右作用の定義\eqref{def:ractionP}から $u = \varphi_\lambda^{-1} \bigl(\pi(u),\, g(u)\bigr) = \varphi_\lambda^{-1} \bigl( \pi(u),\, 1_G \bigr)  \btl g(u) = s_\lambda \bigl( \pi(u) \bigr)  \btl g(u)$ が成り立ち,$u \times_G f = \Bigl( s_\lambda \bigl( \pi(u) \bigr) \btl g(u) \Bigr) \times_G f =  s_\lambda \bigl( \pi(u) \bigr) \times_G \bigl(g(u) \btr f\bigr)$ と書くことができる.
        よって \textbf{$\psi_\lambda$ の定義\eqref{eq:loctriv-Borel}において} $\bm{\psi_\lambda (u \times_G f) = \bigl( \pi(u),\, g(u) \btr f \bigr)}$ \textbf{であり},全ての $q^{-1}(U_\lambda)$ の元の行き先が定義されていることがわかった.
        次に $u \times_G f = u' \times_G f' \in q^{-1}(U_\lambda)$ であるとする.このとき右作用 $\textcolor{red}{\btl}$ の定義からある $h \in G$ が存在して $u' = \varphi_\lambda^{-1} \bigl( \pi(u'),\, g(u') \bigr) = u \btl h = \varphi_\lambda^{-1} \bigl( \pi(u),\, g(u) h \bigr),\; f' = h^{-1} \btr f$ が成り立つので,$\pi(u') = \pi(u),\; g(u') = g(u)h,\, f' = h^{-1} \btr f$ が言える.
        従って $\psi_\lambda(u' \times_G f') = \bigl( \pi(u'),\, g(u') \btr f' \bigr) = \Bigl( \pi(u),\, \bigl(g(u) h\bigr) \btr \bigl( h^{-1} \btr f \bigr) \Bigr) = \bigl( \pi(u),\, g(u) \btr h \btr h^{-1} \btr f \bigr) = \bigl( \pi(u),\, g(u) \btr f \bigr) = \psi_\lambda (u \times_G f)$ が成り立ち,$\psi_\lambda$ がwell-definedであることが示された.
    }
    微分同相写像になる
    \footnote{
        $\pi \colon P \lto M,\; g \coloneqq \mathrm{proj}_2 \circ \varphi_\lambda \colon q^{-1}(U_\lambda) \lto G,\; \btr \colon G \times F \lto F$ は全て \cinfty 写像の合成の形をしているので \cinfty 写像であり,$\psi_\lambda \coloneqq \bigl(\pi \times (\btr \circ (g \times \mathrm{id}_F))\bigr)$ もこれらの合成として書けている(写像 $\times,\, \mathrm{id}_F$ はもちろん \cinfty 級である)ので \cinfty 写像である.
        well-definednessの証明と同じ議論で $\psi_\lambda$ の単射性がわかる.全射性は定義\eqref{eq:loctriv-Borel}より明らか.
        逆写像 $(x,\, f) \lmto s_\lambda(x) \times_G f$ も,\cinfty 写像たちの合成 $q \circ (s_\lambda \times \mathrm{id}_F)$ なので \cinfty 写像である.
    }
    ので,族 
    \begin{align}
        \bigl\{ \psi_\lambda \colon q^{-1}(U_\lambda) \lto U_\lambda \times F \bigr\}_{\lambda \in \Lambda}
    \end{align}
    を $F \hookrightarrow P \times_G F \xrightarrow{q} M$ の局所自明化にとる.
    すると $\forall \alpha,\, \beta \in \Lambda,\; \forall (x,\, f) \in (U_\alpha \cap U_\beta) \times F$ に対して
    \begin{align}
        \psi_\beta^{-1} (x,\, f) &= s_\beta(x) \times_G f \\
        &= \varphi_\beta^{-1}(x,\, 1_G) \times_G f \\
        &= \varphi_\alpha^{-1}(x,\, t_{\alpha\beta}(x)1_G) \times_G f \\
        &= \varphi_\alpha^{-1}(x,\, 1_G t_{\alpha\beta}(x)) \times_G f \\
        &= \bigl(\varphi_\alpha^{-1}(x,\, 1_G) \btl t_{\alpha\beta}(x)\bigr) \times_G f \\
        &= \bigl(s_\alpha (x) \btl t_{\alpha\beta}(x)\bigr) \times_G f \\
        % &= \varpi\bigl(x  t_{\alpha\beta}(x),\, f\bigr) \\
        &= \Bigl( \bigl(s_\alpha (x) \btl t_{\alpha\beta}(x)\bigr) \btl t_{\alpha\beta}(x)^{-1} \Bigr) \times_G \bigl(t_{\alpha\beta}(x) \btr f\bigr) \\
        &= s_\alpha (x) \times_G \bigl( t_{\alpha\beta}(x) \btr f \bigr) \\
        &= \psi_\alpha^{-1} (x,\, t_{\alpha\beta}(x) \btr f)
    \end{align}
    が成り立つので $F \hookrightarrow P \times_G F \xrightarrow{q} M$ の変換関数は
    \begin{align}
        \bigl\{\, t_{\alpha\beta} \colon M \lto G  \,\bigr\}_{\alpha,\, \beta \in \Lambda}
    \end{align}
    である.
\end{proof}

\begin{myexample}[label=def:associated-vect]{同伴ベクトル束}
    \hyperref[def.fiber-1]{主束} $G \hookrightarrow P \xrightarrow{\pi} \mathcal{M}$ を任意に与える.
    Lie群 $G$ の,$N$ 次元 $\mathbb{K}$ ベクトル空間 $V$ への\hyperref[def:Lie-action]{左作用}とは,Lie群 $G$ の $N$ 次元表現 $\rho \colon G \lto \LGL (V)$ のことに他ならない\footnote{$\End V$ に標準的な \cinfty 構造を入れてLie群と見做したものを $\LGL (V)$ と書いた.}.
    このとき,命題\ref{prop:Borelconst}の方法によって構成される階数 $N$ の\hyperref[def:vect]{ベクトル束}のことを $\bm{P \times_\rho V}$ と書き,\textbf{同伴ベクトル束} (associated vector bundle) と呼ぶ.
\end{myexample}

これでゲージ場を導入する準備が整った.
つまり,この節の冒頭で考えた内部対称性を持つ場 $\varphi \colon \mathcal{M} \lto \mathbb{K}^N$ とは,厳密には\hyperref[def.PFD]{主束}
\begin{align}
    G \hookrightarrow P \xrightarrow{\pi} \mathcal{M}
\end{align}
の,線型Lie群 $G$ の $N$ 次元表現
\begin{align}
    \rho \colon G \lto \LGL (\mathbb{K}^N),\; U \lmto (\bm{v} \lmto U\bm{v})
\end{align}
による\hyperref[def:associated-vect]{同伴ベクトル束}
\begin{align}
    \mathbb{K}^N \hookrightarrow P \times_\rho \mathbb{K}^N \xrightarrow{q} \mathcal{M}
\end{align}
の\hyperref[def.section]{局所切断} $\phi \colon V_\alpha \lto P \times_\rho \mathbb{K}^N$ を,ある一つの\hyperref[def.fiber-1]{局所自明化} $\sigma_\alpha \colon q^{-1}(V_\alpha) \lto V_\alpha \times \mathbb{K}^N$ によって座標表示したもの(の第2成分を取り出してきたもの)
\begin{align}
    \varphi = \mathrm{proj}_2 \circ \sigma_\alpha \circ \phi \colon V_\alpha \lto \mathbb{K}^N
\end{align}
のことだと見做せる.と言うのも,こう考えることで場の変換性\eqref{eq:2-inner1}
\begin{align}
    \varphi(x) \lto \tilde{\varphi}(x) \coloneqq U(x) \varphi (x)
\end{align}
が,時空 $\mathcal{M}$ の2つのチャート $\bigl(V,\, (x^\mu)\bigr),\, \bigl(\tilde{V},\, (\tilde{x}^\mu)\bigr)$ の共通部分 $V \cap \tilde{V}$ 上における,局所自明化 $\sigma,\, \tilde{\sigma} \colon q^{-1} (V \cap \tilde{V}) \lto (V \cap \tilde{V}) \times \mathbb{K}^N$ の取り替え(内部自由度に関する一般座標変換のようなもの)に伴う変換関数 $U_{\tilde{V},\, V} \colon \mathcal{M} \lto G$ の作用
\begin{align}
    \tilde{\sigma} \circ \sigma^{-1} \colon (V \cap \tilde{V}) \times \mathbb{K}^N &\lto (V \cap \tilde{V}) \times \mathbb{K}^N, \\
    \bigl(x,\, \bm{\varphi(x)}\bigr) &\lmto \Bigl(x,\, \rho\bigl(U_{\tilde{V},\, V}(x)\bigr)\bigl( \varphi(x) \bigr) \Bigr) = \Bigl( x,\, \bm{U_{\tilde{V},\, V} (x) \varphi(x)} \Bigr) \label{eq:field-locsym}
\end{align}
として上手く定式化できているのである
\footnote{
    物理では変換性によって場を定義するので,数学的定式化はこれで良い.なお,この定式化は主束の全空間 $P$ の情報を一切使っていないが,これは命題\ref{prop.cocycle}の表れである.実際,この節の冒頭の議論で顕に登場したのは時空 $\mathcal{M}$,場の配位を記述する空間 $\mathbb{K}^N$,内部対称性を表すLie群 $G$ とその表現 $\rho \colon G \lto \gGL{N}{\mathbb{K}}$,場の局所的変換を表す \cinfty 写像 $U \colon \mathcal{M} \lto G$ だけだったので,その数学的定式化が $P$ によらないのは妥当だと思う.
}.

% \begin{myprop}[]{}
% 	主 $G$-束 $\xi = (P,\, \pi,\, M,\, G)$ が自明束になるための必要十分条件は,それが切断(定義\ref{def.section})を持つことである.
% \end{myprop}

% \begin{proof}
% 	\textbf{($\bm{\Longrightarrow}$)} $\xi$ が自明束ならば切断をもつことは明らか.

% 	\textbf{($\bm{\Longleftarrow}$)} 切断 $s \colon M \to P$ が存在するとする.命題\ref{prop.PFD_right}より $G$ は $P$ に右から自由に作用する.従って $p \in M$ のファイバー $\pi^{-1}(p)$ 上の任意の2点 $\forall u,\, v \in \pi^{-1}(p)$ に対して,ただ一つの $g \in G$ が存在して $v = u \cdot g$ となる.
	
% 	ここで,写像 $\tilde{f} \colon P \to M \times G$ を次のように定義する:

% 	$u,\, s(\pi(u)) \in \pi^{-1}(\pi(u))$ だから
% 	\begin{align}
% 		\exists! g \in G,\, u = s(\pi(u)) \cdot g
% 	\end{align}
% 	であり,この $g$ を用いて
% 	\begin{align}
% 		\tilde{f}(u) \coloneqq \bigl( \pi(u),\, g \bigr)
% 	\end{align}
% 	とする.この $\tilde{f}$ が下図を可換図式にすることは明らかであり,$(P,\, \pi,\, M,\, G) \cong (M \times G,\, \mathrm{proj}_1,\, M,\, G)$ が示された.
% 	\begin{figure}[H]
% 		\centering
% 		\begin{tikzcd}[column sep=small]
% 				P \arrow[dr, "\pi_1"'] \arrow[red]{rr}[red]{\tilde{f}} &	& M \times G \arrow{dl}{\mathrm{proj}_1} \\
% 				& M &
% 		\end{tikzcd}
% 		\caption{主 $G$-束の同型}
% 	\end{figure}%
% \end{proof}

\subsection{Lie群の指数写像と基本ベクトル場}

% ベクトル空間 $V$ に値をとる微分形式とは,通常の $\mathcal{M}$ 上の微分形式 $\Omega^\bullet (\mathcal{M})$ に $V$ をテンソル積してできる系列 $V \otimes \Omega^\bullet (\mathcal{M})$ のことを言う.
主束の接続の話に入る前に,Lie群のLie代数について考察する.
この小節は~\cite[Chapter 20]{Lee2012smooth},~\cite[第6章]{Imai2013diff}による.

Lie群 $G$ の上の\underline{微分同相写像}\footnote{従って,命題\ref{prop:diffeo-Frelated}から $L_g,\, R_g$ による\hyperref[prop:diffeo-Frelated]{ベクトル場の押し出し}が一意的に存在する.}
\begin{align}
    \bm{L_g} \colon G &\lto G,\, x \lmto gx, \\
    \bm{R_g} \colon G &\lto G,\, x \lmto xg, \\
\end{align}
のことをそれぞれ\textbf{左移動},\textbf{右移動}と言う.

\begin{mydef}[label=def:left-invariant]{左不変ベクトル場}
    Lie群 $G$ の\textbf{左不変ベクトル場} (left-invariant vector field) とは,$\mathbb{R}$-ベクトル空間
    \begin{align}
        \bm{\Lie{\mathfrak{X}} (G)} \coloneqq \bigl\{\, X \in \mathfrak{X}(G) \bigm| \forall g \in G,\; (L_g)_* X = X \,\bigr\} 
    \end{align}
    の元のこと.i.e. $\forall g \in G$ に対して自分自身と \hyperref[def:F-related]{$L_g$-related} な \cinfty ベクトル場のことを言う.
\end{mydef}

$\forall g \in G$ と $\forall X,\, Y \in \Lie{\mathfrak{X}}(G)$ をとる.このとき $(L_g)_* X = X,\; (L_g)_* Y = Y$ なので,命題\ref{prop:Lie-bracket-natural}の後半から
\begin{align}
    (L_g)_* \comm{X}{Y} = \comm{(L_g)_* X}{(L_g)_* Y} = \comm{X}{Y}
\end{align}
が言える.i.e. 
% \textbf{左不変}\hyperref[def:vecf]{ベクトル場}とは,$\mathbb{R}$-ベクトル空間
\underline{$\Lie{\mathfrak{X}} (G)$ は\hyperref[def:Lie-bracket]{Lieブラケット}について閉じるので,体 $\mathbb{R}$ 上のLie代数になる}.

\begin{myprop}[label=prop:LieAlg]{}
    $G$ をLie群とする.このとき評価写像
    \begin{align}
        \evalunit \colon \Lie{\mathfrak{X}}(M) \lto T_{1_G} G,\; X \lmto X_{1_G}
    \end{align}
    はベクトル空間の同型写像である.
    % よって $\dim \Lie{\mathfrak{X}}(G) = \dim G$ である.
\end{myprop}

\begin{proof}
    $\evalunit$ が $\mathbb{R}$-線型写像であることは明らか.
    \begin{description}
        \item[\textbf{$\bm{\evalunit}$ が単射}] 
        
        $\Ker \evalunit = \{ 0 \}$ を示す.$\forall X \in \Ker \evalunit$ に対して $\evalunit (X) = X_{1_G} = 0$ が成り立つ.
        一方 $X \in \Lie{\mathfrak{X}}(G)$ でもあるので,$\forall g \in G$ に対して $X_g = X_{L_g(1_G)} = (T_{1_G} L_g)(X_{1_G}) = 0$ が言える\footnote{2つ目の等号で\hyperref[def:F-related]{$L_g$-related}の定義を使った.}.
        \item[\textbf{$\bm{\evalunit}$ が全射}] 
        
        $\forall v \in T_{1_G} G$ を1つとり,\hyperref[def:vecf]{\cinfty ベクトル場} $\Lie{v} \in \mathfrak{X}(G)$ を
        \begin{align}
            \label{eq:vL}
            \Lie{v} \colon G \lto TG,\; g \lmto T_{1_G} (L_g) (v)
        \end{align}
        と定義する
        \footnote{
            $\Lie{v}$ が \cinfty であることは次のようにしてわかる:$\forall f \in C^\infty (G)$ をとる.$\gamma(0) = 1_G,\; \dot{\gamma}(0) = v$ を充たす \cinfty 曲線 $\gamma \colon (-\delta,\, \delta) \lto G$ をとると,$\forall g \in G$ に対して $(\Lie{v} f)(g) = v (f \circ L_g) = \dot{\gamma}(0) (f \circ L_g) = \eval{\dv{}{t}}_{t=0} (f \circ L_g \circ \gamma)(t)$ と書ける.$f \circ L_g \circ \gamma \colon (-\delta,\, \delta) \times G \lto \mathbb{R}$ と見做すとこれは \cinfty 写像の合成なので \cinfty 写像であり,右辺は $g$ に関して \cinfty 級である.
        }.
        $\forall g \in G$ に対して $\Lie{v}$ が自分自身と $L_g$-related であることを示す.
        実際,$\forall h \in G$ に対して
        \begin{align}
            T_h(L_g) (\Lie{v}|_h) &= T_h(L_g) \circ T_{1_G}(L_h)(v) = T_{1_G}(L_g \circ L_h)(v) = T_{1_G}(L_{gh})(v) = \Lie{v}|_{gh} = \Lie{v}|_{L_g(h)}
        \end{align}
        が言える.i.e. $\Lie{v} \in \Lie{\mathfrak{X}}(G)$ である.
        従って $\Lie{v}$ に $\evalunit$ を作用させることができ,$\evalunit (\Lie{v}) = \Lie{v}|_{1_G} = T_{1_G} L_{1_G} (v) = v \in \Im \evalunit$ が言えた.
    \end{description}
\end{proof}



% 線型写像
% \begin{align}
%     \iota \colon \Lie{\mathfrak{X}}(G) \lto T_{1_G} G,\; X \lmto X_{1_G}
% \end{align}
% を考える.$\forall X \in \Lie{\mathfrak{X}}(G),\; \forall g \in G$ に対して $X_g = \bigl((L_g)_* X\bigr)_{1_G} = X_{1_G}$ であるから $X \in \Lie{\mathfrak{X}} (G)$ は $X_{1_G}$ だけで完全に決まる.i.e. $\iota$ は単射である.
% 逆に $\forall v \in T_{1_G} G$ に対して,ベクトル場 $X \colon G \lto TG,\; g \lmto \bigl( g,\, T_{1_G}(L_g) v \bigr)$ は $v = \iota(X)$ でかつ $X \in \Lie{\mathfrak{X}}(G)$ を充たすので $\iota$ は全射である.

ここで $\bm{\mathfrak{g}} \coloneqq T_{1_G} G$ とおき,命題\ref{prop:LieAlg}の\eqref{eq:vL}を使って $\mathfrak{g}$ 上のLieブラケットを
\begin{align}
    \comm{X}{Y} \coloneqq \comm{\Lie{X}}{\Lie{Y}}_{1_G} \in \mathfrak{g}
\end{align}
と定義すれば $\evalunit$ はLie代数の同型写像となる.この意味で $\mathfrak{g}$ のことを\textbf{Lie群} $\bm{G}$ \textbf{のLie代数}と呼ぶ.
% さて,慣例に従って $X\in \mathfrak{g}$ に対して $\bm{X^\#} \coloneqq \iota^{-1}(X)$ と書く.

\begin{myexample}[label=ex:gl]{一般線型群とそのLie代数}
    一般線型群 $\LGL(n,\, \mathbb{K}) \subset \Mat{n}{\mathbb{K}}$ のLie代数 $\Lgl(n,\, \mathbb{K}) \coloneqq T_{\unity_n} \LGL(n,\, \mathbb{K})$ を,$\LGL(n,\,\mathbb{K})$ のチャート $\bigl( \LGL(n,\,\mathbb{K}),\, (x^\mu{}_\nu) \bigr)$ の下で考える.
    まず,$\mathbb{K}$-線型写像
    \begin{align}
        \alpha \colon \Lgl(n,\, \mathbb{K}) &\lto \Mat{n}{\mathbb{K}}, \\
        a^\mu{}_\nu \eval{\pdv{}{x^\mu{}_\nu}}_{\unity_n} &\lmto \bigl[\, a^\mu{}_\nu \,\bigr]_{1 \le \mu,\, \nu \le n}
    \end{align}
    は明らかに $\mathbb{K}$-ベクトル空間の同型写像である.
    $\forall a = a^\mu{}_\nu \eval{\pdv{}{x^\mu{}_\nu}}_{\unity_n} ,\, b = b^\mu{}_\nu \eval{\pdv{}{x^\mu{}_\nu}}_{\unity_n} \in \Lgl(n,\, \mathbb{K})$ 
    をとる.このとき $\forall g = \bigl[ \, g^{\mu}{}_\nu \bigr]_{1 \le \mu,\, \nu \le n} \in \LGL(n,\, \mathbb{K})$ に関する左移動は
    \begin{align}
        L_g \bigl(\bigl[\, x^\mu{}_\nu \,\bigr]_{1 \le \mu,\, \nu \le n} \bigr) 
        = \bigl[\, g^\mu{}_\rho x^\rho{}_\nu \,\bigr]_{1 \le \mu,\, \nu \le n}
    \end{align}
    なる $C^\infty$ 写像だから
    \begin{align}
        \Lie{a}{}_g = T_{1_G} (L_g) (a) 
        &= a^\mu{}_\nu\, T_{1_G} (L_g) \left( \eval{\pdv{}{x^\mu{}_\nu}}_{\unity_n} \right) \\
        &= a^\mu{}_\nu\, \pdv{[\, L_g \,]^\rho{}_\sigma}{x^\mu{}_\nu}()(\unity_n) \eval{\pdv{}{x^\rho{}_\sigma}}_{L_g(\unity_n)} \\
        &= g^\rho{}_\mu a^\mu{}_\nu \eval{\pdv{}{x^\rho{}_\nu}}_{g}
    \end{align}
    と計算できる.i.e. 第 $(\mu,\, \nu)$ 成分を取り出す $C^\infty$ 関数を $\mathrm{pr}^\mu{}_\nu \colon \LGL(n,\, \mathbb{K}) \lto \mathbb{K}$ とおくと  $\forall f \in C^\infty\bigl(\LGL(n,\, \mathbb{K})\bigr)$ に対して $\Lie{a} f \in C^\infty \bigl(\LGL(n,\, \mathbb{K})\bigr)$ は
    \begin{align}
        \Lie{a} f(g) = a^\mu{}_\nu\, \mathrm{pr}^\rho{}_\mu(g) \pdv{f}{x^\rho{}_\nu}()(g)
    \end{align}
    と書ける.
    よって
    \begin{align}
        \comm{a}{b} f
        &= \comm{\Lie{a}}{\Lie{b}} f (\unity_n)\\
        &= a^\mu{}_\nu\, \mathrm{pr}^\rho{}_\mu(\unity_n) \eval{\pdv{}{x^\rho{}_\nu}\left(b^\alpha{}_\beta\, \mathrm{pr}^\gamma{}_\alpha \pdv{f}{x^\gamma{}_\beta} \right)}_{\unity_n} \\
        &\quad - b^\mu{}_\nu\, \mathrm{pr}^\rho{}_\mu(\unity_n) \eval{\pdv{}{x^\rho{}_\nu}\left(a^\alpha{}_\beta\, \mathrm{pr}^\gamma{}_\alpha \pdv{f}{x^\gamma{}_\beta} \right)}_{\unity_n}\\
        &= a^\mu{}_\nu b^\nu{}_\beta\, \pdv{f}{x^\mu{}_\beta}()(\unity_n) + \cancel{a^\mu{}_\nu b^\alpha{}_\beta \pdv[2]{f}{x^\mu{}_\nu}{x^\alpha{}_\beta}()(\unity_n)} \\
        &\quad - b^\mu{}_\nu a^\nu{}_\beta \pdv{f}{x^\mu{}_\beta}()(\unity_n) - \cancel{b^\mu{}_\nu a^\alpha{}_\beta \pdv[2]{f}{x^\mu{}_\nu}{x^\alpha{}_\beta}()(\unity_n)} \\
        &= \left( (a^\mu{}_\rho b^\rho{}_\nu - b^\mu{}_\rho a^\rho{}_\nu) \eval{\pdv{}{x^\mu{}_\nu}}_{\unity_n}\right) f
    \end{align}
    であり,
    \begin{align}
        \alpha (\comm{a}{b}) = \bigl[\, a^\mu{}_\rho b^\rho{}_\nu - b^\mu{}_\rho a^\rho{}_\nu \,\bigr]_{1\le \mu,\, \nu \le n}
    \end{align}
    i.e. $\alpha$ はLie代数の同型写像だと分かった.
\end{myexample}


\begin{mytheo}[label=thm:induced-LieAlg-hom]{誘導されるLie代数の準同型}
    Lie群 $G,\, H$ とLie群の準同型 $F \colon G \lto H$ を与える.
    \begin{enumerate}
        \item このとき,$\forall X \in \mathfrak{g}$ に対して $Y \in \mathfrak{h}$ がただ一つ存在して,$\Lie{X}$ と $\Lie{Y}$ が\hyperref[def:F-related]{$F$-related}になる.i.e. $\Lie{Y} = F_* \Lie{X}$ である.
        \item $T_{1_G} F \colon \mathfrak{g} \lto \mathfrak{h},\; X \lmto T_{1_G}F(X)$ はLie代数の準同型である.
    \end{enumerate}
\end{mytheo}

\begin{proof}
    \begin{enumerate}
        \item $Y = T_{1_G} F(X) \in \mathfrak{h}$ に対して $\Lie{X}$ と $\Lie{Y}$ が $F$-relatedであることを示す.
        実際,$F$ がLie群の準同型であることから
        $\forall g,\, h \in G$ について
        \begin{align}
            F \circ L_g (h) = F(gh) = F(g)F(h) = L_{F(g)} \circ F(h)
        \end{align}
        が成り立つこと,i.e. $F \circ L_g = L_{F(g)} \circ F$ に注意すると
        $\forall g \in G$ に対して
        \begin{align}
            T_g F (\Lie{X}|_g) &= T_g F \bigl( T_{1_G} L_g(X) \bigr) \\
            &= T_{1_G}(F \circ L_g)(X) \\
            &= T_{1_G}(L_{F(g)} \circ F)(X) \\
            &= T_{1_H}(L_{F(g)}) \circ T_{1_G} F(X) \\
            &= T_{1_H}(L_{F(g)})(Y) \\
            &= Y_{F(g)}
        \end{align}
        が言える.系\ref{col:pushforward}より $F_* \Lie{X} = \Lie{Y}$ がわかるので $Y$ は一意的に定まる.
        \item $\forall X,\, Y \in \mathfrak{g}$ をとる.(1) と命題\ref{prop:Lie-bracket-natural}-(1) より $\comm{F_* \Lie{X}}{F_* \Lie{Y}}$ は $\comm{\Lie{X}}{\Lie{Y}}$ と $F$-related であるが,(1) で示した一意性から
        \begin{align}
            F_*\comm{\Lie{X}}{\Lie{Y}} = \comm{F_* \Lie{X}}{F_* \Lie{Y}}
        \end{align}
        が言える.両辺の $1_H \in H$ における値をとることで
        \begin{align}
            T_{1_G}F (\comm{X}{Y}) = (F_*\comm{\Lie{X}}{\Lie{Y}})_{1_G} = (\comm{F_* \Lie{X}}{F_* \Lie{Y}})_{1_G} = \comm{X}{Y}
        \end{align}
        が示された.
    \end{enumerate}
\end{proof}


\begin{mydef}[label=def:one-parameter-subgroup]{1パラメータ部分群}
    Lie群の準同型写像 $\mathbb{R} \lto G$ のことをLie群 $G$ の\textbf{1パラメータ部分群} (one-parameter subgroup) と呼ぶ\footnote{1パラメータ部分群自身は\underline{部分Lie群ではない}.}.
\end{mydef}

\begin{myprop}[label=prop:one-parameter-basic]{1パラメータ部分群の特徴付け}
    Lie群 $G$ を与える.
    \begin{enumerate}
        \item $G$ の任意の\hyperref[def:one-parameter-subgroup]{1パラメータ部分群} $\gamma \colon \mathbb{R} \lto G$ に対して,$\gamma$ を初期条件 $\gamma(0) = 1_G$ を充たす\hyperref[def:local-flow]{極大積分曲線}として持つ\hyperref[def:left-invariant]{左不変ベクトル場} $X \in \Lie{\mathfrak{X}}(G)$ が一意的に存在する.
        \item $\forall X \in \Lie{\mathfrak{X}}(G)$ に対して,初期条件 $\gamma(0) = 1_G$ を充たす唯一の $X$ の極大積分曲線 $\gamma \colon \mathbb{R} \lto G$ は $G$ の1パラメータ部分群である.
    \end{enumerate}
    \tcblower
    上述の対応によって $X \in \Lie{\mathfrak{X}}(G)$ から一意的に定まる1パラメータ部分群のことを\textbf{$\bm{X}$ が生成する1パラメータ部分群}と呼ぶ.
\end{myprop}

\begin{marker}
    命題\ref{prop:LieAlg}の同型と併せると
    \begin{align}
        \bigl\{\, G\; \text{の1パラメータ部分群}  \,\bigr\} \overset{{}_*}{\longleftrightarrow} \Lie{\mathfrak{X}}(G) \overset{\evalunit}{\longleftrightarrow} T_{1_G} G
    \end{align}
    の1対1対応がある.i.e. $G$ の任意の1パラメータ部分群 $\gamma$ は,その初速度 $\dot{\gamma}(0) \in T_{1_G} G$ により完全に決定される.
\end{marker}


\begin{proof}
    \begin{enumerate}
        \item $G$ の1パラメータ部分群 $\gamma \colon \mathbb{R} \lto G$ を与える.
        $\dv{}{t} \in \Lie{\mathfrak{X}}(\mathbb{R})$ なので,定理\ref{thm:induced-LieAlg-hom}より,$X \coloneqq \gamma_* \bigl(\dv{}{t}\bigr) \in \Lie{\mathfrak{X}}(G)$ は $\dv{}{t}$ と $\gamma$-related な唯一の左不変ベクトル場である.
        このとき $\forall t_0 \in \mathbb{R}$ に大して
        \begin{align}
            X_{\gamma(t_0)} = T_{t_0} \gamma \left( \eval{\dv{}{t}}_{t=t_0} \right) = \dot{\gamma}(t_0)
        \end{align}
        が成り立ち,$\gamma$ は初期条件 $\gamma(0) = 1_G$ を充たす$X$ の極大積分曲線である.
        \item 定理\ref{thm:left-invariant-complete}より $\forall X \in \Lie{\mathfrak{X}}(G)$ は\hyperref[def:complete-vecf]{完備}なので,
        $X$ は\hyperref[def:global-flow]{大域的なフロー}を\hyperref[thm:fundamental-flow]{生成する}.従って $\gamma(0) = 1_G$ を充たす $X$ の極大積分曲線 $\gamma$ が唯一存在し,その定義域が $\mathbb{R}$ になる.
        
         $\forall g \in G$ をとる.\hyperref[def:left-invariant]{左不変ベクトル場の定義}より $X \in \Lie{\mathfrak{X}}(G)$ は $X$ 自身と \hyperref[def:F-related]{$L_g$-related}なので,命題\ref{prop:natural-integral-curve}から $L_g \circ \gamma \colon \mathbb{R} \lto G$ もまた $X$ の積分曲線である.
        従って $\forall s \in \mathbb{R}$ に対して曲線 $L_{\gamma(s)} \circ \gamma \colon t \lmto L_{\gamma(s)} \bigl( \gamma(t) \bigr) = \gamma(s) \gamma(t)$ は $t=0$ において点 $\gamma(s) \in G$ を通過する $X$ の積分曲線である.
        然るに補題\ref{lem:affine-integral-curve}-(2) より曲線 $t \colon t \lmto \gamma(s+t)$ もまた同一の初期条件を充たす $X$ の積分曲線なので,定理\ref{thm:fundamental-flow}よりこれらは $\forall t \in \mathbb{R}$ において一致しなくてはならない:
        \begin{align}
            \gamma(s) \gamma(t) = \gamma(s+t)
        \end{align}
        i.e. $\gamma \colon \mathbb{R} \lto G$ は1パラメータ部分群である.
    \end{enumerate}
    
\end{proof}

\begin{mydef}[label=def:exp]{指数写像}
    Lie群 $G$ を与える.$\mathfrak{g}$ を $G$ のLie代数とする.
    $G$ の\textbf{指数写像} (exponential map) を
    \begin{align}
        \exp \colon \mathfrak{g} \lto G,\; X \lmto \gamma_{(X)}(1)
    \end{align}
    と定義する.ただし,$\gamma_{(X)} \colon \mathbb{R} \lto G$ は $\Lie{X} \in \Lie{\mathfrak{X}}(G)$ が \hyperref[prop:one-parameter-basic]{生成する1パラメータ部分群}である.
\end{mydef}

\begin{myprop}[label=prop:exp]{指数写像の性質}
    Lie群 $G$ を与える.$\mathfrak{g}$ を $G$ のLie代数とする.
    \begin{enumerate}
        \item $\exp \colon \mathfrak{g} \lto G$ は \cinfty 写像 
        \item $\forall X \in \mathfrak{g}$ に対して,
        \begin{align}
            \gamma_{(X)} \colon \mathbb{R} \lto G,\; t \lmto \exp (tX)
        \end{align}
        は $\Lie{X} \in \Lie{\mathfrak{X}}(G)$ が\hyperref[prop:one-parameter-basic]{生成する1パラメータ部分群}である.
        \item $\forall X \in \mathfrak{g},\; \forall s,\, t \in \mathbb{R}$ に対して
        \begin{align}
            \exp \bigl( (s+t) X\bigr) = \exp (sX) \exp(tX)
        \end{align}
        
        \item $\forall X \in \mathfrak{g}$ に対して
        \begin{align}
            (\exp X)^{-1} = \exp (-X)
        \end{align}
        
        % \item $T_0 \exp \colon T_0 \mathfrak{g} \lto T_{1_G} G \cong \mathfrak{g}$

        \item $H$ を別のLie群,$\textcolor{blue}{F} \colon G \lto H$ を任意のLie群の準同型とするとき,以下の図式が可換になる:
        \begin{center}
            \begin{tikzcd}[row sep=large, column sep=large]
            &\mathfrak{g} \ar[r, "T_{1_G} \textcolor{blue}{F}"]\ar[d, "\exp"'] &\mathfrak{h} \ar[d, "\exp"] \\
            &G \ar[r, blue, "F"] &H
        \end{tikzcd}
        \end{center}
        
        \item $\forall X \in \mathfrak{g}$ に対して,左不変ベクトル場 $\Lie{X} \in \Lie{\mathfrak{X}}(G)$ が\hyperref[thm:fundamental-flow]{生成するフロー} $\theta_{(X)} \colon \mathbb{R} \times G \lto G$ に対して 
        \begin{align}
            \theta_{(X)}(t,\, g) = g \exp(tX) \, \bigl(\,\eqqcolon R_{\exp(tX)}(g)\,\bigr)
        \end{align}
        が成り立つ.

        \item $T_0 (\exp) \colon T_0 \mathfrak{g} \lto \mathfrak{g}$ は恒等写像である.
        \item 点 $0 \in \mathfrak{g}$ の近傍 $U \subset \mathfrak{g}$ および点 $1_G \in G$ の近傍 $V \subset G$ が存在して,$\exp|_U \colon U \lto V$ が微分同相写像になる.
    \end{enumerate}
\end{myprop}

\begin{proof}
    \begin{enumerate}
        \item ~\cite[p.519, Proposition 20.8-(1)]{Lee2012smooth}
        \item $\gamma \colon \mathbb{R} \lto G$ を $\Lie{X} \in \Lie{\mathfrak{X}}(G)$ が\hyperref[prop:one-parameter-basic]{生成する1パラメータ部分群}とする.これは命題\ref{prop:one-parameter-basic}-(2) により $\gamma(0) = 1_G$ を充たす $\Lie{X}$ の唯一の\hyperref[thm:fundamental-flow]{極大積分曲線}である.
        
         $\forall t \in \mathbb{R}$ をとる.このとき補題\ref{lem:affine-integral-curve}-(1) より,\cinfty 曲線 $\tilde{\gamma} \colon \mathbb{R} \lto G,\; s \lmto \gamma(ts)$ は初期条件 $\tilde{\gamma}(0) = 1_G$ を充たすベクトル場 $t \Lie{X}$ の極大積分曲線なので,その一意性から
        \begin{align}
            \gamma_{(X)}(t) = \exp (tX) = \tilde{\gamma}(1) = \gamma(t)
        \end{align}
        が成り立つ.i.e. $\gamma_{(X)} = \gamma$ が言えた.
        \item (2) より $\gamma_{(X)}$ が1パラメータ部分群なので
        \begin{align}
            \exp \bigl( (s+t)X \bigr) = \gamma_{(X)}(s+t) = \gamma_{(X)}(s)\gamma_{(X)}(t) = \exp(sX)\exp(tX)
        \end{align}
        \item (2) より $\gamma_{(X)}$ が1パラメータ部分群なので
        \begin{align}
            \exp X \exp(-X) &= \gamma_{(X)} (1) \gamma_{(X)}(-1) = \gamma_{(X)}(0) = 1_G \\
            \exp (-X) \exp X &= \gamma_{(X)} (-1) \gamma_{(X)}(1) = \gamma_{(X)}(0) = 1_G
        \end{align}
        が言える.i.e. $(\exp X)^{-1} = \exp (-X)$ である.
        \item $\forall X \in \mathfrak{g}$ を1つ固定する.
        % $\forall t \in \mathbb{R}$ を1つとる.
        (2) より \cinfty 写像 $t \lmto \exp \bigl(t\, T_{1_G} \textcolor{blue}{F}(X)\bigr)$ は左不変ベクトル場 $\Lie{\bigl(T_{1_G} \textcolor{blue}{F}(X)\bigr)} = \textcolor{blue}{F}_*(\Lie{X}) \in \Lie{\mathfrak{X}}(G)$ が生成する1パラメータ部分群である.
        ここで,$\sigma \colon \mathbb{R} \lto H,\; t \lmto \textcolor{blue}{F}\bigl(\exp (tX)\bigr)$ とおいたとき
        \begin{align}
            \dot{\sigma}(0)
            &= T_{0} \bigl( \textcolor{blue}{F} \circ \exp (tX) \bigr) \left( \eval{\dv{}{t}}_{t=0} \right) \\
            &= T_{1_G} \textcolor{blue}{F} \circ T_0 \bigl( \exp (tX)  \bigr)  \left( \eval{\dv{}{t}}_{t=0} \right) \\
            &= T_{1_G} \textcolor{blue}{F} \bigl( \dot{\gamma_{(X)}}(0) \bigr) \\
            &= T_{1_G} \textcolor{blue}{F} (X)
        \end{align}
        が成り立つので $\sigma$ もまた左不変ベクトル場 $\Lie{\bigl(T_{1_G} \textcolor{blue}{F} (X)\bigr)} \in \Lie{\mathfrak{X}}(G)$  が生成する1パラメータ部分群であり,その一意性から $\sigma(t) = \exp \bigl( t T_{1_G} \textcolor{blue}{F} (X) \bigr)$ が言える.
        \item $\forall (t,\, g) \in \mathbb{R} \times G$ をとる.\hyperref[def:left-invariant]{左不変ベクトル場の定義}より $\Lie{X} \in \Lie{\mathfrak{X}}(G)$ は $\Lie{X}$ 自身と \hyperref[def:F-related]{$L_g$-related}なので,
        命題\ref{prop:natural-integral-curve}から $L_g \circ \gamma_{(X)} \colon \mathbb{R} \lto G,\; t \lmto \exp (tX)$ もまた $\Lie{X}$ の極大積分曲線である.$L_g \circ \gamma_{(X)}(0) = g$ なので,極大積分曲線の一意性から $L_g \circ \gamma_{(X)} = \theta_{(X)}^{(g)}$ が言える.
        従って
        \begin{align}
            g \exp(tX) = L_g(\exp(tX)) = L_g \circ \gamma_{(X)}(t) = \theta_{(X)}^{(g)}(t) = \theta_{(X)} (t,\, g).
        \end{align}

        \item $\forall X \in T_0 \mathfrak{g}$ を1つとる.$\mathfrak{g}$ 上の $C^\infty$ 曲線 $\gamma \colon \mathbb{R} \lto \mathfrak{g},\; t \lmto t X$ は $\dot{\gamma}(0) = X$ を充たすので
        \begin{align}
            T_0 (\exp)(X)
            &= T_0(\exp) \bigl( \dot{\gamma}(0) \bigr) \\
            &= T_0 (\exp) \circ T_0 \gamma \left( \eval{\dv{}{t}}_{t=0} \right) \\
            &= T_0 (\exp \circ \gamma)  \left( \eval{\dv{}{t}}_{t=0} \right) \\
            &= \eval{\dv{}{t}}_{t=0} \exp(tX) \\
            &= X
        \end{align}
        
        \item (7) より点 $0 \in \mathfrak{g}$ において $T_0 (\exp) \colon T_0 \mathfrak{g} \lto \mathfrak{g} = T_{1_G} G$ が全単射なので,\hyperref[thm:inverse-function-b]{$C^\infty$ 多様体に関する逆関数定理}が使える.
    \end{enumerate}
\end{proof}


% \begin{myprop}[label=prop:Exp,breakable]{指数写像の性質}
%     $G$ をLie群,$\mathfrak{g}$ をそのLie代数とする.このとき以下が成り立つ:
%     \begin{enumerate}
%         \item $\forall X \in \mathfrak{g}$ に対して,\hyperref[def:vecf]{ベクトル場} $X^\# \in \Lie{\mathfrak{X}} (G)$ は\hyperref[def:complete-vecf]{完備}である.従って\hyperref[def:global-flow]{大域的な流れ}\footnote{つまり,Lie群 $\mathbb{R}$ の\hyperref[def:Lie-action]{作用}.} $\theta \colon \mathbb{R} \times G \lto G$ を生成する.
%         \item $\forall X \in \mathfrak{g},\; \forall t \in \mathbb{R}$ に対して $\exp_G (t X) \coloneqq \theta(t,\, 1_G)$ と書くと,$\theta_t = R_{\exp_G (tX)} \colon G \lto G$ が成り立つ.
%         \item $\forall X \in \mathfrak{g},\; \forall s,\, t \in \mathbb{R}$ に対して $\exp_G (sX) \exp_G (tX) = \exp_G ((s+t)X) \in G$ が成り立つ.i.e.
%         $\{\exp_G (tX)\}_{t \in \mathbb{R}}$ 
%         は $G$ の1パラメータ部分群である.
%         \item $\forall X \in \mathfrak{g}$ に対して
%         $\eval{\dv{}{t}}_{t=0} \bigl(\exp_G (tX)\bigr) = X \in \mathfrak{g}$ が成り立つ.
%         \item 対応 $X \lmto \{\exp_G (tX)\}_{t \in \mathbb{R}}$ は $\mathfrak{g}$ から $G$ の1パラメータ部分群全体の全単射である.
%     \end{enumerate}
%     従って,\textbf{指数写像} $\exp_G \colon \mathfrak{g} \lmto G$ が定義される.
% \end{myprop}

% \begin{proof}
%     \begin{enumerate}
%         \item 命題\ref{prop:existence-integral}より,ベクトル場 $X^{\#}$ の\hyperref[def:integral-curve]{積分曲線} $\gamma_{1_G} \colon (-\varepsilon,\, \varepsilon) \lto G$ で,$\gamma_\varepsilon (0) = 1_G$ を充たすものが存在する.
%         このとき $\forall g \in G$ に対して \cinfty 曲線を
%         \begin{align}
%             \gamma_g \colon (-\varepsilon,\, \varepsilon) \lto G,\; t \lmto g \gamma_{1_G} (t)
%         \end{align}
%         と定義すると
%         \begin{align}
%             \eval{\dv{}{s}}_{s=0} \gamma_g(t + s) 
%             &= \eval{\dv{}{s}}_{s=0}\bigl( L_g \circ \gamma_{1_G}(t + s)\bigr) \\
%             &= T_{\gamma_{1_G}(t)} (L_g) \left( \eval{\dv{}{s}}_{s=0} \gamma_{1_G}(t + s) \right) \\
%             &= T_{\gamma_{1_G}(t)} (L_g) \left( (X^\#)_{\gamma_{1_G} (t)} \right) \\
%             &= (X^\#)_{\gamma_{g} (t)}
%         \end{align}
%         が成り立つので,$\gamma_g$ は初期条件 $\gamma_g (0) = g$ を充たす $X^\#$ の積分曲線である.

%          ここで $g_1 \coloneqq \gamma_{1_G} (\varepsilon/2),\; g_2 \coloneqq \gamma_{1_G}(-\varepsilon/2)$ とおいて
%         \begin{align}
%             \gamma_{1_G}(t) \coloneqq 
%             \begin{cases}
%                 \gamma_{g_1} (t - \frac{\varepsilon}{2}), & t \in [\varepsilon/2,\, 3\varepsilon/2) \\
%                 \gamma_{1_{G}} (t), & t \in [-\varepsilon/2,\, \varepsilon/2] \\
%                 \gamma_{g_2} (t + \frac{\varepsilon}{2}), & t \in (-3\varepsilon/2,\, -\varepsilon/2] 
%             \end{cases}
%         \end{align}
%         と定義すると $\gamma_{1_G} \colon (-3\varepsilon/2,\, 3\varepsilon/2) \lto G$ はwell-definedで,$X^\#$ の積分曲線となる.同様の議論で定義域を拡張すれば,初期条件 $\gamma_{1_G} (0) = 1_G$  を充たす $X^\#$ の積分曲線 $\gamma_{1_G} \colon \mathbb{R} \lto G$ が得られる.

%          次に,上の議論により得られた $\gamma_{1_G}$ を使って $\forall g \in G$ に対して
%         \begin{align}
%             \gamma_g \colon \mathbb{R} \lto G,\; t \lmto g \gamma_{1_G}(t)
%         \end{align}
%         と定義するとこれは $\gamma_g (0) = g$ を充たす $X^\#$ の積分曲線である.よって $X^\#$ は大域的な流れ
%         \begin{align}
%             \theta \colon \mathbb{R} \times G \lto G,\; (t,\, g) \lmto \gamma_g(t)
%         \end{align}
%         を生成する.
%         \item (1) の証明より,$\forall t \in \mathbb{R},\, \forall g \in G$ に対して
%         \begin{align}
%             \exp_G (tX) &= \theta(t,\, 1_G) = \gamma_{1_G}(t), \\
%             \theta_t (g) &\coloneqq \theta(t,\, g) = \gamma_g(t) = g \gamma_{1_G} (t) = R_{\exp_G (tX)}(g)
%         \end{align}
%         が言える.
%         \item (1) で得た $\theta$ が大域的な流れなので
%         \begin{align}
%             \exp_G (sX) \exp_G (tX) &= \theta_t \bigl( 1_G\exp_G (sX) \bigr)  = \theta_t \circ \theta_s (1_G) = \theta_{s+t}(1_G) = \exp_G \bigl( (s+t) X\bigr) 
%         \end{align}
%         が成り立つ.
%         \item 
%         \begin{align}
%             \eval{\dv{}{t}}_{t=0} \bigl( \exp_G(tX) \bigr) &= \eval{\dv{}{t}}_{t=0} \gamma_{1_G} (t) = (X^\#)_{1_G} = \iota (X^\#) = X
%         \end{align}
%         \item 
%     \end{enumerate}
    
% \end{proof}


\begin{mydef}[label=def:diff-rep]{微分表現}
    $V$ を $\mathbb{K}$-ベクトル空間とする.Lie群 $G$ の表現 $\rho \colon G \lto \LGL (V)$ の,$1_G \in G$ における微分
    $T_{1_G} \rho \colon \mathfrak{g} \lto \mathfrak{gl}(V)$ はLie代数の表現である.この $T_{1_G} \rho$ のことを $\rho$の\textbf{微分表現} (differential representation) と呼ぶ.
\end{mydef}

\begin{myexample}[label=def:Lie-adj]{随伴表現}
    $\forall g \in G$ に対して準同型 $F_g \colon G \lto G,\; x \lmto gxg^{-1}$ を考えると $F_{gh} = F_g \circ F_h$ が成り立つ.
    故に,$1_G \in G$ における微分
    \begin{align}
        T_{1_G} (F_g) \colon \mathfrak{g} \lto \mathfrak{g}
    \end{align}
    は,$T_{1_G}$ の関手性から $T_{1_G}(F_{gh}) = T_{1_G}(F_g) \circ T_{1_G}(F_h)$ を充たす.よって
    \begin{align}
        \Adj \colon G \lto \LGL (\mathfrak{g}),\; g \lmto T_{1_G} (F_g)
    \end{align}
    はLie群 $G$ の表現となる\footnote{厳密には $\Adj$ の \cinfty 性を示さなくてはならない.証明は~\cite[p.534, Proposition 20.24]{Lee2012smooth}を参照.}.
    これをLie群 $G$ の\textbf{随伴表現} (adjoint representation) と呼ぶ.
    
     $\Adj$ の微分表現を\hyperref[def:exp]{指数写像}を使って計算してみよう.
    $\forall X \in \mathfrak{g}$ をとる.命題\ref{prop:exp}-(2) により曲線 $\gamma_{(X)} \colon t \lmto \exp(tX)$ は $X$ が生成する1パラメータ部分群なので,
    命題\ref{prop:infinitesimal-generator-local}から $\dot{\gamma_{(X)}}(0) = X$ である.従って $\forall Y \in \mathfrak{g}$ に大して
    \begin{align}
        \bigl(T_{1_G}(\Adj) (X)\bigr)Y 
        &= \Bigl( T_{1_G} (\Adj)\bigl(\dot{\gamma_{(X)}}(0)\bigr) \Bigr) Y \\
        &= T_{1_G} (\Adj) \circ T_0 \gamma_{(X)} \left( \eval{\dv{}{t}}_{t=0} \right)  Y \\
        &= T_{0} (\Adj \circ \gamma_{(X)}) \left( \eval{\dv{}{t}}_{t=0}  \right) Y \\
        &= \left(\eval{\dv{}{t}}_{t=0} \Adj \bigl(\exp (tX)\bigr) \right) Y \\
        &= \left(\eval{\dv{}{t}}_{t=0} \Adj \bigl(\exp (tX)\bigr)(Y) \right) \\
        &= \eval{\dv{}{t}}_{t=0} \Bigl(T_{1_G}\bigl( F_{\exp (tX)}  \bigr)(Y) \Bigr) \\
        &= \eval{\dv{}{t}}_{t=0} \Bigl(T_{1_G}\bigl( R_{(\exp (tX))^{-1}} \circ L_{\exp (tX)} \bigr)(\Lie{Y}_{1_G}) \Bigr)\\
        &= \eval{\dv{}{t}}_{t=0} \Bigl(T_{L_{\exp (tX)}(1_G)}\bigl( R_{\exp (-tX)} \bigr) \circ T_{1_G} \bigl( L_{\exp (tX)} \bigr)(\Lie{Y}_{1_G}) \Bigr)\\
        &= \eval{\dv{}{t}}_{t=0} \Bigl(T_{\exp (tX)}\bigl( R_{\exp (-tX)} \bigr) \bigl(\Lie{Y}_{\exp (tX)}\bigr) \Bigr)
    \end{align}
    ここで,命題\ref{prop:exp}-(6) より $\Lie{X} \in \Lie{\mathfrak{X}}(G)$ が\hyperref[thm:fundamental-flow]{生成するフロー}が $\theta_t (g) = R_{\exp(tX)} (g)$ と書かれることを思い出すと,
    \begin{align}
        &\eval{\dv{}{t}}_{t=0} \Bigl(T_{\exp (tX)}\bigl( R_{\exp (-tX)} \bigr) \bigl(\Lie{Y}_{\exp (tX)}\bigr) \Bigr) \\
        &= \eval{\dv{}{t}}_{t=0} T_{\theta_t(1_G)} \bigl(\theta_{-t}\bigr) \bigl( \Lie{Y}_{\theta_t (1_G)} \bigr) \\
        &= \lim_{t \to 0} \frac{T_{\theta_t(1_G)} \bigl(\theta_{-t}\bigr) \bigl( \Lie{Y}_{\theta_t (1_G)} \bigr) - \Lie{Y}_{1_G}}{t} \\
        &= (\Liedv{\Lie{X}} \Lie{Y})_{1_G} \\
        &= \comm{\Lie{X}}{\Lie{Y}}_{1_G} \\
        &= \comm{X}{Y}
    \end{align}
    となる.ただし3つ目の等号で\hyperref[def:Liedv]{Lie微分の定義}を使った.結局
    \begin{align}
        \ad \coloneqq T_{1_G} (\Adj) \colon \mathfrak{g} \lto \mathfrak{gl}(\mathfrak{g}),\; X \lmto (Y \mapsto \comm{X}{Y})
    \end{align}
    であることが分かった.
    % 実際,$F_g = R_{g^{-1}} \circ L_g$ で,命題\ref{prop:Exp}-(2) より $\{R_{\exp_G (tX)}\}_{t \in \mathbb{R}}$ は $X^\#$ の生成する1パラメータ変換群だから,
    % \begin{align}
    %     \ad(X)(Y) &= \eval{\dv{}{t}}_{t=0}()
    % \end{align}
    
\end{myexample}

\begin{myexample}[label=ex:gl-adj]{一般線型群の随伴表現}
    $G = \LGL (n,\, \mathbb{K})$ としたときの\hyperref[def:Lie-adj]{随伴表現}を考える.$\LGL(n,\, \mathbb{K})$ のチャート $\bigl( \LGL(n,\, \mathbb{K}),\, (x^\mu{}_\nu) \bigr)$ をとると $\forall g = \bigl[\, g^\mu{}_\nu \,\bigr]_{1 \le \mu,\, \nu \le n} \in \LGL(n,\, \mathbb{K})$ に関して $C^\infty$ 写像 $F_g \colon G \lto G$ は
    \begin{align}
        F_g \bigl( \bigl[\, x^\mu{}_\nu \,\bigr]_{1 \le \mu,\, \nu \le n}  \bigr) = \bigl[ \, g^\mu{}_\rho x^\rho{}_\sigma [g^{-1}]^\sigma{}_\nu \, \bigr]_{1 \le \mu,\, \nu \le n}
    \end{align}
    と座標表示されるので,$\forall c^\mu{}_\nu \eval{\pdv{}{x^\mu{}_\nu}}_{\unity_n} \in \Lgl (n,\, \mathbb{K})$ の自然基底に関して
    \begin{align}
        \Ad(g) \left( c^\mu{}_\nu \eval{\pdv{}{x^\mu{}_\nu}}_{\unity_n} \right) 
        &= c^\mu{}_\nu\, T_{\unity_n} \left( \eval{\pdv{}{x^\mu{}_\nu}}_{\unity_n} \right) \\
        &= c^\mu{}_\nu\, \pdv{[\, F_g \,]^\rho{}_\sigma}{x^\mu{}_\nu}()(\unity_n) \eval{\pdv{}{x^\rho{}_\sigma}}_{F_g(\unity_n)} \\
        &= g^\rho{}_\mu c^\mu{}_\nu [g^{-1}]^\nu{}_\sigma \eval{\pdv{}{x^\rho{}_\sigma}}_{\unity_n}
    \end{align}
    がわかる.\exref{ex:gl}のLie代数の同型写像 $\alpha \colon \Lgl(n,\, \mathbb{K}) \lto \Mat{n}{\mathbb{K}}$ を使うと,これは行列の積の意味で
    \begin{align}
        \alpha \circ \Ad (g)  \circ \alpha^{-1} (X) = g X g^{-1}
    \end{align}
    を意味する.
    以上の議論は $G$ が $\LGL(n,\, \mathbb{K})$ の部分Lie群の場合にも成立するが,\underline{大抵の場合Lie代数の同型写像 $\alpha$ は省略される}.
\end{myexample}

定理\ref{thm:fundamental-flow}によって,\cinfty 多様体 $M$ 上の\hyperref[def:vecf-complete]{完備}なベクトル場 $X$ がLie群 $\mathbb{R}$ の $M$ への作用 $\theta \colon \mathbb{R} \times M \lto M$ を一意に定めることが分かる.
そしてこのような状況を指して,ベクトル場 $X$ はLie群 $\mathbb{R}$ の作用 $\theta$ の無限小生成子であると言うのだった.
この考えを任意のLie群 $G$ の,任意の $M$ への\underline{右作用}に拡張することができる.
つまり,任意のLie群 $G$ の任意の右作用 $\btl \colon M \times G \lto M$ は,ただ一つの無限小生成子を持つ.

\begin{mydef}[label=def:fundamental-vecf]{基本ベクトル場}
    Lie群 $G$ が\cinfty 多様体 $M$ に\hyperref[def:Lie-action]{右から作用}しているとする.この右作用を $\btl \colon M \times G \lto M$ と書く.
    \begin{itemize}
        \item $\forall X \in \mathfrak{g}$ に対して,\textbf{基本ベクトル場} (fundamental vector field) $\bm{X^{\#}} \in \mathfrak{X}(M)$ を次のように定める:
        \begin{align}
            X^{\#}_x \coloneqq \eval{\dv{}{t}}_{t=0} \bigl(x \btl \exp(tX)\bigr) \in T_x M
        \end{align}
        \item 写像
        \begin{align}
            \btl^{\#} \colon \mathfrak{g} \lto \mathfrak{X}(M),\; X \lmto X^\#
        \end{align}
        のことを\textbf{右作用 $\btl$ の無限小生成子}と呼ぶ.
    \end{itemize}
    \tcblower
    上の状況下で
    \begin{itemize}
        \item $\forall \textcolor{blue}{g} \in G$ に対して\textbf{右作用移動} $R_{\textcolor{blue}{g}} \colon M \lto M$ を $R_g (x) \coloneqq x \btl \textcolor{blue}{g}$ と定義する.
        \item $\forall \textcolor{red}{x} \in M$ に対して\textbf{右作用軌道} $R^{(\textcolor{red}{x})} \colon G \lto M$ を $R^{(x)} (g) \coloneqq \textcolor{red}{x} \btl g$ と定義する.
    \end{itemize}
\end{mydef}

$\forall X \in \mathfrak{g}$ に対して,\cinfty 写像\footnote{これは命題\ref{prop:exp}-(6)からの類推だと言える.}
\begin{align}
    \theta_{(X)} \colon \mathbb{R} \times M \lto M,\; (t,\, x) \lmto x \btl \exp(tX) = R_{\exp(tX)} (x)
\end{align}
は\hyperref[def:flow-global]{大域的フロー}である
\footnote{
    実際,命題\ref{prop:exp}から
    \begin{align}
        \theta_{(X)}(0,\, x) &= x \btl 1_G = x, \\
        \theta_{(X)}(t+s,\, x) &= x \btl \exp \bigl( (s+t) X \bigr) \\
        &= x \btl \bigl(\exp(sX)\exp(tX)\bigr) \\
        &= x \btl \exp(sX) \btl \exp(tX) \\
        &= \theta_{(X)} \bigl(t ,\, \theta_{(X)}(s,\, x)\bigr)
    \end{align}
    が成り立つ.
}
.
この大域的フローの\hyperref[def:thm:fundamental-flow]{無限小生成子}はベクトル場
\begin{align}
    x \lmto \Bigl( x,\, \dot{\theta_{(X)}^{(x)}} (0) \Bigr) = \left( x,\, T_0 \bigl( \theta_{(X)}^{(x)} \bigr) \left( \eval{\dv{}{t}}_{t=0} \right)   \right) 
\end{align}
であるが\footnote{強引に書くと $\theta^{(x)}_{(X)} = R^{(x)} \circ \exp (\mhyphen X) \colon \mathbb{R} \lto M$ と言うことになる.},これがまさに $X^\# \in \mathfrak{X}(M)$ になっている.つまり,基本ベクトル場は $\forall x \in M$ において,$\forall f \in C^\infty (M)$ に
\begin{align}
    X^\#_x f = T_0 \bigl( \theta_{(X)}^{(x)} \bigr) \left( \eval{\dv{}{t}}_{t=0} \right)f = \eval{\dv{}{t}}_{t=0} \bigl( f \circ \theta_{(X)}^{(x)}  \bigr) (t) = \eval{\dv{}{t}}_{t=0} f \bigl(x \btl \exp(tX)\bigr)
\end{align}
と作用する.

もしくは,次のように考えることもできる:
曲線 $\gamma_{(X)} \colon t  \lmto \exp(tX)$ は初速度 $\dot{\gamma_{(X)}}(0) = X$ なので,
\begin{align}
    T_{1_G}(R^{(x)}) (X) 
    &= T_{1_G} (R^{(x)}) \bigl( \dot{\gamma_{(X)}}(0) \bigr) \\
    &= T_{1_G} (R^{(x)}) \circ T_0 (\gamma_{(X)}) \left( \eval{\dv{}{t}}_{t=0} \right) \\
    &= T_0 (R^{(x)} \circ \gamma_{(X)}) \left( \eval{\dv{}{t}}_{t=0} \right) \\
    &= T_0 (\theta_{(X)}^{(x)})\left( \eval{\dv{}{t}}_{t=0} \right) \\
    &= X^\#_x. \label{eq:fundamental-vecf}
\end{align}
このことから $\btl^\#$ が $\mathbb{R}$-線型写像だとわかる.なお,等式\eqref{eq:fundamental-vecf}は\hyperref[thm:connection-basic]{主束の接続形式を調べる}際に極めて重要な役割を果たす.

\begin{mylem}[label=lem:fundamental-vecf]{}
    Lie群 $G$ の\cinfty 多様体 $M$ への右作用 $\btl \colon M \times G \lto M$ を与える.
    
    このとき $\forall x \in M$ および $\forall X \in \mathfrak{g}$ に対して,$\Lie{X} \in \Lie{\mathfrak{X}}(G)$ とその\hyperref[def:fundamental-vecf]{基本ベクトル場} $X^\#$ は $R^{(x)}$\hyperref[def:F-related]{-related}である
\end{mylem}

\begin{proof}
    $\forall g,\, h \in G$ に対して
    \begin{align}
        R^{(x \btl g)}(h) = x \btl g \btl h = x \btl(gh) = x \btl L_g(h) = R^{(x)} \circ L_g(h)
    \end{align}
    が成り立つことに注意する.
    $\forall g \in G$ をとり,$y \coloneqq R^{(x)}(g) = x \btl g$ とおく.$\Lie{X}$ が\hyperref[def:left-invariant]{左不変ベクトル場}であることから
    \begin{align}
        X^\#_y &= T_{1_G}(R^{(y)})(X) \\
        &= T_{1_G} (R^{(x \btl g)}) (\Lie{X}_{1_G}) \\
        &= T_{1_G} (R^{(x)} \circ L_g)(\Lie{X}_{1_G}) \\
        &= T_{L_g(1_G)} (R^{(x)}) \circ T_{1_G}(L_g)(\Lie{X}_{1_G}) \\
        &= T_{g} (R^{(x)}) (\Lie{X}_g)
    \end{align}
    が言えた.
\end{proof}


\begin{myprop}[label=prop:infinitesimal-generator-R]{$\btl^\#$ はLie代数の準同型}
    Lie群 $G$ の\cinfty 多様体 $M$ への右作用 $\btl \colon M \times G \lto M$ を与える.
    このとき\hyperref[def:fundamental-vecf]{右作用 $\btl$ の無限小生成子}
    \begin{align}
        \btl^\# \colon \mathfrak{g} \lto \mathfrak{X}(M),\; X \lmto X^\#
    \end{align}
    はLie代数の準同型である.
\end{myprop}

\begin{proof}
    % まず,$\forall x \in M$ および $\forall X \in \mathfrak{g}$ に対して,$\Lie{X} \in \Lie{\mathfrak{X}}(M)$ とその\hyperref[def:fundamental-vecf]{基本ベクトル場} $X^\#$ が\hyperref[def:F-related]{$R^(x)$-related}であることを示す.
    % $\forall g \in G$ をとり,$y \coloneqq R^{(x)}(g) = x \btl g$ とおく.$\Lie{X}$ が\hyperref[def:left-invariant]{左不変ベクトル場}であることから
    % \begin{align}
    %     X^\#_y &= T_{1_G}(R^{(y)})(X) \\
    %     &= T_{1_G} (R^{(x \btl g)}) (\Lie{X}_{1_G}) \\
    %     &= T_{1_G} (R^{(x)} \circ L_g)(\Lie{X}_{1_G}) \\
    %     &= T_{g} (R^{(x)}) \circ T_{1_G}(L_g)(\Lie{X}_{1_G}) \\
    %     &= T_{g} (R^{(x)}) (\Lie{X}_g)
    % \end{align}
    % が言えた.
    $\forall X,\, Y \in \mathfrak{g}$ をとる.
    補題\ref{lem:fundamental-vecf}と\hyperref[prop:Lie-bracket-natural]{Lieブラケットの自然性}から $\forall x \in M$ に対して $\comm{\Lie{X}}{\Lie{Y}}$ と $\comm{X^\#}{Y^\#}$ が $R^{(x)}$-relatedだと分かる.i.e.
    \begin{align}
        \comm{X^\#}{Y^\#}_x = \comm{X^\#}{Y^\#}_{R^{(x)}(1_G)} = T_{1_G} (R^{(x)}) (\comm{\Lie{X}}{\Lie{Y}}_{1_G}) = T_{1_G} (R^{(x)}) (\comm{X}{Y}) = \comm{X}{Y}^\#_x
    \end{align}
    が言えた.
\end{proof}

しばらくの間Lie群 $G$ (もしくはその部分群)のLie代数を $\mathrm{Lie}(G) \coloneqq \mathfrak{g}$ と書くことにする\footnote{例えば~\cite{Lee2012smooth}では,$\mathrm{Lie}(G) \coloneqq \Lie{\mathfrak{X}}(G)$ と定義しているので注意.同型なので然程問題にはならないが...}.

\begin{myprop}[label=prop:fundamental-vecf-basic]{基本ベクトル場の零点}
    Lie群 $G$ の\cinfty 多様体 $M$ への右作用 $\btl \colon M \times G \lto M$ を与える.
    このとき,以下の2つは同値である:
    \begin{enumerate}
        \item $X \in \mathfrak{g}$ の\hyperref[def:fundamental-vecf]{基本ベクトル場} $X^\#$ が点 $x \in M$ において $X^\#_x = 0$ になる
        \item $X \in \mathrm{Lie}\bigl(\Stab (x)\bigr)$
    \end{enumerate}
    ただし,$\Stab (x) \subset G$ は点 $x \in M$ の安定化部分群\footnote{つまり,$\Stab (x) \coloneqq \bigl\{\, g \in G \bigm| x \btl g = x \,\bigr\}$ }である.
\end{myprop}

\begin{proof}
    \begin{description}
        \item[\textbf{(1) $\bm{\Longleftarrow}$ (2)}] 
        
        $X \in \mathrm{Lie}\bigl(\Stab (x)\bigr)$ ならば $\forall t \in \mathbb{R}$ に対して $\exp (tX) \in \mathrm{Stab} (x)$ である.従って $x \in M$ の近傍上で定義された任意の \cinfty 関数 $f$ に対して
        \begin{align}
            X^\#_x f = \eval{\dv{}{t}}_{t=0} f\bigl(x \btl \exp(tX)\bigr) = \eval{\dv{}{t}}_{t=0} f(x) = 0.
        \end{align}
        と計算できる\footnote{$f\bigl(x \btl \exp(tX)\bigr)$ が $t$ に関して定数関数なので.}
        \item[\textbf{(1) $\bm{\Longrightarrow}$ (2)}] 
        
        $X^\#_x = 0$ とする.このとき定数写像 $\gamma \colon \mathbb{R} \lto M,\; t \lmto x$ が
        \begin{align}
            \dot{\gamma}(t) = 0 = X^\#_{\gamma(t)}
        \end{align}
        を充たすので,初期条件 $\gamma(0) = x$ を充たす $X^\# \in \mathfrak{X}(M)$ の極大積分曲線となる.
        一方,$\theta_{(X)}^{(x)} \colon \mathbb{R} \lto M,\; t \lmto x \btl \exp(tX)$ もまた同一の初期条件をみたす $X^\#$ の極大積分曲線だったので,その一意性から $\theta_{(X)}^{(x)} = \gamma \IFF x \btl \exp(tX) = x\quad \forall t \in \mathbb{R} \IFF \exp(tX) \in \Stab(x)\quad \forall t \in \mathbb{R}$ が言えた.従って $X \in \mathrm{Lie}(\Stab(x))$ である.

    \end{description}
\end{proof}

\begin{mycol}[label=col:fundamental-vecf-basic]{}
    Lie群 $G$ の\cinfty 多様体 $M$ への右作用 $\btl \colon M \times G \lto M$ を与える.

    このとき,$\forall x \in M$ の\hyperref[def:fundamental-vecf]{右作用軌道} $R^{(x)} \colon G \lto M$ の微分
    \begin{align}
        T_{1_G} (R^{(x)}) \colon \mathfrak{g} \lto T_x M
    \end{align}
    に対して
    \begin{align}
        \Ker \bigl( T_{1_G}(R^{(x)}) \bigr) = \mathrm{Lie} \bigl( \Stab(x) \bigr) 
    \end{align}
    が成り立つ.
\end{mycol}

\begin{proof}
    $\forall X \in \mathfrak{g}$ をとる.\eqref{eq:fundamental-vecf}と命題\ref{prop:fundamental-vecf-basic}から
    \begin{align}
        X \in \Ker \bigl( T_{1_G}(R^{(x)}) \bigr) 
        &\IFF  T_{1_G}(R^{(x)}) (X) = 0 \\
        &\IFF X^\#_x = 0 \\
        &\IFF X \in \mathrm{Lie} \bigl( \Stab(x) \bigr) 
    \end{align}
    
\end{proof}


% $\forall X,\, Y \in \mathfrak{g}$ に対して
% \begin{align}
%     \comm{X^{\#}}{Y^{\#}} = \comm{X}{Y}^{\#}
% \end{align}
% が成り立つ.

\subsection{主束の接続}

与えられた \cinfty 多様体 $M$ 上の $\bm{k}$-\textbf{形式} ($k$-form) とは,外積代数束(これは\hyperref[def:vect]{ベクトル束}になる)
\begin{align}
    \extp^k T^* M \coloneqq \coprod_{p \in M} \left( \extp^k T^*_p M \right) 
\end{align}
の\hyperref[def.section]{\cinfty 切断}のことである.$k$-形式全体の集合を
\begin{align}
    \bm{\Omega^k (M)} \coloneqq \Gamma \Bigl( \extp^k T^* M \Bigr)
\end{align}
と書く.

任意の $\mathbb{K}$-ベクトル空間 $V,\, W$ に関して,自然な $\mathbb{K}$-ベクトル空間の同型
\begin{align}
    \underbrace{V^* \otimes \cdots \otimes V^*}_k \otimes W \cong \bigl\{\, f \colon \underbrace{V \times \cdots \times V}_k \lto W  \bigm| \text{多重線型写像} \,\bigr\} 
\end{align}
がある.
$M$ を底空間とする任意の\hyperref[def:vect]{ベクトル束} $E \xrightarrow{\pi} M$ が与えられたとき,この同型を念頭において,$\bm{E}$ \textbf{値 $\bm{k}$-形式} ($E$-valued $k$ form) をテンソル積束
\begin{align}
    \left(\extp^k T^* M\right) \otimes E
\end{align}
の \cinfty 切断として定義する.$E$ 値 $k$-形式全体の集合を
\begin{align}
    \label{eq:E-valued}
    \bm{\Omega^k (M;\, E)} \coloneqq \Gamma \Bigl( \left(\extp^k T^* M\right) \otimes E \Bigr) 
\end{align}
と書く\footnote{$\bm{\Omega^0(M;\, E) = \Gamma(E)}$ に注意.}.特に $E$ があるベクトル空間 $V$ に対して $E = M \times V$ の形をした自明束の場合,代わりに
\begin{align}
    \bm{\Omega^k (M;\, V)} \coloneqq \Omega^k (M;\, M \times V)
\end{align}
と書き,\textbf{$\bm{V}$ 値 $\bm{k}$-形式}と呼ぶ\footnote{$V$ が有限次元ベクトル空間ならば,$\Omega^k (M;\, V) \cong \Omega^k (M) \otimes_{\mathbb{R}} V$ が成り立つ.}.

さて,Lie群に関する準備が終わったのでいよいよ\hyperref[def.PFD]{主束}の接続を定義する.この小節の内容は~\cite[第6章]{Imai2013diff}, \cite[\S 28]{Tu2017differential}が詳しい.


\begin{mydef}[label=def:connection,breakable]{主束の接続}
    $G \hookrightarrow P \xrightarrow{\pi} M$ を\hyperref[def.PFD]{主束}とする.$\forall g \in G$ に対して,命題\ref{prop.PFD_right}の右作用によって\hyperref[def:fundamental-vecf]{右作用移動}を $R_g \colon P \lto P,\; u \lmto u \btl g$ と定義する.
    \begin{itemize}
        \item 分布 $\bigl\{\, H_u \subset T_u P \bigm| u \in P \,\bigr\}$ が $P$ 上の\textbf{接続} (connection) であるとは,以下の2条件が成り立つことを言う:
        \begin{description}
            \item[\textbf{(C-1)}]  $\forall u \in P$ に対して
            \begin{align}
                T_u P = \Ker (T_u \pi) \oplus H_u
            \end{align}
            \item[\textbf{(C-2)}] $\forall u \in P,\; \forall g \in G$ に対して
            \begin{align}
                T_u(R_g) (H_u) = H_{R_g(u)}
            \end{align}
            が成り立つ(分布 $\{H_u\}$ は\textbf{$G$-不変}).
        \end{description}
        $\Ker T_u (\pi),\, H_u$ をそれぞれ $T_u P$ の\textbf{垂直部分空間}, \textbf{水平部分空間}と呼ぶ.
        \item $\mathfrak{g}$ \hyperref[eq:E-valued]{値1-形式} $\omega \in \Omega^1(P;\, \mathfrak{g})$ が\textbf{接続形式}であるとは,次の2条件を充たすことをいう:
        \begin{description}
            \item[\textbf{(CF-1)}] $\forall X \in \mathfrak{g}$ に対して
            \begin{align}
                \omega(X^\#) = X
            \end{align}
            \item[\textbf{(CF-2)}] $\forall g \in G$ に対して
            \begin{align}
                (R_g)^* \omega = \Adj (g^{-1})(\omega)
            \end{align}
            ただし $\Adj \colon G \lto \LGL (\mathfrak{g})$ はLie群 $G$ の\hyperref[def:Lie-adj]{随伴表現}である.
        \end{description}
    \end{itemize}
    
\end{mydef}

本題に入る前に,微分幾何学の風習への注意をしておく.境界あり/なし \cinfty 多様体 $M$ とその部分多様体 $S \subset M$ を与える.
このとき包含写像を $\iota \colon S \hookrightarrow M$ と書くと,$\forall p \in S \subset M$ に対して $T_p S$ を $T_p \iota (T_p S)$ と同一視する
\footnote{
    つまり $\forall v \in T_p S$ は $\forall f \in C^\infty (S)$ に $v(f)$ として作用するが,$v \in T_p S \subset T_p M$ と見做す時は $\forall f \in C^\infty (M)$ に,$\bm{T_p \iota(v)}f = v(f \circ \iota) = v(f|_S)$ として作用する.
}
ことで $T_p M$ の部分ベクトル空間と見做すのである~\cite[p.116]{Lee2012smooth}.

さて,主束 $G \hookrightarrow P \xrightarrow{\pi} M$ において $\forall u \in P$ を1つ固定する.
$G_{\pi(u)} \coloneqq \pi^{-1}(\{\pi(u)\})$ とおいたとき,
$\forall X \in T_u G_{\pi(u)} \subset T_u P$ (i.e. 点 $u \in \bm{P}$ におけるファイバー方向の接空間)の,$\pi \colon P \lto M$ の微分による像 $T_u \pi (X) \in T_{\pi(u)} M$ は,上述の注意より勝手な \cinfty 関数 $f \in C^\infty (M)$ に対して
\begin{align}
    T_u \pi (X)f = X(f \circ \pi|_{G_{\pi(u)}})
\end{align}
と作用する.然るに \cinfty 写像 $f \circ \pi|_{G_{\pi(u)}}$ は常に値 $f(\pi(u))$ を返す定数写像なので,$T_u \pi (X)f = 0$ が言える\footnote{定数関数に接ベクトルを作用させると $0$ になる:Leibniz則より,定数関数 $1 \colon M \lto \mathbb{R},\;p \lmto 1$ に対して $v(1) = v(1\cdot 1) = v(1) + v(1) \IMP v(1) = 0$.$v$ の線型性から一般の定数関数に対しても $0$ になることが言える.}.i.e. $X \in \Ker (T_u \pi)$ であり,
\begin{align}
    T_u G_{\pi(u)} \subset \Ker (T_u \pi)
\end{align}
が言えた.
一方,$T_u \pi \colon T_u P \lto T_{\pi(u)} M$ は明らかに全射なので $\dim \Im (T_u \pi) = \dim T_{\pi(u)} M$ であり,故にファイバー束の局所自明性と階数-退化次元の定理から
\begin{align}
    \label{eq:dim}
    \dim \Ker (T_u \pi) = \dim T_u P - \dim T_{\pi(u)} M = \dim T_u G_{\pi(u)} = \dim G
\end{align}
が言える.結局
\begin{align}
    \label{eq:tangent-fiber}
    T_u G_{\pi(u)} = \Ker (T_u \pi)
\end{align}
だと分かった.さらに次の非常に重要な補題がある.この補題のために\hyperref[def:fundamental-vecf]{基本ベクトル場}を導入したと言っても過言ではない:
\begin{mylem}[label=lem:connection]{}
    $G \hookrightarrow P \xrightarrow{\pi} M$ を\hyperref[def.PFD]{主束}とする.
    命題\ref{prop.PFD_right}で与えたLie群 $G$ の全空間 $P$ への\hyperref[def:fundamental-vecf]{右作用 $\btl \colon P \times G \lto P$ の無限小生成子} $\btl^\# \colon \mathfrak{g} \lto \mathfrak{X}(P),\; X \lmto X^\#$ について,
    \begin{align}
        \forall u \in P,\quad \bm{\Ker (T_{u} \pi) = \bigl\{\, X^\#_{u} \in T_{u} P \bigm| X \in \mathfrak{g} \,\bigr\}}
    \end{align}
    が成り立つ.
\end{mylem}

\begin{proof}
    $\forall \textcolor{red}{u} \in P$ を1つ固定する.
    \begin{description}
        \item[\textbf{$\bm{\Ker (T_{\textcolor{red}{u}} \pi) \supset \bigl\{\, X^\#_{\textcolor{red}{u}} \in T_{\textcolor{red}{u}} P \bigm| X \in \mathfrak{g} \,\bigr\}}$}] 
        
        $\forall X \in \mathfrak{g}$ をとる.
        このとき\eqref{eq:fundamental-vecf}より $X^\#_{\textcolor{red}{u}} = T_{1_G}(R^{(\textcolor{red}{u})})(X)$ だが,$\pi \circ R^{(\textcolor{red}{u})}$ は定数写像なので
        \begin{align}
            T_{\textcolor{red}{u}}\pi (X^\#_{\textcolor{red}{u}}) &= T_{\textcolor{red}{u}}\pi \circ T_{1_G}(R^{(\textcolor{red}{u})})(X) \\
            &= T_{1_G}(\pi \circ R^{(\textcolor{red}{u})}) (X) \\
            &= 0
        \end{align}
        が分かる.i.e. $X^\#_{\textcolor{red}{u}} \in \Ker (T_{\textcolor{red}{u}} \pi)$ である.
    
        \item[\textbf{$\bm{\Ker (T_{\textcolor{red}{u}} \pi) \subset \bigl\{\, X^\#_{\textcolor{red}{u}} \in T_{\textcolor{red}{u}} P \bigm| X \in \mathfrak{g} \,\bigr\}}$}] 
        
        まず,$\mathbb{R}$-線型写像
        \begin{align}
            T_{1_G}\bigl(R^{(\textcolor{red}{u})} \bigr) \colon \mathfrak{g} \lto \Ker (T_{\textcolor{red}{u}} \pi)
        \end{align}
        がベクトル空間の同型写像であることを示す.
        系\ref{col:fundamental-vecf-basic}から $\Ker T_{1_G}\bigl(R^{(\textcolor{red}{u})} \bigr) = \mathrm{Lie}\bigl( \Stab (\textcolor{red}{u}) \bigr)$ だが,命題\ref{prop.PFD_right}より右作用 $\btl$ は\hyperref[def:Lie-action]{自由}なので $ \Stab (\textcolor{red}{u}) = \{1_G\}$ である.従って $\Ker T_{1_G}\bigl(R^{(\textcolor{red}{u})} \bigr) = \{0\}$ であり,$T_{1_G}\bigl(R^{(\textcolor{red}{u})} \bigr)$ は単射.
        % 上の議論から $\Im T_{1_G}\bigl(R^{(\textcolor{red}{u})} \bigr) \s\textcolor{red}{u}bset \Ker (T_{\textcolor{red}{u}} \pi)$ も分かる.このときベクトル空間の短完全列
        % \begin{align}
        %     0 \lto \Ker (T_{\textcolor{red}{u}} \pi) \lto T_{\textcolor{red}{u}} P \xrightarrow{T_{\textcolor{red}{u}}\pi} T_{\pi(\textcolor{red}{u})} M \lto 0
        % \end{align}
        \eqref{eq:dim}より $\dim \mathfrak{g} = \dim G = \dim \Ker (T_{\textcolor{red}{u}} \pi)$ なので $T_{1_G}\bigl(R^{(\textcolor{red}{u})} \bigr)$ はベクトル空間の同型写像である.
        
         以上より,$\forall v \in \Ker (T_{\textcolor{red}{u}} \pi)$ に対して $\bigl(T_{1_G} (R^{(\textcolor{red}{u})})\bigr)^{-1}(v) \in \mathfrak{g}$ であり,\eqref{eq:fundamental-vecf}から
        \begin{align}
            v = T_{1_G} (R^{(\textcolor{red}{u})}) \Bigl( \bigl(T_{1_G} (R^{(\textcolor{red}{u})})\bigr)^{-1}(v)\Bigr) = \Bigl(\bigl(T_{1_G} (R^{(\textcolor{red}{u})})\bigr)^{-1}(v)\Bigr)^\#_{\textcolor{red}{u}}
        \end{align}
        が言えた.
    \end{description}

\end{proof}

\hyperref[def:connection]{接続の定義}は幾何学的イメージがわかりやすいが,計算は絶望的である.
幸いにして主束の接続を与えることと,全空間上の\hyperref[def:connection]{接続形式}を与えることは同値なのでなんとかなる:

\begin{mytheo}[label=thm:connection-basic]{接続と接続形式の関係}
    $G \hookrightarrow P \xrightarrow{\pi} M$ を\hyperref[def.PFD]{主束}とする.
    \begin{enumerate}
        \item $\omega \in \Omega^1(P;\, \mathfrak{g})$ が\hyperref[def:connection]{接続形式}ならば,分布 
        \begin{align}
            \bigl\{\, \Ker \omega_u \subset T_u P\bigm| u \in P \,\bigr\} 
        \end{align}
        は $P$ 上の\hyperref[def:connection]{接続}である.
        \item (1) は $P$ 上の接続形式全体の集合から $P$ 上の接続全体の集合への1対1対応を与える.
    \end{enumerate}
\end{mytheo}

\begin{proof}
    \begin{enumerate}
        \item 
        $\forall u \in P$ を1つ固定する.命題\ref{prop.PFD_right}で与えたLie群 $G$ の全空間 $P$ への\hyperref[def:fundamental-vecf]{右作用 $\btl \colon P \times G \lto P$ の無限小生成子} $\btl^\# \colon \mathfrak{g} \lto \mathfrak{X}(P),\; X \lmto X^\#$ を考える.
        \begin{description}
            \item[\textbf{(C-1)}] 
            
             接続形式の定義から $\forall X \in \mathfrak{g}$ に対して $\omega (X^\#) = X$ が成り立つ.i.e. $\omega_{u} \colon T_{u}P \lto \mathfrak{g}$ は全射であり,
            $\mathbb{R}$-線型写像の系列\footnote{$i$ は包含準同型なので $\Ker \omega_{u} = \Im i$.$\omega_{u}$ が全射なので $\Im \omega_{u} = \mathfrak{g} = \Ker 0$.}
            % よって $T_{u} P = \Ker (T_{u}\pi) \oplus \Ker \omega_{u}$ である.
            \begin{align}
                0 \lto \Ker \omega_{u} \xrightarrow{i} T_{u} P \xrightarrow{\omega_{u}} \mathfrak{g} \lto 0
            \end{align}
            は短完全列になる.
            さらにこれは補題\ref{lem:connection}の証明で与えた線型写像 $T_{1_G} (R^{(u)}) \colon \mathfrak{g} \lto \Ker (T_{u} \pi) \subset T_{u}P$ によって分裂するので
            \begin{align}
                T_{u} \cong \mathfrak{g} \oplus \Ker \omega_{u} \cong \Ker (T_{u} \pi) \oplus \Ker \omega_{u}
            \end{align}
            がわかる.

            \item[\textbf{(C-2)}] 
            
            $\forall v \in \Ker \omega_{u}$ をとる.このとき\hyperref[def:connection]{\textbf{\textsf{(CF-2)}}}より $\forall g \in G$ に対して
            \begin{align}
                \omega_{u \btl g} \bigl( T_{u}(R_g)(v) \bigr) = ((R_g)^* \omega)_{u} (v) = \Adj (g^{-1})\bigl(\omega_{u}(v)\bigr) = 0
            \end{align}
            が従い,$T_{u}(R_g)(\Ker \omega_{u}) \subset \Ker \omega_{u \btl g}$ が言えた.
            両辺の次元が等しいので $T_u(R_g)(\Ker \omega_u) = \Ker \omega_{u \btl g}$ が言えた.
        \end{description}

        \item 
        
        \begin{description}
            \item[\textbf{(単射性)}] 
            
            \hyperref[def:connection]{接続形式} $\omega,\, \eta \in \Omega^1(P;\, \mathfrak{g})$ に対して $\bigl\{\, \Ker \omega_u \bigm| u \in P \,\bigr\} = \bigl\{\, \Ker \eta_u \bigm| u \in P \,\bigr\}$ が成り立つとする.
            このとき $\forall u \in P$ に対して $\Ker \omega_u = \Ker \eta_u$ が成り立つ.
            補題\ref{lem:connection}および (1) から $T_u P = \Ker (T_u \pi) \oplus \Ker \omega_u = \bigl\{\, X^\#_u \bigm| X \in \mathfrak{g} \,\bigr\} \oplus \Ker \omega_u$ の直和分解があり,$\forall v \in T_u P$ に対して $V \in \mathfrak{g},\; v^H \in \Ker \omega_u = \Ker \eta_u$ が一意的に存在して $v = V^\#_u + v^H$ と書ける.
            よって\textbf{\textsf{(CF-1)}}から
            \begin{align}
                \omega_u(v) = \omega_u(V^\#_u) = V = \eta_u(V^\#_u) = \eta_u(v)
            \end{align}
            が分かった.i.e. $\omega_u = \eta_u$ である.$u$ は任意だったので $\omega = \eta$ が言えた.

            \item[\textbf{(全射性)}] 
            
            主束 $G \hookrightarrow P \xrightarrow{\pi} M$ の\hyperref[def:connection]{接続} $\bigl\{\, H_u \subset T_u P \bigm| u \in P \,\bigr\}$ を与える.$\forall u \in P$ に対して
            直和分解 $T_u P = \Ker (T_u \pi) \oplus H_u$ の垂直,水平部分空間成分への射影をそれぞれ 
            \begin{align}
                i_1 \colon T_u P &\lto \Ker (T_u \pi),\; v^V + v^H \lmto v^V \\
                i_2 \colon T_u P &\lto H_u,\; v^V + v^H \lmto v^H
            \end{align}
            と書き,
            $\omega \in \Omega^1(P;\, \mathfrak{g})$ を,$\forall u \in P$ に対して補題\ref{lem:connection}の証明で与えた同型写像 $T_{1_G} (R^{(u)}) \colon \mathfrak{g} \lto \Ker (T_u \pi)$ を用いて
            \begin{align}
                \omega_u \coloneqq \bigl(T_{1_G} (R^{(u)})\bigr)^{-1} \circ i_1 \colon T_u P \xrightarrow{i_1} \Ker (T_u \pi) \xrightarrow{\bigl(T_{1_G} (R^{(u)})\bigr)^{-1}} \mathfrak{g}
            \end{align}
            と定義する.この $\omega$ が\hyperref[def:connection]{\textbf{\textsf{(CF-1)}}, \textbf{\textsf{(CF-2)}}}を充たすことを示す:
            \begin{description}
                \item[\textbf{(CF-1)}] 
                $\forall u \in P$ を1つ固定する.
                $\forall X \in \mathfrak{g}$ をとる.補題\ref{lem:connection}より $X_u^\# \in \Ker (T_u\pi)$ だから,\eqref{eq:fundamental-vecf}より
                \begin{align}
                    \omega_u (X^\#_u) &= \bigl(T_{1_G} (R^{(u)})\bigr)^{-1} \circ i_1(X^\#_u) \\
                    &= \bigl(T_{1_G} (R^{(u)})\bigr)^{-1} (X^\#_u) \\
                    &= \bigl(T_{1_G} (R^{(u)})\bigr)^{-1} \bigl( T_{1_G}(R^{(u)})(X) \bigr)  \\
                    &= X \label{eq:CF-1}
                \end{align}
                が言えた.
                \item[\textbf{(CF-2)}] 
                
                $\forall u \in P$ を1つ固定する.
                $\forall g \in G$ をとる.示すべきは $\forall v \in T_u P$ に対して
                \begin{align}
                    \bigl((R_g)^* \omega\bigr)_u (v) = \omega_{R_g(u)} \bigl( T_u (R_g)(v) \bigr) = \Adj (g^{-1})\bigl( \omega_u (v)\bigr)
                \end{align}
                が成り立つことである.
                実際 $\textcolor{red}{i_1(v)} \in \Ker(T_u \pi)$ に対しては,補題\ref{lem:connection}からある $V \in \mathfrak{g}$ が一意的に存在して $i_1(v) = V^\#_u$ と書けるので
                \begin{align}
                    \omega_{R_g(u)} \Bigl( T_u (R_g)\bigl(\textcolor{red}{i_1(v)}\bigr) \Bigr)
                    &= \omega_{u \btl g} \Bigl( T_u (R_g)\bigl(V^\#_u\bigr) \Bigr) &&\\
                    &= \omega_{u \btl g} \Bigl( T_u (R_g)\circ T_{1_G}(R^{(u)})(V)\Bigr) &&\because\quad\text{\eqref{eq:fundamental-vecf}}\\
                    &= \omega_{u \btl g} \Bigl( T_{1_G}(R_g \circ R^{(u)})(V)\Bigr) &&
                \end{align}
                となるが,$\forall x \in G$ に対して
                \begin{align}
                    R_g \circ R^{(u)}(x) 
                    &= u \btl x \btl g 
                    = u \btl (xg) 
                    = u \btl (gg^{-1}xg)  \\
                    &= (u \btl g)  \btl (g^{-1}xg) 
                    = R^{(u \btl g)} \circ  F_{g^{-1}}(x) 
                \end{align}
                が成り立つ\footnote{記号は\exref{def:Lie-adj}を参照}ことから
                \begin{align}
                    \omega_{u \btl g} \Bigl( T_{1_G}(R_g \circ R^{(u)})(V)\Bigr) 
                    &= \omega_{u \btl g} \bigl(T_{F_{g^{-1}}(1_G)} (R^{(u \btl g)}) \circ T_{1_G}(F_{g^{-1}}) (V) \bigr) &&\\
                    &= \omega_{u \btl g} \bigl(T_{1_G} (R^{(u \btl g)}) \circ \Adj (g^{-1})(V)\bigr) &&\because\quad \Adj\; \text{の\hyperref[def:Lie-adj]{定義}}\\
                    &= \omega_{u \btl g} \Bigl( \bigl( \Adj (g^{-1})(V) \bigr)^\#_{u \btl g} \Bigr) &&\because\quad \text{\eqref{eq:fundamental-vecf}}\\
                    &= \bigl( T_{1_G}(R^{(u\btl g)}) \bigr)^{-1} \circ T_{1_G} (R^{(u \btl g)}) \circ \Adj (g^{-1})(V) &&\because\quad \text{\eqref{eq:CF-1}}\\
                    &= \Adj (g^{-1})(V) &&\\
                    &= \Adj (g^{-1})\omega_u(V^\#_u) &&\because\quad \text{\eqref{eq:CF-1}}\\
                    &= \Adj (g^{-1})\omega_u\bigl(\textcolor{red}{i_1(v)}\bigr) &&
                \end{align}
                が言える.$\textcolor{blue}{i_2(v)} \in H_u$ に関しては,\hyperref[def:connection]{\textbf{\textsf{(C-2)}}}から $T_u (R_g)\bigl(\textcolor{blue}{i_2(v)}\bigr) \in H_{u \btl g}$ なので
                \begin{align}
                    \omega_{R_g(u)} \Bigl( T_u (R_g)\bigl(\textcolor{blue}{i_2(v)}\bigr) \Bigr) 
                    &= \bigl( T_{1_G}(R^{(u \btl g)}) \bigr)^{-1} \circ i_1 \Bigl(T_u (R_g)\bigl(\textcolor{blue}{i_2(v)}\bigr)\Bigr) \\
                    &= 0 \\
                    &= \Adj(g^{-1}) \Bigl( \omega_u \bigl(\textcolor{blue}{i_2(v)}\bigr) \Bigr) 
                \end{align}
                が言える.$v = \textcolor{red}{i_1(v)} + \textcolor{blue}{i_2(v)}$ なので証明が完了した.
                
                
            \end{description}
            
        \end{description}
        
    \end{enumerate}
\end{proof}

\subsection{同伴ベクトル束上の共変微分とゲージ場の定義}

$V$ を有限次元 $\mathbb{K}$-ベクトル空間,$\rho \colon G \lto \LGL(V)$ をLie群 $G$ の $\dim V < \infty$ 次元表現とする.

\begin{mydef}[label=def:tensorial-form,breakable]{tensorial form}
    \hyperref[def.PFD]{主束} $G \hookrightarrow P \xrightarrow{\pi} M$ を与える.
    $\forall g \in G$ に対して,命題\ref{prop.PFD_right}の右作用によって\hyperref[def:fundamental-vecf]{右作用移動}を $R_g \colon P \lto P,\; u \lmto u \btl g$ と定義する.
    全空間 $P$ 上の\hyperref[eq:E-valued]{$V$-値 $k$ 形式} $\omega \in \Omega^k(P;\, V)$ を与える.
    \begin{itemize}
        \item $\omega$ が\textbf{水平} (horizontal) であるとは,$\forall X \in \mathfrak{g}$ に対して
        \begin{align}
            i_{X^\#} (\omega) = 0
        \end{align}
        が成り立つことを言う\footnote{$i_{X^\#} \colon \Omega^k (P;\, V) \lto \Omega^{k-1}(P;\, V)$ は微分形式の\textbf{内部積} (interior product) である.}.
        \item $\omega$ が $\bm{\rho}$ \textbf{型の右同変}
        \footnote{
            equivalent の訳語には\textbf{同変}があてられることが多い.何かしらの群作用を考えると言う意味があるようだ:\url{http://pantodon.jp/index.rb?body=equivariant_topology}
        }
        (right equivalent of type $\rho$) であるとは,$\forall g \in G$ に対して
        \begin{align}
            (R_g)^* \omega = \rho(g^{-1})(\omega)
        \end{align}
        が成り立つことを言う.
        \item $\omega$ が $\bm{\rho}$ \textbf{型のtensorial $\bm{k}$-form} (tensorial form of type $\rho$) であるとは,
        $\omega$ が水平かつ$\rho$ 型の右同変であることを言う.
    \end{itemize}
    \tcblower
    $\rho$ 型のtensorial $k$-form全体がなす $\mathbb{K}$-ベクトル空間を $\bm{\Omega^k_\rho (P;\, V)}$ と書く.
\end{mydef}

\begin{marker}
    補題\ref{lem:fundamental-vecf}より,$\omega \in \Omega^k(P;\, V)$ が水平であることは任意の $k$-個の \cinfty ベクトル場 $X_1,\, \dots,\, X_k \in \mathfrak{X}(P)$ に対して
    \begin{align}
        \exists l\; \mathrm{s.t.}\; X_l\;\text{が\hyperref[def:connection]{垂直}} \IMP \omega (X_1,\, \dots,\, X_k) = 0
    \end{align}
    が成り立つことと同値である.
    また,$P$ 上の任意の$0$-形式は引数を持たないので,自動的に水平ということになる.
\end{marker}

命題\ref{prop:Borelconst}を思い出すと,主束 $G \hookrightarrow P \xrightarrow{\pi} M$ の開被覆,局所自明化,変換関数をそれぞれ $\bigl\{ U_\alpha \bigr\}_{\alpha \in \Lambda},\; \bigl\{ \varphi_\alpha \colon \pi^{-1}(U_\alpha) \lto U_\alpha \times G \bigr\}_{\alpha \in \Lambda},\; \bigl\{  t_{\alpha\beta} \colon M \lto G  \bigr\}_{\alpha,\, \beta \in \Lambda}$ とおき,
$\forall \alpha \in \Lambda$ に対して\hyperref[def.section]{局所切断} $s_\alpha \in \Gamma(P|_{U_\alpha})$ を
\begin{align}
    \label{eq:assoc-sec}
    s_\alpha \colon U_\alpha \lto \pi^{-1}(U_\alpha),\; x \lmto \varphi_\alpha^{-1}(x,\, 1_G)
\end{align}
と定めるとき,\hyperref[def:associated-vect]{同伴ベクトル束} $V \hookrightarrow P \times_\rho V \xrightarrow{q} M$ の局所自明化を
\begin{align}
    \label{eq:assoc-loctriv}
    \psi_\alpha \colon q^{-1}(U_\alpha) \lto U_\alpha \times V,\; s_\alpha(x) \times_\rho v \lmto (x,\, v)
\end{align}
として定義するのだった.

\begin{mylem}[label=lem:assoc-basic,breakable]{}
    \hyperref[def.PFD]{主束} $G \hookrightarrow P \xrightarrow{\pi} M$ の\hyperref[def:associated-vect]{同伴ベクトル束} $V \hookrightarrow P \times_\rho V \xrightarrow{q} M$ を考える.
    % \begin{enumerate}
    %     \item $\forall u \in P$ に対して,$\mathbb{K}$-線型写像
    %     \begin{align}
    %         f_u \colon V \lto (P \times_\rho V)_{\pi(u)},\; v \lmto u \times_\rho v
    %     \end{align}
    %     は\footnote{$(P \times_\rho V)_{\pi(u)} \coloneqq q^{-1}(\{\pi(u)\})$ は点 $\pi(u) \in M$ の\hyperref[def.fiber-1]{ファイバー}.}ベクトル空間の同型写像である.
    %     \item 引き戻し束
    %     \begin{center}
    %         \begin{tikzcd}[row sep=large, column sep=large]
    %             &\textcolor{red}{\pi^*(P \times_\rho V)} \ar[d, red, "\mathrm{proj}_1"']\ar[r, "\mathrm{proj}_2"] &P \times_\rho V \ar[d, "q"] \\
    %             &\textcolor{red}{P} \ar[r, "\pi"] &M
    %         \end{tikzcd}
    %     \end{center}
    %     は,\hyperref[def.bundlemap]{束写像}
    %     \begin{align}
    %         \iota \colon P \times V \lto \pi^*(P \times_\rho V),\; (u,\, v) \lmto (u,\, u \times_\rho v)
    %     \end{align}
    %     によって自明束 $V \hookrightarrow P \times V \xrightarrow{\mathrm{proj}_1} P$ と同型になる:
    %     \begin{center}
    %         \begin{center}
    %             \begin{tikzcd}[row sep=large, column sep=large]
    %                 &P \times V \ar[r, red, "\cong"] \ar[dr, "\mathrm{proj}_1"'] &\textcolor{red}{\pi^*(P \times_\rho V)} \ar[d, red, "\mathrm{proj}_1"']\ar[r, "\mathrm{proj}_2"] &P \times_\rho V \ar[d, "q"] \\
    %                 & &\textcolor{red}{P} \ar[r, "\pi"] &M
    %             \end{tikzcd}
    %         \end{center}
    %     \end{center}
    %     % \item 
    % \end{enumerate}
    このとき $\forall u \in P$ に対して,$\mathbb{K}$-線型写像
    \begin{align}
        f_u \colon V \lto (P \times_\rho V)_{\pi(u)},\; v \lmto u \times_\rho v
    \end{align}
    は\footnote{$(P \times_\rho V)_{\pi(u)} \coloneqq q^{-1}(\{\pi(u)\})$ は点 $\pi(u) \in M$ の\hyperref[def.fiber-1]{ファイバー}.}ベクトル空間の同型写像である.
\end{mylem}

\begin{proof}
    % \begin{enumerate}
    %     \item 
    % \end{enumerate}
    $v,\, w \in V$ について $u \times_\rho v = u \times_\rho w$ とする.このとき\hyperref[prop:Borelconst]{$\times_\rho$ の定義}から,ある $g \in G$ が存在して $(u,\, w) = (u \btl g,\, g^{-1} \btr v)$ とかける.\hyperref[prop.PFD_right]{右作用 $\btl$ は自由}なので $u = u \btl g \IMP g = 1_G$ が言える.従って $w = 1_G^{-1} \btr v = v$ が言えた.i.e. $f_u$ は単射である.

     $\forall x \times_\rho w \in (P \times_\rho V)_{\pi(u)}$ を1つとる.このとき $x \in \pi^{-1}(\{\pi(u)\})$ でもあるので,ある $g \in G$ が存在して $x = u \btl g$ と書ける.従って
    \begin{align}
        x \times_\rho w = (u \btl g) \times_\rho w = u \times_\rho g \btr w = f_u(g \btr w)
    \end{align}
    だとわかる.i.e. $f_u$ は全射である.
\end{proof}

\begin{myprop}[label=prop:assoc-basic]{}
    \hyperref[def.PFD]{主束} $G \hookrightarrow P \xrightarrow{\pi} M$ の\hyperref[def:associated-vect]{同伴ベクトル束} $V \hookrightarrow P \times_\rho V \xrightarrow{q} M$ を考える.
    $\forall \omega \in \Omega^k (M;\, P \times_\rho V)$ に対して
    % \footnote{つまり,$\omega$ は\hyperref[eq:E-valued]{$P \times_\rho V$ 値 $k$-形式}である.},
    $\omega^\sharp \in \Omega^k (P;\, V)$ を
    % $\forall u \in \textcolor{red}{P}$ に対して,$\omega_u^\sharp \in \Omega^k_\rho (P;\, V)$ を
    \begin{align}
        \omega^\sharp \colon u \lmto f_u^{-1} \circ (\pi^* \omega)_u
    \end{align}
    と定義する.
    \begin{enumerate}
        \item $\omega^\sharp \in \Omega^k_\rho (P;\, V)$ である.
        \item 写像
        \begin{align}
            \sharp \colon \Omega^k (M;\, P \times_\rho V) \lto \Omega^k_\rho(P;\, V),\; \omega \lmto \omega^\sharp    
        \end{align}
        はベクトル空間の同型写像である.
        \item $\forall s \in \Omega^k(M),\; \forall \eta \in \Omega^l(M;\, P \times_\rho V)$ に対して
        \begin{align}
            \sharp (s \wedge \eta) = (\pi^* s)\wedge \sharp \eta
        \end{align}
        が成り立つ.
        % \footnote{この意味で,(2) と併せると $\sharp$ は $C^\infty(M),\, C^\infty(P)$-加群の同型写像のように振る舞うと言える.}.
        \item 
        $\forall \omega \in \Omega^k (M;\, P \times_\rho V)$ に対して
        \begin{align}
            \mathrm{proj}_2 \circ \psi_\alpha \circ \omega = s_\alpha^* (\omega^\sharp) \in \Omega^k (U_\alpha;\, V)
        \end{align}
        が成り立つ.ただし $\psi_\alpha \colon q^{-1} (U_\alpha) \lto U_\alpha \times V$ は\eqref{eq:assoc-loctriv}で定義された局所自明化,$s_\alpha \colon M \lto \pi^{-1}(U_\alpha)$ は\eqref{eq:assoc-sec}で定義された\hyperref[def.section]{局所切断}とする.
    \end{enumerate}
\end{myprop}

\begin{proof}
    $\forall \omega \in \Omega^k(M;\, P \times_\rho V)$ を1つ固定する.
    \begin{enumerate}
        \item 
        $\omega^\sharp$ が\hyperref[def:tensorial-form]{$\rho$ 型のtensorial $k$-form}であることを示す.
        \begin{description}
            \item[\textbf{$\bm{\omega^\sharp}$ が水平であること}] 
            
            $\forall u \in P$ を1つ固定すると,$\forall X \in \mathfrak{g}$ および $\forall v_2,\, \dots,\, v_k \in T_u P$ に対して
            \begin{align}
                \bigl(i_{X^\#}(\omega^\sharp)\bigr)_u(v_2,\, \dots,\, v_k)
                &= f_u^{-1} \circ (\pi^* \omega)_u (X^\#_u,\, v_2,\, \dots,\, v_k)&& \\
                &= f_u^{-1} \Bigl( \omega_u \bigl(T_u \pi (X^\#_u),\, T_u \pi (v_2),\, \dots,\, T_u \pi(v_k)\bigr) \Bigr)&& \\
                &= f_u^{-1} \Bigl( \omega_u \bigl(0,\, T_u \pi (v_2),\, \dots,\, T_u \pi(v_k)\bigr) \Bigr) &&\because\quad \text{補題\ref{lem:connection}}
            \end{align}
            が成り立つが,$\omega_u$ は多重線型写像なので最右辺は $0$ になる.よって $i_{X^\#}(\omega^\sharp) = 0$ が言えた.
        
            \item[\textbf{$\bm{\omega^\sharp}$ が $\rho$-型の右同変であること}] 
            
            $\forall u \in P$ を1つ固定し,$\forall v_1,\, v_2,\, \dots,\, v_k \in T_u P$ をとる.
            \hyperref[def:tensorial-form]{右作用移動の定義}を思い出すと $\forall g \in G$ に対して $\pi \circ R_g = \pi$ が成り立つので
            \begin{align}
                &\bigl((R_g)^* \omega^\sharp\bigr)_u(v_1,\, \dots,\, v_k) \\
                &= (\omega^\sharp)_{R_g(u)} \bigl( T_u (R_g)(v_1),\, \dots,\, T_u (R_g)(v_k) \bigr) \\
                &= f^{-1}_{u \btl g} \Bigl( \omega_{\pi(u \btl g)} \bigl(  T_{u \btl g} \pi \circ T_u (R_g)(v_1),\, \dots,\, T_{u \btl g} \pi \circ T_u (R_g)(v_k) \bigr)  \Bigr) \\
                &= f^{-1}_{u \btl g} \Bigl( \omega_{\pi(u)} \bigl(  T_{u} (\pi \circ R_g)(v_1),\, \dots,\, T_{u} (\pi \circ R_g)(v_k) \bigr)  \Bigr) \\
                &= f^{-1}_{u \btl g} \Bigl( \omega_{\pi(u)} \bigl(  T_{u} \pi (v_1),\, \dots,\, T_{u} \pi (v_k) \bigr)  \Bigr) \\
                &= f^{-1}_{u \btl g} \bigl( (\pi^*\omega)_u ( v_1,\, \dots,\, v_k)  \bigr) \\
                &= f^{-1}_{u \btl g} \bigl( f_u \circ f^{-1}_u \circ (\pi^*\omega)_u ( v_1,\, \dots,\, v_k )  \bigr) \\
                &= f^{-1}_{u \btl g} \Bigl( f_u \bigl(\omega^\sharp_u (v_1,\, \dots,\, v_k)  \bigr) \Bigr) &&\because \quad \omega^\sharp\; \text{の定義}\\
                &= f^{-1}_{u \btl g} \bigl( u \times_\rho \omega^\sharp_u (v_1,\, \dots,\, v_k)  \bigr) &&\because \quad f_u\; \text{の\hyperref[lem:assoc-basic]{定義}} \\
                &= f^{-1}_{u \btl g} \Bigl( (u \btl g) \times_\rho \rho(g)^{-1} \bigl(\omega^\sharp_u (v_1,\, \dots,\, v_k) \bigr)  \Bigr) &&\because \quad \times_\rho\; \text{の\hyperref[prop:Borelconst]{定義}} \\
                &= \rho(g^{-1}) \bigl(\omega^\sharp_u (v_1,\, \dots,\, v_k) \bigr) &&\because\quad f_u\; \text{の\hyperref[lem:assoc-basic]{定義}}
            \end{align}
            i.e. $(R_g)^* \omega^\sharp = \rho(g^{-1})(\omega^\sharp)$ が言えた.
        \end{description}
        
        \item $\mathbb{K}$-線型写像 $\sharp \colon \Omega^k (M;\, P \times_\rho V) \lto \Omega^k_\rho(P;\, V)$ がベクトル空間の同型写像であることを示す.
        \begin{description}
            \item[\textbf{$\bm{\sharp}$ の単射性}] 
            
            $\omega,\, \eta \in \Omega^k(M;\, P \times_\rho V)$ に対して $\omega^\sharp = \eta^\sharp$ が成り立つとする.このとき $\forall u \in P$ を1つ固定すると,$f_u$ は全単射なので $(\pi^* \omega)_u = (\pi^* \eta)_u$ が言える.i.e. 
            $\forall v_1,\, \dots,\, v_k \in T_u P$ に対して
            \begin{align}
                0 = \bigl(\pi^* (\omega - \eta)\bigr)_u(v_1,\, \dots,\, v_k) = (\omega_{\pi(u)} - \eta_{\pi(u)})\bigl(T_u \pi (v_1),\, \dots,\, T_u \pi(v_k)\bigr)
            \end{align}
            が成り立つ.$T_u \pi \colon T_u P \lto T_{\pi(u)} M$ は全射なので $\omega - \eta = 0 \IFF \omega = \eta$ が言えた.
            \item[\textbf{$\bm{\sharp}$ の全射性}] 
            
            $\forall \tilde{\omega} \in \Omega^k_\rho (P;\, V)$ を1つ固定する.このとき
            $\omega \in \Omega^k (M;\, P \times_\rho V)$ を,$\forall x \in \textcolor{red}{M},\; \forall w_1,\, \dots,\, w_k \in T_x \textcolor{red}{M}$ に対して
            \begin{align}
                \omega_x(w_1,\, \dots,\, w_k) \coloneqq u \times_\rho \tilde{\omega}_u (v_1,\, \dots,\, v_k)
            \end{align}
            と定義する.ただし $u \in \pi^{-1}(\{x\}) \AND v_i \in (T_u \pi)^{-1}(\{w_i\})$ とする.
            \begin{description}
                \item[\textbf{$\bm{\omega_x}$ はwell-defined}] 

                まず $u \in \pi^{-1}(\{x\})$ を1つ固定する.このとき $v_i,\, v'_i \in (T_u \pi)^{-1}(\{w_i\})$ に対して $v'_i - v_i \in \Ker (T_u \pi)$ なので,
                $\tilde{\omega} \in \Omega_\rho^k (P;\, V)$ が\hyperref[def:tensorial-form]{水平}であることおよび $\tilde{\omega}_u$ の多重線型性から
                \begin{align}
                    \tilde{\omega}_u (v'_1,\, \dots,\, v'_k) 
                    &= \tilde{\omega}_u  \bigl(v_1 + (v'_1 - v_1),\, \dots,\, v_k + (v'_k - v_k)\bigr) \\
                    &= \tilde{\omega}_u  (v_1,\, \dots,\, v_k) + (\text{少なくとも1つの}\, i\, \text{について引数が}\; v'_i - v_i) \\
                    &= \tilde{\omega}_u (v_1,\, \dots,\, v_k)
                \end{align}
                が言える.i.e. $\tilde{\omega}_u$ は $v_i$ の取り方によらない.
                
                 次に,他の $u' \in \pi^{-1}(\{x\})$ をとる.このときある $h \in G$ が存在して $u' \btl h = u$ となる.
                \footnote{
                    $x \in U_\alpha \subset M$ に対する $P$ の局所自明化 $\varphi_\alpha$ について $g' \coloneqq \mathrm{proj}_2 \circ \varphi_\alpha (u'),\; g \coloneqq \mathrm{proj}_2 \circ \varphi_\alpha (u)$ とおけば,$h \coloneqq g'{}^{-1}g \in G$ に対して $u' \btl h = u$ となる.
                }
                $T_{u \btl h} \pi \circ T_{u} (R_{h})(v_i) = T_{u} \pi (v_i) = w_i$ かつ $\tilde{\omega}$ が $v_i$ の取り方によらないこと,および $\tilde{\omega}$ の\hyperref[def:tensorial-form]{右同変性}を使うと
                \begin{align}
                    \tilde{\omega}_{u'} (v_1,\, \dots,\, v_k)
                    &= \tilde{\omega}_{u \btl h} \bigl(T_{u}(R_h)(v_1),\, \dots,\, T_{u}(R_h)(v_k)\bigr) \\
                    &= \bigl((R_h)^* \tilde{\omega}\bigr)_u (v_1,\, \dots,\, v_k) \\
                    &= \rho(h^{-1})\bigl(\tilde{\omega}_u (v_1,\, \dots,\, v_k)\bigr)
                \end{align}
                だとわかるので,\hyperref[prop:Borelconst]{$\times_\rho$ の定義}から
                \begin{align}
                    u' \times_\rho \tilde{\omega}_{u'} (v_1,\, \dots,\, v_k)
                    &= u ' \times_\rho \rho(h^{-1})\bigl(\tilde{\omega}_u (v_1,\, \dots,\, v_k)\bigr) \\
                    &= (u ' \btl h) \times_\rho \rho(h) \circ \rho(h^{-1})\bigl(\tilde{\omega}_u (v_1,\, \dots,\, v_k)\bigr) \\
                    &= u \times_\rho \tilde{\omega}_u (v_1,\, \dots,\, v_k)
                \end{align}
                が言える.i.e. $\omega_x$ は $u \in \pi^{-1}(\{x\})$ の取り方にもよらない.
            \end{description}
            \hyperref[lem:assoc-basic]{$f_u$ の定義}および $\omega_x$ のwell-definednessから,
            \begin{align}
                &f_u \bigl(\tilde{\omega}_u(v_1,\, \dots,\, v_k)\bigr) = \omega_{\pi(u)}(w_1,\, \dots,\, w_k) = (\pi^* \omega)_u (v_1,\, \dots,\, v_k) \\
                \IFF &\tilde{\omega}_u = f_u^{-1} \circ (\pi^* \omega)_u
            \end{align}
            i.e. $\tilde{\omega} = \omega^\sharp$ が言えた.
        \end{description}
        \item $\forall u \in P$ および $\forall v_1,\, \dots,\, v_{k+l} \in T_u P$ に対して
        \begin{align}
            &\sharp (s \wedge \eta)_u (v_1,\, \dots,\, v_{k+l}) \\
            &= f^{-1}_u \circ \bigl(\pi^* (s \wedge \eta)\bigr)_u (v_1,\, \dots,\, v_{k+l}) \\
            &= f^{-1}_u \Bigl( (\pi^* s \wedge \pi^* \eta)_u(v_1,\, \dots,\, v_{k+l}) \Bigr) \\
            &= f^{-1}_u \left( \frac{1}{k! l!} \sum_{\sigma \in \mathfrak{S}_{k+l}} \sgn \sigma\, \underbrace{(\pi^* s)_u(v_{\sigma(1)},\, \dots,\, v_{\sigma(k)})}_{\in \mathbb{K}} \otimes (\pi^* \eta)_u(v_{\sigma(k+1)},\, \dots,\, v_{\sigma(k+l)}) \right) \\
            &= \frac{1}{k! l!} \sum_{\sigma \in \mathfrak{S}_{k+l}} \sgn \sigma\, (\pi^* s)_u(v_{\sigma(1)},\, \dots,\, v_{\sigma(k)}) f^{-1}_u \left(  (\pi^* \eta)_u(v_{\sigma(k+1)},\, \dots,\, v_{\sigma(k+l)}) \right) \\
            &= \bigl((\pi^* s) \wedge \sharp \eta\bigr)_u (v_1,\, \dots,\, v_{k+l})
        \end{align}
        が成り立つ.ただし最後から2番目の等号では $f_u^{-1}$ の $\mathbb{K}$-線型性を使った.
        \item 
        % $\forall \omega \in \Omega^k(M;\, P \times_\rho V)$ を1つ固定する.
        % このとき
        % \eqref{eq:assoc-sec}で定義した\hyperref[def.section]{局所切断} $s_\alpha \Gamma(P|_{U_\alpha})$ に対して,
        $\forall x \in U_\alpha,\; \forall w_1,\, \dots,\, w_k \in T_x M$ に対して
        \begin{align}
            &(\mathrm{proj}_2 \circ \psi_\alpha \circ \omega)_x (w_1,\, \dots,\, w_k) \\
            &= f_{s_\alpha(x)}^{-1}\bigl( \omega_x (w_1,\, \dots,\, w_k)\bigr) &&\because\quad f_{s_\alpha(x)}\; \text{の\hyperref[lem:assoc-basic]{定義}}\\
            &= f_{s_\alpha(x)}^{-1}\Bigl( \omega_x \bigl(T_x(\pi \circ s_\alpha)(w_1),\, \dots,\, T_x(\pi \circ s_\alpha)(w_k)\bigr) \Bigr) &&\because\quad \pi \circ s_\alpha = \mathrm{id}_{U_\alpha}\\
            &= f_{s_\alpha(x)}^{-1}\Bigl( \omega_x \bigl(T_{s_\alpha(x)} \pi \circ T_x(s_\alpha)(w_1),\, \dots,\, T_{s_\alpha(x)} \pi \circ T_x(s_\alpha)(w_k)\bigr) \Bigr) \\
            &= f_{s_\alpha(x)}^{-1} \circ (\pi^* \omega)_{s_\alpha(x)}\bigl(T_x(s_\alpha)(w_1),\, \dots,\, T_x(s_\alpha)(w_k)\bigr) \\
            &= (\omega^\sharp)_{s_\alpha(x)} \bigl(T_x(s_\alpha)(w_1),\, \dots,\, T_x(s_\alpha)(w_k)\bigr) &&\because\quad \omega^\sharp\;\text{の\hyperref[prop:assoc-basic]{定義}}\\
            &= \bigl(s_\alpha^* (\omega^\sharp)\bigr)_x (w_1,\, \dots,\, w_k)
        \end{align}
        が成り立つ.
    \end{enumerate}
    
\end{proof}


\begin{mydef}[label=def:connection-vect]{ベクトル束上の接続}
    $V \hookrightarrow  E \xrightarrow{\pi} M$ を\hyperref[def:vect]{ベクトル束}とする.
    \begin{itemize}
        \item ベクトル束 $E$ 上の\textbf{接続} (connection) とは,$\mathbb{K}$-線型写像
        \begin{align}
            \bm{\nabla^E} \colon \Gamma(E) \lto \Omega^1(M;\, E)
        \end{align}
        であって,$\forall f \in C^\infty (M) = \Omega^0 (M),\; \forall s \in \Gamma(E) = \Omega^0 (M;\, E)$ に対してLeibniz則
        \begin{align}
            \nabla^E (fs) = \dd{f} \otimes s + f \nabla^E s
        \end{align}
        を充たすもののこと.
        \item $X \in \Gamma (TM) = \mathfrak{X}(M)$ に対して定まる $\mathbb{K}$-線型写像
        \begin{align}
            \bm{\nabla^E_X} \colon \Gamma(E) &\lto \Gamma(E), \\
            s &\lmto (\nabla^E s)(X)
        \end{align}
        のことを\textbf{$\bm{X}$ に沿った共変微分} (covariant derivative along $X$) と呼ぶ.
        \item $\forall \omega \in \Omega^\bullet(M),\; \forall s \in \Gamma(E)$ に対して
        \begin{align}
            \dd^{\nabla^E}(\omega \otimes s) \coloneqq \dd{\omega} \otimes s + (-1)^{\deg \omega} \omega \wedge \nabla^E s
        \end{align}
        と定義することで定まる写像
        \begin{align}
            \bm{\mathrm{d}^{\nabla^E}} \colon \Omega^\bullet (M;\, E) \lto \Omega^{\bullet+1} (M;\, E)
        \end{align}
        のことを\textbf{共変外微分} (exterior covariant derivative) と呼ぶ.
    \end{itemize}
\end{mydef}

\exref{def:associated-vect}を思い出すと,主束に与えられた\hyperref[def:connection]{接続形式}が自然に同伴\hyperref[def:connection-vect]{ベクトル束上の接続}と結び付くような気がしてくる.
実際それは正しい~\cite[p.150, 命題6.3.3]{Imai2013diff}:

\begin{mytheo}[label=thm:connection-assoc,breakable]{同伴ベクトル束上の接続}
    \begin{itemize}
        \item \hyperref[def.PFD]{主束} $G \hookrightarrow P \xrightarrow{\pi} M$
        \item \hyperref[def:connection]{接続形式} $\omega \in \Omega^1(P;\, \mathfrak{g})$
        \item 有限次元 $\mathbb{K}$-ベクトル空間 $V$ とLie群 $G$ の $\dim V$ 次元表現 $\rho \colon G \lto \LGL (V)$ 
        \item \hyperref[def:associated-vect]{同伴ベクトル束} $V \hookrightarrow P \times_\rho V \xrightarrow{q} M$
    \end{itemize}
    を与える.$\rho_* \coloneqq T_{1_G} \rho \colon \mathfrak{g} \lto \mathfrak{gl}(V)$ を $\rho$ の\hyperref[def:diff-rep]{微分表現}とする.
    このとき,次が成り立つ:
    \begin{enumerate}
        \item \begin{align}
            \bigl(\dd + \rho_* (\omega) \wedge\bigr) \Omega^k_\rho(P;\, V) \subset \Omega^{k+1}_\rho (P;\, V)
        \end{align}
        \item $E \coloneqq P \times_\rho V$ とおく.命題\ref{prop:assoc-basic}で定めた同型写像 $\sharp \colon \Omega^k(M;\, E) \lto \Omega^k_\rho(P;\, V)$ を用いて定義した写像
        \begin{align}
            \nabla^E \coloneqq \sharp^{-1} \circ \bigl(\dd + \rho_*(\omega) \bigr) \circ \sharp\colon \Gamma(E) \lto \Omega^1(M;\, E)
        \end{align}
        は\hyperref[def:connection-vect]{ベクトル束 $E$ 上の接続}である.
        \item \eqref{eq:assoc-loctriv}によって定義された局所自明化 $\psi_\alpha \colon E|_{U_\alpha} \lto U_\alpha \times V$ に対して
        \begin{align}
            \mathrm{proj}_2 \circ \psi_\alpha \circ \nabla^E = \dd + \rho_* (s_\alpha^* \omega)
        \end{align}
        とかける.
        \item (2) の接続について\hyperref[def:connection-vect]{共変外微分} $\dd^{\nabla^E} \colon \Omega^k(M;\, E) \lto \Omega^{k+1}(M;\, E)$
        を考えると,以下の図式が可換になる:
        \begin{center}
            \begin{tikzcd}[row sep=large, column sep=large]
                &\Omega^k(M;\, E) \ar[r, "\dd^{\nabla^E}"]\ar[d, "\sharp"'] &\Omega^{k+1}(M;\, E) \ar[d, "\sharp"] \\
                &\Omega^k_\rho(P;\, V) \ar[r, "\dd + \rho_*(\omega) \wedge"'] &\Omega^{k+1}_\rho(P;\, V)
            \end{tikzcd}
        \end{center}
        
    \end{enumerate}
    
\end{mytheo}

\begin{marker}
    $\omega \in \Omega^1(P;\, \mathfrak{g}),\; \tilde{s} \in \Omega^k(P;\, V)$ に対して $\rho_*(\omega) \wedge \tilde{s} \in \Omega^{k+1}(P;\, V)$ の意味するところは,\textbf{通常の $\bm{\wedge}$ とは微妙に異なる}ことに注意.
    正確には $\forall X_1,\, \dots,\, X_{k+1} \in \mathfrak{X}(P)$ に対して
    \begin{align}
        \bigl(\rho_*(\omega) \wedge \tilde{s} \bigr)(X_1,\, \dots,\, X_{k+1})
        \coloneqq \frac{1}{1!k!} \sum_{\sigma \in \mathfrak{S}_{k+1}} \sgn \sigma \, \underbrace{\rho_*\bigl(\omega(X_{\sigma(1)})\bigr)}_{\in \mathfrak{gl}(V)} \bigl( \underbrace{\tilde{s}(X_{\sigma(2)},\, \dots,\, X_{\sigma(k+1)})}_{\in V} \bigr) 
    \end{align}
    として新しく定義したものである.
\end{marker}


\begin{proof}
    \begin{enumerate}
        \item $\forall \tilde{s} \in \Omega_\rho^k (P;\, V)$ を1つ固定する.
        \begin{description}
            \item[\textbf{$\bm{\bigl(\mathrm{d} + \rho_* (\omega)\wedge\bigr) \tilde{s}}$ が\hyperref[def:tensorial-form]{水平}}] 
            
            $\forall X \in \mathfrak{g}$ を1つとる.\hyperref[def:fundamental-vecf]{基本ベクトル場} $X^\# \in \Gamma(P)$ が\hyperref[thm:fundamental-flow]{生成するフロー}は $\theta \colon \mathbb{R} \times M \lto M,\; (t,\, u) \lmto R_{\exp(tX)}(u)$ だったので,
            \hyperref[def:Liedv]{Lie微分の定義}から
            \begin{align}
                \Liedv{X^\#} \tilde{s}
                &= \eval{\dv{}{t}}_{t=0} (R_{\exp(tX)})^* \tilde{s} \\
                &= \eval{\dv{}{t}}_{t=0} \rho\bigl(\exp(tX)^{-1}\bigr) (\tilde{s}) &&\because\quad \tilde{s}\; \text{の\hyperref[def:tensorial-form]{右同変性}}\\
                &= \left(\eval{\dv{}{t}}_{t=0} \rho\bigl(\exp(-tX)\bigr)\right) \tilde{s} \\
                &= -T_{1_G}\rho(X)(\tilde{s}) \\
                &= -\rho_*(X)(\tilde{s})
            \end{align}
            がわかる.従ってLie微分の公式 (Cartan magic formula) から 
            \begin{align}
                i_{X\#} \Bigl( \bigl(\mathrm{d} + \rho_* (\omega)\wedge\bigr) \tilde{s} \Bigr) 
                &= i_{X^\#} (\dd{\tilde{s}}) + i_{X^\#}\bigl(\rho_* (\omega)\bigr) \wedge \tilde{s} + (-1)^{\deg \omega} \rho_*(\omega) \wedge \cancel{i_{X^\#} (\tilde{s})} \\
                &= \Liedv{X^\#} \tilde{s} - \dd \bigl( \cancel{i_{X^\#}(\tilde{s})} \bigr)  + \rho_*\bigl(i_{X^\#}(\omega)\bigr) \wedge \tilde{s} \\
                &= -\rho_*(X)(\tilde{s}) + \rho_* (X) (\tilde{s}) \\
                &= 0
            \end{align}
            が言える.
            \item[\textbf{$\bm{\bigl(\mathrm{d} + \rho_* (\omega)\wedge\bigr) \tilde{s}}$ が\hyperref[def:tensorial-form]{右同変}}] 
            
            $\forall g \in G$ をとる.
            % $\forall X \in \mathfrak{X}(P)$ に対して
            % \begin{align}
            %     \rho_* \bigl(\Adj(g^{-1})(\omega)  \bigr)(X)
            %     &= \rho_* \Bigl( \Adj(g^{-1}) \bigl( \omega(X) \bigr)  \Bigr) \\
            %     &= T_{1_G} \rho \circ T_{1_G} (F_{g^{-1}}) \bigl( \omega(X) \bigr)   \\
            %     &= T_{1_G}(\rho \circ F_{g^{-1}})\bigl( \omega(X) \bigr) \\
            %     &= \rho(g^{-1}) \circ \rho_* \bigl( \omega(X) \bigr) \circ \rho(g)
            % \end{align}
            % が成り立つ,i.e. 
            $\rho_* \bigl( \Adj(g^{-1})(\omega) \bigr) = \rho(g^{-1}) \circ \rho_*(\omega) \circ \rho(g)$ なので
            \footnote{
                \exref{def:diff-rep}と大体同じ議論をすれば良い:
                $\forall X \in \mathfrak{g}$ をとる.このとき命題\ref{prop:exp}-(2) より \cinfty 曲線 $\gamma \colon t \lmto \exp(tX)$ は $X$ が生成する1パラメータ部分群なので,
                \begin{align}
                    \rho_* \bigl(\Adj(g^{-1})(X)  \bigr)
                    &= T_{1_G} \rho \Bigl(\Adj(g^{-1})\bigl(\dot{\gamma}(0)\bigr) \Bigr)   \\
                    &= T_{1_G} \rho \circ T_{1_G}(F_{g^{-1}}) \circ T_0 \gamma \left( \eval{\dv{}{t}}_{t=0} \right)  \\
                    &= T_{0}(\rho \circ F_{g^{-1}} \circ \gamma) \left( \eval{\dv{}{t}}_{t=0} \right) \\
                    &= \eval{\dv{}{t}}_{t=0} \rho\bigl(g^{-1} \exp(tX) g\bigr) \\
                    &= \rho(g^{-1}) \circ \eval{\dv{}{t}}_{t=0}\Bigl(\rho\bigl(\exp(tX)\bigr) \Bigr) \circ \rho(g) \\
                    &= \rho(g^{-1}) \circ \rho_*(X) \circ \rho(g)
                \end{align}
                ただし最後から2番目の等号では $\rho \colon G \lto \LGL (V)$ がLie群の準同型であることを使った.
                $\omega$ は $\mathfrak{g}$ に値を取るので示された.
            }
            % \footnote{
            %     $\rho_*(\omega)$ の意味は,$\forall X \in \mathfrak{X}(P)$ に対して得られる $\omega(X) \in \mathfrak{g}$ に $\rho_* \coloneqq T_{1_G}\rho \colon \mathfrak{g} \lto \Lgl (V)$ を作用させるという意味である.従って $\rho_* \bigl( \omega(X) \bigr) \in \Lgl (V)$ であるから,
            %     このことを踏まえると,
            %     \begin{align}
            %         T_{1_G}(\rho \circ F_{g^{-1}}) \bigl( \omega(X_1,\, \dots,\, X_k) \bigr)v =  \omega(X_1,\, \dots,\, X_k)(f \circ \rho \circ F_{g^{-1}})
            %     \end{align}
                
            % }
            \begin{align}
                (R_g)^* \Bigl(\bigl( \dd + \rho_*(\omega) \wedge\bigr) \tilde{s}\Bigr)
                &= (R_g)^*\dd\tilde{s} + (R_g)^* \bigl( \rho_*(\omega) \wedge \tilde{s} \bigr) \\
                &= \dd \bigl( (R_g)^* \tilde{s} \bigr) + \rho_* \bigl( (R_g)^* \omega \bigr) \wedge (R_g)^* \tilde{s} \\
                &= \dd \bigl( \rho(g^{-1})\tilde{s} \bigr) + \rho_* \bigl(\Adj(g^{-1})(\omega)  \bigr) \wedge \rho(g^{-1}) \tilde{s} \\
                &= \rho(g^{-1}) \Bigl( \bigl(\dd + \rho_* (\omega) \wedge\bigr) \tilde{s} \Bigr) 
            \end{align}
            が言える.
        \end{description}
        
        \item 
        $\forall f \in C^\infty(M),\; \forall s \in \Gamma(E) = \Omega^0(M;\, E)$ に対して
        \hyperref[def:connection-vect]{Leibniz則}が成り立つことを示す.
        % $\forall \omega \in \Omega^k(M),\, \forall \eta \in \Omega^l(M;\, E)$ に対して
        % \begin{align}
        %     \sharp (\omega \wedge \eta)_u (v_1,\, \dots,\, v_{k+l})
        %     &= f^{-1}_u \circ \bigl(\pi^* (\omega \wedge \eta)\bigr)_u (v_1,\, \dots,\, v_{k+l}) \\
        %     &= f^{-1}_u \Bigl( (\pi^* \omega \wedge \pi^* \eta)_u(v_1,\, \dots,\, v_{k+l}) \Bigr) \\
        %     &= f^{-1}_u \left( \frac{1}{k! l!} \sum_{\sigma \in \mathfrak{S}_{k+l}} \sgn \sigma\, \underbrace{(\pi^* \omega)_u(v_{\sigma(1)},\, \dots,\, v_{\sigma(k)})}_{\in \mathbb{K}} \otimes (\pi^* \eta)_u(v_{\sigma(k+1)},\, \dots,\, v_{\sigma(k+l)}) \right) \\
        %     &= \frac{1}{k! l!} \sum_{\sigma \in \mathfrak{S}_{k+l}} \sgn \sigma\, (\pi^* \omega)_u(v_{\sigma(1)},\, \dots,\, v_{\sigma(k)}) f^{-1}_u \left(  (\pi^* \eta)_u(v_{\sigma(k+1)},\, \dots,\, v_{\sigma(k+l)}) \right) \\
        %     &= \bigl((\pi^* \omega) \wedge \sharp \eta\bigr)_u
        % \end{align}
        % が成り立つ
        % \begin{align}
        %     \sharp (fs)_u 
        %     &= f_u^{-1} \circ \bigl(\pi^* (fs) \bigr)_u \\
        %     &= f_u^{-1} \circ \Bigl((f s) \circ \pi(u)\Bigr) \\
        %     &= f_u^{-1} \Bigl( f\bigl(\pi(u)\bigr)s\bigl(\pi(u) \bigr) \Bigr) \\
        %     &= \bigl(f \circ \pi(u)\bigr) f_u^{-1} \circ (s \circ \pi)(u) &&\because\quad f_u^{-1}\; \text{の}\; \mathbb{K}\text{-線型性} \\
        %     &= (\pi^* f)_u \,f_u^{-1} \circ (\pi^* s)_u \\
        %     &= (\pi^* f)_u \,s^\sharp_u
        % \end{align}
        % が成り立つので $\sharp (fs) = (\pi^* f)\, s^\sharp$ である.よって
        外微分と引き戻しが可換であることに注意すると
        \begin{align}
            \nabla^E (fs)
            &= \sharp^{-1} \circ \Bigl( \dd\bigl(\sharp (fs)\bigr) + \rho_* (\omega) \bigl( \sharp(fs) \bigr)   \Bigr) \\
            &= \sharp^{-1} \circ \Bigl( \dd\bigl((\pi^* f)\, \sharp s\bigr) + \rho_* (\omega) \bigl( (\pi^* f)\, \sharp s \bigr)   \Bigr) & &\because\quad\text{命題\ref{prop:assoc-basic}-(3)}\\
            &= \sharp^{-1} \circ \Bigl(\dd(\pi^* f) \otimes \sharp s + (\pi^* f)\, \dd (\sharp s) + (\pi^* f)\, \rho_* ( \omega ) (\sharp s)   \Bigr) \\
            &= \sharp^{-1} \circ \Bigl(\pi^* (\dd f) \otimes \sharp s + (\pi^* f)\, \bigl(\dd + \rho_* ( \omega )\bigr) (\sharp s)   \Bigr) \\
            &= \sharp^{-1} \circ \Bigl(\sharp(\dd f \otimes s) + (\pi^* f)\, \sharp \bigl(\nabla^E(s)\bigr) \Bigr) & &\because\quad\text{命題\ref{prop:assoc-basic}-(3)} \\
            &= \dd f \otimes s + f \nabla^E s & &\because\quad\text{命題\ref{prop:assoc-basic}-(3)}
        \end{align}
        が言えた.
        \item $\forall s \in \Gamma(E)$ に対して,命題\ref{prop:assoc-basic}-(4) より
        \begin{align}
            \mathrm{proj}_2 \circ \psi_\alpha \circ (\nabla^E s)
            &= s_\alpha^*\bigl(\sharp (\nabla^E s)\bigr) \\
            &= s_\alpha^* \Bigl( \bigl(\dd + \rho_* (\omega)\bigr)(\sharp s) \Bigr) \\
            &= s_\alpha^* \bigl(\dd(\sharp s)\bigr) + s_\alpha^* \bigl( \rho_*(\omega) (\sharp s)\bigr) \\
            &= \dd\bigl(s_\alpha^*(\sharp s)\bigr) + \rho_* \bigl( s_\alpha^*\omega \bigr)\bigl( s_\alpha^* (\sharp s)\bigr) \\
            &= \bigl( \dd + \rho_* (s_\alpha^* \omega) \bigr) s_\alpha^*(\sharp s) \\
            &= \bigl( \dd + \rho_* (s_\alpha^* \omega) \bigr) \bigl( \mathrm{proj}_2 \circ \psi_\alpha (s) \bigr)
        \end{align}
        が言える.
        \item $\forall s \in \Omega^k(M;\, E)$ をとる.
        局所自明化 $\psi_\alpha \colon E|_{U_\alpha} \lto U_\alpha \times V$ について,命題\ref{prop:assoc-basic}-(4) より
        \begin{align}
            &\mathrm{proj}_2 \circ \psi_\alpha \circ \bigl( \sharp^{-1} \circ (\dd + \rho_*(\omega) \wedge) \circ \sharp(s) \bigr) \\
            &= s_\alpha^* \bigl( (\dd + \rho_*(\omega) \wedge)(\sharp s)\bigr) \\
            &= s_\alpha^* \bigl( \dd(\sharp s) \bigr) + s_\alpha^* \bigl( \rho_*(\omega) \wedge \sharp s \bigr) \\
            &= \dd\bigl( s_\alpha^*(\sharp s) \bigr) + \rho_* \bigl( s_\alpha^*(\omega) \bigr) \wedge s_\alpha^*(\sharp s)  \\
            &= \bigl( \dd + \rho_*(s_\alpha^*\omega) \wedge \bigr)  s_\alpha^*(\sharp s) \\
            &= \bigl( \dd + \rho_*(s_\alpha^*\omega) \wedge \bigr)  \bigl(\mathrm{proj}_2 \circ \psi_\alpha(s) \bigr) 
        \end{align}
        が言える.
        % \footnote{この命題の注で述べたことから,$s_\alpha^* \bigl( \rho_*(\omega) \wedge \sharp s \bigr) =  \rho_* \bigl( s_\alpha^*(\omega) \bigr) \wedge s_\alpha^*(\sharp s)$ が言える.}.
        また,$\forall s \in \Omega^k(M),\; \forall t \in \Gamma(E)$ に対して
        \begin{align}
            &\mathrm{proj}_2 \circ \psi_\alpha \bigl( \dd{s} \otimes t + (-1)^{\deg s} s \wedge \nabla^E t \bigr) \\
            &= s_\alpha^* \bigl( \sharp(\dd{s} \otimes t)\bigr) + (-1)^{\deg s} s_\alpha^*\bigl( \sharp (s \wedge \nabla^E t)\bigr) \\
            &= s_\alpha^* \bigl( \pi^*(\dd{s}) \otimes \sharp t\bigr) + (-1)^{\deg s} s_\alpha^*\bigl( \pi^*(s) \wedge \sharp(\nabla^E t)\bigr) \qquad \because\quad \text{命題\ref{prop:assoc-basic}-(3)}\\
            &= \dd \bigl( s_\alpha^* (\pi^* s) \bigr) \otimes s_\alpha^* (\sharp t)  + (-1)^{\deg s}s_\alpha^* (\pi^* s) \wedge s_\alpha^* \bigl( \sharp (\nabla^E t) \bigr) \\
            &= \dd \bigl( s_\alpha^* (\pi^* s) \bigr) \otimes s_\alpha^* (\sharp t)  + (-1)^{\deg s} s_\alpha^* (\pi^* s) \wedge \dd \bigl( s_\alpha^*(\sharp t) \bigr) + (-1)^{\deg s} s_\alpha^* (\pi^* s)  \wedge \rho_* (s_\alpha^* \omega)(s_\alpha^*(\sharp t)) \\
            &= \dd \bigl( s_\alpha^* (\pi^* s) \otimes s_\alpha^* (\sharp t) \bigr) + \rho_* (s_\alpha^* \omega) \wedge (s_\alpha^* (\pi^* s) \otimes s_\alpha^*(\sharp t)) \\
            &= \bigl( \dd + \rho_*(s_\alpha^*\omega) \wedge \bigr)  s_\alpha^* \bigl((\pi^* s) \otimes (\sharp t)  \bigr) \\
            &= \bigl( \dd + \rho_*(s_\alpha^*\omega) \wedge \bigr)  s_\alpha^* \bigl(\sharp(s \otimes t)  \bigr) \qquad \because\quad \text{命題\ref{prop:assoc-basic}-(3)}\\
            &= \bigl( \dd + \rho_*(s_\alpha^*\omega) \wedge \bigr)  \bigl(\mathrm{proj}_2 \circ \psi_\alpha (s \otimes t) \bigr)
        \end{align}
        であるが,\hyperref[def:connection-vect]{共変外微分の定義}から最左辺は $\mathrm{proj}_2 \circ \psi_\alpha \circ \dd^{\nabla^E}(s \otimes t)$ と等しい.よって
        \begin{align}
            \dd^{\nabla^E} (s \otimes t) = \sharp^{-1} \circ \bigl( \dd + \rho_*(\omega) \wedge \bigr) \circ \sharp (s \otimes t)
        \end{align}
        が言えた.
        共変外微分の定義から,$d^{\nabla^E}$ が一般の $\Omega^k(M;\, E)$ の元に作用する場合についても示された.
    \end{enumerate}
\end{proof}

命題\ref{thm:connection-assoc}-(3) がまさに我々の良く知るゲージ場になっていることを確認しよう.
% そのためには,局所自明化の取り替えに伴って $\nabla^E|_{U_\alpha \cap U_\beta}$ が\eqref{eq:gauge-transform}の変換性を充たすことを示せば良い.

まずは状況設定である.\hyperref[def.PFD]{主束}
\begin{align}
    G \hookrightarrow P \xrightarrow{\pi} M
\end{align}
は
\begin{itemize}
    \item 開被覆 $\Familyset[\big]{U_\alpha}{\alpha \in \Lambda}$
    \item 局所自明化 $\Familyset[\big]{\varphi_\alpha \colon \pi^{-1}(U_\alpha) \lto U_\alpha \times G}{\alpha \in \Lambda}$ 
    \item 変換関数 $\Familyset[\big]{t_{\alpha\beta} \colon U_\alpha \cap U_\beta \lto G}{\alpha,\, \beta \in \Lambda}$
\end{itemize}
を持つとする.\underline{$P$ の}\hyperref[def.section]{局所切断} $\Familyset[\big]{s_\alpha \colon U_\alpha \lto \pi^{-1}(U_\alpha)}{\alpha \in \Lambda}$ を,\eqref{eq:assoc-loctriv}の通り $s_\alpha (x) \coloneqq \varphi_\alpha^{-1}(x,\, 1_G)$ と定義する.
このとき主束 $P$ の\hyperref[def:associated-vect]{同伴ベクトル束}
\begin{align}
    V \hookrightarrow P \times_\rho V \xrightarrow{q} M
\end{align}
はその構成から
\begin{itemize}
    \item 開被覆 $\Familyset[\big]{U_\alpha}{\alpha \in \Lambda}$
    \item 変換関数 $\Familyset[\big]{t_{\alpha\beta} \colon U_\alpha \cap U_\beta \lto G}{\alpha,\, \beta \in \Lambda}$
\end{itemize}
を持ち,局所自明化 $\Familyset[\big]{\psi_\alpha \colon q^{-1}(U_\alpha) \lto U_\alpha \times V}{\alpha \in \Lambda}$ は\eqref{eq:assoc-loctriv}の通りに $\psi_\alpha \bigl( s_\alpha(x) \times_\rho v \bigr) \coloneqq (x,\, v)$ と構成された.なお,命題\ref{prop:Borelconst}の証明の脚注で述べたようにこれは $\psi_\alpha(u \times_\rho v) \coloneqq \bigl( \pi(u),\, \mathrm{proj}_2 \circ \varphi_\alpha(u) \btr v \bigr)$ と定義することと同値である.
以下では便宜上 $E \coloneqq P \times_\rho V$ とおく.

さて,\underline{主束 $P$ 上の}\hyperref[def:connection]{接続形式} $\omega \in \Omega^1(P;\, \mathfrak{g})$ を任意に1つ与えよう.このとき定理\ref{thm:connection-assoc}-(2) により,\underline{同伴ベクトル束 $E$ 上の}\hyperref[def:connection-vect]{接続} $\nabla^E \colon \Gamma(E) \lto \Omega^1(M;\, E)$ が $\nabla^E \coloneqq \sharp^{-1}\circ \bigl(\dd + \rho_* (\omega)\bigr) \circ \sharp$ として誘導される.

\eqref{eq:field-locsym}が示すように,Lie群 $G$ で記述される内部対称性を持つ場 $\phi$ は\underline{$E$ の}切断 $\phi \in \Gamma(E)$ を局所自明化 $\psi_\alpha$ によって表示した $\phi_\alpha \coloneqq \mathrm{proj}_2 \circ \psi_\alpha \circ \phi \colon U_\alpha \lto V$ と同一視された.
ここで,$\phi \in \Gamma(E) = \Omega^0 (M;\, E)$ なので,ある\underline{$M$ 上の}\hyperref[def:vecf]{ベクトル場} $X \in \mathfrak{X}(M) = \Gamma(TM)$ に\hyperref[def:connection-vect]{沿った共変微分} $\nabla_X^E \colon \Gamma(E) \lto \Gamma(E),\; \phi \lmto \nabla^E\phi(X)$ を取ることができる.そして $\nabla^E_X \phi \in \Gamma (E)$ の局所自明化による表示 $(\nabla^E_X \phi)_\alpha \coloneqq \mathrm{proj}_2 \circ \psi_\alpha \circ \nabla^E_X \phi \colon U_\alpha \lto V$ もまた,$U_\alpha \cap U_\beta$ における $\psi_\alpha$ から $\psi_\beta$ への局所自明化の取り替えに伴って\eqref{eq:field-locsym}の変換を受ける:
\begin{align}
    \psi_\beta \circ \psi_\alpha^{-1} \colon (U_\alpha \cap U_\beta) \times V &\lto (U_\alpha \cap U_\beta) \times V, \\
    \Bigl( x,\, \bm{(\nabla_X^E\phi)_\alpha (x)} \Bigr) &\lmto \Bigl( x,\, \bm{(\nabla_X^E\phi)_\beta (x)}  \Bigr) = \Bigl( x,\, \bm{\rho\bigl(t_{\beta\alpha}(x)\bigr) \bigl( (\nabla_X^E\phi)_\alpha (x)\bigr)}  \Bigr) 
\end{align}
一方で,定理\ref{thm:connection-assoc}-(3)から
\begin{align}
    (\nabla_X^E\phi)_\alpha
    &= (\nabla^E\phi)_\alpha(X)
    = \bigl( \dd + \rho_* (s^*_\alpha \omega) \bigr)( \phi_\alpha ) (X)
    % &= \dd \phi_\alpha (X) + \rho_* \bigl( (s^*_\alpha \omega)(X) \bigr) (\phi_\alpha)
\end{align}
なので,$\forall x \in U_\alpha \cap U_\beta$ において
\begin{align}
    &(\nabla_X^E\phi)_\beta (x) \\
    &= \eval{\bigl( \dd + \rho_* (s^*_\beta \omega) \bigr) ( \phi_\beta )}_x (X_x) \\
    &= \eval{\rho \bigl( t_{\beta\alpha}(x) \bigr) \circ \bigl( \dd + \rho_* (s^*_\alpha \omega) \bigr) ( \phi_\alpha )}_x (X_x) \\
    &= \eval{\rho \bigl( t_{\beta\alpha}(x) \bigr) \circ \bigl( \dd + \rho_* (s^*_\alpha \omega) \bigr) \circ \rho\bigl(t_{\beta\alpha}(x)\bigr)^{-1} (\phi_\beta)}_x (X_x) \\
    &= \eval{\rho \bigl( t_{\beta\alpha}(x) \bigr) \circ \bigl( \dd + \rho_* (s^*_\alpha \omega) \bigr) \circ \rho\bigl(t_{\beta\alpha}(x)^{-1}\bigr) (\phi_\beta)}_x (X_x)
\end{align}
だとわかる.i.e.
\begin{align}
    \bigl(\dd + \rho_* (s^*_\beta \omega)\bigr)_x &= \rho \bigl( t_{\beta\alpha}(x) \bigr) \circ \bigl( \dd + \rho_* (s^*_\alpha \omega) \bigr)_x \circ \rho\bigl(t_{\beta\alpha}(x)^{-1}\bigr)
\end{align}
となって\eqref{eq:def-codv-transform}の変換則を再現する.
また,$V$ の基底を $e_1,\, \dots e_{\dim V}$ として $\phi_\beta(x) = \phi_\beta{}^i (x) e_i$ と展開し,$\rho (t_{\beta\alpha}(x)^{-1}) \in \LGL (V)$ をこの基底に関して $[\, \rho (t_{\beta\alpha}(x)^{-1})\, ]^{i}{}_j$ と行列表示ときに
\begin{align}
    \eval{\dd \circ \rho \bigl( t_{\beta\alpha}(x)^{-1} \bigr) (\phi_\beta)}_x 
    &= \dd\underbrace{\bigl([\, \rho (t_{\beta\alpha}(x)^{-1})\, ]^{i}{}_j \phi_\beta{}^i (x) \bigr)}_{\in \Omega^0(\textcolor{red}{M}) = C^\infty(\textcolor{red}{M})} e_i \\
    &= \partial_\mu \bigl([\, \rho (t_{\beta\alpha}(x)^{-1})\, ]^{i}{}_j \bigr) \phi_{\beta}{}^i (x) \dd{x}^\mu e_i + [\, \rho (t_{\beta\alpha}(x)^{-1})\, ]^{i}{}_j \partial_{\mu} \phi_\beta{}^j (x) \dd{x}^\mu e_i \\
    &\eqqcolon \eval{\Bigl(\dd \bigl(\rho \bigl( t_{\beta\alpha}(x)^{-1} \bigr) \bigr) + \rho \bigl( t_{\beta\alpha}(x)^{-1} \bigr) \circ \dd\Bigr) (\phi_\beta)}_x
\end{align}
が成り立つと言う意味で
\begin{align}
    \eval{\rho_* (s_\beta^* \omega)}_x = \rho \bigl( t_{\beta\alpha}(x) \bigr) \circ \dd \bigl(\rho \bigl( t_{\beta\alpha}(x)^{-1} \bigr)\bigr) + \rho \bigl( t_{\beta\alpha}(x) \bigr) \circ \rho_* (s_\alpha \omega) \circ \rho \bigl( t_{\beta\alpha}(x)^{-1} \bigr)
\end{align}
と書けてゲージ変換\eqref{eq:def-gauge-transform}を再現する.つまり,\textbf{ゲージ場}とは,接続形式 $\omega$ の局所切断による引き戻し $s_\alpha^* \omega \in \Omega^1(U_\alpha;\, \mathfrak{g})$ のことだったのである.

\subsection{局所接続形式}

先に進む前に,Maurer-Cartan形式についての注意をしておく.

\begin{mydef}[label=def:Maurer-Cartan]{Maurer-Cartan形式}
    Lie群 $G$ の\textbf{Maurer-Cartan形式}とは,
    \begin{align}
        \theta_g \coloneqq T_g(L_{g^{-1}}) \colon T_g G \lto \mathfrak{g}
    \end{align}
    によって定義される\underline{$G$ 上の} $\mathfrak{g}$ 値1-形式 $\theta \in \Omega^1(G;\, \mathfrak{g})$ のことを言う.
\end{mydef}
特に $G$ が $\LGL (V)$ の部分Lie群である場合,Maurer-Cartan形式 $\theta \colon TG \lto \mathfrak{g}$ はよく $\bm{g^{-1} \mathrm{d} g}$ と略記される.これの解釈は次の通りである:
まず $\forall g \in G$ と $G$ 上の恒等写像 $\mathrm{id} \colon G \lto G,\; g \lmto g$ を同一視する.このとき $\dd g \colon TG \lto TG$ は,$\forall g \in G$ に対して $T_g (\mathrm{id}) = \mathrm{id}_{T_g G} \colon T_g G \lto T_g G$ と解釈する.
すると,\exref{ex:gl}, \exref{ex:gl-adj}の議論から行列Lie群の場合 $T_g(L_{g^{-1}}) = L_{g^{-1}}$ と見做して良いので,$\forall v \in T_g G$ に対して
\begin{align}
    \theta_g(v) = T_g(L_{g^{-1}})(v) = L_{g^{-1}} \circ \mathrm{id}_{T_g G}(v) = g^{-1} \dd g(v)
\end{align}
と書けるのである.

さて,ここで任意の\cinfty 写像 $t \colon M \lto G$ をとる.このとき,$G$ が $\LGL (V)$ の部分Lie群ならば,Maurer-Cartan形式の $t$ による引き戻しが $t^* \theta = \bm{t^{-1} \mathrm{d} t} \in \Omega^1(M;\, \mathfrak{g})$ と表記できることを確認しよう.
$\forall g \in G$ を1つ固定し,$\forall v \in T_p M$ をとる.すると
\begin{align}
    (t^* \theta)_p (v)
    &= \theta_{t(p)} \bigl( T_p t (v) \bigr) \\
    &= T_{t(p)} (L_{t(p)^{-1}}) \circ T_p t (v) \\
    &= T_p (L_{t(p)^{-1}} \circ t)(v) \\
    &= T_p (L_{t(p)^{-1}} \circ t)\bigl(\dot{\gamma}(0)\bigr) \\
    &= T_p (L_{t(p)^{-1}} \circ t) \circ T_0 \gamma \left( \eval{\dv{}{t}}_{t=0} \right) \\
    &= \eval{\dv{}{t}}_{t=0} \bigl( L_{t(p)^{-1}} \circ t \circ \gamma(t) \bigr) \\
    &= \eval{\dv{}{t}}_{t=0} t(p)^{-1} t \bigl( \gamma(t) \bigr) \\
    &= t(p)^{-1} \eval{\dv{}{t}}_{t=0} t\bigl( \gamma(t) \bigr) \\
    &= t(p)^{-1} T_p t (\dot{\gamma}(0)) \\
    &= t(p)^{-1} \dd t|_p (v)
\end{align}
ただし $\gamma$ は $v$ が\hyperref[thm:fundamental-flow]{生成する積分曲線}である.


~\cite[第10章-1]{Nakahara2018topo2}に倣って,先ほど導入したゲージ場をより扱いやすい形にしよう.

\begin{mytheo}[label=thm:local-connection]{貼り合わせによる接続形式の構成}
    \hyperref[def.PFD]{主束} $G \hookrightarrow P \xrightarrow{\pi} M$ と,その開被覆 $\Familyset[\big]{U_\alpha}{\alpha \in \Lambda}$,局所自明化 $\Familyset[\big]{\varphi_\alpha \colon \pi^{-1}(U_\alpha) \lto U_\alpha \times G}{\alpha \in \Lambda}$,変換関数 $\Familyset[\big]{t_{\alpha\beta} \colon U_\alpha \cap U_\beta \lto G}{\alpha,\, \beta \in \Lambda}$ を与える.
    \hyperref[def.section]{局所切断}の族 $\Familyset[\big]{s_\alpha \colon U_\alpha \lto \pi^{-1}(U_\alpha)}{\alpha \in \Lambda}$ を\eqref{eq:assoc-sec}で定義する. 
    $\theta \in \Omega^1(G;\, \mathfrak{g})$ を\hyperref[def:Maurer-Cartan]{Maurer-Cartan形式}とする.
    \begin{enumerate}
        \item $\forall \alpha,\, \beta \in \Lambda$ および $\forall x \in U_\alpha \cap U_\beta$ において,
        $\forall v \in T_x \textcolor{red}{M}$ は\footnote{~\cite[p.36, 補題10.1]{Nakahara2018topo2}には誤植がある}
        \begin{align}
            T_x s_\beta (v) = T_{s_\alpha(x)}(R_{t_{\beta\alpha}(x)}) \circ T_x s_\alpha (v) + \eval{\bigl((t_{\beta\alpha}^*\theta)_x(v)\bigr)^\#}_{s_\beta(x)}
        \end{align}
        を充たす.
        \item  $\mathfrak{g}$ 値1-形式の族 $\Familyset[\big]{A_\alpha \in \Omega^1(\textcolor{red}{U_\alpha};\, \mathfrak{g})}{\alpha \in \Lambda}$ が $\forall \alpha,\, \beta \in \Lambda$ について
        \begin{align}
            \label{eq:compatible}
            A_\beta|_x = \Adj \bigl( t_{\beta\alpha}(x)^{-1} \bigr) \bigl( A_\alpha|_x \bigr) + (t_{\beta\alpha}^* \theta)_x, \quad \forall x \in U_\alpha \cap U_\beta
        \end{align}
        を充たすならば,$\forall \alpha \in \Lambda$ に対して $A_\alpha = s_\alpha^* \omega$ を充たす\hyperref[def:connection]{接続形式} $\omega \in \Omega^1(P;\, \mathfrak{g})$ が存在する.
    \end{enumerate}
\end{mytheo}

\begin{proof}
    \begin{enumerate}
        \item $\forall \alpha,\, \beta \in \Lambda$ を固定し,$\forall x \in U_\alpha \cap U_\beta$ を取る.
        $\forall v \in T_x M$ を1つ固定する.
        $\gamma \colon J \lto U_\alpha \cap U_\beta$ を,初期条件 $\gamma(0) = x$ を充たす,$v$ が\hyperref[thm:fundamental-flow]{生成する極大積分曲線}\footnote{従って開区間 $J \subset \mathbb{R}$ は $0$ を含む.}とする.
        \hyperref[def.fiber-1]{変換関数の定義}および\hyperref[prop.PFD_right]{ $P$ への右作用 $\btl$ の定義}から $s_\beta (x) = \varphi_\beta^{-1}(x,\, 1_G) = \varphi_\alpha^{-1}\bigl(x,\, t_{\beta\alpha}(x)\bigr) = s_\alpha(x) \btl t_{\beta\alpha}(x)$ が成り立つ.i.e. \cinfty 写像 $\Delta \colon M \lto M \times M,\; x \lmto (x,\, x)$ について $s_\beta = \btl \circ (s_\alpha \times t_{\beta\alpha}) \circ \Delta$ と書けるので,
        \begin{align}
            T_x s_\beta (v)
            &= T_x s_\beta \bigl( \dot{\gamma}(0) \bigr) \\
            &= T_x s_\beta \circ T_0 (\gamma) \left( \eval{\dv{}{t}}_{t=0} \right) \\
            &= T_0 (s_\beta \circ \gamma) \left( \eval{\dv{}{t}}_{t=0} \right) \\
            &= T_0 \bigl(\btl \circ (s_\alpha \times t_{\beta\alpha}) \circ \Delta \circ \gamma\bigr) \left( \eval{\dv{}{t}}_{t=0} \right) \\
            &= T_{\bigl(s_\alpha(\gamma(0)),\, t_{\beta\alpha}(\gamma(0))\bigr)} \btl \circ T_0 \bigl((s_\alpha \times t_{\beta\alpha}) \circ \Delta \circ \gamma\bigr) \left( \eval{\dv{}{t}}_{t=0} \right) \\
            &= T_{\bigl(s_\alpha(x),\, t_{\beta\alpha}(x)\bigr)} \btl \circ T_0 \bigl((s_\alpha \times t_{\beta\alpha}) \circ \Delta \circ \gamma\bigr) \left( \eval{\dv{}{t}}_{t=0} \right)
            % &= \eval{\dv{}{t}}_{t=0} s_\beta \bigl( \gamma(t) \bigr) \\
            % &= \eval{\dv{}{t}}_{t=0} s_\alpha \bigl( \gamma(t) \bigr) \btl t_{\beta\alpha} \bigl( \gamma(t) \bigr) \\
            % &= \eval{\dv{}{t}}_{t=0} R_{t_{\beta\alpha}(\gamma(t))} \circ s_\alpha \bigl(\gamma(t) \bigr) 
        \end{align}
        ここで $T_0\bigl((s_\alpha \times t_{\beta\alpha}) \circ \Delta \circ \gamma \bigr) \colon T_0 J \lto T_{\bigl( s_\alpha(x),\, t_{\beta\alpha}(x) \bigr)} (P \times G)$ であるが,ベクトル空間の同型
        \begin{align}
            &\iota \colon T_{(u,\, g)}(P \times G) \lto T_u M \oplus T_g G,\; v \lmto \bigl( T_{(u,\, g)}(\mathrm{proj}_1) (v),\, T_{(u,\, g)}(\mathrm{proj}_2)(v) \bigr) \\
            &\iota^{-1} \colon T_u P \oplus T_g G \lto T_{(u,\, g)}(M \times G),\; (v,\, w) \lmto T_u (\mathrm{inj}_1)(v) + T_g (\mathrm{inj})(w)
        \end{align}
        を使う\footnote{$\mathrm{inj}_1 \colon P \lto P \times G,\; x \lmto (x,\, g),\quad \mathrm{inj}_2 \colon G \lto P \times G,\; g \lmto (u,\, g)$}と
        \begin{align}
            &T_0 \bigl((s_\alpha \times t_{\beta\alpha}) \circ \Delta \circ \gamma\bigr) \left( \eval{\dv{}{t}}_{t=0} \right) \\
            &= \iota^{-1} \circ \iota \circ T_0 \bigl((s_\alpha \times t_{\beta\alpha}) \circ \Delta \circ \gamma\bigr) \left( \eval{\dv{}{t}}_{t=0} \right) \\
            &= \iota^{-1} \left( T_{0}(\mathrm{proj}_1 \circ (s_\alpha \times t_{\beta\alpha}) \circ \Delta \circ \gamma) \left( \eval{\dv{}{t}}_{t=0} \right) ,\, T_{0}(\mathrm{proj}_2 \circ (s_\alpha \times t_{\beta\alpha}) \circ \Delta \circ \gamma)\left( \eval{\dv{}{t}}_{t=0} \right)  \right)\\
            &= T_{s_\alpha(x)} (\mathrm{inj}_1) \circ T_{0}(s_\alpha \circ \gamma) \left( \eval{\dv{}{t}}_{t=0} \right) + T_{t_{\beta\alpha}(x)} (\mathrm{inj}_2) \circ T_{0}(t_{\beta\alpha} \circ \gamma)\left( \eval{\dv{}{t}}_{t=0} \right) \\
            &= T_{0} (\mathrm{inj}_1 \circ s_\alpha \circ \gamma) \left( \eval{\dv{}{t}}_{t=0} \right) + T_{0} (\mathrm{inj}_2 \circ t_{\beta\alpha} \circ \gamma)\left( \eval{\dv{}{t}}_{t=0} \right)
        \end{align}
        と計算できるので,
        \begin{align}
            &T_{\bigl(s_\alpha(x),\, t_{\beta\alpha}(x)\bigr)} \btl \circ T_0 \bigl((s_\alpha \times t_{\beta\alpha}) \circ \Delta \circ \gamma\bigr) \left( \eval{\dv{}{t}}_{t=0} \right) \\
            &= T_{0} (\btl \circ \mathrm{inj}_1 \circ s_\alpha \circ \gamma) \left( \eval{\dv{}{t}}_{t=0} \right) 
            + T_{0} (\btl \circ \mathrm{inj}_2 \circ t_{\beta\alpha} \circ \gamma)\left( \eval{\dv{}{t}}_{t=0} \right) \\
            &= \eval{\dv{}{t}}_{t=0} \Bigl( s_\alpha \bigl( \gamma(t) \bigr) \btl t_{\beta\alpha}(x) \Bigr)
            + \eval{\dv{}{t}}_{t=0} \Bigl( s_\alpha(x) \btl t_{\beta\alpha} \bigl( \gamma(t) \bigr) \Bigr)\\
            &= \eval{\dv{}{t}}_{t=0} \bigl( R_{t_{\beta\alpha}(x)} \circ s_\alpha \circ \gamma(t) \bigr)
            + \eval{\dv{}{t}}_{t=0} \bigl( R^{(s_\alpha(x))} \circ t_{\beta\alpha} \circ \gamma(t) \bigr) \\
            &= T_0 (R_{t_{\beta\alpha}(x)} \circ s_\alpha \circ \gamma) \left( \eval{\dv{}{t}}_{t=0} \right) 
            + T_0 (R^{(s_\alpha(x))} \circ t_{\beta\alpha} \circ \gamma) \left( \eval{\dv{}{t}}_{t=0} \right) \\
            &= T_{s_\alpha(x)} (R_{t_{\beta\alpha(x)}}) \circ T_{x} s_\alpha \circ T_0 \gamma \left( \eval{\dv{}{t}}_{t=0} \right) 
            + T_{t_{\beta\alpha}(x)} (R^{(s_\alpha(x))}) \circ T_x t_{\beta\alpha} \circ T_0 \gamma \left( \eval{\dv{}{t}}_{t=0} \right) \\
            &= T_{s_\alpha(x)} (R_{t_{\beta\alpha(x)}}) \circ T_{x} s_\alpha (v)
            + T_{x} (R^{(s_\alpha(x))} \circ t_{\beta\alpha}) (v)
        \end{align}
        を得る.一方\eqref{eq:fundamental-vecf}から
        \begin{align}
            \bigl((t_{\beta\alpha}^* \theta)_x (v) \bigr)^\#|_{s_\beta(x)}
            &= \bigl( \theta_{t_{\beta\alpha}(x)}(T_{x} t_{\beta\alpha} v) \bigr)^\#|_{s_\alpha(x) \btl t_{\beta\alpha}(x)} \\
            &= T_{1_G} ( R^{(s_\alpha(x) \btl t_{\beta\alpha}(x) )} ) \circ T_{t_{\beta\alpha}(x)}(L_{t_{\beta\alpha}(x)^{-1}}) \circ T_{x} t_{\beta\alpha} (v) \\
            &= T_{x} ( R^{(s_\alpha(x))} \circ L_{t_{\beta\alpha}(x)} \circ L_{t_{\beta\alpha}(x)^{-1}} \circ t_{\beta\alpha}) (v) \\
            &= T_{x} ( R^{(s_\alpha(x))} \circ t_{\beta\alpha}) (v)
        \end{align}
        と計算できる.
        % $\forall g \in G$ に対して $R^{(s_\alpha(x))}(g) = s_\alpha(x) \btl g = s_\alpha(x) \btl L_g (1_G)$
        \item 
        条件\eqref{eq:compatible}を充たす$\mathfrak{g}$ 値1-形式の族 $\Familyset[\big]{A_\alpha \in \Omega^1(U_\alpha;\, \mathfrak{g})}{\alpha \in \Lambda}$ を与える.
        $\forall \alpha \in \Lambda$ に対して
        \begin{align}
            g_\alpha \coloneqq \mathrm{proj}_2 \circ \varphi_\alpha \colon \pi^{-1}(U_\alpha) \lto G
        \end{align}
        とおく.このとき $\forall u \in \pi^{-1}(U_\alpha)$ に対して $\varphi_\alpha(u) = \bigl( \pi(u),\, g_\alpha(u)\bigr) = s_\alpha \bigl(\pi(u)\bigr) \btl g_\alpha(u)$ が成り立つ.
        
         $\forall \alpha \in \Lambda$ に対して,$\mathfrak{g}$ 値1-形式 $\omega_\alpha \in \Omega^1(P|_{U_\alpha};\, \mathfrak{g})$ を,$\forall u \in \pi^{-1}(U_\alpha)$ において
        \begin{align}
            \omega_\alpha|_u \coloneqq \Adj \bigl( g_\alpha(u)^{-1} \bigr)\bigl((\pi^* A_\alpha)_u\bigr) + (g_\alpha^*\theta)_u
        \end{align}
        と定義する.
        $\forall x \in \textcolor{red}{U_\alpha}$ および $\forall v \in T_{x} M$ に対して
        \begin{align}
            (s_\alpha^* \omega_\alpha)_{x}(v)
            &= \omega_\alpha|_{s_\alpha(x)} \bigl( T_{x} s_\alpha (v) \bigr) \\
            &= \Adj \bigl( g_\alpha(s_\alpha(x))^{-1} \bigr)\bigl((\pi^* A_\alpha)_{s_\alpha(x)}(T_x s_\alpha(v))\bigr) + (g_\alpha^*\theta)_{s_\alpha(x)}\bigl( T_{x} s_\alpha (v) \bigr) \\
            &= \Adj \bigl( 1_G \bigr)\Bigl(A_\alpha|_{x}\bigl(T_{s_\alpha(x)} \pi \circ T_x s_\alpha(v)\bigr)\Bigr) + \theta_{g_\alpha(s_\alpha(x))}\bigl( T_{s_\alpha(x)} g_\alpha \circ T_{x} s_\alpha (v) \bigr) \\
            &= A_\alpha|_{x}\bigl(T_x(\pi \circ s_\alpha)(v)\bigr) + \theta_{1_G}\bigl( \cancel{T_{x} (g_\alpha \circ s_\alpha)} (v) \bigr) \\
            &= A_\alpha|_{x}(v)
        \end{align}
        が成り立つので $s_\alpha^* \omega_\alpha = A_\alpha$ である.ただし,最後の等号で $\pi \circ s_\alpha = \mathrm{id}_{U_\alpha}$ であることと $g_\alpha \circ s_\alpha$ が常に $1_G$ を返す定数写像であることを使った.
        \begin{description}
            \item[\textbf{$\bm{\omega_\alpha}$ が $\bm{U_\alpha}$ 上で\hyperref[def:connection]{接続形式の公理}を充たすこと}] 
            
            \begin{description}
                \item[\textbf{(CF-1)}] 

                $\forall X \in \mathfrak{g}$ を1つとる.補題\ref{lem:connection}から,$\forall u \in \pi^{-1}(U_\alpha)$ に対して $T_u \pi (X^\#_u) = 0$ である.
                \hyperref[prop.PFD_right]{右作用 $\btl$ の定義}から $\forall g \in G$ に対して $L_{g_\alpha(u)^{-1}} \circ g_\alpha \circ R^{(u)}(g) = g_\alpha(u)^{-1} g_\alpha (u) g = g$ が成り立つ,i.e. $L_{g_\alpha(u)^{-1}} \circ g_\alpha \circ R^{(u)} = \mathrm{id}_G$ であることから
                \begin{align}
                    \omega_\alpha (X^\#)
                    &= \Adj \bigl( g_\alpha(u)^{-1} \bigr) \Bigl( A_\alpha|_{\pi(u)}\bigl( \cancel{T_u \pi (X^\#_u)} \bigr)  \Bigr) + \theta_{g_\alpha(u)} \bigl( T_u g_\alpha (X^\#_u) \bigr) \\
                    &= T_{g_\alpha(u)} (L_{g_\alpha(u)^{-1}}) \circ T_u g_\alpha \circ T_{1_G} (R^{(u)})(X) &&\because\quad \text{\eqref{eq:fundamental-vecf}} \\
                    &= T_{1_G} (L_{g_\alpha(u)^{-1}} \circ g_\alpha \circ R^{(u)})(X) \\
                    &= X
                \end{align}
                が言えた.
                \item[\textbf{(CF-2)}] 

                $\forall u \in P$ を1つ固定する.$\forall v \in T_u P,\; \forall g \in G$ に対して
                \begin{align}
                    (R_g^* \omega_\alpha)_u(v)
                    &= \omega_\alpha|_{R_g(u)} \bigl( T_u (R_g) (v)\bigr) \\
                    &= \Adj \bigl( g_\alpha (u \btl g)^{-1} \bigr) \Bigl( A_\alpha|_{\pi(u \btl g)} \bigl(T_{u \btl g} \pi \circ T_u (R_g) (v)\bigr) \Bigr) + \theta_{g_\alpha(u \btl g)} \bigl( T_{u \btl g} g_\alpha \circ T_u (R_g)(v) \bigr)  \\
                    &= \Adj \bigl( g^{-1}g_\alpha (u)^{-1} \bigr) \Bigl( A_\alpha|_{\pi(u)} \bigl(T_{u} (\pi \circ R_g) (v)\bigr) \Bigr) + \theta_{g_\alpha(u)g} \bigl( T_{u \btl g} (g_\alpha \circ R_g)(v) \bigr)  \\
                    &= \Adj \bigl( g^{-1} \bigr) \circ \Adj \bigl(g_\alpha (u)^{-1}\bigr) \Bigl( A_\alpha|_{\pi(u)} \bigl(T_{u} \pi (v)\bigr) \Bigr) + T_{u} (L_{g^{-1}g_\alpha(u)^{-1}} \circ g_\alpha \circ R_g)(v) \\
                    &= \Adj \bigl( g^{-1} \bigr) \circ \Adj \bigl(g_\alpha (u)^{-1}\bigr) \bigl( (\pi^* A_\alpha)_u (v) \bigr) + T_{u} (L_{g^{-1}g_\alpha(u)^{-1}} \circ g_\alpha \circ R_g)(v)
                \end{align}
                一方,$\mathcal{R}_g \colon G \lto G,\; h \lmto hg$ とおくと
                \begin{align}
                    \Adj(g^{-1}) \bigl( (g_\alpha^* \theta)_u (v) \bigr)
                    &= T_{1_G} (F_{g^{-1}}) \circ \theta_{g_\alpha(u)}\bigl(T_u g_\alpha(v)\bigr) \\
                    &= T_{1_G} (L_{g^{-1}} \circ \mathcal{R}_g) \circ T_{g_\alpha(u)} (L_{g_\alpha(u)^{-1}}) \circ T_u g_\alpha(v) \\
                    &= T_u (L_{g^{-1}} \circ \mathcal{R}_g \circ L_{g_\alpha (u)^{-1}} \circ g_\alpha)(v) \\
                    &= T_u (L_{g^{-1}} \circ L_{g_\alpha (u)^{-1}} \circ g_\alpha \circ R_g)(v) \\
                    &= T_u (L_{g^{-1}g_\alpha (u)^{-1}} \circ g_\alpha \circ R_g)(v)
                \end{align}
                であるから,
                \begin{align}
                    R_g^* \omega_\alpha = \Adj \bigl( g^{-1} \bigr)\bigl( \omega_\alpha \bigr) 
                \end{align}
                が示された.
            \end{description}
            \item[\textbf{$\bm{\omega_\alpha} = \bm{\omega_\beta}$}] 
            
            (1) より,$\forall x \in U_\alpha \cap U_\beta$ および $v \in T_x M$ に対して
            \begin{align}
                A_\beta|_x (v)
                &= (s_\beta^* \omega_\beta)_x(v) \\
                &= \omega_\beta|_{s_\beta(x)} \bigl(T_{x} s_\beta (v)\bigr) \\
                &= \omega_\beta|_{s_\alpha(x) \btl t_{\beta\alpha}(x)} \bigl(T_{s_\alpha(x)} (R_{t_{\beta\alpha}(x)}) \circ T_{x} s_\alpha (v)\bigr)
                + \omega_\beta|_{s_\beta(x)} \bigl( (t_{\beta\alpha}^* \theta)_x(v)^\#|_{s_\beta(x)} \bigr) \\
                &= (R_{t_{\beta\alpha}(x)}^*\omega_\beta)_{s_\alpha(x)} \bigl(T_x s_\alpha(v)\bigr)
                + (t_{\beta\alpha}^* \theta)_x(v) \\
                &= \Adj \bigl( t_{\beta\alpha}(x)^{-1} \bigr) \bigl(s_\alpha^* \omega_\beta|_x (v)\bigr) + (t_{\beta\alpha}^* \theta)_x(v) \\
                &= \Adj \bigl( t_{\beta\alpha}(x)^{-1} \bigr) \bigl(A_\alpha|_x(v)\bigr) + (t^*_{\beta\alpha} \theta)_x(v) \\
                &= \Adj \bigl( t_{\beta\alpha}(x)^{-1} \bigr) \bigl(s_\alpha^* \omega_\alpha|_x(v)\bigr) + (t^*_{\beta\alpha} \theta)_x(v)
            \end{align}
            が言える.よって $s_\alpha^* \omega_\beta = s_\alpha^* \omega_\alpha$ である.
            % $\forall w \in T_{s_\alpha(x)}P$ に対して,
            % \begin{align}
            %     \omega_\beta|_{s_\alpha(x)}(w) 
            % \end{align}
            
        \end{description}    
    \end{enumerate}
    
\end{proof}

% 定理\ref{thm:local-connection}の証明で行った計算は大変複雑であったが,以下で種々の局所表示を求める際に必要になる重要な部分だけまとめておく:

% \begin{mylem}[label=lem:differential]{$C^\infty$ 写像の微分の便利公式}
%     $C^\infty$ 多様体 $M,\, N,\, P$ および $C^\infty$ 写像 $f \colon M \lto N,\, g \colon M \lto P$ を与える.対角写像を $\Delta \colon M \lto M \times M,\; x \lmto (x,\, x)$ と書く.

% \end{mylem}


\subsection{水平持ち上げ}

\begin{mydef}[label=def:horizontal-lift]{水平持ち上げ}
    主束 $G \hookrightarrow P \xrightarrow{\pi} M$ およびその\hyperref[def:connection]{接続形式} $\omega \in \Omega^1(P;\, \mathfrak{g})$ を与える.
    $\gamma \colon [0,\, 1] \lto M$ を\underline{$M$ 上の} \cinfty 曲線とする.このとき,$\forall u \in \pi^{-1}(\{\gamma(0)\})$ に対して以下を充たす\underline{$P$ 上の} \cinfty 曲線 $\tilde{\gamma} \colon [0,\, 1] \lto P$ が一意的に存在し,$\gamma$ の\textbf{水平持ち上げ} (horizontal lift) と呼ばれる:
    \begin{description}
        \item[\textbf{(H-1)}] $\pi \circ \tilde{\gamma} = \gamma$
        \item[\textbf{(H-2)}] $\tilde{\gamma}(0) = u$
        \item[\textbf{(H-3)}] $\forall t \in [0,\, 1]$ に対して
        \begin{align}
            \dot{\tilde{\gamma}}(t) \in \Ker \omega_{\tilde{\gamma}(t)}
        \end{align}
    \end{description}
\end{mydef}


% \underline{$M$ の}チャート $U_\alpha$ が $\gamma([0,\, 1]) \subset U_\alpha$ を充しているとしよう\footnote{この様な状況は,$\gamma$ を適当に分割すればいつでも作れる.}.
% このとき,$\forall t \in [0,\, 1]$ に対して
% \begin{align}
%     \tilde{\gamma}(t) = s_\alpha \bigl( \gamma(t) \bigr) \btl g_\alpha (t) = \varphi_\alpha^{-1}\bigl( \gamma(t),\, g_\alpha(t) \bigr)  \in \pi^{-1}(\{\gamma(t)\})
% \end{align}
% を充たす \underline{$G$ 上の} \cinfty 曲線 $g_\alpha \colon [0,\, 1] \lto G$ を求めよう.



\section{特性類とChern-Simons形式}

\end{document}