\documentclass[TQFT_main]{subfiles}

\begin{document}

\setcounter{chapter}{2}

\chapter{位相的場の理論}

この章は~\cite[Chapter7]{Simon2021}および~\cite{Baez2011physics}に相当する.この節で登場する多様体は特に断らない限り常に $C^\infty$ 多様体である.また,体 $\mathbb{K}$ と言ったら $\mathbb{K} = \mathbb{R},\, \mathbb{C},\, \mathbb{H}$ のいずれかを指すことにしよう.

\section{モノイダル圏}

まず手始めに,モノイダル圏とストリング図式の準備をする.
特に,コボルディズム圏と有限次元Hilbert空間の圏が\hyperref[def:compact]{コンパクト}\hyperref[def:braided-monoidal]{対称モノイダル圏}であることの直感的な説明をする.

\subsection{モノイダル圏の定義}

\begin{mydef}[label=def:monoidal-category,breakable]{モノイダル圏}
    \textbf{モノイダル圏} (monidal category) は,以下の5つのデータからなる:
    \begin{itemize}
        \item 圏 $\mathcal{C}$
        \item \textbf{テンソル積} (tensor product) と呼ばれる関手 $\otimes \colon \mathcal{C} \times \mathcal{C} \lto \mathcal{C}$
        \item \textbf{単位対象} (unit object) $I \in \Obj{\mathcal{C}}$
        \item \textbf{associator}と呼ばれる自然同値
        \begin{align}
            \Familyset[\big]{a_{X,\, Y,\, Z} \colon (X \otimes Y) \otimes Z \xrightarrow{\cong} X \otimes (Y \otimes Z)}{X,\, Y,\, Z \in \Obj{\mathcal{C}}}
        \end{align}
        \item \textbf{left/right unitors}と呼ばれる自然同値
        \begin{align}
            &\Familyset[\big]{l_X \colon I \otimes X \xrightarrow{\cong} X}{X \in \Obj{\mathcal{X}}}, \\
            &\Familyset[\big]{r_X \colon X \otimes I \xrightarrow{\cong} X}{X \in \Obj{\mathcal{X}}}
        \end{align}
        
    \end{itemize}
    これらは $\forall X,\, Y,\, Z,\, W \in \Obj{\mathcal{C}}$ について以下の2つの図式を可換にする:
    \begin{description}
        \item[\textbf{(triangle diagram)}] 
        
        \begin{center}
            \begin{tikzcd}[row sep=large, column sep=large]
                &(X \otimes I) \otimes Y \ar[rr, "a_{X,\, I,\, Y}"]\ar[dr, "r_X \otimes 1_Y"'] & &X \otimes (I \otimes Y) \ar[dl, "1_X \otimes l_Y"] \\
                & &X \otimes Y &
            \end{tikzcd}
        \end{center}
        
        \item[\textbf{(pentagon diagram)}] 
        
        \begin{center}
            \begin{tikzcd}[row sep=large, column sep=large]
                & &((W \otimes X) \otimes Y) \otimes Z \ar[ddl, "a_{W \otimes X,\, Y,\, Z}"']\ar[dr, "a_{W,\, X,\, Y} \otimes 1_Z"] & \\
                & & &(W \otimes (X \otimes Y)) \otimes Z \ar[dd, "a_{W,\, X \otimes Y,\, Z}"] \\
                &(W \otimes X) \otimes (Y \otimes Z) \ar[ddr, "a_{W,\, X,\, Y \otimes Z}"'] & & \\
                & & &W \otimes ((X \otimes Y) \otimes Z) \ar[dl, "1_Z \otimes a_{X,\, Y,\, Z}"]\\
                & &W \otimes (X \otimes (Y \otimes Z)) &
            \end{tikzcd}
        \end{center}
        
    \end{description}

\end{mydef}

定義\ref{def:monoidal-category}は,ストリング図式 (string diagram) で理解するのが良い.
\hyperref[def:monoidal-category]{モノイダル圏}の射 $f \colon X \lto Y,\; f' \colon X' \lto Y'$ があったら,そのテンソル積 $f \otimes f' \colon X \otimes X' \lto Y \otimes Y'$ は,ストリング図式上では次のようになる.

また,単位対象 $I \in \Obj{\mathcal{C}}$ は空白として表す.従って例えば射 $f \colon I \lto X$ は次のようになる:


\begin{myexample}[label=Cob-string]{コボルディズム圏}
    厳密な構成\footnote{例えば,$(B,\, f)$-structureの定義から始めるコボルディズムの統一的な扱いは~\cite[CHAPTER 1]{Kochman1996bordism}などを参照.}は後回しにして,\textbf{コボルディズム圏} (cobordism category) を直感的に導入しよう.
    圏 $\COB_{D+1}$ は,
    \begin{itemize}
        \item $D$ 次元多様体を対象
        \item $D+1$ 次元の\textbf{コボルディズム} (cobordism) を射
    \end{itemize}
    とするような圏のことを言う.
    $D+1$ 次元のコボルディズム $\mathcal{M} \colon X \lto Y$ と言うのは,$D+1$ 次元多様体 $\mathcal{M}$ であって,$\partial \mathcal{M} = X \amalg Y$ となっているようなもの(の微分同相類)のことである:

    射 $\mathcal{M} \colon X \lto Y,\; \mathcal{N} \colon Y \lto Z$ の合成 $\mathcal{N} \circ \mathcal{M} \colon X \lto Y \lto Z$ は次の図式が物語る:

    圏 $\COB_{D+1}$ は,disjoint unionに関して\hyperref[def:monoidal-category]{モノイダル圏}になる:
\end{myexample}

\begin{myexample}[label=Hilb]{有限次元Hilbert空間の圏}
    有限次元 $\mathbb{K}$-Hilbert空間の圏 $\HILB$ とは,
    \begin{itemize}
        \item 有限次元 $\mathbb{K}$-Hilbert空間を対象
        \item 線型写像を射
        \item 写像の合成を射の合成
    \end{itemize}
    に持つような圏のことを言う.$\HILB$ はベクトル空間のテンソル積 $V_1 \otimes V_2$ の上に内積を
    \begin{align}
        \langle v_1 \otimes v_2,\, w_1 \otimes w_2 \rangle \coloneqq \langle v_1,\, w_1 \rangle_1 \langle v_2,\, w_2 \rangle_2
    \end{align}
    と定義することで\hyperref[def:monoidal-category]{モノイダル圏}になる.
\end{myexample}

\subsection{組紐付きモノイダル圏}

\begin{mydef}[label=def:braided-monoidal]{組紐付きモノイダル圏}
    \textbf{組紐付きモノイダル圏} (braided monoidal category) とは,以下の2つからなる:
    \begin{itemize}
        \item \hyperref[def:monoidal-category]{モノイダル圏} $\mathcal{C}$
        \item \textbf{組紐} (braiding) と呼ばれる自然同型
        \begin{align}
            \Familyset[\big]{b_{X,\, Y} \colon X \otimes Y \xrightarrow{\cong} Y \otimes X}{X,\, Y \in \Obj{\mathcal{C}}}
        \end{align}
    \end{itemize}
    これらは $\forall X,\, Y,\, Z \in \Obj{\mathcal{C}}$ について以下の図式を可換にする:
    \begin{description}
        \item[\textbf{(hexagon diagrams)}] 
        
        \begin{center}
            \begin{tikzcd}[row sep=large, column sep=large]
                &X \otimes (Y \otimes Z) \ar[r, "a_{X,\, Y,\, Z}^{-1}"]\ar[d, "b_{X,\, Y \otimes Z}"] &(X \otimes Y) \otimes Z \ar[r, "b_{X,\, Y} \otimes 1_Z"] &(Y \otimes X) \otimes Z \ar[d, "a_{Y,\, X,\, Z}"] \\
                &(Y \otimes Z) \otimes X &Y \otimes (Z \otimes X) \ar[l, "a_{Y,\, Z,\, X}^{-1}"] &Y \otimes (X \otimes Z) \ar[l, "1_X \otimes b_{X,\, Z}"]
            \end{tikzcd}
        \end{center}
        
        \begin{center}
            \begin{tikzcd}[row sep=large, column sep=large]
                &(X \otimes Y) \otimes Z \ar[r, "a_{X,\, Y,\, Z}"]\ar[d, "b_{X \otimes Y ,\, Z}"] &X \otimes (Y \otimes Z) \ar[r, "1_X \otimes b_{Y,\, Z}"] &X \otimes (Z \otimes Y) \ar[d, "a_{X,\, Z,\, Y}^{-1}"] \\
                &Z \otimes (X \otimes Y) &(Z \otimes X) \otimes Y \ar[l, "a_{Z,\, X,\, Y}"] &(X \otimes Z) \otimes Y \ar[l, "b_{X,\, Z} \otimes 1_Y"]
            \end{tikzcd}
        \end{center}
    \end{description}
    \tcblower
    組紐付きモノイダル圏 $\mathcal{C}$ であって,$\mathcal{C}$ の組紐が $b_{X,\, Y} = b_{Y,\, X}^{-1}$ を充たすもののことを\textbf{対称モノイダル圏} (symmetric monoidal category) と呼ぶ.
\end{mydef}

ストリング図式で組紐を書く場合は次のようにする:

このとき\hyperref[def:braided-monoidal]{hexagon diagrams}はとてもわかりやすくなる:

\hyperref[def:braided-monoidal]{対称モノイダル圏}の条件も一目瞭然である:


\begin{myexample}[label=ex:Cob-braided]{$\COB_{D+1}$ の組紐}
    $\COB_{D+1}$ の組紐 $b_{X,\, Y} \colon X \otimes Y \lto Y \otimes X$ は,多様体 $(X \times [0,\, 1]) \amalg (Y \times [0,\, 1])$ と微分同相であるような $D+1$ 次元多様体のことを言う:
    図から,$\COB_{D+1}$ は対称モノイダル圏である.
\end{myexample}

\begin{myexample}[label=ex:Hilb-braided]{$\HILB$ の組紐}
    $\HILB$ の組紐は
    \begin{align}
        b_{X,\, Y} \colon X \otimes Y &\lto Y \otimes X, \\
        x \otimes y &\lmto y \otimes x
    \end{align}
    である.これがベクトル空間の同型写像であることが示される.明らかに $b_{X,\, Y} = b_{Y,\, X}^{-1}$ なので $\HILB$ は対称モノイダル圏である.
\end{myexample}

\subsection{閉圏・コンパクト圏・ダガー圏}

圏 $\mathcal{C}$ を与える.\textbf{Hom 関手} (Hom functor) とは,関手
\begin{align}
    \mathrm{Hom}{} \colon \mathcal{C}^{\mathrm{op}} \times \mathcal{C} \lto \SETS
\end{align}
であって
\begin{align}
    (X,\, Y) &\lmto \Hom{\mathcal{C}} (X,\, Y) \\
    \Bigl(\, (f,\, g) \colon (\textcolor{red}{X'},\, Y) \lto (\textcolor{red}{X},\, Y')\, \Bigr)  &\lmto \Bigl(\, \Hom{\mathcal{C}}(X,\, Y) \lto \Hom{\mathcal{C}}(X',\, Y'),\; h \lmto g \circ h \circ f \,\bigr) 
\end{align}
なる対応を与えるもののこと.

\begin{mydef}[label=def:closed-monoidal,breakable]{閉圏}
    \hyperref[def:monoidal-category]{モノイダル圏} $\mathcal{C}$ を与える.
    \begin{itemize}
        \item $\mathcal{C}$ が\textbf{左に閉じている} (left closed) とは,\textbf{internal hom functor}と呼ばれる関手
        \begin{align}
            \multimap \colon \mathcal{C}^{\mathrm{op}} \times \mathcal{C} \lto \SETS
        \end{align}
        と,\textbf{currying}と呼ばれる自然同型
        \begin{align}
            \Familyset[\big]{c_{X,\, Y,\, Z} \colon \Hom{}(X \otimes Y,\, Z) \xrightarrow{\cong} \Hom{}(X,\, Y \multimap Z)}{X,\, Y,\, Z \in \Obj{\mathcal{C}}}
        \end{align}
        の2つが存在することを言う.
        \item $\mathcal{C}$ が\textbf{右に閉じている} (right closed) とは,\textbf{internal hom functor}と呼ばれる関手と,\textbf{currying}と呼ばれる自然同型
        \begin{align}
            \Familyset[\big]{c_{X,\, Y,\, Z} \colon \Hom{}(X \otimes Y,\, Z) \xrightarrow{\cong} \Hom{}(Y,\, X \multimap Z)}{X,\, Y,\, Z \in \Obj{\mathcal{C}}}
        \end{align}
        の2つが存在することを言う.
    \end{itemize}
\end{mydef}

\hyperref[def:braided-monoidal]{対称モノイダル圏}しか考えないので,以降では右に閉じているかどうかしか気にしないことにする.

\begin{mydef}[label=def:dual,breakable]{双対}
    \hyperref[def:monoidal-category]{モノイダル圏} $\mathcal{C}$ およびその任意の対象 $X,\, X^* \in \Obj{\mathcal{C}}$ を与える.
    $X^*$ が $X$ の\textbf{右双対} (right dual) であり,かつ $X$ が $X^*$ の\textbf{左双対} (left dual) であるとは,
    \begin{itemize}
        \item \textbf{unit}と呼ばれる射
        \begin{align}
            i_X \colon I \lto X^* \otimes X
        \end{align}
        \item \textbf{counit}と呼ばれる射
        \begin{align}
            e_X \colon X \otimes X^* \lto I
        \end{align}
    \end{itemize}
    が存在して以下の図式を可換にすることを言う:
    \begin{description}
        \item[\textbf{(zig-zag equations)}] 
        
        \begin{center}
            \begin{tikzcd}[row sep=large, column sep=large]
                &X \otimes I \ar[dd, "r_X"']\ar[rr, "1_X \otimes i_X"] & &X \otimes (X^* \otimes X) \ar[d, "a_{X,\, X^*,\, X}^{-1}"] \\
                & & &(X \otimes X^*) \otimes X \ar[d, "e_X \otimes 1_X"] \\
                &X & &I \otimes X \ar[ll, "l_X"] \\
            \end{tikzcd}
        \end{center}
        

        \begin{center}
            \begin{tikzcd}[row sep=large, column sep=large]
                &I \otimes X^* \ar[dd, "l_{X^*}"']\ar[rr, "i_X \otimes 1_{X^*}"] & &(X^* \otimes X) \otimes X^* \ar[d, "a_{X^*,\, X,\, X^*}"] \\
                & & &X^* \otimes (X \otimes X^*) \ar[d, "1_{X^*} \otimes e_X"] \\
                &X^* & &X^* \otimes I \ar[ll, "r_{X^*}"] \\
            \end{tikzcd}
        \end{center}
    \end{description}
\end{mydef}

双対のストリング図式は,単に矢印を逆にすれば良い:

このとき\hyperref[def:dual]{zig-zag equations}が本当にジグザグしていることがわかる:

\begin{mydef}[label=def:compact]{コンパクト圏}
    \hyperref[def:monoidal-category]{モノイダル圏} $\mathcal{C}$ は,$\forall X \in \Obj{\mathcal{C}}$ が\hyperref[def:dual]{左・右双対}を持つとき\textbf{コンパクト} (compact) であると言われる.
\end{mydef}

\begin{myexample}[label=ex:Cob-closed]{$\COB$ のunitとcounit}
    $\COB_{D+1}$ における $X \in \Obj{\COB_{D+1}}$ の\hyperref[def:dual]{双対}とは,向き付けを逆にした $D$ 次元多様体 $X$ のことである.
    特に $\COB_3$ における\hyperref[def:dual]{unit, counit}はそれぞれU字管とそれを逆さにしたもののような見た目をしている:

    \hyperref[def:closed-monoidal]{internal hom functor}を $X \multimap Y \coloneqq X^* \otimes Y$ とすれば,図から $\COB_{3}$ が\hyperref[def:closed-monoidal]{閉圏}であることを直接確認できる.
\end{myexample}


\begin{myexample}[label=ex:Hilb-closed]{$\HILB$ のunitとcounit}
    $\HILB$ において $I = \mathbb{C}$ である. 
    従って,$X \in \Obj{\HILB}$ の\hyperref[def:dual]{双対}とは双対ベクトル空間 $\Hom{\mathbb{C}} (X,\, \mathbb{C})$ のことである.ブラ空間のことだと言っても良い.
    特に,自然な同型 $X^* \otimes Y \cong \Hom{\mathbb{C}} (X,\, Y)$ を使うと
    $X$ の\hyperref[def:dual]{unit}は
    \begin{align}
        i_X \colon I &\lto X^* \otimes X, \\
        c &\lmto c\, \mathrm{id}_X
    \end{align}
    で,\hyperref[def:dual]{counit}は
    \begin{align}
        e_X \colon X \otimes X^* &\lto I, \\
        x \otimes f &\lmto f(x)
    \end{align}
    であることがわかる.
    \hyperref[def:closed-monoidal]{internal hom functor}を $X \multimap Y \coloneqq X^* \otimes Y \cong \Hom{\mathbb{C}}(X,\, Y)$ とすれば $\HILB$ が\hyperref[def:closed-monoidal]{閉圏}であることを直接確認できる.
\end{myexample}
% 他に,任意のLie群の有限次元表現は\hyperref[def:compact]{コンパクト}\hyperref[def:braided-monoidal]{対称モノイダル圏}を成すらしい.

実は,コンパクト圏は自動的に\hyperref[def:closed-monoidal]{閉圏}になる.これは
\begin{align}
    X \multimap Y \coloneqq X^* \otimes Y
\end{align}
としてinternal hom functorを定義することで確認できる.


\begin{mydef}[label=def:dagger-monoidal]{ダガー圏}
    圏 $\mathcal{C}$ が\textbf{ダガー圏} (dagger category) であるとは,
    関手
    \begin{align}
        \dagger \colon \mathcal{C} \lto \mathcal{C}^{\mathrm{op}}
    \end{align}
    が存在して以下を充たすことを言う:
    \begin{enumerate}
        \item $\forall X \in \Obj{\mathcal{C}}$ に対して $X^\dagger = X$ を充たす.
        \item $\mathcal{C}$ の任意の射 $f \colon X \lto Y$ に対して $(f^\dagger)^\dagger = f$ を充たす.
    \end{enumerate}
\end{mydef}

\begin{myexample}[label=ex:Cob-dagger]{$\COB$ のdagger}
    $\COB_{D+1}$ における $\mathcal{M} \colon X \lto Y$ のダガーは,上下を逆にしてから $\mathcal{M}$ の連結成分毎に向きを逆にすることで得られる.
\end{myexample}


\begin{myexample}[label=ex:Hilb-dagger]{$\HILB$ のdagger}
    $\HILB$ における $f\colon X \lto Y$ のダガーは,$\forall \phi \in X,\; \forall \psi \in Y$ に対して
    \begin{align}
        \langle f^\dagger (\psi),\, \phi \rangle \coloneqq \langle \psi,\, f(\phi) \rangle
    \end{align}
    とすることで定義される.
\end{myexample}

\subsection{モノイダル関手}

モノイダル関手とは,ざっくり言うと\hyperref[def:monoidal-category]{モノイダル圏}の構造を保存するような関手のことである:
\begin{mydef}[label=def:monidal-functor,breakable]{モノイダル関手}
    2つの\hyperref[def:monoidal-category]{モノイダル圏} $\mathcal{C},\, \mathcal{D}$ の間の関手
    \begin{align}
        F \colon \mathcal{C} \lto \mathcal{D}
    \end{align}
    が\textbf{lax monoidal functor}であるとは,
    \begin{itemize}
        \item 射
        \begin{align}
            \varepsilon \colon I_{\mathcal{D}} \lto F(I_{\mathcal{C}})
        \end{align}
        
        \item 自然変換
        \begin{align}
            \Familyset[\big]{\mu_{X,\, Y} \colon F(X) \otimes_{\mathcal{D}} F(Y) \lto F(X \otimes_{\mathcal{C}} Y)}{X,\, Y \in \Obj{\mathcal{C}}}
        \end{align}
    \end{itemize}
    があって,$\forall X,\, Y,\, Z \in \Obj{\mathcal{C}}$ に対して以下の図式が可換になること:
    \begin{description}
        \item[\textbf{(associatibity)}] 
        
        \begin{center}
            \begin{tikzcd}[row sep=large, column sep=large]
                &(F(X) \otimes_{\mathcal{D}} F(Y)) \otimes_{\mathcal{D}} F(Z) \ar[d, "\mu_{X,\, Y} \otimes 1_{F(Z)}"']\ar[r, "a^{\mathcal{D}}_{F(X),\, F(Y),\, F(Z)}"'] &F(X) \otimes_{\mathcal{D}} (F(Y) \otimes_{\mathcal{D}} F(Z)) \ar[d, "1_{F(X)} \otimes \mu_{Y,\, Z}"]\\
                &F(X \otimes_{\mathcal{C}} Y) \otimes_{\mathcal{D}} F(Z) \ar[d, "\mu_{X \otimes_{\mathcal{C}} Y,\, Z}"'] & F(X) \otimes_{\mathcal{D}} F(Y \otimes_{\mathcal{C}} Z) \ar[d, "\mu_{X,\, Y \otimes_{\mathcal{C}} Z}"] \\
                &F((X \otimes_{\mathcal{C}} Y) \otimes_{\mathcal{C}} Z) \ar[r, "F(a_{X,\, Y,\, Z}^{\mathcal{C}})"'] &F(X \otimes_{\mathcal{C}} (Y \otimes_{\mathcal{C}} Z))
            \end{tikzcd}
        \end{center}
        
        \item[\textbf{(unitality)}] 
        
        \begin{center}
            \begin{tikzcd}[row sep=large, column sep=large]
                &I_{\mathcal{D}} \otimes_{\mathcal{D}} F(X) \ar[d, "l^{\mathcal{D}}_{F(X)}"']\ar[r, "\varepsilon \otimes 1_{F(X)}"] &F(I_{\mathcal{C}}) \otimes_{\mathcal{D}} F(X) \ar[d, "\mu_{I_{\mathcal{C}},\, X}"] \\
                &F(X) &F(I_{\mathcal{C}} \otimes_{\mathcal{C}} X) \ar[l, "F(l^{\mathcal{C}}_X)"]
            \end{tikzcd}
        \end{center}

        \begin{center}
            \begin{tikzcd}[row sep=large, column sep=large]
                &F(X) \otimes_{\mathcal{D}}  I_{\mathcal{D}} \ar[d, "r^{\mathcal{D}}_{F(X)}"']\ar[r, "1_{F(X)} \otimes \varepsilon"] &F(X) \otimes_{\mathcal{D}} F(I_{\mathcal{C}})  \ar[d, "\mu_{X,\, I_{\mathcal{C}}}"] \\
                &F(X) &F(X \otimes_{\mathcal{C}} I_{\mathcal{C}}) \ar[l, "F(r^{\mathcal{C}}_X)"]
            \end{tikzcd}
        \end{center}
        
    \end{description}
    \tcblower
    \begin{itemize}
        \item lax monoidal functor $F$ の $\varepsilon$ と $\mu_{X,\, Y}$ が全て同型射ならば,$F$ は\textbf{strong monoidal functor}と呼ばれる.
        \item lax monoidal functor $F$ の $\varepsilon$ と $\mu_{X,\, Y}$ が全て恒等射ならば,$F$ は\textbf{strict monoidal functor}と呼ばれる.
    \end{itemize}
\end{mydef}


\section{TQFTの定義}

\textbf{位相的場の理論} (Topological Quantum Field Theory; TQFT) の枠組みをトップダウンに導入する.

\subsection{Atiyahの公理系}

まず,全ての出発点としてAtiyahの公理系\cite{Atiyah1988tqft}というものがある:

\begin{myaxiom}[label=ax:Atiyah-TQFT,breakable]{Atiyahの公理系(若干簡略版)}
    体 $\mathbb{K}$ 上の\footnote{原論文~\cite{Atiyah1988tqft}では環としていて,ベクトル空間の代わりに環上の有限生成加群を扱っている.今回はHilbert空間しか考えないので体 $\mathbb{K}$ としておいた.},$D$ 次元の\textbf{位相的場の理論} (Topological Quantum Field Theory; TQFT) とは,以下の2つのデータからなる:
    \begin{enumerate}
        \item 向き付けられた (oriented) $D$ 次元の閉多様体 (closed manifold) $\Sigma$ に対応づけられた\underline{有限次元} $\mathbb{K}$-ベクトル空間 $V(\Sigma)$ 
        \item 向き付けられた $D+1$ 次元の境界付き多様体 $M$ に対応づけられたベクトル $Z(M) \in V(\partial M)$ 
    \end{enumerate}
    これらのデータは以下の条件を充たす:
    \begin{description}
        \item[\textbf{(TQFT-1)}] 
        
        $Z$ は向きを保つ微分同相写像について\textbf{関手的} (functorial) に振る舞う.
        
        % i.e. $D$ 次元閉多様体 $\Sigma,\, \Sigma',\, \Sigma''$ の間の向きを保つ微分同相写像 $f \colon \Sigma \lto \Sigma',\, g \colon \Sigma' \lto \Sigma''$  に対して,$Z(f) \colon Z(\Sigma) \lto Z(\Sigma')$ はベクトル空間の同型写像で,$Z(g \circ f) = Z(g) \circ Z(f)$ が成り立つ.
        % $f$ が $D+1$ 次元多様体 $M,\, M'$ であって $\Sigma = \partial M,\; \Sigma' = \partial M'$ を充たすものの上に $f \colon M \lto M'$ と拡張される場合は $Z(f) \colon Z(M) \lmto Z(M')$ を充たす.
        
        \item[\textbf{(TQFT-2)}] 
        
        $Z$ は\textbf{対合的} (involutory) である.
        
        % i.e. $\Sigma$ の逆の向きを $-\Sigma$ と書くとき,$Z(-\Sigma) = Z(\Sigma)^*$ (i.e. $Z(\Sigma)$ の双対ベクトル空間\footnote{$Z(\Sigma)$ は有限次元だと仮定しているので,$Z(\Sigma)$ をケット空間,$Z(\Sigma)^*$ をブラ空間と見做せる.})を充たす.

        \item[\textbf{(TQFT-3)}] 
        
        $Z$ は\textbf{モノイダル的} (multiplicative\footnote{「乗法的」というと語弊がありそうなのでモノイダル的と言った.}) である.

        % i.e. $\Sigma,\, \Sigma'$ について,$Z(\Sigma \amalg \Sigma') = Z(\Sigma) \otimes Z(\Sigma')$ を充たす.
    \end{description}
    % $D$ 次元の\textbf{位相的場の理論} (Topological Quantum Field Theory; TQFT) とは,
    % \hyperref[Cob-string]{コボルディズム圏}からある\hyperref[def:braided-monoidal]{対称モノイダル圏} $\mathcal{D}$ への\hyperref[def:monidal-functor]{strict monoidal functor}\footnote{strong monoidal functorとする場合もある(例えば\url{https://ncatlab.org/nlab/show/cobordism})ようだが,原論文~\cite{Atiyah1988tqft}ではstrict monoidal functorになっていた.}
    % \begin{align}
    %     Z \colon \COB_{D+1} \lto \mathcal{D}
    % \end{align}
    % のこと.
\end{myaxiom}

~\cite{Atiyah1988tqft}に倣って公理の意味を精査していく.
\begin{description}
    \item[\textbf{(TQFT-1)}] 
    
    この公理は2つの要請を持つ:
    \begin{enumerate}
        \item $D$ 次元閉多様体 $\Sigma,\, \Sigma',\, \Sigma''$ の間の向きを保つ微分同相写像 $f \colon \Sigma \lto \Sigma',\, g \colon \Sigma' \lto \Sigma''$  に対して,$V(f) \colon V(\Sigma) \lto V(\Sigma')$ はベクトル空間の同型写像で,$V(g \circ f) = V(g) \circ V(f)$ が成り立つ.
        \item 向きを保つ微分同相写像 $f \colon \Sigma \lto \Sigma'$ が,$D+1$ 次元多様体 $M,\, M'$ であって $\Sigma = \partial M,\; \Sigma' = \partial M'$ を充たすものの上に $f \colon M \lto M'$ と拡張される場合に $V(f) \bigl(Z(M)\bigr) = Z(M')$ を充たす.
    \end{enumerate}

    \item[\textbf{(TQFT-2)}] 
    
    $\Sigma$ の向きを逆にして得られる $D$ 次元閉多様体を $\Sigma^*$ と書く\footnote{\exref{ex:Cob-closed}の意味で,圏 $\COB_{D+1}$ における $\Sigma \in \Obj{\COB_{D+1}}$ の双対となっている.}とき,$V(\Sigma^*) = V(\Sigma)^*$ を充たす\footnote{$V(\Sigma)$ が有限次元なので,$V(\Sigma)^*$ はブラ空間と見做せる.}.

    \item[\textbf{(TQFT-3)}] 
    
    この公理は5つの要請を持つ:
    \begin{enumerate}
        \item $D$ 次元閉多様体 $\Sigma_1,\, \Sigma_2$ に対して
        \begin{align}
            V(\Sigma_1 \amalg \Sigma_2) = V(\Sigma_1) \otimes V(\Sigma_2)
        \end{align}
        が成り立つこと.
        \item $D+1$ 次元多様体 $M,\, M_1,\, M_2$ に対して $\partial M_1 = \Sigma_1 \amalg \Sigma_3,\; \partial M_2 = \Sigma_2 \amalg \Sigma_3^*,\; M = M_1 \cup_{\Sigma_3} M_2$ が成り立つならば,
        \begin{align}
            Z(M) = \braket{Z(M_1)}{Z(M_2)}
        \end{align}
        ただし,
        \begin{align}
            \braket{\,}{\,} \colon V(\partial M_1) \otimes V(\partial M_2) = V(\Sigma_1) \otimes V(\Sigma_3) \otimes V(\Sigma_3)^* \otimes V(\Sigma_2) &\lto V(\partial M) = V(\Sigma_1) \otimes V(\Sigma_2), \\
            \ket{\psi_1} \otimes \ket{\psi_3} \otimes \bra{\varphi_3} \otimes \ket{\psi_2} &\lmto \braket{\varphi_3}{\psi_3} \ket{\psi_1} \otimes \ket{\psi_2}
        \end{align}
        である.
        \item (2) において $\Sigma_3 = \emptyset$ ならば,
        \begin{align}
            Z(M) = Z(M_1) \otimes Z(M_2)
        \end{align}
        \item (1) から\footnote{$\COB_{D+1}$ の単位対象は $\emptyset$ なので $\emptyset = \emptyset \amalg \emptyset$.よって (1) から $V(\emptyset) = V(\emptyset) \otimes V(\emptyset)$.これを充たすのは $V(\emptyset) = 0,\, \mathbb{K}$(\hyperref[def:monoidal-category]{モノイダル圏} $\VEC{\mathbb{K}}$ の単位対象は $\mathbb{K}$ である)のどちらかしかないので,非自明な方を採用する.},
        \begin{align}
            V(\emptyset) = \mathbb{K}
        \end{align}
        \item (3) から\footnote{$\emptyset = \emptyset \amalg \emptyset$ なので (3) から $Z(\emptyset) = Z(\emptyset) \otimes Z(\emptyset)$.これを充たす $V(\emptyset) = \mathbb{K}$ の元は $0,\, 1$ しかないので,非自明な方を採用する.},
        \begin{align}
            Z(\emptyset) = 1
        \end{align}
    \end{enumerate}
\end{description}

今や別の同値な定義ができる.$D+1$ 次元多様体 $M$ の境界 $\partial M$ を
\begin{align}
    \partial M = \Sigma_1^* \amalg \Sigma_2
\end{align}
と分解すると\footnote{どちらか一方が $\emptyset$ になっても良い},\textsf{\textbf{(TQFT-3)}}-(1) より
$Z(M) \in V(\partial M) = Z(\Sigma_1)^* \otimes Z(\Sigma_2) \cong \Hom{\mathbb{K}} \bigl( Z(\Sigma_1),\, Z(\Sigma_2) \bigr)$ が言えるので,$Z(M)$ を線型写像 $Z(M) \colon V(\Sigma_1) \lto V(\Sigma_2)$ と同一視できるのである.
\textsf{\textbf{(TQFT-1)}}もあわせると,結局これまで $V,\, Z$ と書いていたものは\hyperref[def:monidal-functor]{strong monoidal functor}
\begin{align}
    Z \colon \COB_{D+1} \lto \VEC{\mathbb{K}}
\end{align}
の1つに集約することができる.

\begin{mydef}[label=def:TQFT]{TQFTの定義}
    $D$ 次元の\textbf{位相的場の理論} (Topological Quantum Field Theory; TQFT) とは,
    \hyperref[Cob-string]{コボルディズム圏}からある\hyperref[def:braided-monoidal]{対称モノイダル圏} $\mathcal{D}$ への\hyperref[def:monidal-functor]{strict monoidal functor}\footnote{strong monoidal functorとする場合もある(例えば\url{https://ncatlab.org/nlab/show/cobordism})ようだが,原論文~\cite{Atiyah1988tqft}ではstrict monoidal functorになっていた.}
    \begin{align}
        Z \colon \COB_{D+1} \lto \mathcal{D}
    \end{align}
    のこと.
\end{mydef}

興味があるのは $\mathcal{D} = \VEC{\mathbb{K}},\, \HILB$ の場合なので,以下では\hyperref[ax:Atiyah-TQFT]{TQFT}と言ったら\hyperref[def:monidal-functor]{strict monoidal functor}
\begin{align}
    Z \colon \COB_{D+1} \lto \VEC{\mathbb{K}},\, \HILB
\end{align}
を指すことにしよう.

% \section{フュージョンと $V(\Sigma)$ の次元}

% \subsection{粒子とフュージョン-可換な場合}



% \subsection{粒子とフュージョン-非可換な場合}
% \subsection{$F$ 行列}
% \subsection{$R$ 行列}


\end{document}