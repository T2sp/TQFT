\documentclass[TQFT_main]{subfiles}

\begin{document}

% \setcounter{}{}
\chapter{代数の小技集}

\subsection{ベクトル空間の小技}

次数が $n$ 以下の1変数 $\mathbb{K}$-係数多項式環を $\mathbb{K}[t]^{\le n}$ と書く.

\begin{mydef}[label=def:resultant]{終結式}
    $\mathbb{K}$ を体とし,$\mathbb{K}$-係数多項式 $f,\, g \in \mathbb{K}[t]$ を与える.

    \begin{itemize}
        \item $f,\, g$ の\textbf{Sylvester行列} (Sylvester matrix) とは,$\mathbb{K}$-線形写像
        \begin{align}
            \varphi(f,\, g) \colon \mathbb{K}[t]^{\le \deg g} \times \mathbb{K}[t]^{\le \deg f} \lto \mathbb{K}[t]^{\deg f + \deg g},\; (P,\, Q) \lmto fP + gQ
        \end{align}
        の,
        基底 $1,\, t,\, \dots,\, t^{\deg f + \deg g}$
        に関する表現行列のこと.
        \item $f,\, g$ の\textbf{終結式} (resultant) とは,Sylvester行列 $\varphi(f,\, g)$ の行列式
        \begin{align}
            \bm{\mathrm{res}(f,\, g)} \coloneqq \det \varphi(f,\, g)
        \end{align}
        のこと.
    \end{itemize}
\end{mydef}

\begin{myprop}[label=prop:resultant,breakable]{終結式の基本性質}
    $\mathbb{K}$ を体とし,$\mathbb{K}$-係数多項式 $f = \sum_{k=0}^{\deg f} f_k t^k,\, g = \sum_{k=0}^{\deg g} g_k t^k \in \mathbb{K}[t]$ を与える.このとき以下が成り立つ:
    \begin{enumerate}
        \item \begin{align}
            \res (f,\, g) = \mqty|
                f_0     &           &           &g_0        & & &  \\
                f_1     &\ddots     &           &g_1        &\ddots & &  \\
                \vdots  &\ddots     &f_0        &\vdots     &\ddots &\ddots &  \\ 
                \vdots  &           &f_1        &g_{\deg g} &       &\ddots &g_0 \\
                f_{\deg f}&         &\vdots     &           &\ddots &       &g_1 \\
                            &\ddots &\vdots     &           &       &\ddots &\vdots \\
                            &       &f_{\deg f} &           &       &       &g_{\deg g}
            |
        \end{align}
        \item $\mathbb{K}$ が代数閉体ならば,$f,\, g$ の根を重複込みでそれぞれ $\lambda_i,\, \mu_j$ と書くと
        \begin{align}
            \res (f,\, g) = (f_{\deg f})^{\deg g} (g_{\deg g})^{\deg f} \prod_{\substack{1 \le i \le \deg f \\ 1 \le j \le \deg g}} (\lambda_i - \mu_j)
        \end{align}
        が成り立つ.
    \end{enumerate}
\end{myprop}

\begin{proof}
    \begin{enumerate}
        \item $P = \sum_{k=0}^{\deg g} P_k t^k,\; Q = \sum_{k=0}^{\deg f} Q_k t^{k}$ とすると,$f,\, g$ の\hyperref[def:resultant]{Sylvester行列}は
        \begin{align}
            \varphi(f,\, g)(P,\, Q)
            &= \sum_{k = 0}^{\deg f}\sum_{l = 0}^{\deg g} f_k P_{l} t^{k+l} + \sum_{l = 0}^{\deg f}\sum_{k = 0}^{\deg g} g_k Q_{l} t^{k + l}
        \end{align}
        であるから,$P_l = \delta_l^i,\; Q_l = \delta_l^{j} \WHERE 1 \le i \le \deg g,\; 1 \le j \le \deg f$ のときを考えると
        \begin{align}
            \varphi(f,\, g)(t^i,\, 0) 
            &= \sum_{k=0}^{\deg f} f_k t^{k+i} \\
            % = \sum_{k=0}^{\deg f} \elem{\res (f,\, g)}{k+i}{i} t^{k+i}, \\
            \varphi(f,\, g)(0,\, t^{j})
            &= \sum_{k=0}^{\deg g} g_k t^{k + j}
            % = \sum_{k=0}^{\deg g - i} \elem{\res (f,\, g)}{k + \deg f}{\deg f + i} t^{k + \deg f}
        \end{align}
        となり,$(t^i,\, 0) = t^i,\; (0,\, t^j) = t^{\deg g + j}$ と見做して $ \mathbb{K}[t]^{\le \deg g} \times \mathbb{K}[t]^{\le \deg f}$ の基底を $\mathbb{K}[t]^{\deg f + \deg g}$ の基底と同一視することで $\elem{\res (f,\, g)}{k+i}{i} = f_k,\; \elem{\res (f,\, g)}{k + j}{\deg g + j} = g_k$ が分かった.

        \item 解と係数の関係より,
        \begin{align}
            A \coloneqq \frac{1}{(f_{\deg f})^{\deg g} (g_{\deg g})^{\deg f}} \varphi(f,\, g)
        \end{align}
        の非ゼロな行列成分は $1$ または $\lambda_i$ の基本対称式または $\mu_j$ の基本対称式である.i.e. $A$ は $\lambda_i,\, \mu_j$ の多項式である.
        
        $1\le \forall i \le \deg f,\; 1 \le \forall j \le \deg g$ を1つ固定する.$\lambda_i = \mu_j \eqqcolon \lambda$ ならば,
        \begin{align}
            \bm{x} \coloneqq \mqty[1 \\ \lambda \\ \vdots \\ \lambda^{\deg f + \deg g}]
        \end{align}
        とおくと 
    \end{enumerate}
    
\end{proof}


\subsection{環の小技}


% 補題\ref{lem:smooth-vect-chart}により,素材となる $C^\infty$ ベクトル束から新しい $C^\infty$ ベクトル束を作ることができる.

% \begin{myprop}[label=prop:vect-dual]
    
% \end{myprop}


\end{document}