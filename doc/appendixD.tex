\documentclass[TQFT_main]{subfiles}

\begin{document}

% \setcounter{}{}
\chapter{$C^\infty$ 多様体の話}

~\cite[Chapter4, 5, 6]{Lee2012smooth}の内容を自分用に纏める.

\section{沈めこみ・はめ込み・埋め込み}

$\mathbb{R}^n$ における逆関数定理から始める.まず距離空間に関する基本的な補題を用意する.

\begin{mylem}[label=lem:contraction]{Banachの不動点定理}
    空でない完備な距離空間 $(X,\, d)$ を与える.
    このとき,以下の条件を充たす任意の写像 $F \colon X \lto X$ はただ1つの固定点を持つ:
    \begin{description}
        \item[\textbf{(contraction)}]  
        ある定数 $\lambda \in (0,\, 1)$ が存在し,$\forall x,\, y \in X$ に対して $d \bigl( F(x),\, F(y) \bigr) \le \lambda d(x,\, y)$
    \end{description}
\end{mylem}

\begin{proof}
    まず固定点の存在を示す.$\forall x_0 \in X$ を1つとり, $X$ の点列 $\bigl(\, x_i \,\bigr)_{i=0}^\infty$ を漸化式 $x_{i+1} = F(x_i)$ によって帰納的に定める.
    仮定より $\lambda \in (0,\, 1)$ なので,$\forall \varepsilon > 0$ に対して十分大きな $N_\varepsilon \in \mathbb{N}$ を取れば $\lambda^{N_\varepsilon} < \frac{1-\lambda}{d(x_1,\, x_0)} \varepsilon$ が成り立つようにできる.
    このとき $\forall m,\, n \ge N_\varepsilon \WHERE m \ge n$ に対して
    \begin{align}
        d(x_m,\, x_n) 
        &\le d(x_{m},\, x_{m-1}) + \cdots + d(x_{n+1},\, x_n) & &\because \; \text{三角不等式} \\
        &\le \lambda^{m-1} d(x_1,\, x_0) + \cdots + \lambda^n d(x_1,\, x_0) & &\because \; \text{条件\textsf{\textbf{(contraction)}}} \\
        &\le \lambda^n \left(\sum_{i=0}^{\infty} \lambda^i \right) d(x_1,\, x_0) & \\
        &\le \lambda^{N_\varepsilon} \frac{d(x_1,\, x_0)}{1-\lambda} < \varepsilon
    \end{align}
    が成り立つ.i.e. 点列 $\bigl(\, x_i \,\bigr)_{i=0}^\infty$ はCauchy列であり,仮定より $X$ は完備なのでその収束点 $x \coloneqq \lim_{i \to \infty} x_i \in X$ が一意的に存在する.
    $F$ は明らかに連続なので
    \begin{align}
        F(x) = F (\lim_{i \to \infty} x_i) = \lim_{i \to \infty} F(x_i) = x
    \end{align}
    が成り立つ.i.e. $x$ は固定点である.

    次に固定点 $x$ の一意性を示す.別の固定点 $x' \in X$ が存在したとする.このとき条件\textsf{\textbf{(contraction)}}より
    \begin{align}
        d(x,\, x') = d \bigl( F(x),\, F(x') \bigr)  \le \lambda d(x,\, x') \IFF (1- \lambda) d(x,\, x') \le 0
    \end{align}
    が成り立つので $x = x'$ でなくてはいけない.
\end{proof}

\begin{mytheo}[label=thm:inverse-function-Rn]{$\mathbb{R}^n$ における逆関数定理}
    \begin{itemize}
        \item 開集合\footnote{$\mathbb{R}^n$ には通常のEuclid位相を入れる.} $U,\, V \in \mathbb{R}^n$
        \item $C^\infty$ 関数\footnote{$F = (F^1,\, \dots,\, F^n)$ の各成分 $F^i$ が任意回偏微分可能.} $F \colon U \lto V,\; x \lmto \bigl( F^1(x),\, \dots,\, F^n(x) \bigr)$
    \end{itemize}
    を与え,$F$ のJacobi行列を返す $C^\infty$ 写像
    \begin{align}
        DF \colon U \lto \Mat{n}{\mathbb{R}},\; x \lmto \left[\, \pdv{F^\mu}{x^\nu} ()(x) \right]_{1 \le \mu,\, \nu \le n}
    \end{align}
    を定める.このとき,ある点 $p \in U$ において $DF(p) \in \LGL(n,\, \mathbb{R})$ ならば,
    \begin{itemize}
        \item 点 $p \in U$ の\underline{連結な}近傍 $p \in U_0 \subset U$
        \item 点 $F(p) \in V$ の\underline{連結な}近傍 $F(p) \in V_0 \subset V$
    \end{itemize}
    が存在して $F|_{U_0} \colon U_0 \lto V_0$ が微分同相写像\footnote{i.e. $F|_{U_0}$ は $C^\infty$ 級の逆写像を持つ.}になる.
\end{mytheo}

\begin{proof}
    $U' \coloneqq \bigl\{\, x - p \in \mathbb{R}^n \bigm| x \in U \,\bigr\},\; V' \coloneqq \bigl\{\, x - F(p) \in \mathbb{R}^n \bigm| x \in V \,\bigr\}$ とおく.
    このとき写像
    \begin{align}
        F_1 \colon U' \lto V',\; x \lmto F(x + p) - F(p)
    \end{align}
    は $C^\infty$ 級で,かつ $F_1(0) = 0,\; DF_1(0) = DF(p)$ を充たし,
    $0$ の連結な近傍 $0 \in U'_0 \subset U',\; 0 \in V'_0 \subset V'$ が存在して $F_1|_{U'_0} \colon U'_0 \lto V'_0$ が微分同相写像になることと,点 $p \in U$ の連結な近傍 $p \in U_0 \subset U$ と点 $F(p) \in V$ の連結な近傍 $F(p) \in V_0 \subset V$ が存在して $F|_{U_0} \colon U_0 \lto V_0$ が微分同相写像になることは同値である.

    さらに,$V'' \coloneqq \bigl\{\, DF_1(0)^{-1}(x) \in \mathbb{R}^n \bigm| x \in V' \,\bigr\},\; V''_0 \coloneqq \bigl\{\, DF_1(0)^{-1}(x) \in \mathbb{R}^n \bigm| x \in V'_0 \,\bigr\}$ とおくと
    写像\footnote{$DF_1(0)^{-1} \bigl(F_1(x)\bigr)$ と言うのは,列ベクトル $F_1(x) \in \mathbb{R}^n$ に $n \times n$ 行列 $DF_1(0)^{-1} \in \Mat{n}{\mathbb{R}}$ を作用させると言う意味.} 
    \begin{align}
        F_2 \colon U' \lto V'',\; x \lmto DF_1(0)^{-1} \bigl(F_1(x)\bigr)
    \end{align}
    は $C^\infty$ 級で,$DF_2(0) = \unity_n$ かつ $V''_0 \subset V''$ は $0$ の連結な近傍であり\footnote{$V'_0$ が連結であり,行列 $DF_1(0)^{-1}$ をかけると言う写像は連続なので.},
    $F_2|_{U'_0} \colon U'_0 \lto V''_0$ が微分同相写像になることと $F_1|_{U'_0} \colon U'_0 \lto V'_0$ が微分同相写像になることは同値である.
    以上の考察から,
    \begin{itemize}
        \item $p = 0 \in U$
        \item $F(p) = 0 \in V$
        \item $DF(p) = \unity_n$
    \end{itemize}
    を仮定しても一般性を失わないことが分かった.


    ここで $C^\infty$ 写像
    \begin{align}
        H \colon U \lto \mathbb{R}^n,\; x \lmto x - F(x)
    \end{align}
    を考える.まず $DH(0) = \unity_n - \unity_n = 0$ が成り立つことがわかる.
    さらに写像 $DH \colon U \lto \Mat{n}{\mathbb{R}},\; x \lmto DH(x)$ の連続性\footnote{$H$ が $C^\infty$ 級なので $DH$ は連続.}から,
    $B_\delta(0) \subset U$ を充たす $\delta > 0$ が存在して,$\forall x \in \overline{B_\delta(0)}$ に対して\footnote{$\norm*{\cdot} \colon \Mat{n}{\mathbb{R}} \lto \mathbb{R}_{\ge 0}$ はFrobeniusノルム.}
    \begin{align}
        \norm*{DH(x) - DH(0)} = \norm*{DH(x)} \le \frac{1}{2}
    \end{align}
    が成り立つようにできる.

    \begin{description}
        \item[\textbf{$\bm{F}$ は $\bm{\overline{B_\delta(0)}}$ 上単射}] 
        
        $\forall x,\, y \in \overline{B_\delta(0)}$ をとる.
        $\overline{B_\delta(0)}$ は凸集合なので $\forall t \in [0,\, 1]$ に対して $x + t(y - x) \in \overline{B_\delta(0)}$ が成り立つ.よって
        % 連続写像 $\norm*{DH} \colon U \lto \mathbb{R},\; x \lmto \norm*{DH(x)}$ による像もまたコンパクトである.よって $M \coloneqq \sup_{x \in \overline{B_\delta(0)}} \norm{DH(x)}$ が存在する.
        \begin{align}
            \abs{H(y) - H(x)} 
            &= \abs{\int_0^1 \dd{t} \dv{}{t} H\bigl(x + t(y - x)\bigr)} \\
            &= \abs{\int_0^1 \dd{t} DH \bigl( x + t(y - x) \bigr) (y - x)} \\
            &\le \int_0^1 \dd{t} \abs{DH \bigl( x + t(y - x) \bigr) (y - x)} \\
            &\le \int_0^1 \dd{t} \sup_{x \in \overline{B_\delta(0)}} \norm{DH(x)} \abs{y-x} \\
            &\le \frac{1}{2} \abs{y-x} \label{ineq:thm-inv-1}
        \end{align}
        が言える.故に
        \begin{align}
            \abs{y - x} 
            &= \abs{F(y) - F(x) + H(y) - H(x)} \\
            &\le \abs{F(y) - F(x)} + \abs{H(y) - H(x)} \\
            &\le \abs{F(y) - F(x)} + \frac{1}{2}\abs{y - x}
        \end{align}
        が成り立ち,
        \begin{align}
            \label{ineq:thm-inv-2}
            \frac{1}{2} \abs{y - x} \le \abs{F(y) - F(x)}
        \end{align}
        が従う.故にノルムの正定値性から $\overline{B_\delta(0)}$ 上 $F$ が単射だと分かった.
        
        \item[$\bm{\overline{B_{\delta/2}(0)} \subset F(\overline{B_\delta(0)})}$] 
        
        $\forall y \in \overline{B_{\delta/2}(0)}$ に対してある $x_y \in \overline{B_\delta(0)}$ が存在して $F(x_y) = y$ を充たすことを示す.
        $C^\infty$ 写像
        \begin{align}
            G \colon U \lto \mathbb{R}^n,\; x \lmto y + H(x) = y + x - F(x)
        \end{align}
        を考える.
        不等式\eqref{ineq:thm-inv-1}から $\forall x \in \overline{B_\delta(0)}$ に対して
        \begin{align}
            \abs{G(x)} \le \abs{y} + \abs{H(x)} \le \frac{\delta}{2} + \frac{1}{2}\abs{x} \le \delta
        \end{align}
        が成り立つので $G \bigl( \overline{B_\delta(0)} \bigr) \subset \overline{B_{\delta(0)}}$ が分かった.
        その上再度\eqref{ineq:thm-inv-1}から $\forall x,\, y \in \overline{B_\delta(0)}$ に対して
        \begin{align}
            \abs{G(x) - G(y)} = \abs{H(x) - H(y)} \le \frac{1}{2} \abs{x - y}
        \end{align}
        が成り立つので,空でない完備な距離空間 $\overline{B}_\delta(0)$ 上の写像 $G|_{\overline{B}_\delta(0)} \colon \overline{B}_\delta(0) \lto \overline{B}_\delta(0)$ は補題\ref{lem:contraction}の条件 \textsf{\textbf{(contraction)}} を充たし,$G$ はただ1つの固定点 $x_y \in \overline{B}_\delta(0)$ を持つ.
        $G$ の定義より $G(x_y) = x_y \IMP F(x_y) = y$ である.

         以上の議論から,$U_0 \coloneqq \overline{B_\delta(0)} \cap F^{-1}\bigl( \overline{B_{\delta/2}(0)} \bigr),\, V_0 \coloneqq \overline{B_{\delta/2}(0)}$ とおくと $F|_{U_0} \colon U_0 \lto V_0$ が全単射になることが分かった.
        従って逆写像\footnote{厳密には $(F|_{U_0})^{-1}$ と書くべきだが略記した.} $F^{-1} \colon V_0 \lto U_0$ が存在する.
        さらに $\forall x',\, y' \in V_0$ をとり不等式\eqref{ineq:thm-inv-2}において $x = F^{-1}(x'),\, y = F^{-1}(y')$ とおくことで $F^{-1}$ が連続写像であることがわかる.i.e. $F|_{U_0}$ は同相写像である.
        よって $V_0$ が定義から連結なので $U_0$ も連結である.

        \item[\textbf{$\bm{F^{-1}}$ が $\bm{C^\infty}$ 級}] 
        
        まず $F^{-1} \colon V_0 \lto U_0$ が $C^1$ 級であることを示す.
        $\forall y \in V_0$ を1つ固定する.
        偏微分の連鎖律より $D(F^{-1})(y) = DF \bigl( F^{-1}(y) \bigr)^{-1}$ であるから,
        \begin{align}
            \lim_{y' \to y} \frac{F^{-1}(y') - F^{-1}(y) -  DF \bigl( F^{-1}(y) \bigr)^{-1} (y' - y)}{\abs{y'- y}} = 0
        \end{align}
        を示せば良い.$\forall y' \in V_0 \setminus \{y\}$ を取り,$x \coloneqq F^{-1}(y),\; x' \coloneqq F^{-1}(y') \in U_0 \setminus \{x\}$ とおく.すると
        \begin{align}
            &\abs{\frac{F^{-1}(y') - F^{-1}(y) -  DF \bigl( F^{-1}(y) \bigr)^{-1} (y' - y)}{\abs{y'- y}}}  \\
            &= \frac{\abs{x' - x -  DF(x)^{-1} (y' - y)}}{\abs{y'- y}}  \\
            &= \frac{1}{\abs{y'- y}} \abs{DF(x)^{-1} \bigl( DF(x)(x' - x) - (y' - y)  \bigr)} \\
            &= \frac{\abs{x' - x}}{\abs{F(x') - F(x)}} \abs{DF(x)^{-1} \left( \frac{F(x') - F(x) - DF(x)(x' - x)}{\abs{x' - x}} \right)}  \\
            &\le 2 \sup_{x \in U_0} \norm{DF(x)^{-1}} \abs{\frac{F(x') - F(x) - DF(x)(x' - x)}{\abs{x' - x}}} & &\because\; \text{不等式\eqref{ineq:thm-inv-1}}  \\
        \end{align}
        と評価できるが,$F$ が $C^\infty$ 級なので $\sup_{x \in U_0} \norm{DF(x)^{-1}}$ は有限確定値である.
        $F$ の連続性から $y' \to y$ のとき $x' \to x$ であり,仮定より $F$ は $C^\infty$ 級なので,この極限で最右辺が $0$ に収束することが分かった.

         次に $F^{-1}$ が $\forall k \in \mathbb{N}$ について $C^k$ 級であることを数学的帰納法により示す.$k=1$ の場合は先ほど示した.
        $k > 1$ とする.$D(F^{-1}) = {}^{-1} \circ DF \circ F^{-1}$ であるから\footnote{${}^{-1} \colon \LGL(n,\, \mathbb{R}) \lto \LGL(n,\, \mathbb{R}),\; X \lmto X^{-1}$ とおいた.Cramerの公式よりこれは $C^\infty$ 級である.},帰納法の仮定より $D(F^{-1})$ は $C^k$ 級関数の合成で書けているので $C^k$ 級である.よって $F^{-1}$ は $C^{k+1}$ 級であり,帰納法が完成した.

    \end{description}
    
\end{proof}

\begin{mycol}[label=col:implicit]{陰関数定理}
    $\mathbb{R}^n \times \mathbb{R}^k$ の座標を $(x,\, y) \coloneqq (x^1,\, \dots,\, x^n,\, y^1,\, \dots,\, y^k)$ と書く.
    \begin{itemize}
        \item 開集合 $U \subset \mathbb{R}^n \times \mathbb{R}^k$
        \item $C^\infty$ 関数 $\Phi \colon U \lto \mathbb{R}^k,\; (x,\, y) \lmto \bigl( \Phi^1(x,\, y),\, \dots,\, \Phi^k(x,\, y) \bigr) $
    \end{itemize}
    を与える.
    このとき,点 $(a,\, b) \in U$ において
    \begin{align}
        \left[\, \pdv{\Phi^\mu}{y^\nu}()(a,\, b) \,\right]_{1 \le \mu,\, \nu \le k} \in \LGL(k,\, \mathbb{R})
    \end{align}
    が成り立つならば,
    \begin{itemize}
        \item 点 $a$ の近傍 $a \in V_0 \subset \mathbb{R}^n$
        \item 点 $b$ の近傍 $b \in W_0 \subset \mathbb{R}^k$ 
        \item $C^\infty$ 関数 $F \colon V_0 \lto W_0,\; x \lmto \bigl( F^1(x),\, \dots,\, F^k(x) \bigr)$
    \end{itemize}
    の3つ組であって
    \begin{align}
        \Phi^{-1}(\{c\}) \cap (V_0 \times W_0) = \Bigl\{\, \bigl( x,\, F(x) \bigr) \in V_0 \times W_0 \,\Bigr\} 
    \end{align}
    を充たすものが存在する.ただし $c \coloneqq \Phi(a,\, b) \in \mathbb{R}^k$ とおいた.
\end{mycol}

\begin{proof}
    $C^\infty$ 写像
    \begin{align}
        \Psi \colon U \lto \mathbb{R}^n \times \mathbb{R}^k,\; (x,\, y) \lmto \bigl( x,\, \Phi(x,\, y) \bigr) 
    \end{align}
    を考える.仮定より点 $(a,\, b) \in U$ において
    \begin{align}
        D\Psi (a,\, b) 
        &= \mqty[\unity_n & 0 \\ \left[\, \pdv{\Phi^\mu}{x^\nu}()(a,\, b) \,\right]_{1 \le \mu \le k,\, 1 \le \nu \le n} & \left[\, \pdv{\Phi^\mu}{y^\nu}()(a,\, b) \,\right]_{1 \le \mu,\, \nu \le k} ] \in \LGL(n+k,\, \mathbb{R})
    \end{align}
    であるから,\hyperref[thm:inverse-function-Rn]{逆関数定理}より点 $(a,\, b)$ の連結な近傍 $U_0 \subset U$ と点 $(a,\, c)$ の連結な近傍 $Y_0 \subset \mathbb{R}^n \times \mathbb{R}^k$ が存在して $\Psi|_{U_0} \colon U_0 \lto Y_0$ が微分同相写像になる.$U_0,\, Y_0$ を適当に小さくとることで $U_0 = V \times W$ の形をしていると仮定して良い\footnote{開集合の直積は積位相の開基を成すので.}.

    $\forall (x,\, y) \in Y_0$ に対して $\Psi^{-1}(x,\, y) = \bigl( A(x,\, y),\, B(x,\, y) \bigr)$ とおくと $A\colon Y_0 \lto \mathbb{R}^n,\; B \colon Y_0 \lto \mathbb{R}^k$ はどちらも $C^\infty$ 関数で,
    \begin{align}
        (x,\, y)
        &= \Phi \circ \Psi^{-1}(x,\, y) \\
        &= \Bigl( A(x,\, y),\, \Psi \bigl( A(x,\, y),\, B(x,\, y) \bigr)  \Bigr) 
    \end{align}
    が成り立つ.よって
    \begin{align}
        \Psi^{-1}(x,\, y) &= \bigl( x,\, B(x,\, y) \bigr), \\
        y &= \Psi \bigl( x,\, B(x,\, y) \bigr) 
    \end{align}
    が従う.

    ここで $V_0 \coloneqq \bigl\{\, x \in V \bigm| (x,\, c) \in Y_0 \,\bigr\},\; W_0 \coloneqq W$ とおき,
    \begin{align}
        F \colon V_0 \lto W_0,\; x \lmto B(x,\, c)
    \end{align}
    と定義する.すると $\forall x \in V_0$ に対して
    \begin{align}
        c = \Phi \bigl( x,\, B(x,\, c) \bigr) = \Psi \bigl( x,\, F(x) \bigr) 
    \end{align}
    である.i.e. $\Phi^{-1}(\{c\}) \cap (V_0 \times W_0) \supset \Bigl\{\, \bigl( x,\, F(x) \bigr) \in V_0 \times W_0 \,\Bigr\}$ が言えた.
    逆に $\forall (x,\, y) \in \Phi^{-1}(\{c\}) \cap (V_0 \times W_0)$ をとる.
    このとき $\Phi(x,\, y) = c$ なので $\Psi(x,\, y) = \bigl( x,\, \Phi(x,\, y) \bigr) = (x,\, c)$ であり,
    \begin{align}
        (x,\, y) = \Psi^{-1}(x,\, c) = \bigl( x,\, B(x,\, c) \bigr) = \bigl( x,\, F(x) \bigr) 
    \end{align}
    が成り立つ.i.e. $\Phi^{-1}(\{c\}) \cap (V_0 \times W_0) \subset \Bigl\{\, \bigl( x,\, F(x) \bigr) \in V_0 \times W_0 \,\Bigr\}$ が言えた.

\end{proof}



\subsection{局所微分同相写像}

\begin{mydef}[label=def:loc-diffeo]{局所微分同相写像}
    境界なし/あり $C^\infty$ 多様体 $M,\, N$ を与える.
    
    $C^\infty$ 写像 $F \colon M \lto N$ が\textbf{局所微分同相写像} (local diffeomorphism) であるとは,
    $\forall p \in M$ が以下の条件を充たす近傍 $p \in U_p \subset M$ を持つことを言う:
    \begin{enumerate}
        \item $F(U_p) \subset N$ が開集合
        \item $F|_U \colon U \lto F(U)$ が微分同相写像
    \end{enumerate}
\end{mydef}

\begin{mytheo}[label=thm:inverse-function]{境界を持たない多様体における逆関数定理}
    \begin{itemize}
        \item \underline{境界を持たない} $C^\infty$ 多様体 $M,\, N$
        \item $C^\infty$ 写像 $F \colon M \lto N$
    \end{itemize}
    を与える.このとき,ある点 $p \in M$ において $T_p F \colon T_p M \lto T_{F(p)} N$ が全単射ならば,
    \begin{itemize}
        \item 点 $p \in M$ の連結な近傍 $p \in U_0 \subset M$
        \item 点 $F(p) \in N$ の連結な近傍 $p \in V_0 \subset N$
    \end{itemize}
    が存在して $F|_{U_0} \colon U_0 \lto V_0$ が微分同相写像になる.
\end{mytheo}

\begin{proof}
    $T_p F$ が全単射なので,$\dim M = \dim N \eqqcolon n$ である.
    $p$ を含むチャート $(U,\, \varphi)$ と $F(p)$ を含むチャート $(V,\, \psi)$ を,$F(U) \subset V$ を充たすようにとる.すると
    \begin{itemize}
        \item $\mathbb{R}^n$ の開集合 $\varphi(U),\, \psi(V) \subset \mathbb{R}^n$
        \item $C^\infty$ 関数 $\widehat{F} \coloneqq \psi \circ F \circ \varphi^{-1} \colon \varphi(U) \lto \psi(V)$
    \end{itemize}
    の2つ組は,仮定より点 $\varphi(p) \in \varphi(U)$ において $T_{\varphi(p)} \widehat{F} = T_{F(p)} \psi \circ T_{p} F \circ T_{\varphi(p)} (\varphi^{-1}) \in \LGL(n,\, \mathbb{R})$ を充たすので,
    \hyperref[thm:inverse-function-Rn]{$\mathbb{R}^n$ の逆関数定理}が使えて
    \begin{itemize}
        \item 点 $\varphi(p) \in \varphi(U)$ の連結な近傍 $\varphi(p) \in \widehat{U_0} \subset \varphi(U)$
        \item 点 $\widehat{F}\bigl(\varphi(p)\bigr) = \psi \bigl( F(p) \bigr) \in \varphi(U)$ の連結な近傍 $\widehat{F}\bigl(\varphi(p)\bigr) \in \widehat{V_0} \subset \varphi(V)$
    \end{itemize}
    であって $\widehat{F}|_{\widehat{U_0}} \colon \widehat{U_0} \lto \widehat{V_0}$ が微分同相写像となるようなものがある.
    従って $U_0 \coloneqq \varphi^{-1} (\widehat{U_0}) \subset M,\, V_0 \coloneqq \psi^{-1}(\widehat{V_0}) \subset N$ とおけばこれらはそれぞれ点 $p,\, F(p)$ の連結な近傍で,かつ $F|_{U_0} = \psi^{-1}|_{V_0} \circ \widehat{F}|_{\widehat{U_0}} \circ \varphi|_{U_0} \colon U_0 \lto V_0$ は微分同相写像の合成なので微分同相写像である.
\end{proof}



\subsection{ランク定理}

\begin{mydef}[label=def:rank-smooth]{$C^\infty$ 写像のランク}
    $C^\infty$ 写像 $F \colon M \lto N$ を与える.
    \begin{itemize}
        \item 点 $p \in M$ における $F$ の\textbf{ランク} (rank) とは,線型写像 $T_p F \colon T_p M \lto T_{F(p)} N$ のランク,i.e. $\dim \bigl(\Im (T_p F)\bigr) \in \mathbb{Z}_{\ge 0}$ のこと.$\forall p \in M$ における $F$ のランクが等しいとき,$F$ は\textbf{定ランク} (constant rank) であると言い,$\bm{\rank F} \coloneqq \dim \bigl(\Im (T_p F)\bigr)$ と書く.
        \item 点 $p \in M$ における $F$ のランクが $\min \bigl\{\dim M,\, \dim N\bigr\}$ に等しいとき,$F$ は\textbf{点 $\bm{p}$ においてフルランク} (full rank at $p$) であると言う.
        $\rank F = \min \bigl\{\dim M,\, \dim N\bigr\}$ ならば $F$ は\textbf{フルランク} (full rank) であると言う.
    \end{itemize}
\end{mydef}

\begin{mydef}[label=def:submersion-smooth]{$C^\infty$ 沈めこみ・はめ込み}
    \hyperref[def:rank-smooth]{定ランク}の $C^\infty$ 写像 $F \colon M \lto N$ を与える.
    \begin{itemize}
        \item $F$ が\textbf{ $\bm{C^\infty}$ 沈め込み} (smooth submersion) であるとは,$\forall p \in M$ において $T_p F \colon T_p M \lto T_{F(p)} N$ が全射である,i.e. $\rank F = \dim N$ であることを言う.
        \item $F$ が\textbf{ $\bm{C^\infty}$ はめ込み} (smooth immersion) であるとは,$\forall p \in M$ において $T_p F \colon T_p M \lto T_{F(p)} N$ が単射である,i.e. $\rank F = \dim M$ であること\footnote{階数・退化次元の定理から $\dim (\Ker T_p F) + \dim (\Im T_p F) = \dim M$ なので,$\rank F = \dim (\Im T_p F) = \dim M \IMP \dim (\Ker T_p F) = 0 \IFF \Ker T_p F = 0$}を言う.
    \end{itemize}
\end{mydef}

\begin{mytheo}[label=thm:rank]{局所的ランク定理(境界を持たない場合)}
    \begin{itemize}
        \item \underline{境界を持たない} $C^\infty$ 多様体 $M,\, N$ 
        \item \hyperref[def:rank-smooth]{定ランク}の $C^\infty$ 写像 $F \colon M \lto N$
    \end{itemize}
    を与える.
    このとき $\forall p \in M$ に対して
    \begin{itemize}
        \item 点 $p \in M$ を含む $M$ の $C^\infty$ チャート $(U_0,\, \Phi)$ であって $\Phi(p) = 0 \in \mathbb{R}^{\dim M}$ を充たすもの
        \item 点 $F(p) \in N$ を含む $N$ の $C^\infty$ チャート $(V_0,\, \Psi)$ であって $\Psi \bigl( F(p) \bigr) = 0 \in \mathbb{R}^{\dim N}$ かつ $F(U) \subset V$ を充たすもの
    \end{itemize}
    が存在して,$\forall (x^1,\, \dots,\, x^{\dim M}) \in \Phi(U) \subset \mathbb{R}^{\dim M}$ に対して
    \begin{align}
        \label{eq:coord-rep-canonical}
        \Psi \circ F \circ \Phi^{-1} &(x^1,\, \dots,\, x^{\rank F},\, x^{\rank F + 1},\, \dots,\,  x^{\dim M}) \\
        =\; &(x^1,\, \dots,\, x^{\rank F},\, \underbrace{0,\, \dots,\, 0}_{\dim N - \rank F}) \in \Psi(V) \subset \mathbb{R}^{\dim N}
    \end{align}
    を充たす.
    \tcblower
    特に $F$ が\hyperref[def:submersion-smooth]{$C^\infty$ 沈め込み}ならば\eqref{eq:coord-rep-canonical}は
    \begin{align}
        \label{eq:coord-rep-canonical-submersion}
        \Psi \circ F \circ \Phi^{-1} (x^1,\, \dots,\, x^{\dim N},\, x^{\dim N + 1},\, \dots,\,  x^{\dim M}) = (x^1,\, \dots,\, x^{\dim N})
    \end{align}
    の形になり,$F$ が\hyperref[def:submersion-smooth]{$C^\infty$ はめ込み}ならば\eqref{eq:coord-rep-canonical}は
    \begin{align}
        \label{eq:coord-rep-canonical-immersion}
        \Psi \circ F \circ \Phi^{-1} (x^1,\, \dots,\, x^{\dim M}) = (x^1,\, \dots,\, x^{\dim M},\, 0,\, \dots,\, 0)
    \end{align}
    の形になる.
\end{mytheo}    

\begin{proof}
    $\forall p \in M$ を1つ固定する.以降では $p$ を含む $M$ の任意の $C^\infty$ チャート $(U,\, \varphi) = \bigl( U,\, (x^\mu) \bigr) $ および $F(p)$ を含む $N$ の任意の $C^\infty$ チャート $(V,\, \psi) = \bigl( V,\, (x'{}^\mu) \bigr) $ に対して
    \begin{align}
        \widehat{U} &\coloneqq \varphi(U) \subset \mathbb{R}^{\dim M}, \\
        \widehat{V} &\coloneqq \psi(V) \subset \mathbb{R}^{\dim N}, \\
        \widehat{F} &\coloneqq \psi \circ F \circ \varphi^{-1} \colon \widehat{U} \lto \widehat{V}, \\
        \widehat{p} &\coloneqq \varphi(p) \in \widehat{U}
    \end{align}
    とおく.便宜上 $C^\infty$ チャート $\bigl(U,\, (x^\mu)\bigr),\, \bigl(V,\, (x'{}^\mu)\bigr)$ の座標関数をそれぞれ
    \begin{align}
        &(x,\, y) = (x^1,\, \dots,\, x^{\rank F},\, y^1,\, \dots,\, y^{\dim M - \rank F})
        \coloneqq (x^1,\, \dots,\, x^{\dim M})  \\
        % \colon U \lto \widehat{U} \subset \mathbb{R}^{\rank F} \times \mathbb{R}^{\dim M - \rank F} \\
        &(v,\, w) = (v^1,\, \dots,\, v^{\rank F},\, w^1,\, \dots,\, w^{\dim N - \rank F})
        \coloneqq (x'{}^1,\, \dots,\, x'{}^{\dim N}) 
        % \colon V \lto \widehat{V} \subset \mathbb{R}^{\rank F} \times \mathbb{R}^{\dim N - \rank F}
    \end{align}
    とおき直す.また,2つの $C^\infty$ 写像を
    \begin{align}
        Q \colon \widehat{U} &\lto \mathbb{R}^{\rank F},\; (x,\, y) \lmto \bigl( \widehat{F}^1(x,\, y),\, \dots,\, \widehat{F}^{\rank F} (x,\, y) \bigr) \\
        R \colon \widehat{U} &\lto \mathbb{R}^{\dim N - \rank F},\; (x,\, y) \lmto \bigl( \widehat{F}^{\rank F + 1}(x,\, y),\, \dots,\, \widehat{F}^{\dim N} (x,\, y) \bigr)
    \end{align}
    と定義する.このとき $\widehat{F} = (Q,\, R)$ と書ける.
    
    仮定より $F$ は\hyperref[def:rank-smooth]{定ランク}なので,点 $\widehat{p} \in \widehat{U}$ において線型写像 $T_{\widehat{p}} \widehat{F} \colon T_{\widehat{p}} \widehat{U} \lto T_{\widehat{F} (\widehat{p})} \widehat{V}$ の表現行列のランクは $\rank F$ である.
    必要ならば $M,\, N$ の $C^\infty$ チャートを取り替えることで座標関数の順番を好きなように入れ替えることができる
    \footnote{座標を入れ替える写像は微分同相写像なので,$M,\, N$ の $C^\infty$ 構造の中には座標の入れ替えによって互いに移り合えるような $C^\infty$ チャートたちが含まれている.}
    ので,
    $T_{\widehat{p}} \widehat{F}$ の表現行列\footnote{これは $C^\infty$ 関数 $\widehat{F}$ の\hyperref[thm:inverse-function-Rn]{Jacobi行列} $D \widehat{F} (\widehat{p})$ である.}の $\rank F$ 次首座小行列に対して
    \begin{align}
        \label{eq:hypothesis-rank-theorem}
        \left[ \, \pdv{\widehat{F}^\mu}{x^\nu}()(\widehat{p}) \, \right]_{1 \le \mu,\, \nu \le \rank F} = \left[ \, \pdv{Q^\mu}{x^\nu}()(0,\, 0) \,\right]_{1 \le \mu,\, \nu \le \rank F} \in \LGL(\rank F,\, \mathbb{R})
    \end{align}
    が成り立つような $C^\infty$ チャート $\bigl( U,\, (x,\, y) \bigr),\, \bigl( V,\, (v,\, w) \bigr)$ が存在する.
    さらに,$\widehat{U},\, \widehat{V}$ の原点を平行移動することでいつでも $\widehat{p} = (0,\, 0) \in \mathbb{R}^{\dim M},\, \widehat{F}(\widehat{p}) = (0,\, 0) \in \mathbb{R}^{\dim N}$ が成り立つようにできる\footnote{平行移動は微分同相写像なので,$M,\, N$ の $C^\infty$ 構造には平行移動によって互いに移り合えるような $C^\infty$ チャートたちが含まれている.}.

    ここまでの議論の要請を満たす $M,\, N$ の任意の $C^\infty$ チャート $(U,\, \varphi),\, (V,\, \psi)$ をとり, 
    $C^\infty$ 写像
    \begin{align}
        \widehat{\Phi} \colon \widehat{U} \lto \mathbb{R}^{\dim M},\; (x,\, y) \lmto \bigl( Q(x,\, y),\, y \bigr) 
    \end{align}
    を考える.点 $(0,\, 0) \in \widehat{U}$ における $\Psi$ のJacobi行列は
    \begin{align}
        D\Psi(0,\, 0) = 
        \mqty[
            \left[\pdv{Q^\mu}{x^\nu}()(0,\, 0)\right]_{1 \le \mu,\, \nu \le \rank F} & \left[\pdv{Q^\mu}{y^\nu}()(0,\, 0)\right]_{1 \le \mu \le \rank F,\, 1 \le \nu \le \dim M - \rank F} \\ 
            & \\
            0 & \unity_{\dim M - \rank F}
        ]
    \end{align}
    となるが,仮定\eqref{eq:hypothesis-rank-theorem}よりこれは正則行列である.よって\hyperref[thm:inverse-function-Rn]{$\mathbb{R}^{\dim M}$ における逆関数定理}から
    \begin{itemize}
        \item 点 $(0,\, 0) \in \widehat{U}$ の連結な近傍 $\widehat{U_0} \subset \widehat{U}$
        \item 点 $(0,\, 0) \in \mathbb{R}^{\dim M}$ の連結な近傍 $\tilde{U}_0 \subset \mathbb{R}^{\dim M}$
    \end{itemize}
    が存在して $\widehat{\Phi}|_{\widehat{U_0}} \colon \widehat{U_0} \lto \tilde{U}_0$ が微分同相写像になる.$\widehat{U_0},\, \tilde{U}_0$ を適当に小さくとり直すことで $\tilde{U}_0$ が開区間の直積であると仮定して良い.
    $\widehat{\Phi}|_{\widehat{U_0}}$ の逆写像を $\widehat{\Phi}^{-1} \colon \tilde{U}_0 \lto \widehat{U_0},\; (x,\, y) \lmto \bigl( A(x,\, y),\, B(x,\, y) \bigr)$ と書くと $A \colon \tilde{U}_0 \lto \mathbb{R}^{\rank F},\, B \colon \tilde{U}_0 \lto \mathbb{R}^{\dim M -\rank F}$ はどちらも $C^\infty$ 写像で,
    $\forall (x,\, y) \in \tilde{U}_0$ に対して
    \begin{align}
        (x,\, y) 
        = \widehat{\Phi} \circ \widehat{\Phi}^{-1}(x,\, y)
        = \Bigl( Q \bigl( A(x,\, y),\, B(x,\, y) \bigr),\, B(x,\, y)  \Bigr)
    \end{align}
    が成り立つ.よって
    \begin{align}
        \widehat{\Phi}^{-1}(x,\, y) &= \bigl( A(x,\, y),\, y \bigr) , \\
        x &= Q \bigl( A(x,\, y),\, y \bigr) 
    \end{align}
    であり,
    \begin{align}
        \widehat{F} \circ \widehat{\Phi}^{-1} (x,\, y)
        &= \widehat{F} \bigl( A(x,\, y),\, y \bigr)  \\
        &= \Bigl( Q \bigl( A(x,\, y),\, y \bigr),\,  R \bigl( A(x,\, y),\, y \bigr)  \Bigr) \\
        &= \Bigl( x,\,  R \bigl( A(x,\, y),\, y \bigr)  \Bigr)
    \end{align}
    が分かった.
    従って $C^\infty$ 写像 $\tilde{R} \colon \tilde{U}_0 \lto \mathbb{R}^{\dim N -\rank F},\, (x,\, y) \lmto R \bigl( A(x,\, y),\, y \bigr)$ と定義すると,
    $\forall (x,\, y) \in \tilde{U}_0$ における $\widehat{F} \circ \widehat{\Phi}^{-1}$ のJacobi行列は
    \begin{align}
        D (\widehat{F} \circ \Psi^{-1}) (x,\, y) =
        \mqty[
            \unity_{\rank F} & 0 \\
            & \\
            \left[\pdv{\tilde{R}^\mu}{x^\nu}()(x,\, y)\right]_{1 \le \mu \le \dim N - \rank F,\, 1 \le \nu \le \rank F} & \left[\pdv{\tilde{R}^\mu}{y^\nu}()(x,\, y)\right]_{1 \le \mu \le \dim N - \rank F,\, 1 \le \nu \le \dim M - \rank F} \\ 
        ]
    \end{align}
    と計算できる.ところが,仮定より行列 $D(\widehat{F})(x,\, y)$ のランクは $\rank F$ で,かつ $\widehat{\Phi}^{-1}$ は微分同相写像なので $D(\widehat{\Phi}^{-1})(x,\, y) \in \LGL(\dim M,\, \mathbb{R})$ であり,行列 $D (\widehat{F} \circ \widehat{\Phi}^{-1})(x,\, y) = D(\widehat{F})(x,\, y) \circ D(\widehat{\Phi}^{-1})(x,\, y)$ のランクは $\rank F$ に等しい.よって $\forall (x,\, y) \in \tilde{U_0}$ において
    \begin{align}
        \left[\pdv{\tilde{R}^\mu}{y^\nu}()(x,\, y)\right]_{1 \le \mu \le \dim N - \rank F,\, 1 \le \nu \le \dim M - \rank F} = 0
    \end{align}
    でなくてはいけない.$\tilde{U}_0$ は開区間の直積なので,このことから $\tilde{R}$ が $(y^1,\, \dots,\, y^{\dim M - \rank F})$ によらないことが分かった.
    よって $S(x) \coloneqq \tilde{R}(x,\, 0)$ とおくと
    \begin{align}
        \label{eq:thm-rank-1}
        \widehat{F} \circ \widehat{\Phi}^{-1}(x,\, y) = \bigl( x,\, S(x) \bigr) 
    \end{align}
    と書けることが分かった.

    最後に,点 $\widehat{F} (\widehat{p}) = (0,\, 0) \in \widehat{V}$ の適当な近傍を構成する.
    開集合\footnote{写像 $f \colon \widehat{V} \lto \mathbb{R}^{\dim M},\; (v,\, w) \lmto (v,\, 0)$ は連続で,$\tilde{U}_0 \subset \mathbb{R}^{\dim M}$ は開集合なので,$V_0 = \widehat{V} \cap f^{-1}(\tilde{U}_0) \subset \widehat{V}$ もまた開集合.} 
    $\widehat{V_0} \subset \widehat{V}$ を
    \begin{align}
        \widehat{V_0} \coloneqq \bigl\{\, (v,\, w) \in \widehat{V} \bigm| (v,\, 0) \in \tilde{U}_0 \,\bigr\} 
    \end{align}
    と定義すると,$\widehat{F} (\widehat{p}) = (0,\, 0) \in \widehat{V_0}$ なので $\widehat{V_0}$ は点 $\widehat{F} (\widehat{p})$ の近傍であり,
    $\tilde{U}_0$ は開区間の直積なので\eqref{eq:thm-rank-1}から$\widehat{F} \circ \widehat{\Phi}^{-1}(\tilde{U}_0) \subset \widehat{V_0}$ が成り立つ.
    そして $C^\infty$ 写像
    \begin{align}
        \widehat{\Psi} \colon \widehat{V_0} \lto \mathbb{R}^{\dim N},\; (v,\, w) \lmto \bigl( v,\, w - S(v) \bigr) 
    \end{align}
    を考える.$\widehat{\Psi} \colon \widehat{V_0} \lto \widehat{\Psi}(\widehat{V_0})$ は $C^\infty$ 写像
    \begin{align}
        \widehat{\Psi}^{-1} \colon \widehat{\Psi}(\widehat{V_0}) \lto \widehat{V_0},\; (s,\, t) \lmto \bigl( s,\, t + S(s) \bigr) 
    \end{align}
    を逆写像に持つので微分同相写像であり,$(V_0,\, \widehat{\Psi})$ は $N$ の $C^\infty$ チャートである.その上\eqref{eq:thm-rank-1}から $\forall (x,\, y) \in \tilde{U}_0$ に対して
    \begin{align}
        \label{eq:thm-rank-ans}
        \widehat{\Psi} \circ \widehat{F} \circ \widehat{\Phi}^{-1}(x,\, y) = \widehat{\Psi}\bigl(x,\, S(x)\bigr) = (x,\, 0)
    \end{align}
    となる.

    以上をまとめると,
    \begin{align}
        U_0 &\coloneqq \varphi^{-1} (\widehat{U_0}) \subset M, \\
        V_0 &\coloneqq \psi^{-1} (\widehat{V_0}) \subset N, \\
        \Phi &\coloneqq \widehat{\Phi} \circ \varphi \colon U_0 \lto \Phi(U_0), \\
        \Psi &\coloneqq \widehat{\Psi} \circ \psi \colon V_0 \lto \Psi(V_0)
    \end{align}
    とおくと $\Phi,\, \Psi$ は微分同相写像であり,
    \begin{itemize}
        \item 点 $p \in M$ を含む $M$ の $C^\infty$ チャート $(U_0,\, \Phi)$
        \item 点 $F(p) \in N$ を含む $N$ の $C^\infty$ チャート $(V_0,\, \Psi)$
    \end{itemize}
    の2つ組は\eqref{eq:thm-rank-ans}から $\forall (x,\, y) \in \Phi(U_0) = \tilde{U}_0$ に対して
    \begin{align}
        \Psi \circ F \circ \Phi (x,\, y) = \widehat{\Psi} \circ \widehat{F} \circ \widehat{\Phi}^{-1}(x,\, y) = (x,\, 0)
    \end{align}
    を充たすので証明が完了する.
\end{proof}

\begin{mycol}[label=col:rank]{}
    \underline{境界を持たない} $C^\infty$ 多様体 $M,\, N$ と $C^\infty$ 写像 $F \colon M \lto N$ を与える.
    このとき $M$ が連結ならば以下の2つは同値である:
    \begin{enumerate}
        \item $F$ は\hyperref[def:rank-smooth]{定ランク}
        \item $\forall p \in M$ において,$p,\, F(p)$ を含む $M,\, N$ の $C^\infty$ チャート $(U,\, \varphi),\, (V,\, \psi)$ であって $F$ の座標表示 $\psi \circ F \circ \varphi^{-1} \colon \varphi(U) \lto \psi(V)$ が線型写像となるようなものが存在する.
    \end{enumerate}
\end{mycol}

\begin{proof}
    \begin{description}
        \item[\textbf{(1) $\bm{\Longrightarrow}$ (2)}] 
        
        定理\ref{thm:rank}の\eqref{eq:coord-rep-canonical}は $F$ の座標表示が線型写像であることを意味する.

        \item[\textbf{(1) $\bm{\Longleftarrow}$ (2)}] 
        
        任意の線型写像のランクは一意に定まるので,$\forall p \in M$ の近傍において $F$ の\hyperref[def:rank-smooth]{ランク}は一定だが,仮定より $M$ は連結なので $F$ は定ランクである.
    \end{description}
\end{proof}

\begin{mytheo}[label=thm:rank-global]{大域的ランク定理(境界を持たない場合)}
    \begin{itemize}
        \item \underline{境界を持たない} $C^\infty$ 多様体 $M,\, N$ 
        \item \hyperref[def:rank-smooth]{定ランク}の $C^\infty$ 写像 $F \colon M \lto N$
    \end{itemize}
    を与える.
    このとき以下が成り立つ:
    \begin{enumerate}
        \item $F$ が全射 $\IMP$ $F$ は\hyperref[def:submersion-smooth]{$C^\infty$ 沈め込み}
        \item $F$ が単射 $\IMP$ $F$ は\hyperref[def:submersion-smooth]{$C^\infty$ はめ込み}
        \item $F$ が全単射 $\IMP$ $F$ は微分同相写像
    \end{enumerate}
\end{mytheo}

\begin{proof}
    \begin{enumerate}
        \item $F$ が全射だとする.
        もし $\rank F < \dim N$ ならば,\hyperref[thm:rank]{局所的ランク定理}により 
        $\forall p \in M$ に対して
        \begin{itemize}
            \item 点 $p \in M$ を含む $M$ の $C^\infty$ チャート $(U_p,\, \Phi)$ であって $\Phi(p) = 0 \in \mathbb{R}^{\dim M}$ を充たすもの
            \item 点 $F(p) \in N$ を含む $N$ の $C^\infty$ チャート $(V_p,\, \Psi)$ であって $\Psi \bigl( F(p) \bigr) = 0 \in \mathbb{R}^{\dim N}$ かつ $F(U_p) \subset V_p$ を充たすもの
        \end{itemize}
        が存在して,$F$ の座標表示 $\Psi \circ F \circ \Phi^{-1} \colon \Phi(U_p) \lto \Psi(V_p)$ が\eqref{eq:coord-rep-canonical}の形になる.
        必要なら $U_p$ を適当に小さくとることで $\exists r > 0,\; \Phi(U_p) = B_r (0) \subset \mathbb{R}^{\dim M}$ で,かつ $F(\overline{U_p}) \subset V_p$ が成り立つと仮定して良い.
        このとき $F(\overline{U_p})$ はHausdorff空間 $\bigl\{\, y \in V_p \bigm| \Psi(y) = (y^1,\, \dots,\, y^{\rank F},\, 0,\, \dots,\, 0)  \,\bigr\}$ のコンパクト部分集合なので,$N$ の閉集合でかつ $N$ の開集合を部分集合として持たない.i.e. $N$ 上\textbf{疎} (nowhere dense) である.
        多様体の第2可算性より任意の多様体の開被覆は高々可算な部分被覆を持つから,$M$ の開被覆 $\Familyset[\big]{U_p}{p \in M}$ および $F(M)$ の開被覆 $\Familyset[\big]{V_p}{p \in M}$ はそれぞれ高々可算な部分被覆 $\Familyset[\big]{U_{i}}{i \in I},\, \Familyset[\big]{V_i}{i \in I}$ を持つ.i.e. $F(M)$ は高々可算個の疎集合 $F(\overline{U_i})$ たちの和集合なので,Baireのカテゴリー定理から $F(M)$ の $N$ における内部は空集合ということになるが,これは $F$ が全射であることに矛盾する.
        よって背理法から $\rank F = \dim N$ が言えた.
        \item $F$ が単射だとする.
        もし $\rank F < \dim M$ ならば,\hyperref[thm:rank]{局所的ランク定理}により 
        $\forall p \in M$ に対して
        \begin{itemize}
            \item 点 $p \in M$ を含む $M$ の $C^\infty$ チャート $(U_p,\, \Phi)$ であって $\Phi(p) = 0 \in \mathbb{R}^{\dim M}$ を充たすもの
            \item 点 $F(p) \in N$ を含む $N$ の $C^\infty$ チャート $(V_p,\, \Psi)$ であって $\Psi \bigl( F(p) \bigr) = 0 \in \mathbb{R}^{\dim N}$ かつ $F(U_p) \subset V_p$ を充たすもの
        \end{itemize}
        が存在して,$F$ の座標表示 $\Psi \circ F \circ \Phi^{-1} \colon \Phi(U_p) \lto \Psi(V_p)$ が\eqref{eq:coord-rep-canonical}を充たす.
        このことから,十分小さな任意の $\varepsilon \in \mathbb{R}^{\dim M - \rank F}$ に対して $\Psi \circ F \circ \Phi^{-1}(0,\, \dots,\, 0,\, \varepsilon) = \Psi \circ F \circ \Phi^{-1}(0,\, \cdots,\, 0)$ が成り立つことになり $F$ の単射性に矛盾.
        \item (1), (2) より $F$ が全単射なら $F$ は $C^\infty$ 沈め込みかつ $C^\infty$ はめ込みであるから,$F$ は全単射な\hyperref[def:loc-diffeo]{局所微分同相写像}である.よって $F$ は微分同相写像である.
    \end{enumerate}
    
\end{proof}


\begin{mytheo}[label=thm:rank-b]{はめ込みに関する局所的ランク定理(境界付き)}
    \begin{itemize}
        \item \underline{境界付き} $C^\infty$ 多様体 $M$ と\underline{境界を持たない} $C^\infty$ 多様体 $N$
        \item \hyperref[def:submersion-smooth]{$C^\infty$ はめ込み}の $C^\infty$ 写像 $F \colon M \lto N$
    \end{itemize}
    を与える.
    このとき $\forall p \in \textcolor{red}{\partial M}$ に対して
    \begin{itemize}
        \item 点 $p \in M$ を含む $M$ の $C^\infty$ 境界チャート $(U_0,\, \Phi)$ であって $\Phi(p) = 0 \in \mathbb{H}^{\dim M}$ を充たすもの
        \item 点 $F(p) \in N$ を含む $N$ の $C^\infty$ チャート $(V_0,\, \Psi)$ であって $\Psi \bigl( F(p) \bigr) = 0 \in \mathbb{R}^{\dim N}$ かつ $F(U) \subset V$ を充たすもの
    \end{itemize}
    が存在して,$\forall (x^1,\, \dots,\, x^{\dim M}) \in \Phi(U) \subset \mathbb{H}^{\dim M}$ に対して
    \begin{align}
        \label{eq:coord-rep-canonical-b}
        \Psi \circ F \circ \Phi^{-1} &(x^1,\, \dots,\, x^{\dim M}) = (x^1,\, \dots,\, x^{\dim M},\, \underbrace{0,\, \dots,\, 0}_{\dim N - \rank M}) \in \Psi(V) \subset \mathbb{R}^{\dim N}
    \end{align}
    を充たす.
\end{mytheo}

\begin{proof}
    
\end{proof}



\section{部分多様体}
\section{Sardの定理}

\end{document}