\documentclass[TQFT_main]{subfiles}

\begin{document}

% \setcounter{}{}

\chapter{QFTミニマム}

この章は\cite{Kugo1989qft1}による.自然単位系を用いる.

\section{経路積分}

時空を表す多様体を $\mathcal{M}$ と書く.$\mathcal{M}$ は時間方向と空間方向に $\mathcal{M} = \mathbb{R} \times \Sigma$ と書けるとする.
場とはベクトル束 $V \hookrightarrow E \xrightarrow{\pi} \mathcal{M}$ の切断 $\varphi \in \Gamma(E)$ のこととする.

系のラグランジアン密度 $\mathcal{L} \bigl( \varphi(x),\, \partial_\mu \varphi(x) \bigr)$ を与える.
\begin{itemize}
    \item 時刻 $\irm{t}{i}$ における古典場の配位が $\irm{\Psi}{i} \in \Gamma(E|_{\{\irm{t}{i}\} \times \Sigma})$ であることに対応する量子状態\footnote{Heisenberg表示} $\ket{\irm{\Psi}{i},\, \irm{t}{i}}$
    \item 時刻 $\irm{t}{f}$ における古典場の配位が $\irm{\Psi}{f} \in \Gamma(E|_{\{\irm{t}{f}\} \times \Sigma})$ であることに対応する量子状態 $\ket{\irm{\Psi}{f},\, \irm{t}{f}}$
\end{itemize}
の間に
\begin{align}
    \braket{\irm{\Psi}{f},\, \irm{t}{f}}{\irm{\Psi}{i},\, \irm{t}{i}}
    \propto \int [\dd{\varphi}] \exp \left( \iunit \int_{[\irm{t}{i},\, \irm{t}{f}] \times \Sigma} \dd[D+1]{x} \mathcal{L} \bigl( \varphi(x),\, \partial_\mu \varphi(x) \bigr)\right) 
\end{align}
を要請するのが経路積分による場の量子化である.

$\hat{\varphi}(x)\WHERE x \in \mathcal{M}$ を場の演算子とする.与えられた時空点 $x_1,\, \dots,\, x_n \in \mathcal{M}$ および境界条件 $\irm{\Psi}{i} \in \Gamma(E|_{\{\irm{t}{i}\} \times \Sigma}),\; \irm{\Psi}{f} \in \Gamma(E|_{\{\irm{t}{f}\} \times \Sigma})$ に対して,
\textbf{Green関数}を
\begin{align}
    \label{def:Green}
    G^{(n)} (x_1,\, \dots,\, x_n;\,\irm{\Psi}{f},\, \irm{t}{f};\, \irm{\Psi}{i},\, \irm{t}{i} )
    &\coloneqq \frac{\mel{\irm{\Psi}{f},\, \irm{t}{f}}{\mathcal{T}\bigl\{ \hat{\varphi}(x_1) \cdots \hat{\varphi}(x_n) \bigr\}}{\irm{\Psi}{i},\, \irm{t}{i}}}{\braket{\irm{\Psi}{f},\, \irm{t}{f}}{\irm{\Psi}{i},\, \irm{t}{i}}}
\end{align}
で定義する.時刻 $t$ における完全系 $\int_{\Gamma(E|_{\{t\} \times \Sigma})} [\dd{\varphi}] \ketbra{\varphi|_{\{t\} \times \Sigma}}{\varphi|_{\{t\} \times \Sigma}} = 1$ を適当に挿入することで
\begin{align}
    \label{eq:Green}
    G^{(n)} (x_1,\, \dots,\, x_n;\,\irm{\Psi}{f},\, \irm{t}{f};\, \irm{\Psi}{i},\, \irm{t}{i} ) 
    &\propto \int [\dd{\varphi}]\; 
    \irm{\Psi}{f}\bigl[\varphi|_{\{\irm{t}{f}\} \times \Sigma}\bigr]^* \irm{\Psi}{i}\bigl[\varphi|_{\{\irm{t}{i}\} \times \Sigma}\bigr] 
    \varphi(x_1) \cdots \varphi(x_n) \\
    &\qquad \exp \left( \iunit \int_{[\irm{t}{i},\, \irm{t}{f}] \times \Sigma} \dd[D+1]{x} \mathcal{L} \bigl( \varphi(x),\, \partial_\mu \varphi(x) \bigr)\right)  
\end{align}
と計算できる.ただし $\irm{\Psi}{i}\bigl[\varphi|_{\{\irm{t}{i}\} \times \Sigma}\bigr] \coloneqq \braket{\varphi|_{\{\irm{t}{i}\} \times \Sigma}}{\irm{\Psi}{i},\, \irm{t}{i}},\; \irm{\Psi}{f}\bigl[\varphi|_{\{\irm{t}{f}\} \times \Sigma}\bigr] \coloneqq \braket{\varphi|_{\{\irm{t}{f}\} \times \Sigma}}{\irm{\Psi}{f},\, \irm{t}{f}}$ とおいた.煩雑なので以降では $\irm{\Psi}{i},\, \irm{\Psi}{f}$ と略記する.

与えられた境界条件 $\irm{\Psi}{i} \in \Gamma(E|_{\{\irm{t}{i}\} \times \Sigma}),\; \irm{\Psi}{f} \in \Gamma(E|_{\{\irm{t}{f}\} \times \Sigma})$ に対して,\textbf{Green関数の生成汎函数}を
\begin{align}
    \label{def:gen-Green}
    \irm{Z}{fi}[J]
    \coloneqq \frac{\displaystyle\mel{\irm{\Psi}{f},\, \irm{t}{f}}{\mathcal{T} \exp \left( \iunit\int_{[\irm{t}{i},\, \irm{t}{f}] \times \Sigma} \dd[D+1]{x} J(x) \hat{\varphi}(x) \right)}{\irm{\Psi}{i},\, \irm{t}{i}}}{\braket{\irm{\Psi}{f},\, \irm{t}{f}}{\irm{\Psi}{i},\, \irm{t}{i}}}
\end{align}
と定める.実際,
\begin{align}
    \eval{\frac{\delta^n \irm{Z}{fi}[J]}{\delta J(x_1) \cdots \delta J(x_n)}}_{J=0} = \iunit^n G^{(n)} (x_1,\, \dots,\, x_n;\,\irm{\Psi}{f},\, \irm{t}{f};\, \irm{\Psi}{i},\, \irm{t}{i} ) 
\end{align}
が成り立つので
\begin{align}
    \irm{Z}{fi}[J]
    &= \sum_{n=0}^\infty \frac{\iunit^n}{n!} \int_{([\irm{t}{i},\, \irm{t}{f}] \times \Sigma)^n} \dd[D+1]{x_1} \cdots \dd[D+1]{x_n} J(x_1) \cdots J(x_n) G^{(n)} (x_1,\, \dots,\, x_n;\,\irm{\Psi}{f},\, \irm{t}{f};\, \irm{\Psi}{i},\, \irm{t}{i} ) 
\end{align}
だと分かる.\eqref{eq:Green}より
\begin{align}
    \label{eq:gen-Green}
    \irm{Z}{fi}[J] = \frac{\displaystyle \int [\dd{\varphi}] \irm{\Psi}{f}^* \irm{\Psi}{i} \exp \left( \iunit \int_{[\irm{t}{i},\, \irm{t}{f}] \times \Sigma} \dd[D+1]{x} \bigl(\mathcal{L}(x) + J(x) \phi(x) \bigr) \right) }{\displaystyle \int_{[\irm{t}{i},\, \irm{t}{f}] \times \Sigma} [\dd{\varphi}] \irm{\Psi}{f}^* \irm{\Psi}{i} \exp \left( \iunit \int \dd[D+1]{x} \mathcal{L}(x) \right)}
\end{align}
が成り立つ.

$\irm{t}{i} \to -\infty,\, \irm{t}{f} \to \infty$ の極限を上手くとることで,Green関数\eqref{def:Green}の境界条件への依存性を実質的に無くすことができる.
このような極限としてよく使われるものに,$\iunit \epsilon$ 処方がある:
簡単のため実スカラー場を考え,Lagrangian密度の質量項が $-\frac{1}{2} \mu^2 \varphi^2$ であるとする.このとき $\mu^2 \to \mu^2 - \iunit \epsilon$ と置き換えると,これは系のHamiltonianを $H \to H - \iunit \epsilon \int_\Sigma \dd[3]{x} \frac{\varphi(x)^2}{2} \eqqcolon H^{(\epsilon)}$ に置き換えることに相当する.このとき $\hat{H}$ の固有値,固有状態をそれぞれ $E_n,\, \ket{n}$ とおき,$\hat{H^{(\epsilon)}}$ の固有値,固有状態をそれぞれ $E^{(\epsilon)}_n,\, \ket{n}^{(\epsilon)}$ とおくと,
\begin{align}
    \ket{\irm{\Psi}{i},\, \irm{t}{i}}
    &= e^{\iunit \hat{H^{(\epsilon)}} \irm{t}{i}}\ket{\irm{\Psi}{i}} \\
    &= \sum_n  e^{\iunit E^{(\epsilon)} \irm{t}{i}} \ket{n}^{(\epsilon)} {}^{(\epsilon)}\braket{n}{\irm{\Psi}{i}} \\
    &\approx e^{\epsilon \irm{t}{i}\mel{0}{\int_\Sigma \dd[3]{x} \frac{\hat{\varphi}(x)^2}{2}}{0}}\left( \ket{0}^{(\epsilon)} {}^{(\epsilon)}\braket{0}{\irm{\Psi}{i}} + \sum_{n \neq 0} e^{\epsilon \irm{t}{i}(\mel{n}{\int_\Sigma \dd[3]{x} \frac{\hat{\varphi}(x)^2}{2}}{n} - \mel{0}{\int_\Sigma \dd[3]{x} \frac{\hat{\varphi}(x)^2}{2}}{0})} e^{\iunit E_n \irm{t}{i}} \ket{n}^{(\epsilon)} {}^{(\epsilon)}\braket{n}{\irm{\Psi}{i}} \right) \\
    &\xrightarrow{\substack{\irm{t}{i} \to -\infty, \\ \epsilon \to +0}} \ket{0}\braket{0}{\irm{\Psi}{i}}
\end{align}
同様に $\bra{\irm{\Psi}{f},\, \irm{t}{f}} \xrightarrow{\substack{\irm{t}{f} \to +\infty, \\ \epsilon \to +0}} \braket{\irm{\Psi}{f}}{0}\bra{0}$ もわかり,結局\eqref{def:Green}は
\begin{align}
    \label{def:Green-vac}
    \lim_{\substack{\irm{t}{i} \to -\infty, \\ \irm{t}{f} \to +\infty}} 
    G^{(n)} (x_1,\, \dots,\, x_n;\,\irm{\Psi}{f},\, \irm{t}{f};\, \irm{\Psi}{i},\, \irm{t}{i} )
    &= \mel{0}{\mathcal{T}\bigl\{ \hat{\varphi}(x_1) \cdots \hat{\varphi}(x_n) \bigr\}}{0}
    \eqqcolon G^{(n)} (x_1,\, \dots,\, x_n)
\end{align}
となる.このとき\eqref{eq:Green}を
\begin{align}
    \label{eq:Green-vac}
    G^{(n)} (x_1,\, \dots,\, x_n)
    &\propto \int [\dd{\varphi}]\; 
    \varphi(x_1) \cdots \varphi(x_n) \exp \left( \iunit \int_{\mathcal{M}} \dd[D+1]{x} \mathcal{L}(x)\right)  
\end{align}
と書く.








\end{document}