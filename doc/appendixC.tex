\documentclass[TQFT_main]{subfiles}

\begin{document}

% \setcounter{}{}

% \chapter{QFTミニマム}

% この章は\cite{Kugo1989qft1}による.自然単位系を用いる.

% \section{経路積分}

% 時空を表す多様体を $\mathcal{M}$ と書く.$\mathcal{M}$ は時間方向と空間方向に $\mathcal{M} = \mathbb{R} \times \Sigma$ と書けるとする.
% 場とはベクトル束 $V \hookrightarrow E \xrightarrow{\pi} \mathcal{M}$ の切断 $\varphi \in \Gamma(E)$ のこととする.

% 系のラグランジアン密度 $\mathcal{L} \bigl( \varphi(x),\, \partial_\mu \varphi(x) \bigr)$ を与える.
% \begin{itemize}
%     \item 時刻 $\irm{t}{i}$ における古典場の配位が $\irm{\Psi}{i} \in \Gamma(E|_{\{\irm{t}{i}\} \times \Sigma})$ であることに対応する量子状態\footnote{Heisenberg表示} $\ket{\irm{\Psi}{i},\, \irm{t}{i}}$
%     \item 時刻 $\irm{t}{f}$ における古典場の配位が $\irm{\Psi}{f} \in \Gamma(E|_{\{\irm{t}{f}\} \times \Sigma})$ であることに対応する量子状態 $\ket{\irm{\Psi}{f},\, \irm{t}{f}}$
% \end{itemize}
% の間に
% \begin{align}
%     \braket{\irm{\Psi}{f},\, \irm{t}{f}}{\irm{\Psi}{i},\, \irm{t}{i}}
%     \propto \int [\dd{\varphi}] \exp \left( \iunit \int_{[\irm{t}{i},\, \irm{t}{f}] \times \Sigma} \dd[D+1]{x} \mathcal{L} \bigl( \varphi(x),\, \partial_\mu \varphi(x) \bigr)\right) 
% \end{align}
% を要請するのが経路積分による場の量子化である.

% $\hat{\varphi}(x)\WHERE x \in \mathcal{M}$ を場の演算子とする.与えられた時空点 $x_1,\, \dots,\, x_n \in \mathcal{M}$ および境界条件 $\irm{\Psi}{i} \in \Gamma(E|_{\{\irm{t}{i}\} \times \Sigma}),\; \irm{\Psi}{f} \in \Gamma(E|_{\{\irm{t}{f}\} \times \Sigma})$ に対して,
% \textbf{Green関数}を
% \begin{align}
%     \label{def:Green}
%     G^{(n)} (x_1,\, \dots,\, x_n;\,\irm{\Psi}{f},\, \irm{t}{f};\, \irm{\Psi}{i},\, \irm{t}{i} )
%     &\coloneqq \frac{\mel{\irm{\Psi}{f},\, \irm{t}{f}}{\mathcal{T}\bigl\{ \hat{\varphi}(x_1) \cdots \hat{\varphi}(x_n) \bigr\}}{\irm{\Psi}{i},\, \irm{t}{i}}}{\braket{\irm{\Psi}{f},\, \irm{t}{f}}{\irm{\Psi}{i},\, \irm{t}{i}}}
% \end{align}
% で定義する.時刻 $t$ における完全系 $\int_{\Gamma(E|_{\{t\} \times \Sigma})} [\dd{\varphi}] \ketbra{\varphi|_{\{t\} \times \Sigma}}{\varphi|_{\{t\} \times \Sigma}} = 1$ を適当に挿入することで
% \begin{align}
%     \label{eq:Green}
%     G^{(n)} (x_1,\, \dots,\, x_n;\,\irm{\Psi}{f},\, \irm{t}{f};\, \irm{\Psi}{i},\, \irm{t}{i} ) 
%     &\propto \int [\dd{\varphi}]\; 
%     \irm{\Psi}{f}\bigl[\varphi|_{\{\irm{t}{f}\} \times \Sigma}\bigr]^* \irm{\Psi}{i}\bigl[\varphi|_{\{\irm{t}{i}\} \times \Sigma}\bigr] 
%     \varphi(x_1) \cdots \varphi(x_n) \\
%     &\qquad \exp \left( \iunit \int_{[\irm{t}{i},\, \irm{t}{f}] \times \Sigma} \dd[D+1]{x} \mathcal{L} \bigl( \varphi(x),\, \partial_\mu \varphi(x) \bigr)\right)  
% \end{align}
% と計算できる.ただし $\irm{\Psi}{i}\bigl[\varphi|_{\{\irm{t}{i}\} \times \Sigma}\bigr] \coloneqq \braket{\varphi|_{\{\irm{t}{i}\} \times \Sigma}}{\irm{\Psi}{i},\, \irm{t}{i}},\; \irm{\Psi}{f}\bigl[\varphi|_{\{\irm{t}{f}\} \times \Sigma}\bigr] \coloneqq \braket{\varphi|_{\{\irm{t}{f}\} \times \Sigma}}{\irm{\Psi}{f},\, \irm{t}{f}}$ とおいた.煩雑なので以降では $\irm{\Psi}{i},\, \irm{\Psi}{f}$ と略記する.

% 与えられた境界条件 $\irm{\Psi}{i} \in \Gamma(E|_{\{\irm{t}{i}\} \times \Sigma}),\; \irm{\Psi}{f} \in \Gamma(E|_{\{\irm{t}{f}\} \times \Sigma})$ に対して,\textbf{Green関数の生成汎函数}を
% \begin{align}
%     \label{def:gen-Green}
%     \irm{Z}{fi}[J]
%     \coloneqq \frac{\displaystyle\mel{\irm{\Psi}{f},\, \irm{t}{f}}{\mathcal{T} \exp \left( \iunit\int_{[\irm{t}{i},\, \irm{t}{f}] \times \Sigma} \dd[D+1]{x} J(x) \hat{\varphi}(x) \right)}{\irm{\Psi}{i},\, \irm{t}{i}}}{\braket{\irm{\Psi}{f},\, \irm{t}{f}}{\irm{\Psi}{i},\, \irm{t}{i}}}
% \end{align}
% と定める.実際,
% \begin{align}
%     \eval{\frac{\delta^n \irm{Z}{fi}[J]}{\delta J(x_1) \cdots \delta J(x_n)}}_{J=0} = \iunit^n G^{(n)} (x_1,\, \dots,\, x_n;\,\irm{\Psi}{f},\, \irm{t}{f};\, \irm{\Psi}{i},\, \irm{t}{i} ) 
% \end{align}
% が成り立つので
% \begin{align}
%     \irm{Z}{fi}[J]
%     &= \sum_{n=0}^\infty \frac{\iunit^n}{n!} \int_{([\irm{t}{i},\, \irm{t}{f}] \times \Sigma)^n} \dd[D+1]{x_1} \cdots \dd[D+1]{x_n} J(x_1) \cdots J(x_n) G^{(n)} (x_1,\, \dots,\, x_n;\,\irm{\Psi}{f},\, \irm{t}{f};\, \irm{\Psi}{i},\, \irm{t}{i} ) 
% \end{align}
% だと分かる.\eqref{eq:Green}より
% \begin{align}
%     \label{eq:gen-Green}
%     \irm{Z}{fi}[J] = \frac{\displaystyle \int [\dd{\varphi}] \irm{\Psi}{f}^* \irm{\Psi}{i} \exp \left( \iunit \int_{[\irm{t}{i},\, \irm{t}{f}] \times \Sigma} \dd[D+1]{x} \bigl(\mathcal{L}(x) + J(x) \phi(x) \bigr) \right) }{\displaystyle \int_{[\irm{t}{i},\, \irm{t}{f}] \times \Sigma} [\dd{\varphi}] \irm{\Psi}{f}^* \irm{\Psi}{i} \exp \left( \iunit \int \dd[D+1]{x} \mathcal{L}(x) \right)}
% \end{align}
% が成り立つ.

% $\irm{t}{i} \to -\infty,\, \irm{t}{f} \to \infty$ の極限を上手くとることで,Green関数\eqref{def:Green}の境界条件への依存性を実質的に無くすことができる.
% このような極限としてよく使われるものに,$\iunit \epsilon$ 処方がある:
% 簡単のため実スカラー場を考え,Lagrangian密度の質量項が $-\frac{1}{2} \mu^2 \varphi^2$ であるとする.このとき $\mu^2 \to \mu^2 - \iunit \epsilon$ と置き換えると,これは系のHamiltonianを $H \to H - \iunit \epsilon \int_\Sigma \dd[3]{x} \frac{\varphi(x)^2}{2} \eqqcolon H^{(\epsilon)}$ に置き換えることに相当する.このとき $\hat{H}$ の固有値,固有状態をそれぞれ $E_n,\, \ket{n}$ とおき,$\hat{H^{(\epsilon)}}$ の固有値,固有状態をそれぞれ $E^{(\epsilon)}_n,\, \ket{n}^{(\epsilon)}$ とおくと,
% \begin{align}
%     \ket{\irm{\Psi}{i},\, \irm{t}{i}}
%     &= e^{\iunit \hat{H^{(\epsilon)}} \irm{t}{i}}\ket{\irm{\Psi}{i}} \\
%     &= \sum_n  e^{\iunit E^{(\epsilon)} \irm{t}{i}} \ket{n}^{(\epsilon)} {}^{(\epsilon)}\braket{n}{\irm{\Psi}{i}} \\
%     &\approx e^{\epsilon \irm{t}{i}\mel{0}{\int_\Sigma \dd[3]{x} \frac{\hat{\varphi}(x)^2}{2}}{0}}\left( \ket{0}^{(\epsilon)} {}^{(\epsilon)}\braket{0}{\irm{\Psi}{i}} + \sum_{n \neq 0} e^{\epsilon \irm{t}{i}(\mel{n}{\int_\Sigma \dd[3]{x} \frac{\hat{\varphi}(x)^2}{2}}{n} - \mel{0}{\int_\Sigma \dd[3]{x} \frac{\hat{\varphi}(x)^2}{2}}{0})} e^{\iunit E_n \irm{t}{i}} \ket{n}^{(\epsilon)} {}^{(\epsilon)}\braket{n}{\irm{\Psi}{i}} \right) \\
%     &\xrightarrow{\substack{\irm{t}{i} \to -\infty, \\ \epsilon \to +0}} \ket{0}\braket{0}{\irm{\Psi}{i}}
% \end{align}
% 同様に $\bra{\irm{\Psi}{f},\, \irm{t}{f}} \xrightarrow{\substack{\irm{t}{f} \to +\infty, \\ \epsilon \to +0}} \braket{\irm{\Psi}{f}}{0}\bra{0}$ もわかり,結局\eqref{def:Green}は
% \begin{align}
%     \label{def:Green-vac}
%     \lim_{\substack{\irm{t}{i} \to -\infty, \\ \irm{t}{f} \to +\infty}} 
%     G^{(n)} (x_1,\, \dots,\, x_n;\,\irm{\Psi}{f},\, \irm{t}{f};\, \irm{\Psi}{i},\, \irm{t}{i} )
%     &= \mel{0}{\mathcal{T}\bigl\{ \hat{\varphi}(x_1) \cdots \hat{\varphi}(x_n) \bigr\}}{0}
%     \eqqcolon G^{(n)} (x_1,\, \dots,\, x_n)
% \end{align}
% となる.このとき\eqref{eq:Green}を
% \begin{align}
%     \label{eq:Green-vac}
%     G^{(n)} (x_1,\, \dots,\, x_n)
%     &\propto \int [\dd{\varphi}]\; 
%     \varphi(x_1) \cdots \varphi(x_n) \exp \left( \iunit \int_{\mathcal{M}} \dd[D+1]{x} \mathcal{L}(x)\right)  
% \end{align}
% と書く.

\chapter{$\infty$-圏}

この付録では,~\cite{lurie2008higher}, ~\cite{kerodon}, ~\cite{Land2021infinity}, ~\cite{alfonsi2023higher}に従って $(\infty,\, 1)$-圏\footnote{~\cite{kerodon}は創始者本人によって運営されているwebサイトのようだ.}を導入する.
さらに,~\cite{barwick2020unicityhomotopytheoryhigher}, ~\cite{ayala2020factorizationhomologyihigher}をベースに $(\infty,\, n)$-圏の構成を試みる.

\section{圏論の復習}

\subsection{圏と関手}

\begin{mydef}[label=def:category, breakable]{圏}
	\textbf{圏} (category) $\Cat{C}$ とは,以下の4種類のデータからなる:
	\begin{itemize}
		\item \textbf{対象} (object) と呼ばれる要素の集まり\footnote{$\Obj{\Cat{C}}$ は,集合論では扱えないほど大きなものになっても良い.}
		\begin{align}
			\bm{\Obj{\Cat{C}}}
		\end{align}
		
		\item $\forall A,\, B \in \Obj{\Cat{C}}$ に対して,$A$ から $B$ への\textbf{射} (morphism) と呼ばれる要素の\underline{集合}
		\begin{align}
			\bm{\Hom{\Cat{C}}(A,\, B)}
		\end{align}
		
		\item $\forall A \in \Obj{\Cat{C}}$ に対して,$A$ 上の\textbf{恒等射} (identity morphism) と呼ばれる射
		\begin{align}
			\bm{\mathrm{Id}_A} \in \Hom{\Cat{C}}(A,\, A)
		\end{align}
		
		\item $\forall A,\, B,\, C \in \Obj{\Cat{C}}$ と $\forall f \in \Hom{\Cat{C}}(A,\, B),\, \forall g \in \Hom{\Cat{C}}(B,\, C)$ に対して,$f$ と $g$ の\textbf{合成} (composite) と呼ばれる射 $\bm{g \circ f} \in \Hom{\Cat{C}}(A,\, C)$ を対応させる集合の写像
		\begin{align}
			\bm{\circ} \colon \Hom{\Cat{C}}(A,\, B) \times \Hom{\Cat{C}}(B,\, C) \lto \Hom{\Cat{C}}(A,\, C),\; (f,\, g) \lmto g\circ f
		\end{align}
	\end{itemize}
	これらの構成要素は,次の2条件を満たさねばならない:
	\begin{enumerate}
		\item \textbf{(unitality)}:任意の射 $f \colon A \lto B$ に対して
		\begin{align}
			f \circ \mathrm{Id}_A = f,\quad \mathrm{Id}_B \circ f = f
		\end{align}
		が成り立つ.
		\item \textbf{(associativity)}:任意の射 $f \colon A \lto B,\; g \colon B \lto C,\; h \colon C \lto D$ に対して
		\begin{align}
			h \circ (g \circ f) = (h \circ g) \circ f
		\end{align}
		が成り立つ.
	\end{enumerate}
\end{mydef}

\begin{mydef}[label=def:iso]{モノ・エピ・同型射}
	\hyperref[def:category]{圏} $\Cat{C}$ を与える.
	\begin{itemize}
        \item 射 $f \colon A \lto B$ が\textbf{モノ射} (monomorphism) であるとは,$\forall X \in \Obj{\Cat{C}}$ に対して写像
        \begin{align}
            f_* \colon \Hom{\Cat{C}} (X,\, A) &\lto \Hom{\Cat{C}} (X,\, B), \\
            g \lmto f \circ g
        \end{align}
        が集合の写像として単射であること.
        \item 射 $f \colon A \lto B$ が\textbf{エピ射} (epimorphism) であるとは,$\forall  X \in \Obj{\Cat{C}}$ に対して写像
        \begin{align}
            f^* \colon \Hom{\Cat{C}} (B,\, X) &\lto \Hom{\Cat{C}} (A,\, X), \\
            g \lmto g \circ f
        \end{align}
        が集合の写像として単射であること.
		\item 射 $f \colon A \lto B$ が\textbf{同型射} (isomorphism) であるとは,射 $g \colon B \lto A$ が存在して
		$ g \circ f = \mathrm{Id}_A \AND f \circ g = \mathrm{Id}_B$ を充たすこと.
		このとき $f$ と $g$ は互いの\textbf{逆射} (inverse) であると言い,$g = \bm{f^{-1}},\; f = \bm{g}^{-1}$ と書く\footnote{逆射は存在すれば一意である.}.
		\item $A,\, B \in \Obj{\Cat{C}}$ の間に同型射が存在するとき,対象 $A$ と $B$ は\textbf{同型} (isomorphic) であると言い,$\bm{A\cong B}$ と書く.
	\end{itemize}
\end{mydef}

\begin{mydef}[label=def:functor]{関手}
    \hyperref[def:category]{圏} $\Cat{C},\; \Cat{D}$ を与える.
    圏 $\Cat{C}$ から圏 $\Cat{D}$ への\textbf{関手} $F$ とは,以下の2つの対応からなる:
    \begin{itemize}
        \item 圏 $\Cat{C}$ における任意の対象 $X \in \Obj{\Cat{C}}$ に対して,圏 $\Cat{D}$ における対象 $F(X) \in \Obj{\Cat{D}}$ を対応づける
        \item 圏 $\Cat{C}$ における任意の射 $f \colon X \lto Y$ に対して,圏 $\Cat{D}$ における射 $F(f) \colon F(X) \lto F(Y)$ を対応づける
    \end{itemize}
    これらの対応は以下の条件を充たさねばならない:
    \begin{description}
        \item[\textbf{(fun-1)}]  圏 $\Cat{C}$ における任意の射 $f \colon X \lto Y,\; g \colon Y \lto Z$ に対して,
        \begin{align}
            F(g \circ f) \lto F(g) \circ F(f)
        \end{align}
        \item[\textbf{(fun-2)}]  圏 $\Cat{C}$ における任意の対象 $X \in \Obj{\Cat{C}}$ に対して,
        \begin{align}
            F (\Id_{X}) = \Id_{F(X)}
        \end{align}
    \end{description}
    
    \tcblower

    文脈上明らかなときは,圏 $\Cat{C}$ から圏 $\Cat{D}$ への関手 $F$ のことを関手 $\bm{F \colon \Cat{C} \lto \Cat{D}}$ と略記する.
\end{mydef}

\begin{mydef}[label=def:faithful]{忠実・充満・本質的全射}
    \hyperref[def:functor]{関手} $F \colon \Cat{C} \longrightarrow \Cat{D}$ を与える.
    \begin{itemize}
        \item $F$ が\textbf{忠実} (faithful) であるとは, $\forall X,\,Y \in \Obj{\Cat{C}}$ に対して写像
        \begin{align}
            F_{X,\, Y} \colon \Hom{\mathcal{C}}(X,\, Y) \longrightarrow \Hom{\mathcal{D}} \bigl( F(X),\, F(Y) \bigr),\; f \longmapsto F(f)
        \end{align}
        が単射であること.
        \item $F$ が\textbf{充満} (full) であるとは, $\forall X,\,Y \in \Obj{\Cat{C}}$ に対して写像
        \begin{align}
            F_{X,\, Y} \colon \Hom{\mathcal{C}}(X,\, Y) \longrightarrow \Hom{\mathcal{D}} \bigl( F(X),\, F(Y) \bigr),\; f \longmapsto F(f)
        \end{align}
        が全射であること.
        \item $F$ が\textbf{本質的全射} (essentially surjective) であるとは, $\forall Z \in \Obj{\Cat{D}}$ に対して $X \in \Obj{\Cat{C}}$ が存在して $F(X)$ が $Y$ と同型になること.
    \end{itemize}
    忠実充満関手のことを\textbf{埋め込み}と呼ぶ.
\end{mydef}

\begin{mydef}[label=def:nat]{自然変換}
    2つの\hyperref[def:functor]{関手} $F,\, G \colon \Cat{C} \lto \Cat{D}$ を与える.
    $F,\, G$ の間の\textbf{自然変換} (natural transformation) $\bm{\tau \colon F \Longrightarrow G}$ とは,以下の対応からなる:
    \begin{itemize}
        \item 圏 $\Cat{C}$ における任意の対象 $X \in \Obj{\Cat{C}}$ に対して,圏 $\Cat{D}$ における射 $\tau_X \colon F(X) \lto G(X)$ を対応づける
    \end{itemize}
    この対応は以下の条件を充たさねばならない:
    \begin{description}
        \item[\textbf{(nat)}]  
        圏 $\Cat{C}$ における任意の射 $f \colon X \lto Y$ に対して,以下の図式を可換にする:
        \begin{center}
            \begin{tikzcd}[row sep=large, column sep=large]
                &F(X) \ar[d, red, "\tau_X"]\ar[r, "F(f)"] &F(Y) \ar[d, red, "\tau_Y"] \\
                &G(X) \ar[r, "G(f)"] &G(Y)
            \end{tikzcd}
        \end{center}
    \end{description}
    
    \tcblower 

    自然変換 $\tau \colon F \Longrightarrow G$ であって,$\forall X \in \Obj{\Cat{C}}$ に対して射 $\tau_X \colon F(X) \lto G(X)$ が\hyperref[def:iso]{同型射}であるもののことを\textbf{自然同値} (natural equivalence)\footnote{\textbf{自然同型} (natural isomorphism) と言うこともある.} と呼ぶ.
\end{mydef}

自然変換 $\tau \colon F \Longrightarrow G$ を
\begin{center}
    \begin{tikzcd}[row sep=large, column sep=large]
        \Cat{C} \ar[bend left=50,r, "F"{name=U, above}] \ar[bend right=50,r, "G"{name=D, below}] &\Cat{D}
        \ar[Rightarrow, from=U, to=D, "\tau"]
    \end{tikzcd}
\end{center}
と書くことがある.    

\subsection{極限と余極限}

\begin{mydef}[label=def:diagram]{図式}
    圏 $\Cat{C}$ と小圏 $I$(\textbf{添字圏}と呼ばれる)を与える.
    
    $\Cat{C}$ における $\bm{I}$ \textbf{型の図式} (diagram of shape $I$) とは,\hyperref[def:functor]{関手}
    \begin{align}
        I \lto \Cat{C}
    \end{align}
    のこと.
\end{mydef}

\begin{mydef}[label=def:Cone, breakable]{錐の圏}
	$D \colon \Cat{I} \lto \Cat{C}$ を\hyperref[def:diagram]{図式}とする.
	\begin{itemize}
		\item $D$ 上の\textbf{錐} (cone) とは,
		\begin{itemize}
			\item $\Cat{C}$ の対象 $\bm{C} \in \Obj{\Cat{C}}$
			\item $\Cat{C}$ の射の族 $\bm{c}_\bullet \coloneqq \Familyset[\big]{\bm{c}_i \in \Hom{\Cat{C}}\bigl(\bm{C},\, D(i)\bigr)}{i \in \Obj{\Cat{I}}}$
		\end{itemize}
		の組 $(\bm{C},\, \bm{c}_\bullet)$ であって,
		$\forall i,\, j \in \Obj{\Cat{I}}$ および $\forall f \in \Hom{\Cat{I}}(i,\, j)$ に対して
		\begin{align}
			\bm{c}_j = D(f) \circ  \bm{c}_i
		\end{align}
		を充たす,i.e. 以下の図式を可換にするもののこと.
		\begin{center}
			\begin{tikzcd}
				& &\bm{C} \ar[dl, bend right, "\bm{c}_i"']\ar[dr, bend left, "\bm{c}_j"] & \\
				&D(i) \ar[rr, "D(f)"] & &D(j)
			\end{tikzcd}
		\end{center}
		\item \textbf{錐の射} (morphism of cones) 
		\begin{center}
			\begin{tikzcd}
				&(\bm{C},\, \bm{c}_\bullet) \ar[r, red, "u"] &(\bm{C'},\, \bm{c'}_\bullet)
			\end{tikzcd}
		\end{center}
		とは,$\Cat{C}$ の射 $\textcolor{red}{u} \in \Hom{\Cat{C}} (\bm{C},\, \bm{C'})$ であって,$\forall i \in \Obj{\Cat{I}}$ に対して
		\begin{align}
			\bm{c}_i = \bm{c'}_i \circ \textcolor{red}{u}
		\end{align}
		を充たす,i.e. 以下の図式を可換にするもののこと.
		\begin{center}
			\begin{tikzcd}
				& &\bm{C} \ar[ddl, bend right, "\bm{c}_i"']\ar[d, red, "u"]\\
				& &\bm{C'}\ar[dl, bend right, "\bm{c'}_i"] \\
				&D(i) &
			\end{tikzcd}
		\end{center}
	\end{itemize}
	$D$ 上の錐と錐の射を全て集めたものは圏 $\CONE{D}$ を成す.
\end{mydef}

\begin{mydef}[label=def:lim, breakable]{極限}
	\hyperref[def:diagram]{図式} $D \colon \Cat{I} \lto \Cat{C}$ の\textbf{極限} (limit)\footnote{\textbf{普遍錐} (universal cone) とも言う.}とは,
	圏 $\CONE{D}$ の終対象のこと.記号として $\bigl(\bm{\lim_{\Cat{I}} D},\, \bm{p}_\bullet\bigr)$ と書く\footnote{$\bm{\varprojlim D}$ と書くこともある.}.
	i.e. 極限 $\bigl(\bm{\lim_{\Cat{I}} D},\, \bm{p}_\bullet\bigr) \in \Obj{\CONE{D}}$ は,以下の普遍性を充たす:
	\begin{description}
		\item[\textbf{(極限の普遍性)}]  
		
		$\forall (\textcolor{blue}{\bm{C}},\, \textcolor{blue}{\bm{c}}_\bullet) \in \Obj{\CONE{D}}$ に対して,\hyperref[def:Cone]{錐の射} $\textcolor{red}{u} \in \Hom{\CONE{D}} \bigl((\textcolor{blue}{\bm{C}},\, \textcolor{blue}{\bm{c}}_\bullet),\, \bigl(\bm{\lim_{\Cat{I}} D},\, \bm{p}_\bullet\bigr) \bigr)$ が一意的に存在して,$\forall i,\, j \in \Obj{\Cat{I}}$ および $\forall f\in \Hom{\Cat{I}}(i,\, j)$ に対して図式を可換にする.
		\begin{figure}[H]
			\centering
			\begin{tikzcd}[row sep=large, column sep=large]
				& &\forall \textcolor{blue}{\bm{C}} \ar[ddl, bend right, blue, "\bm{c}_i"']\ar[ddr, bend left, blue, "\bm{c}_j"]\ar[d, red, dashed, "\exists! u"] &\\
				& &\bm{\lim_{\Cat{I}} D} \ar[dl, bend right, "\bm{p}_i"]\ar[dr, bend left, "\bm{p}_j"'] & \\
				&D(i) \ar[rr, "D(f)"]& &D(j)
			\end{tikzcd}
			\caption{極限の普遍性}
			\label{cmtd:lim}
		\end{figure}%
	\end{description}
	
\end{mydef}

\begin{mydef}[label=def:coCone, breakable]{余錐の圏}
	$D \colon \Cat{I} \lto \Cat{C}$ を\hyperref[def:diagram]{図式}とする.
	\begin{itemize}
		\item $D$ 上の\textbf{余錐} (cocone) とは,
		\begin{itemize}
			\item $\Cat{C}$ の対象 $\bm{C} \in \Obj{\Cat{C}}$
			\item $\Cat{C}$ の射の族 $\bm{c}_\bullet \coloneqq \Familyset[\big]{\bm{c}_i \in \Hom{\Cat{C}}\bigl(D(i),\, \bm{C}\bigr)}{i \in \Obj{\Cat{I}}}$
		\end{itemize}
		の組 $(\bm{C},\, \bm{c}_\bullet)$ であって,
		$\forall i,\, j \in \Obj{\Cat{I}}$ および $\forall f \in \Hom{\Cat{I}}(i,\, j)$ に対して
		\begin{align}
			\bm{c}_i = D(f) \circ \bm{c}_j
		\end{align}
		を充たす,i.e. 以下の図式を可換にするもののこと.
		\begin{center}
			\begin{tikzcd}
				&D(i) \ar[rr, "D(f)"]\ar[dr, bend right, "\bm{c}_i"'] & &D(j) \ar[dl, bend left, "\bm{c}_j"] \\
                & &\bm{C} &
			\end{tikzcd}
		\end{center}
		\item \textbf{余錐の射} (morphism of cocones) 
		\begin{center}
			\begin{tikzcd}
				&(\bm{C},\, \bm{c}_\bullet) \ar[r, red, "u"] &(\bm{C'},\, \bm{c'}_\bullet)
			\end{tikzcd}
		\end{center}
		とは,$\Cat{C}$ の射 $\textcolor{red}{u} \in \Hom{\Cat{C}} (\bm{C},\, \bm{C'})$ であって,$\forall i \in \Obj{\Cat{I}}$ に対して
		\begin{align}
			\bm{c'}_i = \textcolor{red}{u} \circ \bm{c}_i
		\end{align}
		を充たす,i.e. 以下の図式を可換にするもののこと.
		\begin{center}
			\begin{tikzcd}
                &D(i) \ar[dr, bend right, "\bm{c}_i"]\ar[ddr, bend right, "\bm{c}_j"'] & \\
				& &\bm{C} \ar[d, red, "u"]\\
				& &\bm{C'}
			\end{tikzcd}
		\end{center}
	\end{itemize}
	$D$ 上の余錐と余錐の射を全て集めたものは圏 $\CONE{D}$ を成す.
\end{mydef}

\begin{mydef}[label=def:colim, breakable]{余極限}
	\hyperref[def:diagram]{図式} $D \colon \Cat{I} \lto \Cat{C}$ の\textbf{余極限} (colimit)\footnote{\textbf{普遍余錐} (universal cocone) とも言う.}とは,
	圏 $\coCONE{D}$ の始対象のこと.記号として $\bigl(\underset{\Cat{I}}{\bm{\colim}} D,\, \bm{p}_\bullet\bigr)$ と書く\footnote{$\bm{\varinjlim D}$ と書くこともある.}.
	i.e. 余極限 $\bigl(\bm{\colim_{\Cat{I}} D},\, \bm{p}_\bullet\bigr) \in \Obj{\coCONE{D}}$ は,以下の普遍性を充たす:
	\begin{description}
		\item[\textbf{(余極限の普遍性)}]  
		
		$\forall (\textcolor{blue}{\bm{C}},\, \textcolor{blue}{\bm{c}}_\bullet) \in \Obj{\coCONE{D}}$ に対して,\hyperref[def:coCone]{余錐の射} $\textcolor{red}{u} \in \Hom{\coCONE{D}} \bigl(\bigl(\bm{\colim_{\Cat{I}} D},\, \bm{p}_\bullet\bigr),\, (\textcolor{blue}{\bm{C}},\, \textcolor{blue}{\bm{c}}_\bullet) \bigr)$ が一意的に存在して,$\forall i,\, j \in \Obj{\Cat{I}}$ および $\forall f\in \Hom{\Cat{I}}(i,\, j)$ に対して図式を可換にする.
		\begin{figure}[H]
			\centering
			\begin{tikzcd}[row sep=large, column sep=large]
				&D(i) \ar[rr, "D(f)"]\ar[dr, bend right, "\bm{p}_i"]\ar[ddr, bend right, blue, "\bm{c}_i"']& &D(j) \ar[dl, bend left, "\bm{p}_i"']\ar[ddl, bend left, blue, "\bm{c}_i"] \\
				& &\bm{\colim_{\Cat{I}} D} \ar[d, red, dashed, "\exists ! u"] & \\
                & &\forall \textcolor{blue}{\bm{C}} &
			\end{tikzcd}
			\caption{余極限の普遍性}
			\label{cmtd:colim}
		\end{figure}%
	\end{description}
	
\end{mydef}

\begin{myexample}[label=def:product-coproduct]{積と和}
    \hyperref[def:diagram]{図式}
    \begin{align}
        D \colon \boxdiagram{
            \overset{1}{\bullet} \&\overset{2}{\bullet} 
        } \lto \Cat{C}
    \end{align}
    の\hyperref[def:lim]{極限}を(存在すれば)\textbf{積} (product) と呼び,$\bm{D(1) \times D(2)}$ と書く.
    同じ図式の\hyperref[def:colim]{余極限}を(存在すれば)\textbf{和}\footnote{\textbf{余積}と言うこともある.} (coproduct) と呼び,$\bm{D(1) \amalg D(2)}$ と書く.

     より具体的には,圏 $\Cat{C}$ における2つの対象 $D(1),\, D(2) \in \Obj{\Cat{C}}$ の\textbf{積}とは,
    \begin{itemize}
        \item 圏 $\Cat{C}$ における1つの対象 $D(1) \times D(2) \in \Obj{\Cat{C}}$
        \item 圏 $\Cat{C}$ における2つの射 $p_i \in \Hom{\Cat{C}} \bigl( D(1) \times D(2),\, D(i) \bigr) \WHERE i=1,\, 2$
    \end{itemize}
    の組であって,任意の組 $\Bigl( \textcolor{blue}{C} \in \Obj{\Cat{C}},\, \Familyset[big]{\textcolor{blue}{c_i} \in \Hom{\Cat{C}} \bigl( \textcolor{blue}{C},\, D(i) \bigr)}{i \in \{1,\, 2\}} \Bigr)$ に対して以下の図式を可換にする圏 $\Cat{C}$ の射 $\textcolor{red}{u} \in \Hom{\Cat{C}} \bigl(\textcolor{blue}{C},\, D(1) \times D(2) \bigr)$ が\underline{一意的}に存在するようなもののこと:
    % https://tikzcd.yichuanshen.de/#N4Igdg9gJgpgziAXAbVABwnAlgFyxMJZABgBoBGAXVJADcBDAGwFcYkQARACnIEoQAvqXSZc+QigBMFanSat23SfyEjseAkXKlishizaJOPXgB1TeALbwABEpXCQGdeK2lJe+YZDmcMAB44AMYQjBAATsAARgYCwADCAoKyMFAA5vBEoABm4RCWSADMNDgQSGRyBuxBAPrkIDRRMGBQRcSqILn5RSVliNqVCka1kg0gjPRNjAAKohoSIOFYaQAWOGNNLUgAtIXtjl0FiNIgpeU0+kNOdWMTU7MumkZLq+uNza2Iu-s5eUcnZ36Fy87DQNVG7y2iD2HUOPVOfROjCwYG8UHocBWqTGl285gCWDgODgAEIbMxkgIgA
    \begin{center}
        \begin{tikzcd}
            & D(1)\times D(2) \arrow[ld, "p_1"', bend right] \arrow[rd, "p_2", bend left]                                      &      \\
        D(1) &                                                                                                                  & D(2) \\
            & \textcolor{blue}{C} \arrow[lu, "c_1", blue,bend left] \arrow[ru, "c_2"', blue,bend right] \arrow[uu, "\exists! u", red, dashed] &     
        \end{tikzcd}
    \end{center}
    具体的な圏における積は,例えば以下の通りである:
    \begin{enumerate}
        \item 集合と写像の圏 $\SETS$ における積とは,直積集合と,直積因子への射影の組のこと.
        \item 位相空間の圏 $\TOP$ における積とは,直積位相空間と,直積因子への連続な\footnote{むしろ,直積位相とは射影という写像が連続になるような最弱の位相のことである.}射影の組みのこと.
        \item 体 $\mathbb{K}$ 上のベクトル空間の圏 $\SETS$ における積とは,直積ベクトル空間と射影\footnote{定義から線型写像になるため,圏 $\VEC{\mathbb{K}}$ の射である.}の組みのこと.これは,特に元の図式の対象が有限個である場合は直和ベクトル空間と同型である.
    \end{enumerate}
    
     同様に,圏 $\Cat{C}$ における2つの対象 $D(1),\, D(2) \in \Obj{\Cat{C}}$ の\textbf{和}とは,
    \begin{itemize}
        \item 圏 $\Cat{C}$ における1つの対象 $D(1) \amalg D(2) \in \Obj{\Cat{C}}$
        \item 圏 $\Cat{C}$ における2つの射 $i_j \in \Hom{\Cat{C}} \bigl( D(1) \amalg D(2),\, D(j) \bigr) \WHERE j=1,\, 2$
    \end{itemize}
    の組であって,任意の組 $\Bigl( \textcolor{blue}{C} \in \Obj{\Cat{C}},\, \Familyset[big]{\textcolor{blue}{c_i} \in \Hom{\Cat{C}} \bigl( D(i),\, \textcolor{blue}{C} \bigr)}{i \in \{1,\, 2\}} \Bigr)$ に対して以下の図式を可換にする圏 $\Cat{C}$ の射 $\textcolor{red}{u} \in \Hom{\Cat{C}} \bigl(D(1)\amalg D(2),\, \textcolor{blue}{C} \bigr)$ が\underline{一意的}に存在するようなもののこと:
    % https://tikzcd.yichuanshen.de/#N4Igdg9gJgpgziAXAbVABwnAlgFyxMJZARgBoAGAXVJADcBDAGwFcYkQAdDnGADxwDGERhABOwAEYsYAX2ABhGSBml0mXPkIpypYtTpNW7ACIAKYgEplqkBmx4CRAEy79DaSdNOrKtfc1EZE5uhmyIIGaWXPQAtkwA5gAEZt7K+jBQ8fBEoABmohAxSGQgOBBIOgYe4QIA+sQgNIz0EjCMAArqDlogjDC5OI0grWBQSADMlXAAFlgDFb4g+YVILqXliJXuRjW1TkPNrR1dAeGiWPHTgzQjY4gAtJOLy0WIJWUTNNthIFj1By02p1-I4zhcrkNbkhHuRngVXmsPohxl9Quw-vsmoDjiCen15jcYKMJrCbC9PusKqjqpwOHwsHAcHAAISJZgAo7AjSg3r9a69LBgH5QegzDJpGRAA
    \begin{center}
        \begin{tikzcd}
            & \textcolor{blue}{C}                              &                                                                  \\
        D(1) \arrow[ru, blue,"c_1", bend left] \arrow[rd, "i_1"', bend right] &                                                  & D(2) \arrow[lu, blue,"c_2"', bend right] \arrow[ld, "i_2", bend left] \\
            & D(1)\amalg D(2) \arrow[uu, red,"\exists! u", dashed] &                                                                 
        \end{tikzcd}
    \end{center}
    具体的な圏における和は,例えば以下の通りである:
    \begin{enumerate}
        \item 集合と写像の圏 $\SETS$ における積とは,disjoint unionと,disjoint unionの各成分への包含写像の組のこと.
        \item 位相空間の圏 $\TOP$ における積とは,disjoint union
        \item 体 $\mathbb{K}$ 上のベクトル空間の圏 $\SETS$ における積とは,直積ベクトル空間のこと.これは,特に元の図式の対象が有限個である場合は直和ベクトル空間と同型である.
    \end{enumerate}
\end{myexample}

\begin{myexample}[label=def:eq-coeq]{イコライザとコイコライザ}
    \hyperref[def:diagram]{図式}
    \begin{align}
        D \colon \boxdiagram{
            \overset{1}{\bullet} \ar[r,shift left,"f"]\ar[r,shift right,"g"'] \&\overset{2}{\bullet}
        } \lto \Cat{C}
    \end{align}
    の\hyperref[def:lim]{極限}を(存在すれば)\textbf{イコライザ} (equalizer) と呼び,$\bm{\mathrm{Eq}\bigl(D(1),\, D(2)\bigr)}$ と書く.
    同じ図式の\hyperref[def:colim]{余極限}を(存在すれば)\textbf{コイコライザ} (coequalizer) と呼び,$\bm{\mathrm{Coeq}\bigl(D(1),\, D(2)\bigr)}$ と書く.
    具体的な圏における例は以下の通り:
    \begin{enumerate}
        \item 集合と写像の圏 $\SETS$ におけるイコライザとは,2つの写像 $f,\, g \in \Hom{\SETS} (X,\, Y)$ によって定まる $X$ の部分集合
        \begin{align}
            \mathrm{Eq}(f,\, g) \coloneqq \bigl\{\, x \in X \bigm| f(x) = g(x) \,\bigr\} 
        \end{align}
        と,包含写像 $i \colon \mathrm{Eq}(f,\, g) \lto X$ の組みである.直観的には方程式 $f(x) = g(x)$ の解空間のことである.
        \item 集合と写像の圏 $\SETS$ における2つの写像 $f,\, g \in \Hom{\SETS} (X,\, Y)$ の間のコイコライザとは,$f(x) \sim g(x)\; \forall x \in X$ を充たす $Y$ の最小の同値関係 $\sim \subset Y \times Y$ による商集合 
        \begin{align}
            \mathrm{Coeq} (f,\, g) \coloneqq Y/{\sim}
        \end{align}
        および商写像 $q \colon Y \lto \mathrm{Coeq}(f,\, g)$ の組みである.直観的には,$\forall x \in X$ に対して方程式 $f(x) = g(x)$ が成立するように強引に $Y$ に同値関係を入れて得られる商集合ということになる.
        \item 位相空間の圏 $\TOP$ におけるイコライザとは $\SETS$ におけるイコライザ $\bigl(\mathrm{Eq}(f,\, g),\, i\bigr)$ に,$i$ が連続写像になるような最弱の位相を入れて得られる位相空間のこと.
        $\TOP$ におけるコイコライザとは,$\SETS$ におけるコイコライザ $\bigl(\mathrm{Coeq}(f,\, g),\, p\bigr)$ に,$p$ が連続写像になるような最強の位相を入れて得られる位相空間のこと.
        \item 体 $\mathbb{K}$ 上のベクトル空間の圏 $\VEC{\mathbb{K}}$ における,線型写像 $f \in \Hom{\VEC{\mathbb{K}}}(V,\, W)$ と零射 $0 \in \Hom{\VEC{\mathbb{K}}}(V,\, W)$ の間のイコライザとは,線型写像 $f$ の核 $\Ker f$ および包含準同型 $i \colon \Ker f \hookrightarrow V$ の組みのこと.
        $\VEC{\mathbb{K}}$ における,線型写像 $f \in \Hom{\VEC{\mathbb{K}}}(V,\, W)$ と零射 $0 \in \Hom{\VEC{\mathbb{K}}}(V,\, W)$ の間のコイコライザとは,線型写像 $f$ の余核 $\Coker f \coloneqq W / \Im f$ および標準的射影 $p \colon W \twoheadrightarrow \Coker f$ の組みのこと.
    \end{enumerate}
\end{myexample}

\begin{myexample}[label=def:pullback-pushout]{引き戻しと押し出し}
    \hyperref[def:diagram]{図式}
    \begin{align}
        D \colon \boxdiagram{
            \&\overset{2}{\bullet} \ar[d, "p"] \\
            \overset{1}{\bullet} \ar[r, "f"'] \&\overset{3}{\bullet} 
        } \lto \Cat{C}
    \end{align}
    の\hyperref[def:lim]{極限}を(存在すれば)\textbf{引き戻し}\footnote{\textbf{ファイバー積} (fiber product) と呼ぶこともある.} (pullback) と呼び,$\bm{D(1) \times_{D(3)} D(2)}$ と書く.
    
    \hyperref[def:diagram]{図式}
    \begin{align}
        D \colon \boxdiagram{
            \&\overset{2}{\bullet} \\
            \overset{1}{\bullet}  \ar[from=r, "f"'] \&\overset{3}{\bullet} \ar[u, "i"']
        } \lto \Cat{C}
    \end{align}
    の\hyperref[def:colim]{余極限}を(存在すれば)\textbf{押し出し}と呼び,$\bm{D(1) \amalg_{D(3)} D(2)}$ と書く.
\end{myexample}


\begin{mydef}[label=def:complete]{完備な圏}
    圏 $\Cat{C}$ が\textbf{完備}(resp. \textbf{余完備}) (complete resp. cocomplete) であるとは,$\Cat{C}$ における任意の\hyperref[def:diagram]{図式}が\hyperref[def:lim]{極限}(resp. \hyperref[def:colim]{余極限})を持つことを言う.完備かつ余完備な圏は\textbf{双完備} (bicomplete) であると言われる.
\end{mydef}

$\SETS$ は\hyperref[def:complete]{双完備}である.

\begin{myprop}[label=prop:lim-colim-basic,breakable]{極限とHomの交換}
    圏 $\Cat{C}$ の\hyperref[def:diagram]{図式} $D \colon I \lto \Cat{C}$ を与える.
    \begin{enumerate}
        \item 圏 $\Cat{C}$ は\hyperref[def:complete]{完備}であるとする.
        このとき $\forall X \in \Obj{\Cat{C}}$ に対して,
        集合 $\Hom{\Cat{C}} \bigl( X,\, \lim_I D \bigr) \in \Obj{\SETS}$ は $\SETS$ の図式
        \begin{align}
            \label{eq:diag-post}
            I \xrightarrow{D} \Cat{C} \xrightarrow{\Hom{\Cat{C}}(X,\, \mhyphen)} \SETS
        \end{align}
        の\hyperref[def:lim]{極限}である.
        i.e. 全単射
        \begin{align}
            \lim_{i \in I} \Hom{\Cat{C}} \bigl( X,\, D(i) \bigr) \cong \Hom{\Cat{C}} \bigl( X,\, \lim_I D \bigr) 
        \end{align}
        が存在する.
        \item  圏 $\Cat{C}$ は\hyperref[def:complete]{余完備}であるとする.
        このとき $\forall X \in \Obj{\Cat{C}}$ に対して,
        集合 $\Hom{\Cat{C}} \bigl( \colim_I D,\, X \bigr) \in \Obj{\SETS}$ は $\SETS$ の図式
        \begin{align}
            \label{eq:diag-pre}
            I \xrightarrow{D} \Cat{C} \xrightarrow{\Hom{\Cat{C}}(\mhyphen,\, X)} \SETS
        \end{align}
        の\hyperref[def:colim]{余極限}である.
        i.e. 全単射
        \begin{align}
            \lim_{i \in I} \Hom{\Cat{C}} \bigl( D(i),\, X\bigr) \cong \Hom{\Cat{C}} \bigl(\underset{I}{\colim},\, X \bigr) 
        \end{align}
        が存在する.
    \end{enumerate}
\end{myprop}

\begin{proof}
    \begin{enumerate}
        \item $\Cat{C}$ が完備なので,図式 $D \colon I \lto \Cat{C}$ の\hyperref[def:lim]{極限}
        \begin{center}
            \begin{tikzcd}[row sep=large, column sep=large]
                &&\lim_I D \ar[dl, bend right, "p_i"'] \ar[dr, bend left, "p_j"] & \\
                &D(i) \ar[rr, "D(f)"] & &D(j)
            \end{tikzcd}
        \end{center}
        が存在する\footnote{$i,\, j \in \Obj{I}$ および $f_{ij} \in \Hom{I}(i,\, j)$ は任意にとる.}.示すべきは $\SETS$ の図式
        \begin{center}
            \begin{tikzcd}[row sep=large, column sep=large]
                &&\Hom{\Cat{C}}(X,\, \lim_I D) \ar[dl, bend right, "p_i{}_*"'] \ar[dr, bend left, "p_j{}_*"] & \\
                &\Hom{\Cat{C}}\bigl(X,\, D(i)\bigr) \ar[rr, "D(f)_*"] & &\Hom{\Cat{C}}\bigl(X,\, D(j)\bigr)
            \end{tikzcd}
        \end{center}
        が\hyperref[cmtd:lim]{極限の普遍性}を充たすことである\footnote{$p_i{}_* \colon \Hom{\Cat{C}}(X,\, \lim_I D) \lto \Hom{\Cat{C}} \bigl( X,\, D(i) \bigr),\;  f \lmto p_i \circ f$ などと定義する.このように射に下付きの $*$ を書いた時はpost-composeを表す.上付きの $*$ はpre-composeである.}.

         $\SETS$ の図式\eqref{eq:diag-post}の\hyperref[def:Cone]{錐} $(\textcolor{blue}{Y,\, c_\bullet})$ を任意にとる.すると\hyperref[def:Cone]{錐の定義}および $(\mhyphen)_*$ の定義から,$\forall y \in \textcolor{blue}{Y}$ に対して以下の図式が可換になる:
        \begin{center}
            \begin{tikzcd}[row sep=large, column sep=large]
                &&X \ar[dl, blue, bend right, "c_i(y)"'] \ar[dr, blue, bend left, "c_j(y)"] & \\
                &D(i) \ar[rr, "D(f)"] & &D(j)
            \end{tikzcd}
        \end{center}
        i.e. 組 $\bigl(X,\, \textcolor{blue}{c}_\bullet(y)\bigr)$ は $\Cat{C}$ の図式 $D \colon I \lto \Cat{C}$ の\hyperref[def:Cone]{錐}であるから,\hyperref[def:Cone]{錐の射} $u_y \colon X \lto \lim_I D$ が一意的に存在する.
        ここで写像
        \begin{align}
            \textcolor{red}{u} \colon \textcolor{blue}{Y} \lto \Hom{\Cat{C}} (X,\, \lim_I D),\; y \lmto u_y
        \end{align}
        を考えると,これは $\forall y \in \textcolor{blue}{Y}$ に対して $p_\bullet{}_* \circ \textcolor{red}{u}(y) = p_\bullet \circ u_y = \textcolor{blue}{c}_\bullet (y)$ を充たす.i.e. $\SETS$ の図式
        \begin{center}
            \begin{tikzcd}[row sep=large, column sep=large]
				& &\forall \textcolor{blue}{\bm{Y}} \ar[ddl, bend right, blue, "\bm{c}_i"']\ar[ddr, bend left, blue, "\bm{c}_j"]\ar[d, red, "u"] &\\
				& &\Hom{\Cat{C}}(X,\, \lim_I D) \ar[dl, bend right, "\bm{p}_i{}_*"]\ar[dr, bend left, "\bm{p}_j{}_*"'] & \\
				&\Hom{\Cat{C}}\bigl(X,\, D(i)\bigr) \ar[rr, "D(f)_*"]& &\Hom{\Cat{C}}\bigl(X,\, D(j)\bigr)
			\end{tikzcd}
        \end{center}
        を可換にする.$u_y$ の定義からこのような $\textcolor{red}{u}$ は一意であるから,図式\eqref{eq:diag-post}の錐 $\bigl( \Hom{\Cat{C}}(X,\, \lim_I D),\, p_\bullet{}_* \bigr)$ が\hyperref[def:lim]{極限の普遍性}を充たすことが分かった.極限の一意性より
        \begin{align}
            \lim_{i \in I} \Hom{\Cat{C}} \bigl( X,\, D(i) \bigr) \cong \Hom{\Cat{C}} \bigl( X,\, \lim_I D \bigr) 
        \end{align}
        でなくてはいけない.

        \item 
        $\Cat{C}$ が余完備なので,図式 $D \colon I \lto \Cat{C}$ の\hyperref[def:colim]{余極限}
        \begin{center}
            \begin{tikzcd}[row sep=large, column sep=large]
                &D(i) \ar[rr, "D(f)"]\ar[dr, bend right, "p_i"']&&D(j)\ar[dl, bend left, "p_j"] \\
                & &\colim_I D &
            \end{tikzcd}
        \end{center}
        が存在する\footnote{$i,\, j \in \Obj{I}$ および $f_{ij} \in \Hom{I}(i,\, j)$ は任意にとる.}.示すべきは $\SETS$ の図式
        \begin{center}
            \begin{tikzcd}[row sep=large, column sep=large]
                &&\Hom{\Cat{C}}(\colim_I D,\, X) \ar[dl, bend right, "p_i{}^*"'] \ar[dr, bend left, "p_i{}^*"] & \\
                &\Hom{\Cat{C}}\bigl(D(j),\, X\bigr) \ar[rr, "D(f)^*"] & &\Hom{\Cat{C}}\bigl(D(i),\, X\bigr)
            \end{tikzcd}
        \end{center}
        が\hyperref[cmtd:lim]{極限の普遍性}を充たすことだが,以降の議論は (1) と同様である.

    \end{enumerate}
    
\end{proof}

\subsection{米田埋め込み}

\begin{mydef}[label=def:presheaf-general]{前層}
    圏 $\Cat{C}$ 上の圏 $\Cat{S}$ に値をとる\textbf{前層}とは,\hyperref[def:functor]{関手}
    \begin{align}
        P \colon \OP{\Cat{C}} \lto \Cat{S}
    \end{align}
    のこと.
\end{mydef}

\textbf{前層の圏} $\PSH{\Cat{C}}{\Cat{S}}$ \footnote{$\comm{\OP{\Cat{C}}}{\Cat{S}}$ や $\Cat{S}^{\OP{\Cat{C}}}$ と書くこともある.なお,\hyperref[def:presheaf]{付録A}で登場したものはこれの一例である.}とは,
\begin{itemize}
    \item \hyperref[def:presheaf-general]{前層} $P \colon \OP{\Cat{C}} \lto \Cat{S}$ を対象とする
    \item 前層 $P,\, Q \colon \OP{\Cat{C}} \lto \Cat{S}$ の間の\hyperref[def:nat]{自然変換} $ \tau \colon P \Longrightarrow Q$ を射とする
\end{itemize}
として構成される圏のこと\footnote{$\PSH{\Cat{C}}{\Cat{S}}$ の恒等射は $\forall X \in \Obj{\OP{\Cat{C}}}$ に対して $\Id_X \colon X \lto X$ を対応づける自然変換である.}.

\begin{mydef}[label=def:representable,breakable]{表現可能前層・米田埋め込み}
    圏 $\Cat{C}$ を与える.
    $\forall X \in \Obj{\Cat{C}}$ に対して,以下で定義する\hyperref[def:presheaf-general]{前層}
        \begin{align}
            \bm{\Hom{\Cat{C}}(\,\mhyphen\,,\, X)} \colon \OP{\Cat{C}} &\lto \SETS
        \end{align}
        のことを\textbf{表現可能前層} (representable presheaf) と呼ぶ:
        \begin{itemize}
            \item $\forall Y \in \Obj{\OP{\Cat{C}}}$ に対して
            \begin{align}
                \Hom{\Cat{C}}(\,\mhyphen\,,\, X) (Y) \coloneqq \Hom{\Cat{C}}(Y,\, X) \in \Obj{\SETS}
            \end{align}
            を対応づける
            \item $\OP{\Cat{C}}$ における任意の射 $g \colon Y \lto Z$\footnote{つまり,これは $\Cat{C}$ における射 $g \colon Z \lto Y$ である.} に対して,
            \begin{align}
                \Hom{\Cat{C}}(\,\mhyphen\,,\, X)(g) \coloneqq g^* \colon \Hom{\Cat{C}} (Y,\, X) &\lto \Hom{\Cat{C}} (Z,\, X),\\ 
                h &\lmto h \circ g
            \end{align}
            を対応付ける
        \end{itemize}
    \tcblower
        \textbf{米田埋め込み} (Yoneda embedding) とは,以下で定義する\hyperref[def:functor]{関手}
        \begin{align}
            \yo \colon \Cat{C} \lto \PshSETS{\Cat{C}}
        \end{align}
        のこと\footnote{実際,一部の数学者は米田埋め込みの記号に平仮名の「よ」を使っている.}:
        \begin{itemize}
            \item $\forall X \in \Obj{\OP{\Cat{C}}}$ に対して表現可能前層 $\yo (X) \coloneqq \Hom{\Cat{C}}(\,\mhyphen\,,\, X) \in \Obj{\PshSETS{\Cat{C}}}$ を対応付ける
            \item $\Cat{C}$ における任意の射 $f \colon X \lto Y$ に対して,以下で定義される\hyperref[def:nat]{自然変換} $\yo(f) \colon \Hom{\Cat{C}}(\,\mhyphen\,,\, X) \Longrightarrow \Hom{\Cat{C}}(\,\mhyphen\,,\, Y)$ を対応付ける:
            \begin{itemize}
                \item $\forall Z \in \Obj{\OP{\Cat{C}}}$ に対して,圏 $\SETS$ における射
                \begin{align}
                    \yo(f)_Z \coloneqq f_*\colon \Hom{\Cat{C}}(Z,\, X) &\lto \Hom{\Cat{C}}(Z,\, Y), \\
                    g &\lmto f \circ g
                \end{align}
                を対応付ける.
            \end{itemize}
        \end{itemize}
\end{mydef}

\begin{mylem}[label=lem:Yoneda]{米田の補題}
    \hyperref[def:presheaf-general]{前層} $F \colon \OP{\Cat{C}} \lto \SETS$ および圏 $\Cat{C}$ の対象 $X \in \Obj{\Cat{C}}$ を与える.
    このとき,写像
    \begin{align}
        \Hom{\PshSETS{\Cat{C}}}\bigl( \Hom{\Cat{C}}(\, \mhyphen\,,\, X),\, F \bigr) &\lto F(X), \\
        \tau &\lmto \tau_X(\Id_X)
    \end{align}
    は全単射である.
\end{mylem}

米田の補題の主張は少し込み入っているが,次のように考えれば良い:

$\tau \in \Hom{\PshSETS{\Cat{C}}}\bigl( \Hom{\Cat{C}}(\, \mhyphen\,,\, X)\bigr)$ とは\hyperref[def:nat]{自然変換}
\begin{center}
    \begin{tikzcd}[row sep=large, column sep=large]
        \mathcal{C}^{\mathrm{op}} \ar[bend left=50,r, "\mathrm{Hom}_{\mathcal{C}}(\,\mhyphen\,{,}\, X)"{name=U, above}] \ar[bend right=50,r, "F"{name=D, below}] &\SETS
        \ar[Rightarrow, from=U, to=D, "\tau"]
    \end{tikzcd}
\end{center}
のことであるから,\hyperref[def:representable]{表現可能前層}の定義より $X \in \Obj{\OP{\Cat{C}}}$ に対して圏 $\SETS$ における射(i.e. 写像)$\tau_X \colon \Hom{\Cat{C}}(X,\, X) \lto F(X)$ が定まる.\hyperref[def:category]{圏の定義}より集合 $\Hom{\Cat{C}}(X,\, X)$ には必ず恒等射という元 $\Id_X \in \Hom{\Cat{C}}(X,\, X)$ が含まれるので,それを写像 $\tau_X$ で送った先は $\tau_X(\Id_X) \in F(X)$ としてwell-definedである.

\begin{proof}
    写像
    \begin{align}
        \eta \colon F(X) &\lto \Hom{\PshSETS{\Cat{C}}}\bigl( \Hom{\Cat{C}}(\, \mhyphen\,,\, X),\, F \bigr), \\
        s &\lmto \Familyset[\big]{\eta(s)_Y \colon \Hom{\Cat{C}}(Y,\, X) \lto F(Y),\; f \lmto F(f)(s) }{Y \in \Obj{\Cat{C}}}
    \end{align}
    を考える.$\forall s \in F(X)$ を1つ固定する.
    このとき圏 $\OP{\Cat{C}}$ における任意の射 $Y \lrto Z \colon f$ および $\forall g \in \Hom{\Cat{C}}(Z,\, X)$ に対して
    \begin{align}
        \eta(s)_Y \circ \Hom{\Cat{C}} (\, \mhyphen\, , \, X)(f) (g)
        &= \eta(s)_Y (g \circ f) \\
        &= F(g \circ f)(s) \\
        &= F(f) \circ F(g)(s) \\
        &= F(f) \circ \eta(s)_Z(g)
    \end{align}
    が言える.i.e. $\eta(s)$ は自然変換であり,$\eta$ はwell-definedである.
    
    ところで,$\forall \tau \in \Hom{\PshSETS{\Cat{C}}}\bigl( \Hom{\Cat{C}}(\, \mhyphen\,,\, X),\, F \bigr)$ に対して
    \begin{align}
        \eta (\tau_X(\Id_X))
        &= \Familyset[\big]{\Hom{\Cat{C}}(Y,\, X) \lto F(Y),\; f \lmto F(f)\bigl( \tau_X(\Id_X) \bigr)  }{Y \in \Obj{\Cat{C}}} \\
        &= \Familyset[\big]{\Hom{\Cat{C}}(Y,\, X) \lto F(Y),\; f \lmto F(f)\circ \tau_X(\Id_X)  }{Y \in \Obj{\Cat{C}}} \\
        &= \Familyset[\big]{\Hom{\Cat{C}}(Y,\, X) \lto F(Y),\; f \lmto \tau_Y \circ \Hom{\Cat{C}} (\,\mhyphen\,,\, X)(f)(\Id_X)  }{Y \in \Obj{\Cat{C}}} \\
        &= \Familyset[\big]{\Hom{\Cat{C}}(Y,\, X) \lto F(Y),\; f \lmto \tau_Y (f \circ \Id_X)  }{Y \in \Obj{\Cat{C}}} \\
        &= \Familyset[\big]{\Hom{\Cat{C}}(Y,\, X) \lto F(Y),\; f \lmto \tau_Y(f)  }{Y \in \Obj{\Cat{C}}} \\
        &= \tau
    \end{align}
    が成り立ち,かつ
    \begin{align}
        \eta(s)_X (\Id_X) = F(\Id_X) (s) = \Id_{F(X)}(s) = s
    \end{align}
    が成り立つので,$\eta$ は題意の写像の逆写像である.
\end{proof}

\begin{myprop}[label=prop:Yoneda]{米田埋め込みは埋め込み}
    \hyperref[def:representable]{米田埋め込み} $\yo \colon \Cat{C} \lto \PshSETS{\Cat{C}}$ は\hyperref[def:faithful]{埋め込み}である.
\end{myprop}

\begin{proof}
    $\forall X,\, Y \in \Obj{\Cat{C}}$ を固定する.写像
    \begin{align}
        \Hom{\Cat{C}} (X,\, Y) &\lmto \Hom{\PshSETS{\Cat{C}}} \bigl( \Hom{\Cat{C}}(\, \mhyphen \, , \, X),\, \Hom{\Cat{C}}(\, \mhyphen \, , \, Y) \bigr),\\ 
        f &\lmto \yo(f)
    \end{align}
    が全単射であることを示せば良い.\hyperref[def:representable]{米田埋め込みの定義}から,$\forall f \in \Hom{\Cat{C}} (X,\, Y)$ に対して
    \begin{align}
        \yo(f) 
        &= \Familyset[\big]{\Hom{\Cat{C}}(Z,\, X) \lto \Hom{\Cat{C}}(Z,\, Y),\; g \lmto f \circ g}{Z \in \Obj{\Cat{C}}} \\
        &= \Familyset[\big]{\Hom{\Cat{C}}(Z,\, X) \lto \Hom{\Cat{C}}(Z,\, Y),\; g \lmto \Hom{\Cat{C}}(\, \mhyphen\, ,\, Y)(g)(f)}{Z \in \Obj{\Cat{C}}}
    \end{align}
    が成り立つが,これは\hyperref[lem:Yoneda]{米田の補題}において $F = \Hom{\Cat{C}}(\, \mhyphen\,,\, Y) \in \Obj{\PshSETS{\Cat{C}}}$ としたときの逆写像であり,示された.
\end{proof}

\begin{mycol}[label=col:Yoneda]{同型と表現可能前層の自然同値}
    以下の2つは同値である:
    \begin{enumerate}
        \item $X,\, Y \in \Obj{\Cat{C}}$ は\hyperref[def:iso]{同型}
        \item \hyperref[def:representable]{表現可能前層} $\Hom{\Cat{C}}(\, \mhyphen \,,\, X),\, \Hom{\Cat{C}}(\, \mhyphen \,,\, Y) \in \Obj{\PshSETS{\Cat{C}}}$ が\hyperref[def:nat]{自然同型}
    \end{enumerate}
\end{mycol}

\begin{proof}
    \begin{description}
        \item[\textbf{(1) $\bm{\Longrightarrow}$ (2)}] 
        
        $X \cong Y$ なので $f \in \Hom{\Cat{C}} (X,\, Y),\; g \in \Hom{\Cat{C}}(Y,\, X)$ が存在して $g \circ f = \Id_X \AND f \circ g = \Id_Y$ を充たす.
        このとき $\forall A \in \Obj{\Cat{C}}$ に対して
        \begin{align}
            \tau_A \colon \Hom{\Cat{C}}(A,\, X) \lto \Hom{\Cat{C}}(A,\, Y),\; h \lmto f \circ h
        \end{align}
        と定義するとこれは $\eta_A \colon \Hom{\Cat{C}}(A,\, Y) \lto \Hom{\Cat{C}}(A,\, X),\; h \lmto g \circ h$ を逆射に持つので
        \hyperref[def:iso]{同型射}であり,\hyperref[def:nat]{自然同値}
        \begin{align}
            \tau \colon \Hom{\Cat{C}} (\, \mhyphen \,,\, X) \Longrightarrow \Hom{\Cat{C}} (\, \mhyphen \,,\, Y)
        \end{align}
        を定める.

        \item[\textbf{(1) $\bm{\Longleftarrow}$ (2)}] 
        
        $\PshSETS{\Cat{C}}$ における\hyperref[def:iso]{同型射}とは,2つの\hyperref[def:presheaf-general]{前層}の間の\hyperref[def:nat]{自然同型}である.関手は\hyperref[def:iso]{同型射}を保つので,命題\ref{prop:Yoneda}より示された.
    \end{description}
    
\end{proof}


% 圏 $\PshSETS{\Cat{C}}$ における\hyperref[def:iso]{同型射}とは,2つの\hyperref[def:presheaf-general]{前層}の間の\hyperref[def:nat]{自然同型}である.さらに関手は同型を保つので,命題\ref{prop:Yoneda}から,2つの対象 $X,\, Y \in \Obj{\Cat{C}}$ 
% \begin{mycol}[label=col:Yoneda]{Hom集合が自然に同型ならば対象が同型}
%     $X,\, Y \in \Obj{\Cat{C}}$ に対して,
% \end{mycol}
後の便宜のため,\hyperref[def:lim]{極限}を捉え直そう.勝手な\hyperref[def:diagram]{図式} $D \colon \OP{I} \lto \Cat{C}$ を1つ固定する.
また,\textbf{定数関手} (constant functor) 
\begin{align}
    \label{eq:constant-functor}
    \mathrm{pt} \colon \OP{I} \lto \SETS
\end{align}
を以下のように定める:
\begin{itemize}
    \item $\forall i \in \Obj{\OP{I}}$ に対して,1点集合\footnote{圏 $\SETS$ における終対象である.} $\{\mathrm{pt}\} \in \Obj{\SETS}$ を対応付ける.
    \item $\forall f \in \Hom{\OP{I}}(i,\, j)$ に対して,恒等写像 $\mathrm{pt} \lmto \mathrm{pt}$ を対応付ける.
\end{itemize}
さらに,$\forall C \in \Obj{\Cat{C}}$ に対して,$\SETS$ に値をとる\hyperref[def:presheaf-general]{前層}
\begin{align}
    \Hom{\Cat{C}}\bigl( C,\, D(\, \mhyphen \,) \bigr) \colon \OP{I} \lto \SETS
\end{align}
を,以下のように定義する:
\begin{itemize}
    \item $\forall i \in \Obj{\OP{I}}$ に対して $\Hom{\Cat{C}}\bigl(X, D(i)\bigr) \in \Obj{\SETS}$ を対応付ける.
    \item $\forall f \in \Hom{\OP{I}}(i,\, j)$ に対して写像
        \begin{align}
            D(f)_* \colon \Hom{\Cat{C}}\bigl(X, D(i)\bigr) &\lto \Hom{\Cat{C}}\bigl(X, D(j)\bigr), \\
            g &\lmto F(f) \circ g
        \end{align}
    を対応付ける.
\end{itemize}
これを用いて,$\SETS$ に値をとる前層
\begin{align}
    \Hom{\PSH{I}{\SETS}} \Bigl( \mathrm{pt},\, \Hom{\Cat{C}}\bigl(\, \mhyphen \,,\, D(\, \mhyphen \,) \bigr)  \Bigr) \colon \OP{\Cat{C}} \lto \SETS
\end{align}
を以下のように定義する:
\begin{itemize}
    \item $\forall C \in \Obj{\OP{\Cat{C}}}$ に対して,集合
    \begin{align}
        \Hom{\PSH{I}{\SETS}} \Bigl( \mathrm{pt},\, \Hom{\Cat{C}}\bigl(C,\, D(\, \mhyphen \,) \bigr)  \Bigr) \in \Obj{\SETS}
    \end{align}
    を対応付ける.
    \item $\forall f \in \Hom{\OP{\Cat{C}}} (X,\, Y)$ に対して,写像
    \begin{align}
        \Hom{\PSH{I}{\SETS}} \Bigl( \mathrm{pt},\, \Hom{\Cat{C}}\bigl(X,\, D(\, \mhyphen \,) \bigr)  \Bigr) &\lto \Hom{\PSH{I}{\SETS}} \Bigl( \mathrm{pt},\, \Hom{\Cat{C}}\bigl(Y,\, D(\, \mhyphen \,) \bigr)  \Bigr), \\
        \tau &\lmto \tau \circ f
    \end{align}
    を対応付ける.
\end{itemize}

\begin{myprop}[label=prop:redef-lim]{極限の特徴付け}
    $X \in \Obj{\Cat{C}}$ が図式 $D \colon \OP{I} \lto \Cat{C}$ の極限であるための必要十分条件は,
    $\SETS$ に値をとる前層
    \begin{align}
        \Hom{\PSH{I}{\SETS}} \Bigl( \mathrm{pt},\, \Hom{\Cat{C}}\bigl(\, \mhyphen \,,\, D(\, \mhyphen \,) \bigr)  \Bigr) \colon \OP{\Cat{C}} \lto \SETS
    \end{align}
    が\hyperref[def:representable]{表現可能前層}
    \begin{align}
        \Hom{\Cat{C}}(\, \mhyphen \,, X) \colon \OP{\Cat{C}} \lto \SETS
    \end{align}
    と\hyperref[def:nat]{自然同型}になることである.
\end{myprop}

\begin{proof}
    \begin{description}
        \item[\textbf{($\bm{\Longleftarrow}$)}] 
        
        自然同型
        \begin{align}
            \theta \colon \Hom{\Cat{C}}(\, \mhyphen \,, X) \Longrightarrow \Hom{\PSH{I}{\SETS}} \Bigl( \mathrm{pt},\, \Hom{\Cat{C}}\bigl(\, \mhyphen \,,\, D(\, \mhyphen \,) \bigr)  \Bigr)
        \end{align}
        を与える.$\forall C \in \Obj{\Cat{C}}$ を1つ固定する.
        
         まず,集合 $\Hom{\PSH{I}{\SETS}} \Bigl( \mathrm{pt},\, \Hom{\Cat{C}}\bigl(C,\, D(\, \mhyphen \,) \bigr)  \Bigr)$ が図式 $D$ 上の $C$ を頂点とする\hyperref[def:Cone]{錐}全体の集合と同一視できることに注意する.
        実際,$\forall \tau \in \Hom{\PSH{I}{\SETS}} \Bigl( \mathrm{pt},\, \Hom{\Cat{C}}\bigl(C,\, D(\, \mhyphen \,) \bigr)  \Bigr)$
        は写像の族
        \begin{align}
            \Familyset[Big]{\tau_i \colon \{\mathrm{pt}\} \lto \Hom{\Cat{C}} \bigl( C,\, D(i) \bigr) }{i \in I}
        \end{align}
        からなるが,$\{\mathrm{pt}\}$ は1点集合なので $\tau_i$ を $\tau_i (\mathrm{pt})$ と同一視できる.i.e. $\tau_i \in \Hom{\Cat{C}} \bigl( C,\, D(i) \bigr)$ である.さらに,$\tau$ が\hyperref[def:nat]{自然変換}であることから
        $\forall f \in \Hom{\OP{I}}(i,\, j)$ に対して以下の図式が可換になる:
        \begin{center}
            \begin{tikzcd}
                & \{\mathrm{pt}\} \arrow[ld, "\tau_i"'] \arrow[rd, "\tau_j"] &                                         \\
            {\Hom{\Cat{C}} \bigl( C,\, D(i) \bigr)} \arrow[rr, "D(f)_*"'] &                                                            & {\Hom{\Cat{C}} \bigl( C,\, D(j) \bigr)}
            \end{tikzcd}
        \end{center}
        この図式における $\mathrm{pt} \in \{\mathrm{pt}\}$ の行き先を追跡することで
        \begin{align}
            D(f) \circ \tau_i = \tau_j
        \end{align}
        が分かるが,これはまさに\hyperref[def:Cone]{錐の定義}である.

         以上の考察から,自然同型 $\theta$ は図式 $D$ の勝手な錐 $(C,\, \tau) \in \Obj{\CONE{D}}$ が与えられると,対応する $\theta_C^{-1}(\tau) \in \Hom{\Cat{C}}(C,\, X)$ を一意的に定めるが,これは\hyperref[def:lim]{極限の普遍性}に他ならない.
        特に,\hyperref[lem:Yoneda]{米田の補題}から自然同型 $\theta$ は $\theta_X(\Id_X) \in \Hom{\PSH{I}{\SETS}} \Bigl( \mathrm{pt},\, \Hom{\Cat{C}}\bigl(X,\, D(\, \mhyphen \,) \bigr)  \Bigr)$ と対応付き,$\bigl( X,\, \theta_X(\Id_X) \bigr) \in \Obj{\CONE{D}}$ が図式 $D$ の極限である.

        \item[\textbf{($\bm{\Longrightarrow}$)}] 
        
        上述の議論から明らか.
    \end{description}
\end{proof}

\subsection{重み付き極限・エンド・コエンド}

\textbf{重み付き極限}とは,命題\ref{prop:redef-lim}の一般化である.

\begin{mydef}[label=def:weighted-lim]{重み付き極限}
    \hyperref[def:diagram]{図式} $D \colon \OP{I} \lto \Cat{C}$ 
    および $\SETS$ に値をとる任意の\hyperref[def:presheaf-general]{前層} $W \colon \OP{I} \lto \SETS$ を与える.
    
    図式 $D$ の $\bm{W}$\textbf{-重み付き極限} ($W$-weighted limit) とは,(存在すれば)圏 $\Cat{C}$ の対象 $\bm{\lim^W D} \in \Obj{\Cat{C}}$ であって,$\SETS$ に値をとる前層
    \begin{align}
        \Hom{\PSH{I}{\SETS}} \Bigl( W,\, \Hom{\Cat{C}}\bigl(\, \mhyphen \,,\, D(\, \mhyphen \,) \bigr)  \Bigr) \colon \OP{\Cat{C}} \lto \SETS
    \end{align}
    と\hyperref[def:representable]{表現可能前層}
    \begin{align}
        \Hom{\Cat{C}} (\, \mhyphen \,,\, \lim{}^W D) \colon \OP{\Cat{C}} \lto \SETS
    \end{align}
    が\hyperref[def:nat]{自然同型}となるもののこと.
\end{mydef}

\textbf{Hom関手}
\begin{align}
    \label{def:Hom-funct}
    \Hom{\Cat{C}} \colon \OP{\Cat{C}} \times \Cat{C} \lto \SETS
\end{align}
を以下で定義する:
\begin{itemize}
    \item $\forall (X,\, Y) \in \Obj{\OP{\Cat{C}} \times \Cat{C}}$ に対して $\Hom{\Cat{C}} (X,\, Y) \in \Obj{\SETS}$ を対応付ける.
    \item $\forall (f,\, g) \in \Hom{\OP{\Cat{C}} \times \Cat{C}} \bigl( (X,\, Y),\, (X',\, Y') \bigr)$ に対して,写像
    \begin{align}
        \Hom{\Cat{C}} (X,\, Y) &\lto \Hom{\Cat{C}} (X',\, Y'), \\
        h &\lmto g \circ h \circ f
    \end{align}
    を対応付ける.
\end{itemize}

\begin{mydef}[label=def:end]{エンド}
    関手 $F \colon \OP{\Cat{C}} \times \Cat{C} \lto \Cat{D}$ の\textbf{エンド} (end) とは,(存在すれば)
    \hyperref[def:weighted-lim]{$\Hom{\Cat{C}}$-重み付き極限}
    \begin{align}
        \bm{\int_{C \in \Obj{\Cat{C}}} F(C,\, C)} \coloneqq \lim{}^{\Hom{\Cat{C}}} F \in \Obj{\Cat{D}}
    \end{align}
    のこと.
\end{mydef}

% まず,$\PSH{\Cat{C}}{\SETS}$ に値をとる\hyperref[def:presheaf-general]{前層}
% \begin{align}
%     \Hom{\Cat{C}}\bigl(\, \mhyphen \,, D(\, \mhyphen \,)\bigr) \colon \OP{I} \xrightarrow{D} \Cat{C} \xrightarrow{\yo} \PSH{\Cat{C}}{\SETS}
% \end{align}
% を考える.このとき $\forall X \in \Obj{\Cat{C}}$ に対して,$\SETS$ に値をとる前層
% \begin{align}
%     \Hom{\Cat{C}}\bigl(X, D(\, \mhyphen \,)\bigr) \colon \OP{I} \lto \SETS
% \end{align}
% を次のようにして作ることができる:
% \begin{itemize}
%     \item $\forall i \in \Obj{\OP{I}}$ に対して $\Hom{\Cat{C}}\bigl(X, D(i)\bigr) \in \Obj{\PSH{I}{\SETS}}$ を対応付ける.
%     \item $\forall f \in \Hom{\OP{I}}(i,\, j)$ に対して写像
%     \begin{align}
%         D(f)_* \colon \Hom{\Cat{C}}\bigl(X, D(i)\bigr) &\lto \Hom{\Cat{C}}\bigl(X, D(j)\bigr), \\
%         g &\lmto F(f) \circ g
%     \end{align}
%     を対応付ける.
% \end{itemize}
% 次に,\textbf{定数関手} (constant functor) 
% \begin{align}
%     \mathrm{pt} \colon \OP{I} \lto \SETS
% \end{align}
% を以下のように定める:
% \begin{itemize}
%     \item $\forall i \in \Obj{\OP{I}}$ に対して,1点集合 $\{\mathrm{pt}\} \in \Obj{\SETS}$ を対応付ける.
%     \item $\forall f \in \Hom{\OP{I}}(i,\, j)$ に対して,恒等写像 $\mathrm{pt} \lmto \mathrm{pt}$ を対応付ける.
% \end{itemize}

% 以上の準備の下で,$\SETS$ に値をとる前層 $\widetilde{D} \colon \OP{\Cat{C}} \lto \SETS$ を
% \begin{itemize}
%     \item $\forall X \in \Obj{\Cat{C}}$ に対して,
%     \begin{align}
%         \widetilde{D}(X) \coloneqq \Hom{\PSH{I}{\SETS}} \Bigl( \mathrm{pt},\, \Hom{\Cat{C}}\bigl(X, D(\, \mhyphen \,)\bigr) \Bigr) \in \Obj{\SETS}
%     \end{align}
%     を対応付ける.
%     \item $\forall u \in \Hom{\OP{\Cat{C}}}(X,\, Y)$ に対して,写像
%     \begin{align}
%         \widetilde{D}(u) \colon 
%         \Hom{\PSH{I}{\SETS}} \Bigl( \mathrm{pt},\, \Hom{\Cat{C}}\bigl(X, D(\, \mhyphen \,)\bigr) \Bigr) &\lto \Hom{\PSH{I}{\SETS}} \Bigl( \mathrm{pt},\, \Hom{\Cat{C}}\bigl(Y, D(\, \mhyphen \,)\bigr) \Bigr), \\
%         \tau &\lmto 
%     \end{align}
    
% \end{itemize}


% これは\hyperref[def:presheaf-general]{前層} $D \in \Obj{\PSH{I}{\SETS}}$


% \begin{align}
%     \label{def:const-fun}
%     \Delta C \colon I \lto \Cat{C}
% \end{align}
% を次のように定める:
% \begin{itemize}
%     \item $\forall i \in \Obj{I}$ に対して $\Delta C (i) \coloneqq C \in \Obj{\Cat{C}}$ を対応づける
%     \item $\forall f \in \Hom{I} (i,\, j)$ に対して $\Delta C (f) \coloneqq \Id_C \in \Hom{\Cat{C}} (C,\, C)$ を対応づける
% \end{itemize}
% すると,図式 $D$ 上の\hyperref[def:Cone]{錐}とはまさに\hyperref[def:nat]{自然変換}
% \begin{align}
%     c \colon \Delta C \Longrightarrow D
% \end{align}
% のことに他ならない.実際,このような $c \in \Hom{\Fun (I,\, \Cat{C})} (\Delta C,\, D)$ は $\forall f \in \Hom{I}(i,\, j)$ に対して以下の図式を可換にする:
% \begin{center}
%     \begin{tikzcd}[row sep=huge]
%         \Delta C(i) \arrow[d, "c_i"'] \arrow[r, "\Delta C(f) = \Id_C"] & \Delta C(j) \arrow[d, "c_j"] \\
%         D(i) \arrow[r, "D(f)"]                                             & D(j)                            
%     \end{tikzcd}
% \end{center}

\subsection{随伴}

\begin{mydef}[label=def:adjoint]{随伴}
    \hyperref[def:functor]{関手} $F \colon \Cat{C} \lto \Cat{D},\; G \colon \Cat{D} \lto \Cat{C}$ を与える.
    $F$ が $G$ の\textbf{左随伴} (left adjoint) であり,かつ $G$ が $F$ の\textbf{右随伴} (right adjoint) であるとは,2つの関手
    \begin{align}
        \Hom{\Cat{D}} \bigl( F(\mhyphen),\, \mhyphen \bigr) \colon \OP{\Cat{C}} \times \Cat{D} &\lto \SETS, \\
        \Hom{\Cat{C}} \bigl( \mhyphen,\, G(\mhyphen) \bigr) \colon \OP{\Cat{C}} \times \Cat{D} &\lto \SETS
    \end{align}
    の間に\hyperref[def:nat]{自然同型}
    \begin{center}
        \begin{tikzcd}[row sep=large, column sep=large]
            \OP{\Cat{C}} \times \Cat{D} \ar[bend left=50,r, "\Hom{\Cat{D}} \bigl( F(\mhyphen){,}\, \mhyphen \bigr)"{name=U, above}] \ar[bend right=50,r, "\Hom{\Cat{C}} \bigl( \mhyphen{,}\, G(\mhyphen) \bigr)"{name=D, below}] &\SETS
            \ar[Rightarrow, from=U, to=D]
        \end{tikzcd}
    \end{center}
    が存在することを言う.

    \tcblower

    $F$ が $G$ の左随伴である(全く同じことだが,$G$ が $F$ の右随伴である)ことを $\bm{F \dashv G}$ と書く.図式中では
    \begin{center}
        \begin{tikzcd}
            \Cat{C}\ar[r,bend left,"F",""{name=A, below}] & \Cat{D}\ar[l,bend left,"G",""{name=B,above}] 
            \ar[from=A, to=B, symbol=\dashv]
            \end{tikzcd}
    \end{center}
    のように書く.
\end{mydef}

さて,圏 $\Cat{C}$ 上の\hyperref[def:diagram]{図式} $D \colon I \lto \Cat{C}$ が\hyperref[def:colim]{余極限}を持つとする:
\begin{center}
    \begin{tikzcd}[row sep=large, column sep=large]
        & &D(\forall i) \ar[dl, bend right]\ar[dr, blue, bend left] & \\
        &\textcolor{red}{\colim_I D} \ar[rr, red, dashed, "\exists !"] & &\forall \textcolor{blue}{X}
    \end{tikzcd}
\end{center}
このとき,$\Cat{D}$ 上の図式として
\begin{center}
    \begin{tikzcd}[row sep=large, column sep=large]
        & &F\bigl(D(\forall i)\bigr) \ar[dl, bend right]\ar[dr, bend left] & \\
        &\textcolor{red}{\colim_I F(D)} \ar[rr, red, dashed, "\exists ! u"] & &F(\colim_I D)
    \end{tikzcd}
\end{center}
を考えることができる.特に,一意に定まる射 $\textcolor{red}{u} \colon \colim_I F(D) \lto F(\colim_I D)$ が\hyperref[def:iso]{同型}のとき,関手 $F$ は\textbf{余極限を保つ}\label{def:preserve-colim}という.

同様に,圏 $\Cat{D}$ 上の\hyperref[def:diagram]{図式} $D \colon I \lto \Cat{C}$ が\hyperref[def:lim]{極限}を持つとする:
\begin{center}
    \begin{tikzcd}[row sep=large, column sep=large]
        &\textcolor{red}{\lim_I D}\ar[dr, bend right] & &\forall \textcolor{blue}{X} \ar[ll, red, dashed, "\exists !"]\ar[dl, bend left, blue] \\
        & &D(\forall i) &
    \end{tikzcd}
\end{center}
このとき,$\Cat{D}$ 上の図式として
\begin{center}
    \begin{tikzcd}[row sep=large, column sep=large]
        &\textcolor{red}{\lim_I F(D)}\ar[dr, bend right] & &F(\lim_{I} D) \ar[ll, red, dashed, "\exists ! u"]\ar[dl, bend left] \\
        & &F\bigl(D(\forall i)\bigr) &
    \end{tikzcd}
\end{center}
を考えることができる.特に,一意に定まる射 $\textcolor{red}{u} \colon F(\lim_I D) \lto \lim_I F(D)$ が\hyperref[def:iso]{同型}のとき,関手 $F$ は\textbf{極限を保つ}\label{def:preserve-lim}という.

\begin{myprop}[label=prop:adj-lim]{随伴と極限・余極限}
    \hyperref[def:functor]{関手} $F \colon \Cat{C} \lto \Cat{D},\; G \colon \Cat{D} \lto \Cat{C}$ が\hyperref[def:adjoint]{$F \dashv G$}であるとする.
    このとき,$F$ は\hyperref[def:preserve-colim]{余極限を保ち},$G$ は\hyperref[def:preserve-lim]{極限を保つ}.
\end{myprop}

\begin{proof}
    \hyperref[def:colim]{余極限}を持つ任意の $\Cat{C}$ の図式 $D \colon I \lto \Cat{C}$ を1つ固定する.
    \hyperref[def:adjoint]{随伴の定義}および命題\ref{prop:lim-colim-basic}より,$\forall Y \in \Obj{\Cat{D}}$ に対して
    \begin{align}
        \Hom{\Cat{D}} \bigl( F(\underset{I}{\colim} D),\, Y \bigr) 
        &\cong \Hom{\Cat{C}} \bigl( \underset{I}{\colim} D,\, G(Y) \bigr) \\
        &\cong \lim_I \Hom{\Cat{C}} \bigl( D(i),\, G(Y) \bigr) \\
        &\cong \lim_I \Hom{\Cat{D}} \bigl( F(D(i)),\, Y \bigr) \\
        &\cong \Hom{\Cat{D}} \bigl( \underset{I}{\colim} F(D),\, Y \bigr) 
    \end{align}
    が言える.i.e. \hyperref[def:iso]{自然同型}
    \begin{center}
    \begin{tikzcd}[row sep=large, column sep=large]
        \Cat{D} \ar[bend left=50,r, "\Hom{\Cat{D}}\bigl( F(\colim_I D){,}\, \mhyphen \bigr) "{name=U, above}] \ar[bend right=50,r, "\Hom{\Cat{D}}\bigl( \colim_I F(D){,}\, \mhyphen \bigr)"{name=D, below}] &\SETS
        \ar[Rightarrow, from=U, to=D]
    \end{tikzcd}
    \end{center}
    があるので,\hyperref[col:Yoneda]{米田の補題の系}より
    \begin{align}
        F(\underset{I}{\colim} D) \cong \underset{I}{\colim} F(D)
    \end{align}
    が示された.
\end{proof}

\subsection{Kan拡張}

\begin{mydef}[label=def:slice-category]{スライス圏}
    圏 $\Cat{D}$ およびその対象 $X \in \Obj{\Cat{D}}$ を与える.
    \textbf{スライス圏} (slice category) $\bm{\Cat{D}_{/X}}$ とは,以下のデータからなる圏のこと:
    \begin{itemize}
        \item $\Cat{D}$ の対象と射の組 $(D \in \Obj{\Cat{D}},\, \alpha \colon D \lto \textcolor{red}{X})$ を対象に持つ
        \item $(D,\, \alpha),\; (D',\, \alpha')$ の間の射は,$\Cat{D}$ における射 $\beta \colon D \lto D'$ であって $\Cat{D}$ における図式
        \begin{center}
            \begin{tikzcd}[row sep=large, column sep=large]
                &D \ar[rr, "\beta"]\ar[dr, "\alpha"] & &D' \ar[dl, "\alpha'"'] \\
                & &X &
            \end{tikzcd}
        \end{center}
        を可換にするものとする
    \end{itemize}
    
    \tcblower 

    \textbf{双対スライス圏} (dual slice category) $\bm{\Cat{D}_{X/}}$ とは,以下のデータからなる圏のこと:
    \begin{itemize}
        \item $\Cat{D}$ の対象と射の組 $(D \in \Obj{\Cat{D}},\, \alpha \colon \textcolor{red}{X} \lto D)$ を対象に持つ
        \item $(D,\, \alpha),\; (D',\, \alpha')$ の間の射は,$\Cat{D}$ における射 $\beta \colon D \lto D'$ であって $\Cat{D}$ における図式
        \begin{center}
            \begin{tikzcd}[row sep=large, column sep=large]
                &D \ar[rr, "\beta"] &&D'\\
                & &X \ar[ul, "\alpha"']\ar[ur, "\alpha'"] &
            \end{tikzcd}
        \end{center}
        を可換にするものとする
    \end{itemize}
\end{mydef}

関手\footnote{対象の対応のみ明示した.} $\Cat{D}_{/X} \lto \Cat{D},\; (D,\, \alpha) \lmto D$ のことを\textbf{標準的忘却関手} (canonical forgetful functor) と呼ぶ.標準的関手は図式中でも記号で明記しないことが多い.

関手 $F \colon \Cat{C} \lto \Cat{D}$ および\underline{圏 $\Cat{D}$ における}対象 $X \in \Cat{D}$ を与える.このとき\textbf{関手 $\bm{F}$ に関するスライス圏}を,圏全体がなす圏 $\CAT$ における\hyperref[def:pullback-pushout]{引き戻し}
\begin{center}
    \begin{tikzcd}[row sep=large, column sep=large]
        &\bm{F_{/X}} \ar[r]\ar[d]&\Cat{D}_{/X}\ar[d] \\
        &\Cat{C} \ar[r, "F"] &\Cat{D}
    \end{tikzcd}
\end{center}
として定義する.i.e. $F_{/X}$ の対象は $(C \in \Cat{C},\, \alpha \colon F(C) \lto X)$ であり,$(C,\, \alpha),\; (C',\, \alpha')$ の間の射とは,$\Cat{C}$ における射 $\beta \colon C \lto C'$ であって \underline{$\Cat{D}$ における}図式
\begin{center}
    \begin{tikzcd}[row sep=large, column sep=large]
        &F(C) \ar[rr, "F(\beta)"]\ar[dr, "\alpha"] &&F(C')\ar[dl, "\alpha'"']\\
        & &X &
    \end{tikzcd}
\end{center}
を可換にするものである.

\begin{mytheo}[label=thm:density-cat]{density theorem}
    \hyperref[def:presheaf-general]{前層} $F \colon \OP{\Cat{C}} \lto \SETS$ を与える.関手 $\yo \colon \Cat{C} \lto \PshSETS{\Cat{C}}$ を\hyperref[def:representable]{米田埋め込み}とする.
    
    このとき,前層の圏 $\PshSETS{\Cat{C}}$ における\hyperref[def:iso]{同型}
    \begin{align}
        F \cong \underset{X \in \yo_{/F}}{\colim} \Hom{\Cat{C}} (\, \mhyphen \,,\, X)
    \end{align}
    が成り立つ.
\end{mytheo}

\begin{proof}
    \hyperref[col:Yoneda]{米田の補題の系}により,示すべきは\hyperref[def:nat]{自然同型}
    \begin{align}
        \Hom{\PshSETS{\Cat{C}}} \bigl( \underset{X \in \yo_{/F}}{\colim} \Hom{\Cat{C}} (\, \mhyphen \,,\ X),\, \, \mhyphen \, \bigr) \Longrightarrow \Hom{\PshSETS{\Cat{C}}} \bigl( F,\, \, \mhyphen \, \bigr) 
    \end{align}
    である.このとき,$\forall G \in \Obj{\PshSETS{\Cat{C}}}$ に対して
    \begin{align}
        \Hom{\PshSETS{\Cat{C}}} \bigl( \underset{X \in \yo_{/F}}{\colim} \Hom{\Cat{C}} (\, \mhyphen \,,\ X),\, G \bigr)
        &\cong \lim_{X \in \yo_{/F}} \Hom{\PshSETS{\Cat{C}}} \bigl( \Hom{\Cat{C}}(\, \mhyphen \,,\, X),\, G \bigr) &&\because\quad \text{命題\ref{prop:lim-colim-basic}} \\
        &\cong \lim_{X \in \yo_{/F}} G(X) &&\because\quad \text{\hyperref[lem:Yoneda]{米田の補題}}
    \end{align}
    なる自然同型がある.
    さらに,命題\ref{prop:redef-lim}と同様の議論により\hyperref[eq:constant-functor]{定数関手} $\mathrm{pt} \in \Obj{\PshSETS{\yo_{/F}}}$ による自然な同型
    \begin{align}
        \lim_{X \in \yo_{/F}} G(X) \cong \Hom{\PshSETS{\yo_{/F}}} (\mathrm{pt},\, G)
    \end{align}
    があることが分かる.よって自然な同型
    \begin{align}
        \Hom{\PshSETS{\yo_{/F}}} (\mathrm{pt},\, G) \cong \Hom{\PshSETS{\Cat{C}}} \bigl( F,\, G \bigr)
    \end{align}
    を示せば十分である.

    ところで,\hyperref[def:nat]{自然変換} $\tau \in \Hom{\PshSETS{\yo_{/F}}} (\mathrm{pt},\, G)$ は,写像の族
    \begin{align}
        \Familyset[\big]{\tau_{(X,\, \alpha)} \colon \{\mathrm{pt}\} \lto G(X)}{(X,\, \alpha) \in \Obj{\yo_{/F}}}
    \end{align}
    からなるが,$\{\mathrm{pt}\}$ は一点集合なので $\tau_{(X,\, \alpha)}$ と $\tau_{(X,\, \alpha)}(\mathrm{\mathrm{pt}}) \in G(X)$ を同一視して良い.
    一方で $\forall (X,\, \alpha) \in \Obj{\yo_{/F}}$ について $\alpha \in \Hom{\PSH{\Cat{C}}{\SETS}} \bigl( \yo (X),\, F \bigr) = \Hom{\PSH{\Cat{C}}{\SETS}} \bigl( \Hom{\Cat{C}} (\, \mhyphen \,, X),\, F \bigr)$ であるから,\hyperref[lem:Yoneda]{米田の補題}からこれは $\alpha_X(\Id_X) \in F(X)$ と一対一対応する.
    この対応により $\forall X \in \OP{\Cat{C}}$ について写像
    \begin{align}
        \eta(\tau)_X \colon F(X) \lto G(X),\; \alpha_X(\Id_X) \lmto \tau_{(X,\, \alpha)}
    \end{align}
    が得られる.$\alpha$ は自然変換なので $\eta(\tau)_X$ を全て集めたものは自然変換 $\eta(\tau) \colon F \Longrightarrow G$ になる.
    よって写像
    \begin{align}
        \Hom{\PshSETS{\yo_{/F}}} (\mathrm{pt},\, G) \lto \Hom{\PshSETS{\Cat{C}}}(F,\, G),\; \tau \lmto \eta(\tau)
    \end{align}
    は自然な同型であり,証明が完了した.
\end{proof}

\begin{mydef}[label=def:Kanext]{Kan拡張}
    $i \colon \Cat{C}_0 \hookrightarrow \Cat{C}$ を $\Cat{C}$ の小部分圏,$\Cat{D}$ を\hyperref[def:complete]{双完備}な圏とする.
    \begin{itemize}
        \item 関手 $F \colon \Cat{C}_0 \lto \Cat{D}$ の,関手 $i$ に沿った\textbf{左Kan拡張} (left Kan extention) とは,
        \begin{align}
            i_{!} (F)(x) \coloneqq \underset{c \in (C_0)_{/x}}{\colim} F(c)
        \end{align}
        によって定義される関手
        \begin{align}
            i_{!} \colon \mathrm{Fun} (\Cat{C}_0,\, \Cat{D}) \lto \mathrm{Fun} (\Cat{C},\, \Cat{D})
        \end{align}
        によって定まる関手 $i_!(F) \colon \Cat{C} \lto \Cat{D}$ のこと. 
        \item   関手 $F \colon \Cat{C}_0 \lto \Cat{D}$ の,関手 $i$ に沿った\textbf{右Kan拡張} (right Kan extention) とは,
        \begin{align}
            i_{*} (F)(x) \coloneqq \underset{c \in (C_0)_{x/}}{\lim} F(c)
        \end{align}
        によって定義される関手
        \begin{align}
            i_{*} \colon \mathrm{Fun} (\Cat{C}_0,\, \Cat{D}) \lto \mathrm{Fun} (\Cat{C},\, \Cat{D})
        \end{align}
        によって定まる関手 $i_*(F) \colon \Cat{C} \lto \Cat{D}$ のこと. 
    \end{itemize}
    
\end{mydef}
制限関手
\begin{align}
    i^* \colon \mathrm{Fun} (\Cat{C},\, \Cat{D}) \lto \mathrm{Fun} (\Cat{C}_0,\, \Cat{D})
\end{align}
について \hyperref[def:adjoint]{$i_{!} \dashv i^*$ かつ $i_* \vdash i^*$}である.

\section{単体的集合}

higher geometry において重要な役割を果たす\hyperref[def:SimpSet]{単体的集合の圏}を定義する.

\begin{mydef}[label=def:simplex-cat,breakable]{単体圏}
    \begin{itemize}
        \item $\forall n \in \mathbb{N} \cup \{0\}$ に対して,全順序付集合 $\bm{[n]} \coloneqq \{0,\, 1,\, \dots,\, n\}$ のことを $\bm{n}$\textbf{-単体} ($n$-simplex) と呼ぶ.
        \item \textbf{単体圏} (simplex category) $\Delta$ とは,
        \begin{itemize}
            \item $\forall n \in \mathbb{N} \cup \{0\}$ に対して,$n$-単体 $[n]$ を対象とする.
            \item 順序を保つ写像を射とする
        \end{itemize}
        圏のこと.
    \end{itemize}
    
    \tcblower

    特に $\Hom{\Delta} ([n-1],\, [n])$ の元のうち
    \begin{align}
        d^n_i \colon [n-1] \hookrightarrow [n],\; x \lmto 
        \begin{cases}
            x, &x < i \\
            x+1 &x \ge i
        \end{cases}
        \WHERE i = 0,\, \dots,\, n
    \end{align}
    のことを\textbf{面写像} (face map) と呼び,
    $\Hom{\Delta}([n+1],\, [n])$ の元のうち
    \begin{align}
        s^n_i \colon [n+1] \twoheadrightarrow [n],\; x \lmto 
        \begin{cases}
            x, &x \le i \\
            x-1 &x > i
        \end{cases}
        \WHERE i=0,\, \dots,\, n
    \end{align}
    のことを\textbf{縮退写像} (degeneracy map) と呼ぶ.
\end{mydef}

\begin{mydef}[label=def:SimpSet,breakable]{単体的集合}
    \begin{itemize}
        \item \textbf{単体的集合} (simplicial set)  とは,\hyperref[def:presheaf-general]{前層}
        \begin{align}
            K \colon \OP{\Delta} \lto \SETS
        \end{align}
        のこと.
        特に $n$-単体 $[n] \in \Obj{\OP{\Delta}}$ の\hyperref[def:representable]{表現可能前層}を $\bm{\Delta^n} \coloneqq \Hom{\Delta} (\, \mhyphen \,,\, [n]) \in \Obj{\sSet}$ と書く.

        \item \textbf{余単体的集合} (cosimplicial set) とは,関手
        \begin{align}
            K \colon \Delta \lto \SETS
        \end{align}
        のこと.
        \item 単体的集合 $S \colon \OP{\Delta} \lto \SETS$ が単体的集合 $K \colon \OP{\Delta} \lto \SETS$ の\textbf{単体的部分集合} (simplicial subset) であるとは,以下の2つの条件を充たすことを言う
        \footnote{条件 \textsf{\textbf{(sub-2)}} により\textbf{包含写像} $i \coloneqq \Familyset[\big]{i_{[n]} \colon S([n]) \lto K([n]),\; x \lmto x}{n \ge 0}$ が\hyperref[def:nat]{自然変換}になり,$i \in \Hom{\sSet} (S,\, K)$ と見做せる.このことから,以降では包含写像を $S \hookrightarrow K$ と略記する.}
        :
        \begin{description}
            \item[\textbf{(sub-1)}] $\forall n \ge 0$ に対して $S([n]) \subset K([n])$
            \item[\textbf{(sub-2)}] \hyperref[def:simplex-cat]{圏 $\Delta$} における任意の射 $\alpha \colon [n] \lto [m]$ に対して $K(\alpha) \bigl( S([m]) \bigr) \subset S([n])$ で,かつ $S(\alpha) = K(\alpha)|_{S([m])}$ が成り立つ.
        \end{description}
        誤解の恐れがないときは,単体的部分集合を $\bm{S \subset K}$ と書く.
        \item \textbf{単体的集合の圏} $\sSet$ とは,\hyperref[def:presheaf-general]{前層}の圏
        \begin{align}
            \sSet \coloneqq \PshSETS{\Delta}
        \end{align}
        のこと.
    \end{itemize}

    \tcblower

    $K_n \coloneqq K([n])$ とおく.
    \begin{itemize}
        \item $K_n$ の元のことを\textbf{$n$-単体} ($n$-simplex)
        \item $\partial_i \coloneqq K(d_i) \colon K_{n} \twoheadrightarrow K_{n-1}$ のことを\textbf{面写像} (face map)
        \item $\sigma_i \coloneqq K(s_i) \colon K_n \hookrightarrow K_{n+1}$ のことを\textbf{縮退写像} (degeneracy map) と呼ぶ.
    \end{itemize}
    と呼ぶ.これらは以下の\textbf{単体的恒等式} (simplicial identities) を充たす:
    \begin{align}
        \partial^{n-1}_i \circ \partial^{n}_j &= \partial^{n-1}_{j-1} \circ \partial^{n}_i&& (i<j), \label{eq:Simp1}\\
        \partial^{n+1}_i \circ \sigma^{n}_j &= \sigma^{n-1}_{j-1} \circ \partial^{n}_i && (i<j), \label{eq:Simp2}\\
        \partial^{n+1}_i \circ \sigma^{n}_j &= \sigma^{n-1}_j \circ \partial^{n}_{i-1} &&(i > j+1), \label{eq:Simp3}\\
        \partial^{n+1}_i \circ \sigma^{n}_j &= \mathrm{id} &&(i = j,\, j+1), \label{eq:Simp4}\\
        \sigma^{n+1}_i \circ \sigma^{n}_j &= \sigma^{n+1}_{j+1} \circ \sigma^{n}_i &&(i \le j) \label{eq:Simp5}
    \end{align}
\end{mydef}

逆に,
\begin{itemize}
    \item 対象の族 $\Familyset[\big]{K_n}{n \ge 0}$
    \item 射の族 $\Familyset[\big]{\partial^n_i \colon K_n \lto K_{n-1}}{0 \le i \le n,\, n \ge 0}$
    \item 射の族 $\Familyset[\big]{\sigma^n_i \colon K_n \lto K_{n+1}}{0 \le i \le n,\, n \ge 0}$
\end{itemize}
の組であって単体的恒等式を充たすものは単体的集合を一意に定める\cite[\href{https://kerodon.net/tag/04FW}{Tag 04FW}]{kerodon}.

\hyperref[def:SimpSet]{単体的集合} $K \colon \OP{\Delta} \lto \SETS$ を図示する方法がある.
\begin{enumerate}
    \item $K_0$ の元(i.e. $0$-単体)を点と見做し,$K_0 = \{ \bullet,\, \dots,\, \bullet\}$ のように書く.
    \item $K_1$ の元(i.e. $1$-単体) $e \in K_1$ を
    \begin{center}\label{fig:1-simp}
        \begin{tikzpicture}
            \node[bullet={below left:$\partial^1_1 (e)$}] (x) {};
            \node[bullet={below right:$\partial^1_0 (e)$}] (y) at ($(x) + (3,0)$) {};
            \draw[->-=.5] (x) -- (y) node[midway,below] {$e$};
        \end{tikzpicture}
    \end{center}
    のように有向辺として図示する.
    \item $K_2$ の元(i.e. $2$-単体) $\sigma \in K_2$ を
    \begin{center}\label{fig:2-simp}
        \begin{tikzpicture}
            \node[bullet={below left:$\partial^1_1 \partial^2_1(\sigma) = \partial^1_1 \partial^2_2 (\sigma)$}] (x) {};
            \node[bullet={below right:$\partial^1_0 \partial^2_1(\sigma) = \partial^1_0 \partial^2_0 (\sigma)$}] (y) at ($(x) + (2,0)$) {};
            \node[bullet={above:$\partial^1_0 \partial^2_2(\sigma) = \partial^1_1 \partial^2_0 (\sigma)$}] (z) at ($(x) + (1, {sqrt(3)})$) {};
            \node (s) at ($(1,0)!0.33!(z)$) {$\sigma$};
            \draw[->-=.5] (x) -- (y) node[midway,below] {$\partial^2_1(\sigma)$};
            \draw[->-=.5] (x) -- (z) node[midway,above left] {$\partial^2_2(\sigma)$};
            \draw[->-=.5] (z) -- (y) node[midway,above right] {$\partial^2_0(\sigma)$};
        \end{tikzpicture}
    \end{center}
    のように向きづけられた三角形として図示する.この図は単体的恒等式\eqref{eq:Simp1}を表している.
    \item $K_3$ の元(i.e. $3$-単体)$t \in K_3$ を,
    \begin{center}\label{fig:3-simp}
        \begin{tikzpicture}[
            3d/install view={phi=45,psi=0,theta=60},
            declare function={
                a=7;
                g=0.45;
            }
        ]
            \path 
                (-a/2,0,0) coordinate (P_1)
                (a/2,0,0) coordinate (P_2)
                (0,{a*sqrt(3)/2},0) coordinate (P_3)
                (0,{a/(2*sqrt(3))},{a*sqrt(2/3)}) coordinate (P_4)
                (0,{a/(2*sqrt(3))},{a/(2*sqrt(6))}) coordinate (P_5)
            ;
        
            % faces
            \fill[fill=green,opacity=0.1,save named path=f_134] (P_1) -- (P_3) -- (P_4) -- cycle;
            \fill[fill=red,opacity=0.1,save named path=f_123] (P_1) -- (P_2) -- (P_3) -- cycle;
            \fill[fill=blue,opacity=0.1,save named path=f_124] (P_1) -- (P_2) -- (P_4) -- cycle;
            \fill[fill=DarkOrange,opacity=0.1,save named path=f_234] (P_2) -- (P_3) -- (P_4) -- cycle;
         
            % edge
            \draw[->-=.5,save named path=e_12] (P_1) -- (P_2);
            \draw[-<-=.5,dashed,save named path=e_13] (P_1) -- (P_3);
            \draw[-<-=.5,save named path=e_14] (P_1) -- (P_4);
            \draw[-<-=.5,save named path=e_23] (P_2) -- (P_3);
            \draw[-<-=.5,save named path=e_24] (P_2) -- (P_4);
            \draw[-<-=.5,save named path=e_34] (P_3) -- (P_4);
            
            
            % labels
            \path[red,save named path=d_31, plane x={($(P_2)-(P_1)$)}, plane y={($(P_3)-(P_1)$)}, canvas is plane] ($(P_1) + (0.35,0.35)$) node {$\partial^3_0(t)$};
            \path[DarkGreen,save named path=d_32, plane x={($(P_3)-(P_1)$)}, plane y={($(P_4)-(P_1)$)}, canvas is plane] ($(P_1) + (0.35,0.35)$) node {$\partial^3_3(t)$};
            \path[blue,save named path=d_33, plane x={($(P_2)-(P_1)$)}, plane y={($(P_4)-(P_1)$)}, canvas is plane] ($(P_1) + (0.35,0.35)$) node {$\partial^3_1(t)$};
            \path[DarkOrange,save named path=d_34, plane x={($(P_3)-(P_2)$)}, plane y={($(P_4)-(P_2)$)}, canvas is plane] ($(P_2) + (0.35,0.35)$) node {$\partial^3_2(t)$};
        
            \foreach \i in {1,...,4} {
                \node[bullet] at (P_\i) {};
            }
            \node[left] at (P_1) {2};
            \node[below] at (P_2) {3};
            \node[right] at (P_3) {1};
            \node[above] at (P_4) {0};
            \fill[white] (P_5) circle[radius=5pt];
            \node at (P_5) {$t$};
        \end{tikzpicture}        
    \end{center}
    のように向き付けられた四面体として図示する.四面体の面の繋がり方が単体的恒等式\eqref{eq:Simp1}を表している.
    \item $K_n$ の元(i.e. $n$-単体)は,単体的恒等式\eqref{eq:Simp1}によって帰納的に図示する.
\end{enumerate}


\begin{myprop}[label=prop:SimpSet-basic]{単体的集合の圏の基本性質}
    $\yo \colon \Delta \lto \sSet$ を\hyperref[def:representable]{米田埋め込み}とする.
    \begin{enumerate}
        \item 任意の\hyperref[def:SimpSet]{単体的集合} $K \colon \OP{\Delta} \lto \SETS$ に対して,\hyperref[def:nat]{自然な同型}
        \begin{align}
            \Hom{\sSet} (\Delta^n,\, K) \cong K_n
        \end{align}
        が成り立つ.
        \item 圏 \hyperref[def:SimpSet]{$\sSet$} は\hyperref[def:complete]{双完備}である.
        \item 任意の\hyperref[def:SimpSet]{単体的集合} $K \colon \OP{\Delta} \lto \SETS$ に対して,
        \begin{align}
            K \cong \underset{[n] \in \yo_{/K}}{\colim} \Delta^n
        \end{align}
        が成り立つ.
    \end{enumerate}
    
\end{myprop}

\begin{marker}
    命題\ref{prop:SimpSet-basic}-(1) によって,\hyperref[def:SimpSet]{$n$-単体} $\sigma \in K_n$ を\hyperref[def:nat]{自然変換} $\sigma \in \Hom{\sSet}(\Delta^n,\, K)$ と同一視できる!
\end{marker}


\begin{proof}
    \begin{enumerate}
        \item \hyperref[lem:Yoneda]{米田の補題}より
        \begin{align}
            \Hom{\sSet} (\Delta^n,\, K) 
            =\Hom{\PshSETS{\Delta}} (\Hom{\Delta}(\mhyphen,\, [n]),\, K)
            \cong K([n]) = K_n
        \end{align}
        \item \hyperref[def:diagram]{図式} $D \colon I \lto \sSet$ を与える.
        $\SETS$ が双完備であることから,\hyperref[def:SimpSet]{単体的集合}
        \begin{align}
            \lim_I D \colon \OP{\Delta} &\lto \SETS, \\
            [n] &\lmto \lim_I D(\, \mhyphen \,)([n])
        \end{align}
        がwell-definedである.これがちょうど図式 $D$ の\hyperref[def:lim]{極限}を与える.
        
         同様に,\hyperref[def:SimpSet]{単体的集合}
        \begin{align}
            \colim_I D \colon \OP{\Delta} &\lto \SETS, \\
            [n] &\lmto \colim_I D(\, \mhyphen \,)([n])
        \end{align} 
        が図式 $D$ の\hyperref[def:colim]{余極限}を与える.
        
        \item 定理\ref{thm:density-cat}より
        \begin{align}
            K \cong \underset{[n] \in \yo_{/K}}{\colim}\Hom{\Delta}(\mhyphen,\, [n]) = \underset{[n] \in \yo_{/K}}{\colim} \Delta^n
        \end{align}
        
    \end{enumerate}
    
\end{proof}

\subsection{幾何学的実現}

定義\ref{def:SimpSet}を再現する具体的な構成をする.

\begin{mydef}[label=def:simplicial-top,breakable]{幾何学的 $n$-単体}
    \begin{itemize}
        \item \textbf{幾何学的 $\bm{n}$-単体} $\irm{\Delta}{top}^n$ とは,位相空間
        \begin{align}
            \irm{\Delta}{top}^n \coloneqq \bigl\{\, (x^0,\, \dots,\, x^n) \in \mathbb{R}^{n+1}\bigm| x^i \ge 0,\; \sum_{i=0}^n x^i = 1 \,\bigr\} 
        \end{align}
        のこと.
        \item \hyperref[def:SimpSet]{余単体的集合} 
        \begin{align}
            \irm{\Delta}{top} \colon \Delta \lto \TOP
        \end{align}
        とは,
        \begin{itemize}
            \item \hyperref[def:simplex-cat]{$n$-単体} $[n] \in \Obj{\Delta}$ に対して幾何学的 $n$-単体 $\irm{\Delta}{top}^n$ を対応づける
            \item \hyperref[def:simplex-cat]{圏 $\Delta$} における任意の射 $\alpha \colon [n] \lto [m]$ に対して,連続写像
            \begin{align}
                \irm{\Delta}{top}(\alpha) \colon \irm{\Delta}{top}^n \lto \irm{\Delta}{top}^m,\; (x^0,\, \dots,\, x^n) \lmto \left( \sum_{j,\; \alpha(j) = 0} x^j,\, \dots ,\, \sum_{j,\; \alpha(j) = m} x^j \right) 
            \end{align}
            を対応付ける
        \end{itemize}
        \hyperref[def:functor]{関手}のこと.
        \item 位相空間 $X \in \Obj{\TOP}$ の\textbf{特異単体} (singular simplicial set) とは,\hyperref[def:SimpSet]{単体的集合}
        \begin{align}
            S(X) \colon \OP{\Delta} \lto \SETS,\; [n] \lmto \Hom{\TOP} (\irm{\Delta}{top}^n ,\, X)
        \end{align}
        のこと.
        \item \hyperref[def:simplicial-top]{特異複体}とは,関手 $S \colon \TOP \lto \sSet,\; X \lmto S(X)$ のこと.
    \end{itemize}
    
\end{mydef}


\begin{mydef}[label=def:geometric-realization]{幾何学的実現}
    $\yo \colon \Delta \lto \sSet$ を\hyperref[def:representable]{米田埋め込み}とする.
    \textbf{幾何学的実現} (geometric realization) とは,\hyperref[def:colim]{余極限}を保つ関手
    \begin{align}
        \abs{\mhyphen} \colon \sSet \lto \TOP,\; K \lmto \underset{[n] \in \yo_{/K}}{\colim} \irm{\Delta}{top}([n])
    \end{align}
    のこと.
    \tcblower
    $\TOP$ における $\colim$ の公式を使うと
    \begin{align}
        \abs{K} =  \left( \coprod_{[n] \in \Obj{\Delta}} (K_n \times \irm{\Delta}{top}^n)  \right) \bigg/ \Bigl\{\, \bigl(\alpha^*(x),\, t\bigr) \sim \bigl(x,\, \irm{\Delta}{top}(\alpha)(t)\bigr) \bigm| \substack{x \in K_n,\, t \in \irm{\Delta}{top}^m, \\ \alpha \in \Hom{\Delta}([m],\, [n])} \,\Bigr\} 
    \end{align}
    となる.
\end{mydef}

\begin{myprop}[label=prop:geometry-realization]{}
    \hyperref[def:simplicial-top]{特異複体} $S \colon \TOP \lto \sSet,\; X \lmto S(X)$ は\hyperref[def:geometric-realization]{幾何学的実現} $\abs{\mhyphen} \colon \sSet \lto \TOP$ の\hyperref[def:adjoint]{右随伴}である.
\end{myprop}

\begin{proof}
    \hyperref[def:adjoint]{右随伴の定義}を思い出すと,$\forall (K,\, X) \in \Obj{\OP{\sSet} \times \TOP}$ に対して\hyperref[def:nat]{自然同型}
    \begin{align}
        \Hom{\TOP} (\abs{K},\, X) \cong \Hom{\OP{\sSet}} \bigl(K,\, S(X)\bigr)
    \end{align}
    が成り立つことを示せば良い.実際,命題\ref{prop:lim-colim-basic}より
    \begin{align}
        \Hom{\TOP} (\abs{K},\, X) = \Hom{\TOP} \bigl( \underset{[n] \in \yo_{/K}}{\colim} \irm{\Delta}{top}^n,\, X \bigr) \cong \underset{[n] \in \yo_{/K}}{\lim} \Hom{\TOP}(\irm{\Delta}{top}^n,\, X)
    \end{align}
    が,命題\ref{prop:SimpSet-basic}-(3) より
    \begin{align}
        \Hom{\OP{\sSet}} \bigl(K,\, S(X)\bigr) \cong \Hom{\OP{\sSet}} \bigl( \underset{[n] \in \yo_K}{\colim} \Delta^n,\, S(X) \bigr) \cong \underset{[n] \in \yo_{/K}}{\lim} \Hom{\TOP}(\Delta^n,\, X)
    \end{align}
    が言える.
\end{proof}

\subsection{境界・角・背骨}

\begin{mydef}[label=def:horn,breakable]{境界・角・背骨・骨格}
    \begin{itemize}
        \item $\Delta^n \in \Obj{\sSet}$ の\textbf{単体的境界} (simplicial boundary) $\bm{\partial \Delta^n}$ とは,$\Delta^n$ の\hyperref[def:SimpSet]{単体的部分集合}
        \begin{align}
            \partial \Delta^n \colon \OP{\Delta} \lto \SETS
        \end{align}
        であって 
        \begin{align}
            \partial \Delta^n ([k]) &\coloneqq 
            \begin{cases}
                \Delta^n ([k]), &k \neq n \\
                \Delta^n ([k]) \setminus \{\Id_{[n]}\}, &k=n
            \end{cases}
        \end{align}
        を充たすもののこと.
        
        \item 任意の部分集合 $S \subset [n]$ を与える.
        \textbf{$\bm{S}$-角} ($S$-horn) とは,$\Delta^n$ の\hyperref[def:SimpSet]{単体的部分集合}
        \begin{align}
            \bm{\Lambda^n_S} \colon \OP{\Delta} \lto \SETS
        \end{align}
        であって,
        \begin{align}
            \Lambda^n_S ([k]) &\coloneqq \bigl\{\, f \in \Delta^n([k]) \bigm| [n] \setminus \bigl(f([k]) \cup S \bigr) \neq \emptyset  \,\bigr\}
        \end{align}
        を充たすもののこと.特に $\bm{\Lambda^n_j} \coloneqq \Lambda^n_{\{j\}}$ は $0 < j < n$ のとき\textbf{内部角} (inner horn),$j = 0,\, n$ のとき\textbf{外部角} (outer horn) と呼ばれる.

        \item \textbf{背骨} (spine) とは,$\Delta^n$ の\hyperref[def:SimpSet]{単体的部分集合}
        \begin{align}
            \bm{I^n} \colon \OP{\Delta} \lto \SETS
        \end{align}
        であって,
        \begin{align}
            I^n ([k]) \coloneqq \bigl\{\, f \in \Delta^n([k]) \bigm| f([k]) = \{j\}\; \mathrm{or}\; f([k]) = \{j,\, j+1\} \,\bigr\} \subset \Delta^n([k])
        \end{align}
        を充たすもののこと.
        \item \hyperref[def:SimpSet]{単体的集合} $K \colon \OP{\Delta} \lto \SETS$ の\textbf{$\bm{n}$-骨格} ($n$-skelton) とは,
        濃度 $n+1$ 以下の対象からなる $\Delta$ の\hyperref[def:faithful]{充満}部分圏 $i \colon \Delta_{\le n} \hookrightarrow \Delta$ に沿った,関手 $i^*(K) \colon \OP{(\Delta_{\le n})} \lto \SETS$ の\hyperref[def:Kanext]{左Kan拡張} $i_! \bigl(i^* (K)\bigr) \colon \OP{\Delta} \lto \SETS$ のこと.

        \item \hyperref[def:SimpSet]{単体的集合} $K \colon \OP{\Delta} \lto \SETS$ の\textbf{$\bm{n}$-余骨格} ($n$-coskelton) とは,
        濃度 $n+1$ 以下の対象からなる $\Delta$ の\hyperref[def:faithful]{充満}部分圏 $i \colon \Delta_{\le n} \hookrightarrow \Delta$ に沿った,関手 $i^*(K) \colon \OP{(\Delta_{\le n})} \lto \SETS$ の\hyperref[def:Kanext]{右Kan拡張} $i_* \bigl(i^* (K)\bigr) \colon \OP{\Delta} \lto \SETS$ のこと.
    \end{itemize}
    
\end{mydef}

\begin{myexample}[label=ex:horn]{角 $\Lambda^2_j$ の構造}
    \hyperref[def:horn]{角} $\Lambda^2_0 \colon \OP{\Delta} \lto \SETS$ とはどのようなものだろうか.
    まず,\hyperref[def:representable]{表現可能前層} $\Delta^2$ の定義から
    \begin{align}
        \Lambda^2_0([0]) = \Bigl\{\, f \in \Hom{\Delta} ([0],\, [2]) \Bigm| [2] \setminus \bigl( f([0]) \cup \{0\}\bigr) \neq \emptyset \,\Bigr\} 
    \end{align}
    であるが,$\forall f \in \Hom{\Delta} ([0],\, [2])$ は定数写像なので $\Lambda^2_0([0]) = \Hom{\Delta} ([0],\, [2])$ である.これらを
    \begin{align}
        \Lambda^2_0 ([0]) \eqqcolon \bigl\{ \underset{\{0\}}{\bullet},\, \underset{\{1\}}{\bullet},\, \underset{\{2\}}{\bullet} \bigr\}
    \end{align}
    と書く.次に
    \begin{align}
        \Lambda^2_0([1]) = \Bigl\{\, f \in \Hom{\Delta} ([1],\, [2]) \Bigm| [2] \setminus \bigl( f([1]) \cup \{0\}\bigr) \neq \emptyset  \,\Bigr\} 
    \end{align}
    を調べる.6点集合 $\Hom{\Delta} ([1],\, [2])$ の元を全て書き出すと
    \begin{align}
        f_0 \colon 0 &\lmto 0,\; 1 \lmto 0 \\
        f_1 \colon 0 &\lmto 0,\; 1 \lmto 1 \\
        f_2 \colon 0 &\lmto 0,\; 1 \lmto 2 \\
        f_3 \colon 0 &\lmto 1,\; 1 \lmto 1 \\
        f_4 \colon 0 &\lmto 1,\; 1 \lmto 2 \\
        f_5 \colon 0 &\lmto 2,\; 1 \lmto 2
    \end{align}
    であるから,$f_4$ のみが除外される.さらに,
    \begin{align}
        \partial_1^1 (f_0) &= \partial_0^1 (f_0) = \{0\}, \\
        \partial_1^1 (f_1) &=\{0\},\; \partial_0^1 (f_1) = \{1\}, \\
        \partial_1^1 (f_2) &=\{0\},\; \partial_0^1 (f_2) = \{2\}, \\
        \partial_1^1 (f_3) &= \partial_0^1 (f_3) = \{1\}, \\
        \partial_1^1 (f_4) &=\{1\},\; \partial_0^1 (f_4) = \{2\}, \\
        \partial_1^1 (f_5) &= \partial_0^1 (f_5) = \{2\}, \\
    \end{align}
    と計算できるため $f_0 = \sigma^0_0(\{0\}),\; f_3 = \sigma^0_0(\{1\}),\; f_5 = \sigma^0_0(\{2\})$ であり,図\ref{fig:1-simp}に則り
    \begin{align}
        \Lambda^2_0([1])
        = \sigma^0_0 \bigl( \Lambda^2_0([0]) \bigr) \cup
        \bigl\{ 
            \begin{tikzpicture}[baseline={([yshift=-.5ex]current bounding box.center)}]
                \draw[->-=.5] (0,0) node[bullet]{} node[below]{$\{0\}$} -- node[midway,above]{$f_1$} (1,0) node[bullet]{} node[below]{$\{1\}$};
            \end{tikzpicture},\;
            \begin{tikzpicture}[baseline={([yshift=-.5ex]current bounding box.center)}]
                \draw[->-=.5] (0,0) node[bullet]{} node[below]{$\{0\}$} -- node[midway,above]{$f_2$} (1,0) node[bullet]{} node[below]{$\{2\}$};
            \end{tikzpicture}
         \bigr\} 
    \end{align}
    と書ける.
    次に
    \begin{align}
        \Lambda^2_0([2]) = \Bigl\{\, f \in \Hom{\Delta} ([2],\, [2]) \Bigm| [2] \setminus \bigl( f([2]) \cup \{0\}\bigr) \neq \emptyset \,\Bigr\} 
    \end{align}
    であるが,縮退写像の像に含まれない $\Hom{\Delta} ([2],\, [2])$ の元は $\Id_{[2]}$ のみであり,かつ $\Id_{[2]} \notin \Lambda^2_0([2])$ となっている.
    \begin{align}
        \partial_0^2(\Id_{[2]}) = \bigl( 0 \lmto 1,\, 1 \lmto 2 \bigr) = f_4
    \end{align}
    となっていることに注目すべきである.ここから,$\Lambda^n_j([n-1]) \subset \Hom{\Delta}([n-1],\, [n])$ において\textbf{除外される要素がちょうど $\partial_j^n(\Id_{[n-1]})$ なのではないかという予想}が立つのである.
    この予想は系\ref{col:horn-coeq}および\hyperref[lem:Yoneda]{米田の補題}によって正当化される.

     $k \ge 3$ に関する $\Lambda^2_0([k])$ は必ず\hyperref[def:SimpSet]{縮退写像}の像に入ってしまう.
    以上の考察より,図\ref{fig:2-simp}に則り $\Lambda^2_0$ を次のように図示する:
    \begin{align}
        \Lambda^2_0 &=\quad
        \begin{tikzpicture}[baseline={([yshift=-.5ex]current bounding box.center)}]
            \path coordinate[bullet,label=below left:$\{0\}$] (0)
            +(0:1) coordinate[bullet,label=below right:$\{2\}$] (2)
            +(60:1) coordinate[bullet,label=above:$\{1\}$] (1)
            ;
            \draw[->-=.5] (0) --node[midway,above left] {$f_1$} (1);
            \draw[->-=.5] (0) --node[midway,below] {$f_2$} (2);
        \end{tikzpicture}
    \end{align}
    同様に,$\Lambda^2_1,\, \Lambda^2_2$ も次のように図示できる:
    \begin{align}
        \Lambda^2_1 &=\quad 
        \begin{tikzpicture}[baseline={([yshift=-.5ex]current bounding box.center)}]
            \path coordinate[bullet,label=below left:$\{0\}$] (0)
            +(0:1) coordinate[bullet,label=below right:$\{2\}$] (2)
            +(60:1) coordinate[bullet,label=above:$\{1\}$] (1)
            ;
            \draw[->-=.5] (0) -- (1);
            \draw[->-=.5] (1) -- (2);
        \end{tikzpicture} &
        \Lambda^2_2 &=\quad 
        \begin{tikzpicture}[baseline={([yshift=-.5ex]current bounding box.center)}]
            \path coordinate[bullet,label=below left:$\{0\}$] (0)
            +(0:1) coordinate[bullet,label=below right:$\{2\}$] (2)
            +(60:1) coordinate[bullet,label=above:$\{1\}$] (1)
            ;
            \draw[->-=.5] (0) -- (2);
            \draw[->-=.5] (1) -- (2);
        \end{tikzpicture}
    \end{align}
\end{myexample}



\begin{myprop}[label=prop:boundary-coeq]{コイコライザとしての境界}
    \hyperref[def:horn]{境界} $\partial \Delta^n$ は圏\hyperref[def:SimpSet]{$\sSet$}における\hyperref[def:eq-coeq]{コイコライザ}である:
    \begin{center}
        \begin{tikzcd}[row sep=large, column sep=large]
             &\coprod_{0 \le i < j \le n} \Delta^{n-2} \ar[r, shift left, "u"]\ar[r, shift right, "v"'] &\coprod_{0 \le k \le n} \Delta^{n-1} \ar[r, "w"] &\partial \Delta^n
        \end{tikzcd}
    \end{center}
\end{myprop}

\begin{proof}
    $0 \le \forall \textcolor{red}{k} \le n$ に対して,圏 $\sSet$ における射(i.e. \hyperref[def:nat]{自然変換})
    \begin{align}
        u_{\textcolor{red}{k}} \colon \coprod_{0 \le i < \textcolor{red}{k}} \Delta^{n-2} &\lto \Delta^{n-1} \\
        v_{\textcolor{red}{k}} \colon \coprod_{\textcolor{red}{k} < j \le n} \Delta^{n-2} &\lto \Delta^{n-1}
    \end{align}
    を
    \begin{align}
        u_{\textcolor{red}{k}} &\coloneqq \Familyset[\bigg]{u_{\textcolor{red}{k}}{}_{[m]} \colon \coprod_{0 \le i < \textcolor{red}{k}} \Delta^{n-2}\bigl( [m] \bigr)  \lto \Delta^{n-2}\bigl([m]\bigr),\; (i,\, \alpha) \lmto d_i^{n-1} \circ \alpha}{[m] \in \Obj{\OP{\Delta}}} \\
        v_{\textcolor{red}{k}} &\coloneqq \Familyset[\bigg]{u_{\textcolor{red}{k}}{}_{[m]} \colon \coprod_{\textcolor{red}{k} < j \le n} \Delta^{n-2}\bigl( [m] \bigr)  \lto \Delta^{n-2}\bigl([m]\bigr),\; (j,\, \alpha) \lmto d_{j-1}^{n-1} \circ \alpha}{[m] \in \Obj{\OP{\Delta}}}
    \end{align}
    により定義する\footnote{$\coprod_{i} \Delta^{n-2}\bigl( [m] \bigr) = \coprod_{i} \Hom{\Delta}([m],\, [n-2])$ は集合と写像の圏 $\SETS$ における余積なので,集合としては $\bigcup_{i}\bigl\{\, (i,\, \alpha) \bigm| \alpha \in \Hom{\Delta}([m],\, [n-2]) \,\bigr\}$ と1対1対応する.}.
    そして圏 $\sSet$ における2つの\hyperref[def:product-coproduct]{余積}を
    \begin{center}
        \begin{tikzcd}[row sep=large, column sep=large]
            &\coprod_{0 \le i < j \le n} \Delta^{n-2} = \coprod_{0 \le \textcolor{red}{k} \le n}\coprod_{0 \le i < \textcolor{red}{k}} \Delta^{n-2} \ar[r, red, dashed, "\exists! u"] &\coprod_{0 \le \textcolor{red}{k} \le n}\Delta^{n-1} \\
            &\coprod_{0 \le i < \textcolor{red}{k}} \Delta^{n-2} \ar[u, hookrightarrow] \ar[r, "u_{\textcolor{red}{k}}"'] &\Delta^{n-1} \ar[u, hookrightarrow]
        \end{tikzcd}
        \begin{tikzcd}[row sep=large, column sep=large]
            &\coprod_{0 \le i < j \le n} \Delta^{n-2} = \coprod_{0 \le \textcolor{red}{k} \le n}\coprod_{\textcolor{red}{k} < j \le n} \Delta^{n-2} \ar[r, red, dashed, "\exists! v"] &\coprod_{0 \le \textcolor{red}{k} \le n}\Delta^{n-1} \\
            &\coprod_{\textcolor{red}{k} < j \le n} \Delta^{n-2} \ar[u, hookrightarrow] \ar[r, "v_{\textcolor{red}{k}}"'] &\Delta^{n-1} \ar[u, hookrightarrow]
        \end{tikzcd}
    \end{center}
    のようにとる.さらに射
    \begin{align}
        w \colon \coprod_{0 \le \textcolor{red}{k} \le n} \Delta^{n-1} \lto \partial \Delta^n
    \end{align}
    を
    \begin{align}
        w \coloneqq \Familyset[\Big]{w_{[m]} \colon \coprod_{0 \le \textcolor{red}{k} \le n} \Delta^{n-1} \bigl( [m] \bigr) \lto \partial \Delta^n \bigl( [m] \bigr),\; (k,\, \beta) \lmto d^n_k \circ \beta  }{[m] \in \Obj{\OP{\Delta}}}
    \end{align}
    で定義する\footnote{$\forall (k,\, \beta) \in  \coprod_{0 \le \textcolor{red}{k} \le n} \Delta^{n-1} \bigl( [m] \bigr)  =  \coprod_{0 \le \textcolor{red}{k} \le n} \Hom{\Delta}\bigl( [m],\, [n-1] \bigr)$ に対して $w_{[m]}(k,\, \beta) = d^n_k \circ \beta \in \Hom{\Delta}([m],\, [n]) = \Delta^n \bigl( [m] \bigr)$ であり,$d^n_k \in \Hom{\Delta}\bigl( [n-1],\, [n] \bigr)$ は全射でないため $m=n$ のときも $w_{[m]}(k,\, \beta) \in \Delta^n \bigl( [m] \bigr) \setminus \{\Id_{[n]}\}$ が言える.よって $w$ の像は $\partial \Delta^n$ の\hyperref[def:SimpSet]{単体的部分集合}である.}.
    すると $\forall [m] \in \Obj{\OP{\Delta}},\; \forall \bigl( (i<j),\, \alpha \bigr) \in \coprod_{0 \le i < j \le n} \Delta^{n-2}\bigl( [m] \bigr)$ に対して
    \begin{align}
        (w \circ u)_{[m]} \bigl( (i<j),\, \alpha \bigr)
        &= w_{[m]} \Bigl(j,\, u_j{}_{[m]} \bigl( i,\, \alpha \bigr) \Bigr) \\
        &= d^n_j \circ d^{n-1}_i \circ \alpha, \\
        (w \circ v)_{[m]} \bigl( (i<j),\, \alpha \bigr) 
        &= w_{[m]} \Bigl( i,\, v_i{}_{[m]} \bigl( j,\, \alpha \bigr)  \Bigr) \\
        &= d^n_i \circ d^{n-1}_{j-1} \circ \alpha
    \end{align}
    が成り立ち,単体的恒等式\eqref{eq:Simp1}より $w \circ u = w \circ v$ が分かる.
    よって\hyperref[def:eq-coeq]{コイコライザの普遍性}から圏 $\sSet$ の可換図式
    \begin{center}
        \begin{tikzcd}[row sep=large, column sep=large]
            &\coprod_{0 \le i < j \le n} \Delta^{n-2} \ar[r, shift left, "u"]\ar[r, shift right, "v"'] &\coprod_{0 \le k \le n} \Delta^{n-1} \ar[dr, "w"']\ar[r, "q"] &\mathrm{Coeq}(u,\, v)\ar[d, red, dashed, "\exists! \bar{w}"] \\
            & & &\partial \Delta^n
        \end{tikzcd}
    \end{center}
    が成り立つ.後は $\bar{w} \colon \mathrm{Coeq}(u,\, v) \lto \partial \Delta^n$ が\hyperref[def:nat]{自然同値}であることを示せば良い.

    \begin{description}
        \item[\textbf{($\bar{w}$ はエピ射)}] 
        
        $w$ がエピ射であることを示す.そのためには $\forall [m] \in \Obj{\OP{\Delta}}$ を1つ固定し,写像 $w_{[m]} \colon \coprod_{0 \le k \le n} \Delta^{n-1} \bigl( [m] \bigr)  \lto \partial \Delta^n \bigl( [m] \bigr) $ が全射であることを示せば良い.
        
         $\forall \gamma \in \partial \Delta^n\bigl([m]\bigr)$ を1つ固定する.このとき $\gamma \in \Hom{\Delta} \bigl([m],\, [n]\bigr)$ は全射でない.
        i.e. ある $0 \le i \le n$ が存在して,圏 $\Delta$ において $\gamma$ は
        \begin{center}
            \begin{tikzcd}[row sep=large, column sep=large]
                 &{[m]} \ar[dr, red, dashed, "\exists ! \bar{\gamma}"']\ar[r, "\gamma"] &{[n]} \\
                 & &{[n]} \setminus \{i\} \ar[u, hookrightarrow]
            \end{tikzcd}
        \end{center}
        と一意的に分解する.$d_i^n \bigl([n-1]\bigr) = [n] \setminus \{i\}$ かつ $d_i^n$ は単射なので,ある $(i,\, \beta_i) \in \prod_{0 \le k \le n} \Delta^{n-1} \bigl( [m] \bigr)$ が一意的に存在して $\bar{\gamma} = d_i^n \circ \beta_i =  w_{[m]}(i,\, \beta_i)$ が成り立つ.

        \item[\textbf{($\bar{w}$ はモノ射)}] 
        
        $\forall [m] \in \Obj{\OP{\Delta}}$ を1つ固定し,写像 $\bar{w}_{[m]} \colon \mathrm{Coeq}(u,\, v) \bigl( [m] \bigr)  \lto \partial \Delta^n \bigl( [m] \bigr) $ が単射であることを示す.
        $\bar{w}_{[m]}(x) = \bar{w}_{[m]}(y)$ を仮定する.$q \colon \coprod_{0 \le k \le n} \Delta^{n-1} \lto \mathrm{Coeq} (u,\, v)$ はエピなので,$x = q_{[m]} (i,\, \beta_i),\, y = q_{[m]} (j,\, \beta_j)$ を充たす $(i,\, \beta_i),\, (j,\, \beta_j) \in \coprod_{0 \le k \le n} \Delta^{n-1} \bigl( [m] \bigr)$ が存在する.
        コイコライザの普遍性の図式の可換性から $w_{[m]} (i,\, \beta_i) = \bar{w}_{[m]}(x) = \bar{w}_{[m]}(y) = w_{[m]} (j,\, \beta_j)$ が分かる.
        
         $i=j$ ならば $x = y$ は自明なので,$i < j$ とする.このとき,エピ射であることの証明から $\gamma \coloneqq w_{[m]} (i,\, \beta_i) = w_{[m]} (j,\, \beta_j) \in \partial \Delta^n \bigl( [m] \bigr)$ の像は $[n] \setminus \{i < j\}$ に収まっている.i.e. $\gamma$ は
        \begin{center}
            \begin{tikzcd}[row sep=large, column sep=large]
                 &{[m]} \ar[dr, red, dashed, "\bar{\gamma}"']\ar[r, "\gamma"] &{[n]} \\
                 & &{[n]} \setminus \{i < j\} \ar[u, hookrightarrow]
            \end{tikzcd}
        \end{center}
        と分解する.$d^n_j d^{n-1}_i \bigl( [n-2] \bigr) = d^n_i d^{n-1}_{j-1} \bigl( [n-2] \bigr) = {[n]} \setminus \{i < j\}$ なので,ある $\bigl( (i < j),\, \alpha \bigr) \in \coprod_{0 \le k < l \le n} \Delta^{n-2} \bigl( [m] \bigr)$ が存在して $(i,\, \beta_i) = u_{[m]} \bigl((i < j),\, \gamma\bigr),\; (j,\, \beta_j) = v_{[m]} \bigl((i < j),\, \gamma\bigr)$ と書ける.よって
        \begin{align}
            x &= q_{[m]} (i,\, \beta_i) = (q \circ u)_{[m]} \bigl( (i < j),\, \gamma \bigr) = (q \circ v)_{[m]} \bigl( (i < j),\, \gamma \bigr) = q_{[m]} (j,\, \beta_j) = y
        \end{align}
        が言えた.
    \end{description}
\end{proof}

\begin{mycol}[label=col:coundary-coeq]{境界の公式}
    $\forall K \in \Obj{\sSet}$ に対して,写像
    \begin{align}
        \Hom{\sSet} (\partial \Delta^n,\, K) &\lto \biggl\{\, (\sigma_0,\, \dots,\, \sigma_n) \in \prod_{0 \le k \le n} K_{n-1} \biggm| 0 \le \forall i < j \le n,\;  \partial^{n-1}_i (\sigma_j) = \partial^{n-1}_{j-1} (\sigma_i) \,\biggr\}, \\
        f &\lmto \bigl(f_{[n-1]} \circ d^n_0{}^*(\Id_{[n]}),\, \dots,\, f_{[n-1]} \circ d^n_n{}^*(\Id_{[n]}) \bigr) 
    \end{align}
    は全単射である.
\end{mycol}

\begin{proof}
    命題\ref{prop:boundary-coeq}より,集合
    \begin{align}
        &X\coloneqq\biggl\{\, \bar{f} \in \Hom{\sSet}\Bigl(\coprod_{0 \le k \le n} \Delta^{n-1},\, K \Bigr) \biggm| \bar{f} \circ u = \bar{f} \circ v \,\biggr\}  \\
        &= \biggl\{\, \bar{f} \in \Hom{\sSet}\Bigl(\coprod_{0 \le k \le n} \Delta^{n-1},\, K \Bigr) \biggm| \substack{\forall [m] \in \Obj{\OP{\Delta}},\, \forall \bigl((i < j),\, \alpha\bigr) \in \coprod_{0 \le k < l \le n} \Delta^{n-2} \bigl( [m] \bigr),\\ \bar{f}_{[m]} \bigl( j,\, d^{n-1}_i \circ \alpha \bigr)  = \bar{f}_{[m]} \bigl( i,\, d^{n-1}_{j-1} \circ \alpha \bigr)}  \,\biggr\}
    \end{align}
    の任意の元 $\bar{f}$ に対してコイコライザの普遍性の可換図式
    \begin{center}
        \begin{tikzcd}[row sep=large, column sep=large]
            &\coprod_{0 \le i < j \le n} \Delta^{n-2} \ar[r, shift left, "u"]\ar[r, shift right, "v"'] &\coprod_{0 \le k \le n} \Delta^{n-1} \ar[dr, "\bar{f}"']\ar[r, "q"] &\partial \Delta^n \ar[d, red, dashed, "\exists! f"] \\
            & & &K
        \end{tikzcd}
    \end{center}
    が成り立つ.i.e. 写像
    \begin{align}
        \Hom{\sSet} (\partial \Delta^n,\, K) &\lto X, \\
        f &\lmto f \circ w = \Familyset[\bigg]{\Dpmember[\big]{(f_{[m]} \circ d^n_k)_*}{0 \le k \le n} \colon \coprod_{0 \le k \le n} \Delta^{n-1} \bigl( [m] \bigr) \lto K_m}{[m] \in \Obj{\OP{\Delta}}}
    \end{align}
    は全単射である.命題\ref{prop:lim-colim-basic}-(2) と\hyperref[lem:Yoneda]{米田の補題}から
    \begin{align}
        \Hom{\sSet}\Bigl(\coprod_{0 \le k \le n} \Delta^{n-1},\, K \Bigr) \cong \prod_{0 \le k \le n} \Hom{\sSet} \bigl( \Delta^{n-1},\, K \bigr) \cong \prod_{0 \le k \le n} K_{n-1}
    \end{align}
    が言えるので,示された.
\end{proof}

\begin{myprop}[label=prop:horn-coeq]{コイコライザとしての角}
    \hyperref[def:horn]{角} $\Lambda^n_i$ は圏\hyperref[def:SimpSet]{$\sSet$}における\hyperref[def:eq-coeq]{コイコライザ}である:
    \begin{center}
        \begin{tikzcd}[row sep=large, column sep=large]
             &\coprod_{\substack{j,\, k \in [n] \setminus \{i\} \\ j < k}} \Delta^{n-2} \ar[r, shift left, "u"]\ar[r, shift right, "v"'] &\coprod_{l \in [n] \setminus \{i\}} \Delta^{n-1} \ar[r, "w"] &\Lambda_i^n
        \end{tikzcd}
    \end{center}
\end{myprop}

\begin{proof}
    命題\ref{prop:boundary-coeq}とほぼ同様である.
\end{proof}

\begin{mycol}[label=col:horn-coeq]{角の公式}
    $\forall K \in \Obj{\sSet}$ に対して,写像
    \begin{align}
        \Hom{\sSet} (\Lambda^n_i,\, K) &\lto \biggl\{\, (\sigma_0,\, \dots,\,\sigma_{i-1},\, \sigma_{i+1},\, \dots,\, \sigma_n) \in \prod_{k \in [n] \setminus \{i\}} K_{n-1} \biggm| \substack{\forall j,\, k \in [n] \setminus \{i\}\; \mathrm{s.t.}\;  j < k,\\  \partial^{n-1}_j (\sigma_k) = \partial^{n-1}_{k-1} (\sigma_j)} \,\biggr\}, \\
        f &\lmto \bigl(f_{[n-1]} \circ d^n_0{}^*(\Id_{[n]}),\, \dots,\, \widehat{f_{[n-1]}\circ d^n_{i}{}^*(\Id_{[n]})},\, \dots,\, f_{[n-1]} \circ d^n_n{}^*(\Id_{[n]}) \bigr) 
    \end{align}
    は全単射である.ただし,$\widehat{\cdot}$ は $\cdot$ を除外することを意味する.
\end{mycol}

\begin{proof}
    系\ref{col:coundary-coeq}と同様.
\end{proof}


\begin{myprop}[label=prop:horn-spine-basic]{角と背骨の幾何学的実現}
    \hyperref[def:geometric-realization]{幾何学的実現}は\hyperref[def:horn]{角,背骨}を保つ.
\end{myprop}

\begin{proof}
    $\abs{\Delta^n} = \irm{\Delta}{top}^n$ であることに注意する.
\end{proof}

\section{脈体・$\infty$-亜群・$(\infty,\, 1)$-圏}

$[n] \in \Obj{\Delta}$ に対して,
\begin{itemize}
    \item $\forall i \in [n]$ を対象とする
    \item Hom集合は
    \begin{align}
        \Hom{[n]} (i,\, j)
        \coloneqq
        \begin{cases}
            \{\mathrm{pt}_{ij}\}, &i \le j \\
            \emptyset, &i > j
        \end{cases}
    \end{align}
    とする.ただし,$\mathrm{pt}_{ii} = \Id_i$ である.
    \item 射の合成は $0 \le i \le j \le k \le n$ に対して
    \begin{align}
        \mathrm{pt}_{jk} \circ \mathrm{pt}_{ij} \coloneqq \mathrm{pt}_{ik}
    \end{align}
    と定義する.
\end{itemize}
ことにより $[n]$ 自身が圏になる.

\subsection{脈体}

後に示す命題\ref{prop:nerve-grpd}により,通常の\hyperref[def:category]{圏}は\hyperref[def:SimpSet]{単体的集合}と同一視できる.

\begin{mydef}[label=def:nerve,breakable]{脈体}
        圏 $\Cat{C}$ の\textbf{脈体} (nerve) とは,以下で定義される\hyperref[def:SimpSet]{単体的集合}
        \begin{align}
            \Ner (\Cat{C}) \colon \OP{\Delta} \lto \SETS
        \end{align}
        のことを言う:
        \begin{itemize}
            \item $\forall [n] \in \Obj{\OP{\Delta}}$ に対して
            \begin{align}
                \Ner (\Cat{C}) ([n]) = \mathrm{Fun} ([n],\, \Cat{C})
            \end{align}
            を対応付ける
            \item \hyperref[def:simplex-cat]{圏 $\OP{\Delta}$ }における任意の射 $[n] \xrightarrow{\alpha} [m]$ に対して\underline{写像}
            \begin{align}
                \ner{\Cat{C}}(\alpha) \coloneqq \alpha^* \colon \mathrm{Fun}([n],\, \Cat{C}) &\lto \mathrm{Fun}([m],\, \Cat{C}),\\ 
                X &\lmto X \circ \alpha
            \end{align}
            を対応付ける
        \end{itemize}

        \tcblower

        \textbf{脈体関手} (nerve functor) とは,以下で定義される関手
        \begin{align}
            \Ner  \colon \CAT \lto \sSet
        \end{align}
        のことを言う:
        \begin{itemize}
            \item $\forall \Cat{C} \in \Obj{\Cat{C}}$ に対して $\Ner (\Cat{C}) \in \Obj{\sSet}$ を対応づける.
            \item 任意の関手 $\Cat{C} \xrightarrow{F} \Cat{D}$ に対して\hyperref[def:nat]{自然変換}
            \begin{align}
                \ner{F} \colon &\ner{\Cat{C}} \Longrightarrow \ner{\Cat{D}}, \\
                \WHERE \ner{F} &\coloneqq \Familyset[\big]{\Ner (F)_{[n]} \colon \mathrm{Fun}([n],\, \Cat{C}) \lto \mathrm{Fun}([n],\, \Cat{D}),\; X \lmto F \circ X}{[n] \in \Obj{\OP{\Delta}}}
            \end{align}
            を対応付ける
        \end{itemize}
\end{mydef}

定義\ref{def:SimpSet}の略記に倣い,$\Ner (\Cat{C})_n \coloneqq \Ner  (\Cat{C})([n]) \in \Obj{\SETS}$ と略記する.
このとき,圏 $[n]$ の定義を思い出すと,集合の要素 $X \in \Ner (\Cat{C})_n$ は以下のデータからなる:
\begin{itemize}
    \item 圏 $\Cat{C}$ の対象の族
    \begin{align}
        \Familyset[big]{X_i \coloneqq X (i) \in \Obj{\Cat{C}} }{0 \le i \le n}
    \end{align}
    \item 圏 $\Cat{C}$ の射の族
    \begin{align}
        \Familyset[big]{f_{ij} \coloneqq X(\mathrm{pt}_{ij}) \colon X(i) \lto X(j)}{0 \le i \le j \le n}
    \end{align}
\end{itemize}
$X \colon [n] \lto \Cat{C}$ は関手であるから,$0 \le i < j \le n$ に対して
\begin{align}
    f_{ii} &= X(\Id_i) = \Id_{X_i}, \\
    f_{ij} &= X(\mathrm{pt}_{j-1,j} \circ \cdots \mathrm{pt}_{i+1,i+2}\circ\mathrm{pt}_{i,i+1}) \\
    &= f_{j-1,j} \circ \cdots \circ f_{i+1,i+2} \circ f_{i,i+1}
\end{align}
が成り立つ.故に,$X$ を特徴付けるには,$f_i \coloneqq f_{i,i-1}$ とおいて $\Cat{C}$ の図式
\begin{align}
    \label{eq:nerve-spine}
    X_0 \xrightarrow{f_1} X_1 \xrightarrow{f_2} \cdots \xrightarrow{f_n} X_n
\end{align}
を指定することが必要十分である.このことから,脈体のmorphism-morphism対応は図式の対応
\begin{align}
    (X_0 \to X_1 \to \cdots \to X_n) \lmto (X_{\alpha(0)} \to X_{\alpha(1)} \to \cdots \to X_{\alpha(m)})
\end{align}
と理解できる.

\begin{myexample}[label=ex:nerve-face]{面写像}
    $\forall X \in \Ner (\Cat{C})_n$ を $\Cat{C}$ の図式
    \begin{align}
        X_0 \xrightarrow{f_1} X_1 \xrightarrow{f_2} \cdots \xrightarrow{f_n} X_n
    \end{align}
    として指定する.
    このとき\hyperref[def:simplex-cat]{面写像} $d^n_i \in \Hom{\OP{\Delta}}([n],\, [n-1])$ は,\hyperref[def:nerve]{脈体}によって写像
    \begin{align}
        N(\Cat{C})(d^n_i)\colon 
        X &\lmto \bigl( X_0 \xrightarrow{f_1} \cdots \xrightarrow{f_{i-1}} X_{i-1} \xrightarrow{f_{i+1} \circ f_{i}} X_{i+1} \xrightarrow{f_{i+2}} \cdots \xrightarrow{f_n} X_n \bigr) 
    \end{align}
    と対応付く.
\end{myexample}

\begin{myexample}[label=ex:nerve-degen]{縮退写像}
    $\forall X \in \Ner (\Cat{C})_n$ を $\Cat{C}$ の図式
    \begin{align}
        X_0 \xrightarrow{f_1} X_1 \xrightarrow{f_2} \cdots \xrightarrow{f_n} X_n
    \end{align}
    として指定する.
    このとき\hyperref[def:simplex-cat]{縮退写像} $s^n_i \in \Hom{\OP{\Delta}}([n],\, [n+1])$ は,\hyperref[def:nerve]{脈体}によって写像
    \begin{align}
        N(\Cat{C})(s^n_i)\colon 
        X &\lmto \bigl( X_0 \xrightarrow{f_1} \cdots \xrightarrow{f_{i}} X_{i} \xrightarrow{\Id_{X_i}} X_{i} \xrightarrow{f_{i+1}} \cdots \xrightarrow{f_n} X_{n} \bigr) 
    \end{align}
    と対応付く.
\end{myexample}

\begin{myprop}[label=prop:nerve]{脈体関手は忠実充満}
    \hyperref[def:nerve]{脈体関手}は\hyperref[def:faithful]{忠実充満関手}である.
\end{myprop}
\begin{proof}
    \begin{align}
        \theta \colon \Hom{\CAT}(\Cat{C},\, \Cat{D}) \lto \Hom{\sSet} \bigl( \Ner (\Cat{C}),\, \Ner (\Cat{D}) \bigr),\; F \lmto \Ner (F)
    \end{align}
    が全単射であることを示せば良い.

    \begin{description}
        \item[\textbf{単射}] 
        
        関手 $F,\, G \colon \Cat{C} \lto \Cat{D}$ が $\ner{F} = \ner{G}$ を充たすとする.
        \eqref{eq:nerve-spine}により $\Cat{C}$ における任意の図式
        \begin{align}
            X \xrightarrow{f} Y
        \end{align}
        を $\ner{\Cat{C}}_1$ の元と見做すことができるが,仮定より圏 $\Cat{D}$ において
        \begin{align}
            &\ner{F}_{[1]}(X \xrightarrow{f} Y) = \ner{G}_{[1]}(X \xrightarrow{f} Y) \\
            \IFF & \bigl(F(X) \xrightarrow{F(f)} F(Y)\bigr) =  \bigl(G(X) \xrightarrow{G(f)} G(Y)\bigr)
        \end{align}
        が成り立つ.i.e. $F = G$ である.
        \item[\textbf{全射}] 
        
        $\forall f \in \Hom{\sSet} \bigl( \Ner (\Cat{C}),\, \Ner (\Cat{D}) \bigr)$ を1つ固定する.$f$ は\hyperref[def:nat]{自然変換}だから,$\forall n \ge 0$ に対して\hyperref[def:nat]{自然変換} $f_{[n]} \colon \Ner (\Cat{C})_n \lto \Ner (\Cat{D})_n$ が定まる.
        \eqref{eq:nerve-spine}より $\forall X,\, Y \in \Obj{\Cat{C}}$ を $\ner{\Cat{C}}_0$ の要素と見做し,圏 $\Cat{C}$ における任意の射 $X \xrightarrow{u} Y$ を $\ner{\Cat{C}}_1$ の要素と見做すことができる.すると自然変換 $f$ により
        \begin{align}
            f_{[0]} (X),\, f_{[0]} (Y) \in \Obj{\Cat{D}}
        \end{align}
        が対応付く.その上 $f$ が自然変換であることから\hyperref[def:simplex-cat]{面写像} $d_i^1 \colon [0] \lto [1]$ との間に可換図式
        \begin{center}
            \begin{tikzcd}[row sep=large, column sep=large]
                &\ner{\Cat{C}}_1 \ar[r, "f_{[1]}"]\ar[d, "\ner{\Cat{C}}(d_i^1)=\partial_i^1"'] &\ner{\Cat{D}}_1 \ar[d, "\ner{\Cat{D}}(d_i^1)=\partial_i^1"]\\
                &\ner{\Cat{C}}_0 \ar[r, "f_{[0]}"'] &\ner{\Cat{D}}_0
            \end{tikzcd}
        \end{center}
        が成り立つ.よって
        \begin{align}
            \partial_0^1 \circ f_{[1]}\bigl( X \xrightarrow{u} Y \bigr) 
            &= f_{[0]} \circ \partial_0^1 \bigl( X \xrightarrow{u} Y \bigr)  \\
            &= f_{[0]} (Y), \\
            \partial_1^1 \circ f_{[1]}\bigl( X \xrightarrow{u} Y \bigr) 
            &= f_{[0]} \circ \partial_1^1 \bigl( X \xrightarrow{u} Y \bigr) \\
            &= f_{[0]} (X)
        \end{align}
        が言える.i.e. $f_{[1]}(u)$ は圏 $\Cat{D}$ における射 $f_{[0]}(X) \xrightarrow{f_{[1]}(u)} f_{[0]} (Y)$ である.
        ここで,対応 $F_f \colon \Cat{C} \lto \Cat{D}$ を
        \begin{itemize}
            \item $\forall X \in \Obj{\Cat{C}}$ に対して $f_{[0]}(X) \in \Obj{\Cat{D}}$ を対応付ける
            \item $X \xrightarrow{\forall u} Y$ に対して $f_{[0]}(X) \xrightarrow{f_{[1]}(u)} f_{[0]}(Y)$ を対応付ける
        \end{itemize}
        ものとして定義する.もし $F_f$ が関手ならば明らかに $\theta(F_f) = f$ であるから,$F_f$ が\hyperref[def:functor]{関手}であることを示せば良い:
        \begin{description}
            \item[\textbf{(fun-1)}] 
            
                \eqref{eq:nerve-spine}により圏 $\Cat{C}$ における図式
                \begin{align}
                    X \xrightarrow{u} Y \xrightarrow{v} Z
                \end{align}
                を $\ner{\Cat{C}}_2$ の要素と見做すことができる.$f$ は\hyperref[def:nat]{自然変換}なので,\hyperref[def:simplex-cat]{面写像} $d_i^2 \colon [2] \lto [1]$ について可換図式
                \begin{center}
                    \begin{tikzcd}[row sep=large, column sep=large]
                        &\ner{\Cat{C}}_2 \ar[r, "f_{[2]}"]\ar[d, "\ner{\Cat{C}}(d_i^2)=\partial_i^2"'] &\ner{\Cat{D}}_2 \ar[d, "\ner{\Cat{D}}(d_i^2)=\partial_i^2"]\\
                        &\ner{\Cat{C}}_1 \ar[r, "f_{[1]}"'] &\ner{\Cat{D}}_1
                    \end{tikzcd}
                \end{center}
                が成り立つ.よって
                \begin{align}
                    \partial_0^2 \circ f_{[2]} (X \xrightarrow{u} Y \xrightarrow{v} Z)
                    &= f_{[1]} \circ \partial_0^2  (X \xrightarrow{u} Y \xrightarrow{v} Z) \\
                    &= f_{[1]} (Y \xrightarrow{v} Z) \\
                    &= \bigl(F_f(Y) \xrightarrow{F_f(v)} F_f(Z)\bigr), \\
                    \partial_2^2 \circ f_{[2]} (X \xrightarrow{u} Y \xrightarrow{v} Z)
                    &= f_{[1]} \circ \partial_2^2  (X \xrightarrow{u} Y \xrightarrow{v} Z) \\
                    &= f_{[1]} (X \xrightarrow{u} Y) \\
                    &= \bigl(F_f(X) \xrightarrow{F_f(u)} F_f(Y)\bigr)
                \end{align}
                が分かった.i.e.
                \begin{align}
                    f_{[2]} (X \xrightarrow{u} Y \xrightarrow{v} Z)
                    &=\bigl( F_f(X) \xrightarrow{F_f(u)} F_f(Y) \xrightarrow{F_f(v)} F_f(Z)\bigr)
                \end{align}
                である.故に
                \begin{align}
                    \bigl( F_f(X) \xrightarrow{F_f(v) \circ F_f(u)} F_f(Y)\bigr) 
                    &= \partial_1^2 \circ f_{[2]} \bigl( X \xrightarrow{u} Y \xrightarrow{v} Z \bigr) \\
                    &= f_{[1]} \circ \partial_1^2 \bigl( X \xrightarrow{u} Y \xrightarrow{v} Z \bigr) \\
                    &= f_{[1]} \bigl( X \xrightarrow{v \circ u} Z \bigr) \\
                    &= \bigl(F_f(X) \xrightarrow{F_f(v\circ u)} F_f(Y)\bigr)
                \end{align}
                i.e.
                \begin{align}
                    F(v\circ u) = F(v) \circ F(u)
                \end{align}
                が示された.

            \item[\textbf{(fun-2)}] 
            
                $f$ が\hyperref[def:nat]{自然変換}なので\hyperref[def:simplex-cat]{縮退写像} $s_0^0 \colon [1] \lto [0]$ について可換図式
                \begin{center}
                    \begin{tikzcd}[row sep=large, column sep=large]
                        &\ner{\Cat{C}}_{1} \ar[from=d, "\Ner (\Cat{C})(s_0^0)=\sigma_0^0"] \ar[r, "f_{[1]}"] &\ner{\Cat{D}}_1 \ar[from=d, "\ner{\Cat{D}}(s_0^0)=\sigma_0^0"'] \\
                        &\ner{\Cat{C}}_{0} \ar[r, "f_{[0]}"] &\ner{\Cat{D}}_0
                    \end{tikzcd}
                \end{center}
                が成り立つ.\eqref{eq:nerve-spine}を使うと,これは $\forall X \in \Obj{\Cat{C}} = \ner{\Cat{C}}_0$ に対して
                \begin{align}
                    f_{[1]} \circ \sigma_0^0 (X) 
                    &= f_{[1]} (X \xrightarrow{\Id_X} X) \\
                    &= \bigl(F_f(X) \xrightarrow{F_f(\Id_X)} F_f(X)\bigr) \\
                    &= \sigma_0^0 \circ f_{[0]} (X) \\
                    &= \bigl(F_f(X) \xrightarrow{\Id_{F_f(X)}} F_f(X)\bigr)
                \end{align}
                を意味するので
                \begin{align}
                    F_f(\Id_X) = \Id_{F(X)}
                \end{align}
                が示された.
        \end{description}
    \end{description}
    
\end{proof}

\subsection{$\infty$-亜群・$(\infty,\, 1)$-圏}

\begin{mydef}[label=def:KanCplx,breakable]{Kan条件}
    \textbf{Kan複体} (Kan complex) とは,\hyperref[def:SimpSet]{単体的集合}
    \begin{align}
        K \colon \OP{\Delta} \lto \SETS
    \end{align}
    であって以下の性質を充たすもののこと:

    \begin{description}
        \item[\textbf{(Kan)}] 
        
        $\forall n \ge 1,\; 0 \le \forall j \le n$ および $\forall \textcolor{blue}{f} \in \Hom{\sSet}(\Lambda^n_j,\, K)$に対して,以下の図式を可換にする\hyperref[def:nat]{自然変換} $\textcolor{red}{u} \in \Hom{\sSet}(\Delta^n,\, K)$ が存在する:
        \begin{center}
            \begin{tikzcd}[row sep=large, column sep=large]
                &\Lambda^n_j \ar[r, blue, "f"]\ar[d, hookrightarrow] &K \\
                &\Delta^n \ar[ur, dashed, red, "u"] &
            \end{tikzcd}
        \end{center}
        
    \end{description}
    
    \tcblower

    単体的集合であって,\hyperref[def:horn]{内部角} i.e. $\forall n \ge 2,\, 0 < \forall j < n$ についてのみ \textsf{\textbf{(Kan)}} を充たすもののことを\textbf{弱Kan複体} (weak Kan complex) と呼ぶ.
\end{mydef}

\begin{mydef}[label=def:infinity-1]{{$(\infty,\, 1)$}-圏}
    \begin{itemize}
        \item $\bm{(\infty,\, 1)}$\textbf{-圏}\footnote{\textbf{擬圏} (quasi-category) と呼ばれることもある.}とは,\hyperref[def:KanCplx]{弱Kan複体}のこと.
        $(\infty,\, 1)$-圏の関手とは,\hyperref[def:SimpSet]{$\sSet$}の射のこと.
        \item $\bm{\infty}$\textbf{-groupoid}\footnote{より現代的には\textbf{アニマ} (anima) と呼ばれることもある.}とは,\hyperref[def:KanCplx]{Kan複体}のこと.
    \end{itemize}
    \tcblower
    \begin{itemize}
        \item $(\infty,\, 1)$-圏 $\Cat{C}$ の\textbf{対象}\footnote{$\bm{0}$\textbf{-射},$\bm{0}$\textbf{-セル} ($0$-cell) とも言う.} (object) とは,$\Cat{C}$ の\hyperref[def:SimpSet]{$0$-単体} $\Cat{C}_0$ の元のこと.
        \item $(\infty,\, 1)$-圏 $\Cat{C}$ の\textbf{射}\footnote{\textbf{1-射},\textbf{1-セル} (1-cell) とも言う.同様に,$n$-単体 $\Cat{C}_n$ の元のことを\textbf{$\bm{n}$-射} ($n$-morphism),\textbf{$\bm{n}$-セル} ($n$-cell) と呼ぶ.} (morphism) とは,$\Cat{C}$ の\hyperref[def:SimpSet]{$1$-単体} $\Cat{C}_1$ の元のこと.
        \item $(\infty,\, 1)$-圏の射 $f \in \Cat{C}_1$ に対して定まる $(\infty,\, 1)$-圏の対象 $\partial^1_1(f),\, \partial^1_0(f) \in \Cat{C}_0$ のことをそれぞれ $f$ の\textbf{始点} (source),\textbf{終点} (target) と呼ぶ.
        \item $(\infty,\, 1)$-圏 $\Cat{C},\, \Cat{D}$ の間の\textbf{関手} $F \colon \Cat{C} \lto \Cat{D}$ とは,\hyperref[def:nat]{自然変換} $F \in \Hom{\sSet}(\Cat{C},\, \Cat{D})$ のこと.
    \end{itemize}
\end{mydef}

\begin{marker}
    一般に,$(\infty,\, 1)$-圏のことを単に $\bm{\infty}$\textbf{-圏}と呼ぶことが多い.
\end{marker}

\begin{myexample}[label=def:infty-Fun]{{$(\infty,\, 1)$}-圏の関手が成す {$\infty,\, 1$}-圏}
    \hyperref[def:infinity-1]{$(\infty,\, 1)$-圏} $K,\, L$ を与える.
    このとき,
    % 自然変換の集合 $\Hom{\sSet} (K,\, L)$ を\hyperref[def:infinity-1]{$(\infty,\, 1)$-圏}と見做すことができる.
    \hyperref[def:SimpSet]{単体的集合}
    \begin{align}
        \FUN (K,\, L) \colon \OP{\Delta} &\lto \SETS, \\
        [n] &\lmto \Hom{\sSet} (\Delta^n \times K,\, L), \\
        \bigl( [m] \xrightarrow{\alpha} [n] \bigr) &\lmto \bigl(f \mapsto f \circ (\alpha_* \times \Id_K)\bigr)
    \end{align}
    は $(\infty,\, 1)$-圏であることが知られている~\cite[\href{https://kerodon.net/tag/0066}{Tag 0066}]{kerodon}.
\end{myexample}


\hyperref[def:infinity-1]{$(\infty,\, 1)$-圏}が通常の\hyperref[def:category]{圏}の一般化であることは,次の定理(および命題\ref{prop:nerve-grpd})から分かる.

\begin{mytheo}[label=thm:KanCplx]{Kan条件と脈体}
    任意の\hyperref[def:SimpSet]{単体的集合} $K \colon \OP{\Delta} \lto \SETS$ に対して以下は同値である:
    \begin{enumerate}
        % \item $K$ は\hyperref[def:horn]{背骨}の包含 $I^n \hookrightarrow \Delta^n$ を一意に持ち上げる.
        \item $K$ はある圏の\hyperref[def:nerve]{脈体}と\hyperref[def:iso]{同型}である.
        \item $K$ は\hyperref[def:KanCplx]{弱Kan条件}を充たす\underline{一意解}を持つ
    \end{enumerate}
\end{mytheo}

\begin{proof}
    % ~\cite[p.20, Theorem 1.1.52]{Land2021infinity}を参照.
    \begin{description}
        \item[\textbf{(1)$\bm{\Longrightarrow}$(2)}] 
        
        ある圏 $\Cat{C}$ が存在して $K \cong \Ner (\Cat{C})$ だとする.
        $\forall f \in \Hom{\sSet}\bigl(\Lambda^n_j,\, \Ner (\Cat{C})\bigr)$ を1つ与える.このとき $0 < \forall j < \forall n$ に対して $f$ が $u \in \Hom{\sSet} \bigl( \Delta^n,\, \Ner (\Cat{C}) \bigr)$ へ一意的に拡張できることを示せば良い.

         $0 \le \forall k \le n$ に対して $U_k \coloneqq f_{[0]}(\{k\})$ とおく\footnote{\exref{ex:horn}より,$\{k\} \in \Lambda^n_j ([0])$ である.}.
        さらに $0 < \forall k \le n$ に対して
        \begin{align}
            g_k \coloneqq f_{[1]} \bigl( 
                \begin{tikzpicture}[baseline={([yshift=-.5ex]current bounding box.center)}]
                    \draw[->-=.5] (0,0) node[bullet]{} node[below]{$\{k-1\}$} -- (1,0) node[bullet]{}node[below]{$\{k\}$};
                \end{tikzpicture}
             \bigr) \in \Hom{\Cat{C}} (U_{k-1},\, U_k)
        \end{align}
        とおくと,$\Cat{C}$ の図式
        \begin{align}
            U_0 \xrightarrow{g_1} U_1 \xrightarrow{g_2} \cdots \xrightarrow{g_{n}} U_n
        \end{align}
        が定まる.ここから\eqref{eq:nerve-spine}の方法で $U \in \Ner (\Cat{C})_n$ が一意的に定まり,\hyperref[lem:Yoneda]{米田の補題}により対応する $u \in \Hom{\sSet} \bigl(\Delta^n,\, \Ner (\Cat{C})\bigr)$ が一意的に定まるが,構成からこれが所望の $u$ である\cite[\href{https://kerodon.net/tag/0032}{Tag 0032}]{kerodon}.
        % .この $u$ が\hyperref[def:KanCplx]{弱Kan条件}を充たすことを示す(なお,弱Kan条件を充たす $u$ の一意性は明らかである.実際,$u'|_{\Lambda^n_j} = f$ を充たすもう一つの $u'$ が存在したとすると,$u'$ もまた $U$ に対応付けられるため $u=u'$ だと分かる).
        % $n = 2$ のときは,$0 < j < 2$ なので $j=1$ となる.\exref{ex:horn}より,$0 < \forall k \le 2$ に対して $g_k \coloneqq f_{[1]}(\{k-1\} \lto \{k\})$ とおくと,
        % \begin{align}
            
        % \end{align}
        
        \item[\textbf{(1)$\bm{\Longleftarrow}$(2)}] 
        
        圏 $\Cat{C}$ を以下のように構成する:
        \begin{itemize}
            \item $\Obj{\Cat{C}} \coloneqq K_0$ 
            \item $\Hom{\Cat{C}}(C,\, D) \coloneqq \bigl\{\, f \in K_1 \bigm| \partial^1_1(f) = C,\; \partial^1_0(f) = D \,\bigr\}$
            \item $\forall C \in \Obj{\Cat{C}} = K_0$ に対して\footnote{単体的恒等式\eqref{eq:Simp4}により $\Id_C \in \Hom{\Cat{C}}(C,\, C)$ が分かる.} $\Id_C \coloneqq \sigma^0_0(C)$
            \item $\forall (g,\, f) \in \Hom{\Cat{C}}(D,\, E) \times \Hom{\Cat{C}}(C,\, D)$ に対して射の合成を定義するために,\hyperref[def:KanCplx]{弱Kan条件}
            \begin{center}
                \begin{tikzcd}[row sep=large, column sep=large]
                    &\Lambda^2_1 \ar[r, blue, "{(g,\, f)}"]\ar[d, hookrightarrow] &K \\
                    &\Delta^2 \ar[ur, dashed, red, "\exists u"] &
                \end{tikzcd}
            \end{center}
            および系\ref{col:horn-coeq}を用いる.
            具体的には次のようにする:
            \begin{description}
                \item[\textbf{(STEP1)}]  
                
                $(g,\, f) \in K_1 \times K_1$ かつ $\partial^1_1(g) = \partial^1_{0}(f)$ が成り立つので系\ref{col:horn-coeq}が使えて,$(g,\, f) \in \Hom{\sSet}(\Lambda^2_1,\, K)$ と見做せる.
                \item[\textbf{(STEP2)}]  
                
                仮定より,$(g,\, f) \in \Hom{\sSet}(\Lambda^2_1,\, K)$ に対する弱Kan条件の\underline{一意解} $u \in \Hom{\sSet}(\Delta^2,\, K) \cong K_2$ が存在する.
                \item[\textbf{(STEP3)}]  
                
                $u|_{\Lambda^2_1}$ に対して再度系\ref{col:horn-coeq}を適用することで,弱Kan条件の図式の可換性は $\partial^2_0(u) = g,\; \partial^2_2(u) = f$ を意味していることがわかる.このことから,$f$ と $g$ の合成を $g \circ f \coloneqq \partial^2_1(u)$ と定義する\footnote{単体的恒等式\eqref{eq:Simp1}より,$g \circ f \in \Hom{\Cat{C}}(C,\, E)$ が分かる.}.
            \end{description}
        \end{itemize}
        このように構成したデータの組み $(\Obj{\Cat{C}},\, \Hom{\Cat{C}}(\, \mhyphen\,,\,\mhyphen \,),\, \Id,\, \circ)$ が\hyperref[def:category]{圏}になっていることを示そう.
        \begin{description}
            \item[\textbf{(unitality)}] 
            
            $\forall f \in \Hom{\Cat{C}}(C,\, D)$ をとる.ここで $u \coloneqq s^1_1(f) \in K_2$ に対して単体的恒等式\eqref{eq:Simp2}, \eqref{eq:Simp4}を用いると
            \begin{align}
                \partial^2_0 (u) &= \partial^2_0 s^1_1 (f) = s^0_0 \partial^1_0(f) = \Id_D , \\
                \partial^2_1 (u) &= \partial^2_1 s^1_1 (f) = f, \\
                \partial^2_2 (u) &= \partial^2_2 s^1_1 (f) = f, \\
            \end{align}
            が分かる.i.e. $\Id_D \circ f = f$ である.$v \coloneqq s^1_0(f) \in K_2$ に対して単体的恒等式\eqref{eq:Simp3}, \eqref{eq:Simp4}を用いると
            \begin{align}
                \partial^2_0 (u) &= \partial^2_0 s^1_0 (f) = f, \\
                \partial^2_1 (u) &= \partial^2_1 s^1_0 (f) = f, \\
                \partial^2_2 (u) &= \partial^2_2 s^1_0 (f) = \Id_C
            \end{align}
            が分かる.i.e. $f \circ \Id_C = f$ である.

            \item[\textbf{(associativity)}] 
            
            $\forall (h,\, g,\, f) \in \Hom{\Cat{C}}(E,\, F) \times \Hom{\Cat{C}}(D,\, E) \times \Hom{\Cat{C}}(C,\, D)$ を与える.系\ref{col:horn-coeq}を用いることで,
            \hyperref[def:KanCplx]{弱Kan条件}の3つの一意解を得る:
            \begin{itemize}
                \item $h \circ g \coloneqq \partial^2_1(u_0)$ を与える $u_0 \in K_2$:
                \begin{center}
                    \begin{tikzcd}[row sep=large, column sep=large]
                        &\Lambda^2_1 \ar[r, blue, "{(h,\, g)}"]\ar[d, hookrightarrow] &K \\
                        &\Delta^2 \ar[ur, dashed, red, "\exists! u_0"] &
                    \end{tikzcd}
                \end{center}
                \item $g \circ f \coloneqq \partial^2_1(u_3)$ を与える $u_3 \in K_2$:
                \begin{center}
                    \begin{tikzcd}[row sep=large, column sep=large]
                        &\Lambda^2_1 \ar[r, blue, "{(g,\, f)}"]\ar[d, hookrightarrow] &K \\
                        &\Delta^2 \ar[ur, dashed, red, "\exists! u_3"] &
                    \end{tikzcd}
                \end{center}
                \item $(h \circ g) \circ f \coloneqq \partial^2_1(u_2)$ を与える $u_2 \in K_2$
                \begin{center}
                    \begin{tikzcd}[row sep=large, column sep=large]
                        &\Lambda^2_1 \ar[r, blue, "{(h \circ g,\, f)}"]\ar[d, hookrightarrow] &K \\
                        &\Delta^2 \ar[ur, dashed, red, "\exists! u_2"] &
                    \end{tikzcd}
                \end{center}
            \end{itemize}
            さらに,$(u_0,\, u_2,\, u_3) \in K_2^{\times 3}$ は
            \begin{align}
                \partial^2_1 (u_0) &= h \circ g = \partial^2_0 (u_2), \\
                \partial^2_2 (u_0) &= g = \partial^2_0 (u_3), \\
                \partial^2_2 (u_2) &= f = \partial^2_2 (u_3)
            \end{align}
            を充たすため系\ref{col:horn-coeq}が使えて,$(u_0,\, u_2,\, u_3) \in \Hom{\sSet}(\Lambda^3_1,\, K)$ と見做せる.すると,仮定より\hyperref[def:KanCplx]{弱Kan条件}の一意解
            \begin{center}
                \begin{tikzcd}[row sep=large, column sep=large]
                    &\Lambda^3_1 \ar[r, blue, "{(u_0,\, u_2,\, u_3)}"]\ar[d, hookrightarrow] &K \\
                    &\Delta^3 \ar[ur, dashed, red, "\exists! \tau"] &
                \end{tikzcd}
            \end{center}
            が存在する.i.e. ある $\tau \in K_3$ が\footnote{\hyperref[lem:Yoneda]{米田の補題}より $\tau \in \Hom{\sSet}(\Delta^3,\, K) \cong K_3$ である.}一意的に存在して,
            \begin{align}
                u_0 &= \partial^3_0 (\tau), &
                u_2 &= \partial^3_2 (\tau), &
                u_3 &= \partial^3_3 (\tau)
            \end{align}
            を充たす.この3-単体 $\tau$ を図\ref{fig:3-simp}に則り図示すると次のようになる:
            \begin{center}
                \begin{tikzcd}[row sep=large, column sep=large]
                    &  & C \arrow[rd, "f"] \arrow[lldd, "g \circ f"'] &                                                     \\
                    &  &                                                                                & D \arrow[ldd, "h\circ g"] \arrow[llld, "g"] \\
                 E\arrow[rrd, "h"'] &  &                                                                                &                                                     \\
                    &  & F \arrow[from=uuu, crossing over, "(h\circ g) \circ f"]                                                                         &                                                    
                \end{tikzcd}
            \end{center}
            ここで,$u_1 \coloneqq \partial^3_1 (\tau) \in K_2$ とおく.これは $\tau$ の図式で言うと三角形 $C,\, E,\, F$ が指定する2-単体である.
            図式から明らかなように,
            \begin{align}
                \partial^2_0 (u_1) &= h, \\
                \partial^2_1 (u_1) &= (h\circ g) \circ f, \\
                \partial^2_2 (u_1) &= g \circ f
            \end{align}
            が成り立っている.i.e. $h \circ (g \circ f) = (h\circ g) \circ f$ が分かった.
        \end{description}
        
        さて,$\sSet$ の射
        \begin{align}
            \theta \colon K \lto \Ner (\Cat{C})
        \end{align}
        を構成しよう.$\theta$ は\hyperref[def:nat]{自然変換}であるから,$\forall [n] \in $ に対して写像 $\theta_{[n]} \colon K_n \lto \Ner (\Cat{C})_n$ を定めれば良い.命題\ref{prop:SimpSet-basic}-(1) より $\forall \sigma \in K_n$ は自然に $\Hom{\sSet} (\Delta^n,\, K)$ の元と見做せるため,
        \begin{align}
            \theta_{[n]} \colon K_n &\lto \Ner (\Cat{C})_n, \\
            \sigma &\lmto \bigl( \sigma_{[0]}(\{0\}) \xrightarrow{\sigma_{[1]}(\{0,\, 1\})} \cdots \xrightarrow{\sigma_{[1]}(\{n-1,\, n\})} \sigma_{[0]}(\{n\}) \bigr) 
        \end{align}
        と定義する.構成から $\theta_{[n]}$ は自然である.
        $\forall n \ge 0$ に対して $\theta_{[n]}$ が全単射であることを,$n$ に関する数学的帰納法により示す.
        まず,圏 $\Cat{C}$ の構成から $\theta_{[0]},\, \theta_{[1]}$ が全単射であることは明らかである.
        $n > 1$ のとき,示すべきは命題\ref{prop:SimpSet-basic}-(1) より
        \begin{align}
            \theta_* \colon \Hom{\sSet} (\Delta^n,\, K) &\lto \Hom{\sSet} \bigl(\Delta^n,\, \Ner (\Cat{C})\bigr), \\
            \sigma &\lmto \theta \circ \sigma
        \end{align}
        が全単射であることである.$0 < j < n$ を1つとる.$u \in \Hom{\sSet} (\Lambda^n_j,\, K)$ に対する\hyperref[def:KanCplx]{弱Kan条件}の解を $\overline{u} \in \Hom{\sSet} (\Delta^n,\, K)$ と書くと,解の一意性の仮定より写像
        \begin{align}
            \phi_K \colon \Hom{\sSet} (\Delta^n,\, K) &\lto \Hom{\sSet} (\Lambda^n_j,\, K), \\
            \overline{u} &\lmto u
        \end{align}
        は全単射である.$\Ner (\Cat{C})$ にも同様の構成ができるため,可換図式
        \begin{center}
            \begin{tikzcd}
                {\Hom{\sSet}(\Delta^n,\, K)} \arrow[d, "\phi_K"'] \arrow[rr, "\theta_*"] &  & {\Hom{\sSet}\bigl(\Delta^n,\, \Ner (\Cat{C})\bigr)} \arrow[d, "\phi_{\Ner (\Cat{C})}"] \\
                {\Hom{\sSet}(\Lambda_j^n,\, K)} \arrow[rr, "\theta_*|_{\Lambda^n_j}"]    &  & {\Hom{\sSet}\bigl(\Lambda_j^n,\, \Ner (\Cat{C})\bigr)}                                     
            \end{tikzcd}
        \end{center}
        が書ける.縦の射は全単射なので,$\theta_*|_{\Lambda^n_j}$ が全単射であることを示せば良い.これは系\ref{col:horn-coeq}および帰納法の仮定から従う.
    \end{description}
    
\end{proof}

\begin{mydef}[label=def:groupoid]{亜群}
    \textbf{亜群} (groupoid) とは,\hyperref[def:category]{小圏} $\Cat{C}$ であって,任意の射が可逆であるもののこと.
    i.e. $\forall f \in \Hom{\Cat{C}} (X,\, Y)$ に対して,ある $f^{-1} \in \Hom{\Cat{C}} (Y,\, X)$ が存在して
    \begin{align}
        f^{-1} \circ f &= \Id_X, \\
        f \circ f^{-1} &= \Id_Y
    \end{align}
    を充たすこと.
\end{mydef}


\begin{myprop}[label=prop:nerve-grpd]{脈体が $\infty$-groupoidになる必要十分条件}
    圏 $\Cat{C}$ の\hyperref[def:nerve]{脈体} $\Ner (\Cat{C})$ が\hyperref[def:infinity-1]{$\infty$-groupoid}になる必要十分条件は,$\Cat{C}$ が\hyperref[def:groupoid]{groupoid}であること.
\end{myprop}

\begin{proof}
    % ~\cite[p.23, Lemma 1.1.54]{Land2021infinity}
    \begin{description}
        \item[\textbf{($\bm{\Longrightarrow}$)}] 
        
        $\Ner (\Cat{C})$ が\hyperref[def:infinity-1]{$\infty$-groupoid}だと仮定する.
        このとき,\hyperref[def:KanCplx]{Kan条件}により,$\forall n \ge 1,\; 0 \le \forall j \le n$ に対して全射\footnote{定理\ref{thm:KanCplx}の状況とは異なり,\underline{単射とは限らない}!}
        \begin{align}
            \theta^n_j \colon \Hom{\sSet}\bigl(\Delta^n,\,\Ner (\Cat{C})\bigr) &\lto \Hom{\sSet}\bigl(\Lambda_j^n,\,\Ner (\Cat{C})\bigr)
        \end{align}
        が存在する.

         $\forall f \in \Hom{\Cat{C}}(X,\, Y)$ を1つ固定し,これを\eqref{eq:nerve-spine}の方法により $\Ner (\Cat{C})_1$ の元と見做す.このとき,$\bigl(f,\, \Id_Y = \sigma^0_0(Y)\bigr) \in \Ner (\Cat{C})_1^{\times 2}$ は
        % 単体的恒等式\eqref{eq:Simp4}より
        \begin{align}
            \partial^1_0(f) &= Y = \partial^1_0(\Id_Y)
        \end{align}
        を充たすため系\ref{col:horn-coeq}が使えて,$(f,\, \Id_Y) \in \Hom{\sSet}\bigl(\Lambda^2_2,\, \Ner (\Cat{C})\bigr)$ と見做せる.
        このとき $\theta^2_2$ の全射性により,$\sigma \in \Ner (\Cat{C})_2 \cong \Hom{\sSet} \bigl( \Delta^2,\, \Ner (\Cat{C}) \bigr)$ が存在して $(f,\, \Id_Y) = \theta^2_2(\sigma)$ を充たす.
        $\theta^n_j$ の定義および系\ref{col:horn-coeq}から,これは
        \begin{align}
            \partial^2_0 (\sigma) &= f, \\
            \partial^2_1 (\sigma) &= \Id_Y
        \end{align}
        を意味する.故に命題\ref{prop:nerve}の証明から,$g \coloneqq \partial^2_2 (\sigma) \in \Ner (\Cat{C})_1 = \Hom{\Cat{C}}(Y,\, X)$ とおくことで,
        \begin{align}
            f \circ g = \partial^2_1 (\sigma) = \Id_Y
        \end{align}
        が成り立つことが分かった.
        同様に,$(\Id_X,\, f) \in \Hom{\sSet}\bigl(\Lambda^2_0,\, \Ner (\Cat{C})\bigr)$ と見做せること,および $\theta^2_0$ の全射性から $h \in \Ner (\Cat{C})_1 = \Hom{\Cat{C}} (Y,\, X)$ であって $h \circ f = \Id_X$ を充たすものが存在する.
        命題\ref{prop:nerve}の証明から,
        \begin{align}
            g = \Id_X \circ g = (h \circ f) \circ g = h \circ (f \circ g) = h \circ \Id_Y = h
        \end{align}
        が言える.i.e. $f^{-1} = g = h \in \Hom{\Cat{C}}(Y,\, X)$ である.

        \item[\textbf{($\bm{\Longleftarrow}$)}] 
        
        $\Cat{C}$ がgroupoidだとする.
        定理\ref{thm:KanCplx}から,示すべきは $\forall n \ge 1,\, j = 0,\, n$ に対して\hyperref[def:KanCplx]{Kan条件}が成立していることである.
        
         $n=1,\, j=0$ とする.このとき $\Lambda^n_0 = \bigl\{ \{0\},\, \{1\} \bigr\}$ であるから,
        $\forall \sigma \in \Hom{\sSet}\bigl( \Lambda^n_0,\, \Ner (\Cat{C}) \bigr)$ は $\Obj{\Cat{C}}$ の元である.よって,$\bar{\sigma} \in \Hom{\sSet}\bigl(\Delta^1,\, \Ner (\Cat{C})\bigr)$ として $\bar{\sigma} \in \Hom{\Cat{C}}(X,\, \sigma)$ をとれば $\partial^1_0(\bar{\sigma}) = \sigma$ を充たす.
        このことは系\ref{col:horn-coeq}より $\bar{\sigma}|_{\Lambda^n_0} = \sigma$ を意味する.

         次に,$n=2,\, j=0$ とする.このとき $\forall \sigma \in \Hom{\sSet}\bigl( \Lambda^n_0,\, \Ner (\Cat{C}) \bigr)$ は,系\ref{col:horn-coeq}より $\bigl(\sigma_1 \coloneqq \partial^2_1(\sigma),\, \sigma_2 \coloneqq \sigma^2_2(\sigma)\bigr) \in \Ner (\Cat{C})_{1}^{\times 2}$ であって以下の図式として書けるものと同一視できる:
        \begin{center}
            \begin{tikzcd}
                & Z &   \\
            X \arrow[rr, "\sigma_1"] \arrow[ru, "\sigma_2"] &   & Y
            \end{tikzcd}
        \end{center}
        $\Cat{C}$ は\hyperref[def:groupoid]{groupoid}なので,この図式に $\sigma_0 \coloneqq \sigma_1 \circ \sigma_2 \in \Ner (\Cat{C})_1$ を付け足すことで所望の $\bar{\sigma} \in \Ner (\Cat{C})_2 \cong \Hom{\sSet} \bigl(\Delta^2,\, \Ner (\Cat{C})\bigr)$ を得る.

         次に,$n \ge 3,\, j=0$ とする.このとき定理\ref{thm:KanCplx}の (1) $\Longrightarrow$ (2) の構成によって $\bar{\sigma} \in \Ner (\Cat{C})_n$ を作ろうとする.この構成が可能なのは,勝手な $0 \le i \le j \le k \le n$ に対して $\sigma_{[1]} (\{j,\, k\}) \circ \sigma_{[1]} (\{i,\, j\}) = \sigma_{\{i,\, k\}}$ が成り立つときだが,
        これは3頂点 $\bigl\{\{i,\}\, \{j\},\, \{k\}\bigr\}$ が作る2-単体 $\sigma_{ijk} \in \Delta^n([2])$ が $\Lambda^n_0([2])$ に含まれることと同値である.よって $n=3$ のときのみが非自明である.
        $n=3$ のときは,$\sigma_{ij} \coloneqq \sigma_{[1]} (\{j,\, k\})$ とおくと
        \begin{align}
            (\sigma_{23} \circ \sigma_{12}) \circ \sigma_{01}
            &= \sigma_{23} \circ (\sigma_{12} \circ \sigma_{01}) \\
            &= \sigma_{23} \circ \sigma_{02} \\
            &= \sigma_{03} \\
            &= \sigma_{13} \circ \sigma_{01}
        \end{align}
        と計算できるが,$\Cat{C}$ がgroupoidなので両辺の右から $\sigma_{01}^{-1}$ をかけることで $\sigma_{23} \circ \sigma_{12} = \sigma_{13}$ が示される.
        % $\forall \sigma \in \Hom{\sSet} \bigl( \Lambda^n_0,\, \Ner (\Cat{C}) \bigr)$ から次のようにして $\Ner (\Cat{C})_n$ の元を構成する:

         $j=n$ の場合も同様である.
    \end{description}
    
\end{proof}

\subsection{$(\infty,\, 1)$-圏の圏同値}

\begin{mydef}[label=def:infty-homotopy-morphism,breakable]{1-射の間のホモトピー}
    $K$ を\hyperref[def:infinity-1]{$(\infty,\, 1)$-圏},$f,\, g \in K_1$ を\hyperref[def:infinity-1]{始点と終点}が同一であるような $K$ の1-射とする.
    
    このとき,\textbf{$\bm{f}$ と $\bm{g}$ を繋ぐホモトピー}とは,2-射 $\sigma \in K_2$ であって以下を充たすもののこと:
    \begin{itemize}
        \item $\partial^2_0(\sigma) = \Id$
        \item $\partial^2_1(\sigma) = g$
        \item $\partial^2_2(\sigma) = f$
    \end{itemize}
    $\partial^1_1(f) \eqqcolon X,\, \partial^1_0(f) \eqqcolon Y$ とおいて図式\ref{fig:2-simp}を書くと以下の通り:
    \begin{center}
        % https://tikzcd.yichuanshen.de/#N4Igdg9gJgpgziAXAbVABwnAlgFyxMJZABgBoBGAXVJADcBDAGwFcYkQANEAX1PU1z5CKAEwVqdJq3YBNHnxAZseAkXKliEhizaIQc7hJhQA5vCKgAZgCcIAWyRkQOCEnWSd7EyBqN6AIxhGAAUBFWEQaywTAAsceStbB0QnFyQxD2k9SwSQG3t0mjTEd20skAAdCoBJKAB9A0puIA
\begin{tikzcd}
                                   & Y \arrow[rdd, "\Id_Y"] &   \\
                                   &\sigma  & \\
X \arrow[rr, "g"'] \arrow[ruu, "f"] &                       & Y
\end{tikzcd}
    \end{center}
\end{mydef}

\begin{mylem}[label=def:infty-homotopy-equivalence]{ホモトピックは同値関係}
    $K$ を\hyperref[def:infinity-1]{$(\infty,\, 1)$-圏},$\Hom{K}(X,\, Y) \subset K_1$ を,対象 $X,\, Y \in K_0$ をそれぞれ始点,終点に持つような1-射全体が成す集合とする.
    このとき,2項関係
    \begin{align}
        f \simeq g \DEF f,\, g\; \text{を\hyperref[def:infty-homotopy-morphism]{繋ぐホモトピー}が存在}
    \end{align}
    は $\Hom{K}(X,\, Y)$ 上の同値関係である.
\end{mylem}

\begin{proof}
    
\end{proof}

$K$ を\hyperref[def:infinity-1]{$(\infty,\, 1)$-圏}とする.$\forall X,\, Y,\, Z \in K_0$ に対して
\begin{align}
    \bm{\Hom{\htpy{K}} (X,\, Y)} &\coloneqq \Hom{K} (X,\, Y)/{\simeq}
\end{align}
とおき,さらに
\begin{align}
    \label{eq:htpy-cat-composition}
    \circ \colon \Hom{\htpy{K}}(Y,\, Z) \times \Hom{\htpy{K}}(X,\, Y) &\lto \Hom{\htpy{K}} (X,\, Z), \\
    \bigl([g],\, [f]\bigr) &\lmto [g \circ f]
\end{align}
と定義する\footnote{角の図式 $(g,\, \bullet,\, f) \in K_1^{\times 2}$ を\hyperref[def:KanCplx]{\textsf{\textbf{(Kan)}}}によって埋める2-射 $\sigma \in K_2$ に対して,$g \circ f \coloneqq \partial^2_1 (\sigma)$ とおいた.この写像はwell-definedである.}.
    

\begin{mydef}[label=def:hcat-infty,breakable]{{$(\infty,\, 1)$}-圏のホモトピー圏}
    \hyperref[def:infinity-1]{$(\infty,\, 1)$-圏} $K$ の\textbf{ホモトピー圏} (homotopy category) $\htpy{K}$ とは,次のように定義される $(1,\, 1)$-圏のこと:
    \begin{itemize}
        \item $\Obj{\htpy{K}} \coloneqq K_0$
        \item $\Hom{\htpy{K}} (X,\, Y)$ をHom集合とする.
        \item \eqref{eq:htpy-cat-composition}の写像 $\circ$ を射の合成とする.
        \item $\forall X \in \Obj{\htpy{K}}$ に対して $[\Id_X] \in \Hom{\htpy{K}}(X,\, X)$ を恒等射とする
    \end{itemize}
\end{mydef}

\begin{mydef}[label=def:isom-infty]{{$(\infty,\, 1)$}-圏における対象の同型}
    \hyperref[def:infinity-1]{$(\infty,\, 1)$-圏} $K$ を与える.
    \begin{itemize}
        \item 1-射 $f \in K_1$ が\textbf{同型射} (isomorphism) であるとは,$K$ の\hyperref[def:hcat-infty]{ホモトピー圏} $\htpy{K}$ において $[f]$ が\hyperref[def:iso]{同型射}であることを言う.
        \item 対象 $X,\, Y \in K_0$ が\textbf{同型} (isomorphic) であるとは,$X,\, Y$ をそれぞれ始点・終点に持つ $K$ の同型射が存在することを言う.
    \end{itemize}z
\end{mydef}

\begin{mydef}[label=def:hqCat]{{$\qCat$}のホモトピー圏}
    $(1,\, 1)$-圏 $\htpy{(\qCat)}$ を以下で定義する:
    \begin{itemize}
        \item \hyperref[def:infinity-1]{$(\infty,\, 1)$-圏}を対象とする
        \item \exref{def:infty-Fun}で構成した $(\infty,\, 1)$-圏 $\FUN(K,\, L)$ の\hyperref[def:hcat-infty]{ホモトピー圏} $\htpy{\bigl(\FUN(K,\, L)\bigr)}$ における対象の\hyperref[def:iso]{同型}類を射とする
    \end{itemize}
\end{mydef}

\begin{mydef}[label=def:equiv-infty]{{$(\infty,\, 1)$}-圏の同値}
    \hyperref[def:infinity-1]{$(\infty,\, 1)$-圏の関手} $F \colon K \lto L$ が\textbf{$\bm{(\infty,\, 1)}$-圏同値} (equivalence) であるとは,$[F] \in \Hom{\htpy{(\qCat)}}(K,\, L)$ が同型射であることを言う.
\end{mydef}

\subsection{射の空間}

\begin{mydef}[label=def:Map]{射の空間}
    \hyperref[def:infinity-1]{$(\infty,\, 1)$-圏} $K$ と,その対象 $x,\, y \in K_0$ を与える.
    $x$ から $y$ へ向かう\textbf{射の空間} (morphism space) を,$(1,\, 1)$-圏 $\sSet$ における引き戻し
    \begin{align}
        \bm{\inftyMap{K} (x,\, y)} \coloneqq \{x\} \times_{\FUN (\{0\},\, K)} \FUN (\Delta^1,\, K) \times_{\FUN (\{1\},\, K)} \{y\} \in \Obj{\sSet}
    \end{align}
    として定義する~\cite[\href{https://kerodon.net/tag/01J5}{Tag 01J5}]{kerodon}.
\end{mydef}

$(1,\, 1)$-圏 $\sSet$ における引き戻しとは,命題\ref{prop:SimpSet-basic}-(2) の証明より,$n$-単体毎に\hyperref[def:pullback-pushout]{$\SETS$ における引き戻し}を取ることで構成される.
さらに,\exref{def:infty-Fun}より $\FUN (\Delta^1,\, K)_n = \Hom{\sSet}(\Delta^n \times \Delta^1,\, K)$ であるから,
単体的集合 $\inftyMap{K} (x,\, y)$ の\hyperref[def:SimpSet]{$n$-単体}とは,集合として具体的に
\begin{align}
    \inftyMap{K} (x,\, y)_n = \bigl\{\, f \in \Hom{\sSet}(\Delta^n \times \Delta^1,\, K) \bigm| f|_{\Delta^n \times \{0\}} = x,\; f|_{\Delta^n \times \{1\}} = y \,\bigr\} 
\end{align}
と書ける\footnote{右辺に登場する $x \in \Hom{\sSet}(\Delta^n \times \{0\},\, K),\, y \in \Hom{\sSet}(\Delta^n \times \{1\},\, K)$ は記号の濫用である.正確には,\hyperref[lem:Yoneda]{米田の補題}による全単射 $\Hom{\sSet}(\Delta^n \times \{i\},\, K) \cong \Hom{\sSet}(\Delta^n,\, K) \cong K_n$ を用いて $\sigma^n_0 \circ \cdots \circ \sigma^0_0 (x),\, \sigma^n_0 \circ \cdots \circ \sigma^0_0 (y) \in K_n$ を送った先を意味する.図示すると,全ての頂点に $x,\, y$ が載り,全ての部分単体上に $\Id_x,\, \Id_y$ が載った標準 $n$-単体になっている.}.
特に $\inftyMap{K} (x,\, y)_0 \subset K_1$ と見做すことができるが,これは始点,終点をそれぞれ $x,\, y \in K_0$ に持つような1-射全体の集合と一致する.
この意味で $\inftyMap{K}(x,\, y)$ は,$(1,\, 1)$-圏におけるHom集合の $(\infty,\, 1)$-圏における対応物である.

\begin{myprop}[label=prop:Map-is-Kan]{射の空間はKan複体}
    $(\infty,\, 1)$-圏 $K$ における\hyperref[def:Map]{射の空間} $\inftyMap{K}(x,\, y) \in \Obj{\sSet}$ は\hyperref[def:infinity-1]{Kan複体}である.
\end{myprop}

\begin{proof}
    \cite[\href{https://kerodon.net/tag/01JC}{Tag 01JC}]{kerodon}
\end{proof}

\begin{mydef}[label=def:FF-ES-infty]{忠実充満・本質的全射な{$(\infty,\, 1)$}-圏の関手}
    \hyperref[def:infinity-1]{$(\infty,\, 1)$-圏の関手}
    \begin{align}
        F \colon K \lto L
    \end{align}
    を与える.
    \begin{itemize}
        \item $F$ が\textbf{忠実充満} (fully faithful) であるとは,$\forall x,\, y \in K_0$ に対して,$F$ が誘導する\hyperref[def:Map]{射の空間}の間の関手
        \begin{align}
            \inftyMap{K} (x,\, y) \lto \inftyMap{L} \bigl( F_{[0]}(x),\, F_{[0]}(y) \bigr) 
        \end{align}
        が\hyperref[def:equiv-infty]{$(\infty,\, 1)$-圏同値}であることを言う.
        \item $F$ の\textbf{本質的像} (essential image) とは,$F(x) \in L_0$ の形で書ける $L$ の対象と\hyperref[def:iso-infty]{同型}であるような $L$ の対象全体によって生成される $L$ の\hyperref[def:fullsub-infty]{充満部分 $(\infty,\, 1)$-圏}のこと.
        \item $F$ が\textbf{本質的全射} (essentially surgective) であるとは,$F$ の本質的像が $L$ と一致すること.
    \end{itemize}
\end{mydef}

\begin{myprop}[label=prop:equiv-FF-ES]{忠実充満かつ本質的全射}
        \hyperref[def:infinity-1]{$(\infty,\, 1)$-圏の関手}
    \begin{align}
        F \colon K \lto L
    \end{align}
    が\hyperref[def:equiv-infty]{$(\infty,\, 1)$-圏同値}である必要十分条件は,それが\hyperref[def:FF-ES-infty]{忠実充満}かつ\hyperref[def:FF-ES-infty]{本質的全射}であること.
\end{myprop}

\begin{proof}
    \cite[\href{https://kerodon.net/tag/01JX}{Tag 01JX}]{kerodon}
\end{proof}


\section{単体的豊穣圏とホモトピー論}

\subsection{単体的豊穣圏とホモトピーコヒーレントな脈体}

\begin{mydef}[label=def:enriched,breakable]{豊穣圏}
    \hyperref[redef:monoidal-category]{モノイダル圏} $(V,\, \otimes,\, I)$ を与える.

    $\bm{V}$\textbf{-豊穣圏} ($V$-enriched category) $\Cat{C}$ は,以下のデータからなる:
    \begin{itemize}
        \item 集合 $\Obj{\Cat{C}}$
        \item $\forall x, y \in \Obj{\Cat{C}}$ に対して,\textbf{Hom対象}と呼ばれる\underline{$V$ の}対象 $\Hom{\Cat{C}}(x,\, y) \in \Obj{V}$ を持つ
        \item $\forall x, y,\, z \in \Obj{\Cat{C}}$ に対して,\textbf{合成射}と呼ばれる\underline{$V$ の}射 $\circ_{x,\, y,\, z} \colon \Hom{\Cat{C}}(y,\, z) \otimes \Hom{\Cat{C}}(x,\, y) \lto \Hom{\Cat{C}} (x,\, z)$ を持つ
        \item $\forall x \in \Obj{\Cat{C}}$ に対して,\textbf{恒等素}と呼ばれる\underline{$V$ の}射 $j_x \colon I \lto \Hom{\Cat{C}} (x,\, x)$ を持つ
    \end{itemize}
    これらは以下の図式を可換にしなくてはいけない:
    \begin{description}
        \item[\textbf{(associativity)}] 
        
        $\forall x,\, y,\, z,\, w \in \Obj{\Cat{C}}$ について\footnote{$\cong$ はモノイダル圏 $V$ の \hyperref[def:monoidal-category]{associator}}
        \begin{flushleft}
            \begin{tikzcd}[column sep=large]
                &\bigl( \Hom{\Cat{C}}(x,\,  y) \otimes \Hom{\Cat{C}}(y,\,  z) \bigr) \otimes \Hom{\Cat{C}}(z,\,  w) \ar[dd, "\cong"']\ar[r,"\substack{\circ_{x,y,z} \otimes \Id_{\Hom{\Cat{C}}(z,\,  w)} \\ \phantom{0}}"] &\Hom{\Cat{C}} (x,\,  z) \otimes \Hom{\Cat{C}}(z,\,  w) \ar[d, "\circ_{x,z,w}"] \\
                & &\Hom{\Cat{C}}(x,\,  w) \\
                &\Hom{\Cat{C}}(x,\,  y) \otimes \bigl( \Hom{\Cat{C}}(y,\,  z) \otimes \Hom{\Cat{C}}(z,\,  w) \bigr) \ar[r, "\substack{\phantom{0} \\ \Id_{\Hom{\Cat{C}}(x,\,  y)} \otimes \circ_{y,z,w}}"'] &\Hom{\Cat{C}}(x,\,  y) \otimes \Hom{\Cat{C}}(y,\,  w) \ar[u,"\circ_{x,y,w}"']
            \end{tikzcd}
        \end{flushleft}
        
        \item[\textbf{(unitality)}] 
        
        $\forall x,\, y \in \Obj{\Cat{C}}$ について\footnote{$\cong$ はモノイダル圏 $V$ の\hyperref[def:monoidal-category]{left/right unitor}}
        \begin{center}
            \begin{tikzcd}
                &\Hom{\Cat{C}}(x,\,  x) \otimes \Hom{\Cat{C}}(x,\,  y) \ar[r, "\circ_{x{,}x{,}y}"] &\Hom{\Cat{C}}(x,\,  y) &\Hom{\Cat{C}}(x,\,  y) \otimes \Hom{\Cat{C}}(y,\,  y) \ar[l, "\circ_{x{,}y{,}y}"'] \\
                &I \otimes \Hom{\Cat{C}}(x,\,  y) \ar[u, "j_x \otimes \Id_{\Hom{\Cat{C}}(x,\,  y)}"] \ar[ur, "\cong"'] & &\Hom{\Cat{C}}(x,\,  y) \otimes I \ar[u, "\Id_{\Hom{\Cat{C}}(x,\,  y)} \otimes j_y"'] \ar[ul, "\cong"]
            \end{tikzcd}
        \end{center}
        
    \end{description}
\end{mydef}

\begin{mydef}[label=def:enriched-functor,breakable]{豊穣関手}
    モノイダル圏 $(V,\, \otimes,\, I)$ を与える.

    2つの\hyperref[def:enriched]{$V$-豊穣圏} $\Cat{C},\, \Cat{D}$ の間の $V$\textbf{-豊穣関手} ($V$-enriched functor) 
    \begin{align}
        F \colon \Cat{C} \lto \Cat{D}
    \end{align}
    は,以下のデータからなる:
    \begin{itemize}
        \item 写像 $F_0\colon \Obj{\Cat{C}} \lto \Obj{\Cat{D}},\; x \lmto F_0(x)$
        \item \underline{$V$ の}射の族
        \begin{align}
            \Familyset[\big]{F_{x,\, y} \colon \Hom{\Cat{C}} (x,\, y) \lto \Hom{\Cat{D}} \bigl( F_0(x),\, F_0(y) \bigr) }{x,\, y \in \Obj{\Cat{C}}}
        \end{align}
    \end{itemize}
    これらは以下の図式を可換にしなくてはいけない:
    \begin{description}
        \item[\textbf{(enriched-1)}] 
        
        $\forall x,\, y,\, z \in \Obj{\Cat{C}}$ に対して\footnote{これは,通常の関手において射の合成が保存されることに対応する.}
        \begin{center}
            \begin{tikzcd}[row sep=large, column sep=large]
                &\Hom{\Cat{C}}(x,\,  y) \otimes \Hom{\Cat{C}}(y,\,  z) \ar[r, "\circ_{x{,}y{,}{z}}"]\ar[d, "F_{x{,}y} \otimes F_{y{,}z}"'] &\Hom{\Cat{C}}(x,\,  z) \ar[d, "F_{x{,}z}"] \\
                &\Hom{\Cat{D}}\bigl(F_0(x),\,  F_0(y)\bigr) \otimes \Hom{\Cat{D}}\bigl(F_0(y),\,  F_0(z)\bigr) \ar[r, "\substack{\phantom{0} \\ \circ_{F_0(x){,}F_0(y){,}F_0(z)}}"'] &\Hom{\Cat{D}}\bigl(F_0(x),\,  F_0(z)\bigr)
            \end{tikzcd}
        \end{center}
        \item[\textbf{(enriched-2)}] 
        
        $\forall x \in \Obj{\Cat{C}}$ に対して\footnote{これは,通常の関手において恒等射が保存されることに対応する.}
        \begin{center}
            \begin{tikzcd}[row sep=large, column sep=large]
                &I \ar[dr,"j_{F_0(x)}"']\ar[r, "j_x"] &\Hom{\Cat{C}}(x,\,  x) \ar[d, "F_{x{,}x}"] \\
                & &\Hom{\Cat{D}}\bigl( F_0(x),\,  F_0(x) \bigr) 
            \end{tikzcd}
        \end{center}
    \end{description}
\end{mydef}

\hyperref[def:SimpSet]{単体的集合の圏} $\sSet$ は\hyperref[def:monoidal-category]{モノイダル圏}の構造を持つ.
実際,命題\ref{prop:SimpSet-basic}-(2) より,単体的集合 $S,\, T \colon \OP{\Delta} \lto \SETS$ に対してその\hyperref[def:product-coproduct]{直積}
\begin{align}
    \label{eq:sSet-tensor}
    S \times T \colon \OP{\Delta} &\lto \SETS,\\ 
    [n] &\lmto S_n \times T_n, \\
    \bigl( [n] \xrightarrow{\alpha} [m] \bigr) &\lmto \bigl( S_n \times T_n \xrightarrow{\bigl(S(\alpha),\, T(\alpha)\bigr)} S_{m} \times T_{m} \bigr) 
\end{align}
がテンソル積 $\times \colon \sSet \times \sSet \lto \sSet$ を定めている.

\begin{mydef}[label=def:SimpCat]{単体的豊穣圏}
    \hyperref[def:SimpSet]{単体的集合の圏} $\sSet$ を\eqref{eq:sSet-tensor}により\hyperref[redef:monoidal-category]{モノイダル圏}と見做す.
    このとき,$\sSet$-\hyperref[def:enriched]{豊穣圏}のことを\textbf{単体的豊穣圏} (simplicially enriched category) と呼ぶ.
    \tcblower
    単体的\hyperref[def:enriched]{豊穣圏}と $\sSet$-\hyperref[def:enriched-functor]{豊穣関手}全体が成す圏のことを $\sCat$ と書く.
\end{mydef}

つまり,単体的豊穣圏 $\Cat{C}$ とは以下のデータからなる:
\begin{itemize}
    \item 集合 $\Obj{\Cat{C}}$
    \item $\forall X,\, Y \in \Obj{\Cat{C}}$ に対して,\hyperref[def:SimpSet]{単体的集合} $\Hom{\Cat{C}}(X,\, Y) \in \Obj{\sSet}$
    \item $\forall X,\, Y,\, Z \in \Obj{\Cat{C}}$ に対して,自然変換
    \begin{align}
        \circ_{X,\, Y,\, Z} \in \Hom{\sSet} \bigl( \Hom{\Cat{C}}(Y,\, Z) \times \Hom{\Cat{C}}(X,\, Y),\, \Hom{\Cat{C}}(X,\, Z) \bigr) 
    \end{align}
    \item $\forall X \in \Obj{\Cat{C}}$ に対して,$0$-単体 $j_X \in \Hom{\Cat{C}}(X,\, X)_0 \cong \Hom{\sSet} \bigl( \Delta^0,\, \Hom{\Cat{C}}(X,\, X) \bigr)$
\end{itemize}

\begin{myexample}[label=def:Kan]{Kan複体の圏}
    $(1,\, 1)$-圏 $\Kan$ を,
    \begin{itemize}
        \item \hyperref[def:KanCplx]{Kan複体}を対象に持つ
        \item Kan複体の間の自然変換を射に持つ
    \end{itemize}
    で定義する.$\Kan$ は\hyperref[def:SimpSet]{単体的集合の圏 $\sSet$}の充満部分圏であり,直積\eqref{eq:sSet-tensor}をテンソル積とするモノイダル圏になる.

     さらに,$\Kan$ は\exref{def:infty-Fun}と同様の方法で\hyperref[def:SimpCat]{単体的豊穣圏}と見做すこともできる.
    実際,任意のKan複体 $X,\, Y \in \Obj{\Kan}$ に対して
    \begin{align}
        \FUN(X,\, Y) \colon \OP{\Delta} &\lto \SETS, \\
        [n] &\lmto \Hom{\sSet}(X \times \Delta^n,\, Y), \\
        \bigl( [m] \xrightarrow{\alpha} [n] \bigr) &\lmto \bigl( f \mapsto f \circ (\Id_X \times \alpha_*) \bigr) 
    \end{align}
    なる対応は\hyperref[def:SimpSet]{単体的集合}を成しているため,
    \begin{itemize}
        \item $\Obj{\KAN} \coloneqq \Obj{\Kan}$ 
        \item Hom対象を $\Hom{\KAN}(X,\, Y) \coloneqq \FUN(X,\, Y) \in \Obj{\sSet}$
    \end{itemize}
    と定義することで単体的豊穣圏 $\KAN$ が構成できた.
    特に,
    \begin{align}
        \Hom{\KAN}(X,\, Y)_0 &= \Hom{\sSet}(X \times \Delta^0,\, Y) \cong \Hom{\sSet}(X,\, Y) = \Hom{\Kan}(X,\, Y)
    \end{align}
    が成り立つ.
\end{myexample}


\begin{mydef}[label=def:PathCat,breakable]{単体的豊穣圏 {$\mathfrak{C}[Q]$}}
$(Q,\, \le)$ を半順序集合とする.このとき\hyperref[def:SimpCat]{単体的豊穣圏} $\bm{\mathfrak{C}}[Q] \in \Obj{\sCat}$ を以下で定義する:
\begin{itemize}
    \item $\Obj{\mathfrak{C}[Q]} \coloneqq Q$
    \item $\forall i,\, j \in \Obj{\mathfrak{C}[Q]}$ に対して,
    \begin{align}
        \Hom{\mathfrak{C}[Q]} (i,\, j) \coloneqq 
        \begin{cases}
            \Ner (P_{ij}), &i \le j \\
            \emptyset, &i > j
        \end{cases}
    \end{align}
    ただし,半順序集合 $P_{ij} \coloneqq \bigl\{\, k \in Q \bigm| i \le k \le j \,\bigr\}$ を
    \begin{itemize}
        \item $\forall k \in P_{ij}$ を対象とする.
        \item Hom集合は
        \begin{align}
            \Hom{P_{ij}} (k,\, l) \coloneqq 
            \begin{cases}
                \{\mathrm{pt}_{kl}\}, &k \le l \\
                \emptyset, &k > l
            \end{cases}
        \end{align}
        で定義する.
    \end{itemize}
    ことで圏と見做し,それの\hyperref[def:nerve]{脈体}を取った.
    \item $\forall i,\, j,\, k \in \Obj{\mathfrak{C}[Q]},\; i \le j \le k$ に対して,自然変換
    \begin{align}
        \circ_{ijk} \colon \Hom{\mathfrak{C}[Q]}(j,\, k) \times \Hom{\mathfrak{C}[Q]}(i,\, j) &\lto \Hom{\mathfrak{C}[Q]}(i,\, k)
    \end{align}
    を次で定義する:
    \begin{align}
        \circ_{ijk}{}_{[n]} \colon \Ner (P_{jk})_n \times \Ner (P_{ij})_n &\lto \Ner (P_{ij})_n, \\
        (\sigma,\, \tau) &\lmto \sigma \cup \tau
    \end{align}
\end{itemize}    
\end{mydef}

\begin{myexample}[label=ex:PathCat2]{単体的豊穣圏 {$\mathfrak{C}[2]$}}
    圏 \hyperref[def:PathCat]{$\mathfrak{C}[2]$} の構造を調べよう.
    定義\ref{def:PathCat}の記号に倣うと
    \begin{align}
        P_{ii} &= \{i\}, \\
        P_{01} &= \{0 \le 1\}, \\
        P_{02} &= \{0 \le 1 \le 2\}, \\
        P_{12} &= \{1 \le 2\}
    \end{align}
    であるから,\hyperref[def:nerve]{脈体}の定義により
    \begin{align}
        % \Ner (P_{ii})_0 &= \mathrm{Fun}([0],\, P_{ii}) = \{\bullet_{i}\}, \\
        % \Ner (P_{01})_0 &= \mathrm{Fun}([0],\, P_{01}) = \{\bullet_{0},\, \bullet_{1}\}, \\
        % \Ner (P_{02})_0 &= \mathrm{Fun}([0],\, P_{02}) = \{\bullet_{0},\, \bullet_{1},\, \bullet_{2}\}, \\
        % \Ner (P_{12})_0 &= \mathrm{Fun}([0],\, P_{12}) = \{\bullet_{1},\, \bullet_{2}\}, \\
        \Ner (P_{ii})_n &= \mathrm{Fun}([n],\, P_{ii}) = \{\mathrm{const}_{i} \colon k \lmto i\}, \\
        \Ner (P_{01})_n &= \mathrm{Fun}([n],\, P_{01}) = 
        \Bigl\{ 
            f_{x} \colon k \lmto
            \begin{cases}
                0, &k < x, \\
                1, &k \ge x
            \end{cases}\;
        \Bigm|\; 0 \le x \le n+1
        \Bigr\}, \\
        \Ner (P_{12})_n &= \mathrm{Fun}([n],\, P_{12}) = 
        \Bigl\{ 
            f_{x} \colon k \lmto
            \begin{cases}
                1, &k < x, \\
                2, &k \ge x
            \end{cases}\;
        \Bigm|\; 0 \le x \le n+1
        \Bigr\}, \\
        \Ner (P_{02})_n &= \mathrm{Fun}([n],\, P_{02}) = 
        \Bigl\{ 
            f_{xy} \colon k \lmto 
            \begin{cases}
                0, &k < x, \\
                1, &x \le k < y, \\
                2, &k \ge y
            \end{cases}
        \Bigm|\; 0 \le x \le y \le n+1
         \Bigr\} 
    \end{align}
    だと分かる.
    % 特に,
    % \begin{align}
    %     \circ_{012}{}_{[n]} \colon \Ner (P_{12})_n \times \Ner (P_{01}) \lto \Ner (P_{02})
    % \end{align}
    % である.
\end{myexample}

\begin{mydef}[label=def:nerve-hc,breakable]{homotopy coherentな脈体}
    \hyperref[def:SimpCat]{単体的豊穣圏} $\Cat{C} \in \Obj{\sCat}$ の\textbf{homotopy coherent nerve}
    \begin{align}
        \hcNer (\Cat{C}) \colon \OP{\Delta} \lto \SETS
    \end{align}
    を以下で定義する:
    \begin{itemize}
        \item $\forall [n] \in \Obj{\OP{\Delta}}$ に対して
        \begin{align}
            \hcNer (\Cat{C})_n &\coloneqq \Hom{\sCat} \bigl( \mathfrak{C}[n],\, \Cat{C} \bigr) 
        \end{align}
        を対応づける.
        \item $\forall \alpha \in \Hom{\OP{\Delta}}([n],\, [m])$ に対して,写像
        \begin{align}
            \hcNer (\Cat{C})(\alpha) \coloneqq \alpha^* \colon \Hom{\sCat} \bigl( \mathfrak{C}[n],\, \Cat{C} \bigr) &\lto \Hom{\sCat} \bigl( \mathfrak{C}[m],\, \Cat{C} \bigr), \\
            X &\lmto X \circ \alpha
        \end{align}
        を対応付ける.
    \end{itemize}
    
    \tcblower

    \textbf{homotopy coherent nerve functor}
    \begin{align}
        \hcNer  \colon \sCat \lto \sSet
    \end{align}
    を以下で定義する:
    \begin{itemize}
        \item $\forall \Cat{C} \in \Obj{\sCat}$ に対して $\hcNer (\Cat{C}) \in \Obj{\sSet}$ を対応付ける.
        \item 任意の $\sSet$-\hyperref[def:enriched-functor]{豊穣関手} $\Cat{C} \xrightarrow{F} \Cat{D}$ に対して,\hyperref[def:nat]{自然変換}
        \begin{align}
            \hcNer  (F) \coloneqq \Familyset[big]{\hcNer (F)_{[n]} \colon X \lmto F \circ X}{[n] \in \Obj{\OP{\Delta}}}
        \end{align}
        を対応付ける.
    \end{itemize}
\end{mydef}
% \subsection{単体的圏}

\begin{myexample}[label=def:Spaces]{空間の成す {$(\infty,\, 1)$}-圏}
    \exref{def:Kan}\hyperref[def:SimpCat]{単体的豊穣圏} $\KAN$ を\hyperref[def:nerve-hc]{homotopy coherent nerve}で $\sSet$ に埋め込んだものを\textbf{空間の成す $\bm{(\infty,\, 1)}$-圏} ($\infty$-category of spaces)\footnote{より現代的には,$\bm{(\infty,\, 1)}$\textbf{-category of anima}と呼ばれることがある.} と呼び,
    \begin{align}
         \SPACES \coloneqq \hcNer (\KAN)
    \end{align}
    と書く.実際,\hyperref[def:SimpSet]{単体的集合} $\SPACES \in \Obj{\sSet}$ は\hyperref[def:infinity-1]{$(\infty,\, 1)$-圏}である\cite[\href{https://kerodon.net/tag/01YY}{Tag 01YY}]{kerodon}.
\end{myexample}


\subsection{単体的ホモトピー}

\begin{mydef}[label=def:SimpSet-homotopic,breakable]{単体的ホモトピー}
    % \begin{itemize}
    %     \item $f,\, g \in \Hom{\sSet} (X,\, Y)$ を繋ぐ\textbf{ホモトピー}とは,\hyperref[def:SimpSet]{$\sSet$ の射} $\eta \in \Hom{\sSet}(X \times \Delta^1,\, Y)$ であって,以下の $\sSet$ の図式を可換にするもののこと:
    %     \begin{center}
    %         \begin{tikzcd}[row sep=large, column sep=large]
    %             &X \cong X \times \Delta^0 \ar[r, "\Id \times d^1_1{}_*"]\ar[dr, "f"'] &X \times \Delta^1 \ar[d, "\eta"] &X \times \Delta^0 \ar[l, "\Id \times d^1_0{}_*"'] \ar[dl, "g"] \\
    %             & &Y &
    %         \end{tikzcd}
    %     \end{center}
    %     $f,\, g$ を繋ぐホモトピーが存在するとき,$f,\, g$ は互いに\textbf{ホモトピック}であるという.
    %     \item 基点付き\hyperref[def:KanCplx]{Kan複体} $(K,\, x)$ を与える.このとき $n$ 次の\textbf{単体的ホモトピー群} (simplicial homotopy group) を
    %     \begin{align}
    %         \bm{\pi^\Delta_n (X,\, x)} \coloneqq \Hom{\sSet_*} \bigl( (\Delta^n,\, \partial\Delta^n),\, (X,\, x) \bigr) / \simeq
    %     \end{align}
    %     と定義する.
    % \end{itemize}
    $X,\, Y,\, K\in \Obj{\sSet}$ を,$K$ が $X$ の\hyperref[def:SimpSet]{単体的部分集合}となるようにとる.包含射 $i \colon K \hookrightarrow X$ をとる.
    \begin{itemize}
        \item 
        $f,\, g \in \Hom{\sSet} (X,\, Y)$ を繋ぐ\textbf{ホモトピー}とは,\hyperref[def:SimpSet]{$\sSet$ の射} $\eta \in \Hom{\sSet}(X \times \Delta^1,\, Y)$ であって,以下の $\sSet$ の図式を可換にするもののこと:
        \begin{center}
            \begin{tikzcd}[row sep=large, column sep=large]
                &X \cong X \times \Delta^0 \ar[r, "\Id \times d^1_1{}_*"]\ar[dr, "f"'] &X \times \Delta^1 \ar[d, "\eta"] &X \times \Delta^0 \ar[l, "\Id \times d^1_0{}_*"'] \ar[dl, "g"] \\
                & &Y &
            \end{tikzcd}
        \end{center}
        $f,\, g$ を繋ぐホモトピーが存在するとき,$f,\, g$ は互いに\textbf{ホモトピック}であるという.
        \item $f \circ i = g \circ i \eqqcolon \alpha$ とおく.ホモトピー $\eta \in \Hom{\sSet} (X \times \Delta^1,\, Y)$ が $f$ と $g$ の間の\textbf{$\bm{K}$ に関する相対ホモトピー} (homotopy from $f$ to $g$ (rel $K$)) であるとは,上の可換図式に加えて
        \begin{center}
            \begin{tikzcd}[row sep=large, column sep=large]
                 &K \times \Delta^1 \ar[r, hookrightarrow, "i \times \Id"]\ar[d, "\mathrm{pr}"'] &X \times \Delta^1 \ar[d, "\eta"] \\
                 &K \ar[r, "\alpha"'] &Y
            \end{tikzcd}
        \end{center}
        が成り立つことを言う.
    \end{itemize}
    
    \tcblower

    より具体的には,$f,\, g$ を繋ぐ\textbf{単体的ホモトピー} (simplicial homotopy) とは $\SETS$ の射の族
    \begin{align}
        \Familyset[\big]{h_i \colon X_n \lto Y_{n+1}}{i = 0,\, \dots,\, n,\, n \ge 0}
    \end{align}
    であって以下を充たすもののこと:
    \begin{align}
        \partial_0 \circ h_0 &= f_n, \\
        \partial_{n+1} \circ h_n &= g_n, \\
        \partial_i \circ h_j &=
        \begin{cases}
            h_{j-1} \circ \partial_{i}, &i < j \\
            \partial_i \circ h_{i-1}, &i=j \neq 0 \\
            h_j \circ \partial_{i-1}, &i > j+1
        \end{cases} \\
        \sigma_i \circ h_j &=
        \begin{cases}
            h_{j+1} \circ \sigma_i, &i \le j \\
            h_j \circ \sigma_{i-1}, &i > j
        \end{cases}
    \end{align}
\end{mydef}

一見するとホモトピーと単体的ホモトピーは別のものに見えるが,実は同じものである.\hyperref[def:SimpSet-homotopic]{単体的ホモトピー} $\Familyset[\big]{h_i \colon X_n \lto Y_{n+1}}{i = 0,\, \dots,\, n,\, n \ge 0}$ が与えられたとする.このとき $\sSet$ の射 $\eta \in \Hom{\sSet}(X \times \Delta^1,\, Y)$ を
\begin{align}
    \eta_0 &\coloneqq \partial_0 \circ h_0, \\
    \eta_{n+1} &\coloneqq \partial_{n+1} \circ h_n, \\
    \eta_j &\coloneqq \partial_j \circ h_{j} \quad (1\le j \le n)
\end{align}
と定義すると,\hyperref[def:colim]{和の普遍性}の図式によって $\eta \in \Hom{\sSet}(X \times \Delta^1,\, Y)$ が定まる.

\begin{myprop}[label=prop:simphomotopy-equiv]{$\infty$-groupoidとホモトピー}
    $X,\, Y \in \Obj{\sSet}$ が\hyperref[def:infinity-1]{$\infty$-groupoid}ならば,\hyperref[def:SimpSet-homotopic]{ホモトピック}は $\Hom{\sSet} (X,\, Y)$ の上の同値関係になる.
    ホモトピック (rel $K \subset X$) も同値関係である.
\end{myprop}

\begin{proof}
    ~\cite[p.26, COROLLARY 6.2]{goerss2009simplicial}
\end{proof}

$\infty$-groupoid $X$ を与え,$* \in X_0$ を1つ固定する.このとき集合としての同型
\begin{align}
    \Hom{\sSet} \bigl( (\Delta^n,\, \partial\Delta^n),\, (X,\, *) \bigr) \cong \bigl\{\, x \in X_n \bigm|  0 \le \forall i \le n,\; \partial^n_i(x) = \sigma^{n-2}_0 \circ \cdots \circ \sigma^0_0 (*) \,\bigr\} \eqqcolon Z_n (X,\, *)
\end{align}
がある~\cite{SChen2024anomaly}.$a,\, b \in X_n$ を繋ぐ\textbf{ホモトピー}とは,この場合 $y \in X_{n+1}$ であって
\begin{align}
    \partial_i^{n+1} (y) =
    \begin{cases}
        \sigma^{n-1}_0 \circ \cdots \circ \sigma^0_0 (*), &i < n \\
        a, &i=n \\
        b, &i=n+1
    \end{cases}    
\end{align}
を充たすもののことである.ホモトピック $\sim$ は $Z_n(X,\, *)$ 上の同値関係になる~\cite[p.27, Lemma 3.28]{SChen2024anomaly}.

$a,\, b \in Z_n (X,\, *)$ に対して,系\ref{col:horn-coeq},命題\ref{prop:SimpSet-basic}-(1) および\hyperref[def:infinity-1]{Kan条件}によって
\begin{center}
    \begin{tikzcd}[row sep=large, column sep=large]
        &\Lambda^{n+1}_n \ar[d, hookrightarrow]\ar[r, "{\bigl((\sigma_0)^n(*),\, \dots,\, (\sigma_0)^n(*),\, a,\, b\Bigr)}"] &X \\
        &\Delta^{n+1} \ar[ur, red, dashed, "\exists a\star b"'] &
    \end{tikzcd}
\end{center}
として $a \star b \in X_{n+1}$ をとってくる.
\begin{align}
    \bm{\pi^\Delta_n (X,\, *)} \coloneqq \Hom{\sSet} \bigl( (\Delta^n,\, \partial\Delta^n),\, (X,\, *) \bigr) / {\simeq} \cong Z_n (X,\, *) /{\sim}
\end{align}
とおく.

\begin{myprop}[label=def:simpi]{単体的ホモトピー群}
    写像
    \begin{align}
        \pi^\Delta_n (X,\, *) \times \pi^\Delta_n (X,\, *) \lto \pi^\Delta_n (X,\, *),\; ([a],\, [b]) \lmto [\partial^{n+1}_n (a \star b)]
    \end{align}
    によって $\pi^\Delta_n (X,\, *)$ は群になる.これを\textbf{単体的ホモトピー群}と呼ぶ.
\end{myprop}

\begin{proof}
    
\end{proof}

\begin{mytheo}[label=thm:homotopygrp]{単体的ホモトピー群と幾何学的実現}
    \begin{align}
        \pi^\Delta_n (X,\, *) \cong \pi_n (\abs{X},\, \abs{*})
    \end{align}
\end{mytheo}

\begin{proof}
    ~\cite[p.64, PROPOSITION 11.1]{goerss2009simplicial}
\end{proof}

\section{$\infty$-トポス}

\hyperref[def:SimpSet]{単体圏} $\Delta$ 上の関手
\begin{align}
    \mathrm{OP} \colon \Delta &\lto \Delta, \\
    [n] &\lmto [n], \\
    \bigl( [m] \xrightarrow{\alpha} [n] \bigr) &\lmto (j \mapsto n - \alpha (m-j))
\end{align}
を考える.

\begin{mydef}[label=def:op-infty]{{$(\infty,\, 1)$}-圏の逆}
    \hyperref[def:infinity-1]{$(\infty,\, 1)$-圏} $K$ の\textbf{反対} (opposite) とは,\hyperref[def:SimpSet]{単体的集合}
    \begin{align}
        \bm{\OP{K}} \colon \OP{\Delta} \xrightarrow{\mathrm{OP}} \OP{\Delta} \xrightarrow{K} \SETS
    \end{align}
    のことを言う.$\OP{K}$ もまた $(\infty,\, 1)$-圏である~\cite[\href{https://kerodon.net/tag/003S}{Tag 003S}]{kerodon}.
\end{mydef}

反対圏 $\OP{K}$ における\hyperref[def:SimpSet]{面写像・縮退写像}はそれぞれ
\begin{align}
    (\partial^n_i \colon \OP{K}_n \lto \OP{K}_{n-1}) &= (\partial^n_{n-i} \colon K_n \lto K_{n-1}), \\
    (\sigma^n_i \colon \OP{K}_n \lto \OP{K}_{n+1}) &= (\sigma^n_{n-i} \colon K_n \lto K_{n+1})
\end{align}
となる.従って,\hyperref[def:infinity-1]{$(\infty,\, 1)$-圏 $K$ の射}の始点と終点が入れ替わっている.

\begin{mydef}[label=def:infinity-presheaf]{{$(\infty,\, 1)$}-前層}
    \begin{itemize}
        \item     
        $K$ を\hyperref[def:infinity-1]{$(\infty,\, 1)$-圏}とする.$K$ 上の\textbf{$\bm{(\infty,\, 1)}$-前層}とは,
        自然変換\footnote{i.e. これは\hyperref[def:infinity-1]{$(\infty,\, 1)$-圏の関手}である.}
        \begin{align}
            P \colon \OP{K} \lto \SPACES
        \end{align}
        のこと.
        \item $K$ 上の $\bm{(\infty,\, 1)}$\textbf{-前層が成す} $\bm{(\infty,\, 1)}$\textbf{-圏}とは,$(\infty,\, 1)$-前層全体の集合 $\Hom{\sSet} (\OP{K},\, \SPACES)$ を\exref{def:infty-Fun}の構成と全く同じ方法で\hyperref[def:SimpSet]{単体的集合}と見做した
        \begin{align}
            \inftyPsh{K} \coloneqq \FUN(\OP{K},\, \SPACES) \in \Obj{\sSet}
        \end{align}
        のこと.$\inftyPsh{K}$ は $(\infty,\, 1)$-圏である~\cite[\href{https://kerodon.net/tag/0066}{Tag 0066}]{kerodon}
        \footnote{$\inftyPsh{K}_0 = \Hom{\sSet} (\OP{K} \times \Delta^0,\, \SPACES) \cong \Hom{\sSet}(\OP{K},\, \SPACES)$ より,$\inftyPsh{K}$ の\hyperref[def:infinity-1]{対象} (0-セル) がまさに $K$ 上の $(\infty,\, 1)$-前層となっている.}
    \end{itemize}
\end{mydef}

\begin{marker}
    以降では,$(\infty,\, 1)$-前層のことを $\infty$-前層と呼ぶ.
\end{marker}


% \begin{myprop}[label=prop:nerve-grpd-Psh]{$\infty$-前層の圏のモデル}
%     $\Cat{C}$ を\hyperref[def:enriched]{$\sSet$-豊穣圏}であって,\hyperref[def:KanCplx]{Kan複体}を\hyperref[def:enriched]{Hom対象}に持つものとする.
    
%     このとき\hyperref[def:nerve-hc]{homotopy coherentな脈体} $\irm{\Ner }{hc}$ に対して
%     \begin{align}
%         \inftyPsh{\irm{\Ner }{hc}(\Cat{C})} \cong \irm{\Ner }{hc} \bigl( \comm{\OP{\Cat{C}}}{\irm{\sSet}{Quillen}}_{\mathrm{proj}}^\circ \bigr)
%     \end{align}
%     が成り立つ.
% \end{myprop}

% \begin{proof}
%     \url{https://ncatlab.org/nlab/show/%28infinity%2C1%29-category+of+%28infinity%2C1%29-presheaves}を参照.
% \end{proof}

~\cite{nLab}, ~\cite[p.9]{NSS2012}に従い $(\infty,\, 1)$-トポスの定義を概観する\footnote{ここでの定義は不完全なので,詳細は~\cite{nLab}, ~\cite{lurie2008higher}などを参照.}.

\begin{mydef}[label=def:infinity-topos]{$\infty$-トポス}
    $K$ を\hyperref[def:infinity-1]{$(\infty,\, 1)$-圏}とする.
    
    $K$ 上の $\bm{(\infty,\, 1)}$\textbf{-トポス}とは,$(\infty,\, 1)$-圏 $\inftyPsh{K}$ の部分\hyperref[def:infinity-1]{$(\infty,\, 1)$-圏}
    \begin{align}
        i \colon \bm{\mathrm{H}} \hookrightarrow \inftyPsh{K}
    \end{align}
    であって,包含 \hyperref[def:infinity-1]{$(\infty,\, 1)$-関手} $i$ が\hyperref[def:colim]{有限極限を保つ}\hyperref[def:adjoint]{左随伴} $(\infty,\, 1)$-関手
    \begin{center}
        \begin{tikzcd}
            \bm{\mathrm{H}}\ar[r,bend left, hookrightarrow, "i",""{name=A, below}] & \inftyPsh{K}\ar[l,bend left,"\mathrm{lex}",""{name=B,above}] 
            \ar[from=A, to=B, symbol=\vdash]
            \end{tikzcd}
    \end{center}
    を持つようなもののこと.

    \tcblower

    もしくは,\hyperref[def:complete]{余完全}\footnote{正確には\textbf{presentable}~\cite[p.372, Def5.5.0.18]{lurie2008higher}}な\hyperref[def:infinity-1]{$(\infty,\, 1)$-圏} $\bm{\mathrm{H}}$ であって以下の公理を充たすもののこと~\cite[p.9, Definition5.4]{NSS2012}:
    \begin{description}
        \item[\textbf{(T1)}] $\forall f \in \Hom{\bm{\mathrm{H}}}(X,\, Y)$ および $\bm{\mathrm{H}}$ における\hyperref[def:diagram]{図式} $D \colon I \lto \bm{\mathrm{H}}_{/Y}$ において,自然な同型
        \begin{align}
            \colim_{i \in I} \bigl( X \times_Y D(i) \bigr) \cong X \times_Y \colim_{i \in I} D(i)
        \end{align}
        がある\footnote{$\times_Y$ は引き戻し}.
        \item[\textbf{(T2)}] $\forall X,\, Y \in \Obj{\bm{\mathrm{H}}}$ に対して,図式 $Y \lrto \emptyset \lto X$ の\hyperref[def:colim]{押し出し}
        \begin{center}
            \begin{tikzcd}[row sep=large, column sep=large]
                &\emptyset \ar[r, ""]\ar[d, ""'] &X \ar[d, ""] \\
                &Y \ar[r, ""'] &X \amalg Y
            \end{tikzcd}
        \end{center}
        は図式 $Y \lto X \amalg Y \lrto X$ の引き戻しでもある.i.e. 任意の和がdisjointである.
        \item[\textbf{(T3)}] $\bm{\mathrm{H}}$ における任意のgroupoid objectはdeloopingを持つ.
    \end{description}
    
\end{mydef}

% ~\cite{alfonsi2023higher}は命題\ref{prop:nerve-grpd-Psh}を使って $\infty$-トポスを定義している.


\section{$(\infty,\, n)$-圏}

% \subsection{因子化ホモロジー}

\subsection{Complete Segal space}

\subsection{Theta space}


\end{document}
