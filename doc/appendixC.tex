\documentclass[TQFT_main]{subfiles}

\begin{document}

% \setcounter{}{}
\chapter{微分形式の話}

$C^\infty$ 級の\hyperref[def:vect]{ベクトル束}を構成する便利な補題から出発しよう~\cite[p.253]{Lee2012smooth}

\begin{mylem}[label=lem:smooth-vect-chart]{$C^\infty$ ベクトル束の構成}
    \begin{itemize}
        \item 境界あり/なし $C^\infty$ 多様体 $M$
        \item $n$ 次元実ベクトル空間の族 $\Familyset[\big]{E_p}{p \in M}$ と全射
        \begin{align}
            \pi \colon \coprod_{p \in M} E_p \lto M,\; (p,\, v) \lmto p
        \end{align}
        \item $M$ の開被覆 $\Familyset[\big]{U_\lambda}{\lambda \in \Lambda}$
        \item 全単射の族 $\Familyset[\big]{\psi_\lambda \colon \pi^{-1}(U_\lambda) \lto U_\lambda \times \mathbb{R}^n}{\lambda \in \Lambda}$
        \item \underline{$C^\infty$ 写像}の族 $\Familyset[\big]{t_{\alpha\beta} \colon U_\alpha \cap U_\beta \lto \LGL(n,\, \mathbb{R})}{\alpha,\, \beta \in \Lambda}$
    \end{itemize}
    の5つ組であって以下の条件を充たすものを与える:
    \begin{description}
        \item[\textbf{(DVS-1)}] $\forall \lambda \in \Lambda$ および $\forall p \in U_\lambda$ に対して,制限
        \begin{align}
            \psi_\alpha|_{E_p} \colon E_p \lto \{p\} \times \mathbb{R}^n \cong \mathbb{R}^n
        \end{align}
        はベクトル空間の同型写像である.
        \item[\textbf{(DVS-2)}] $\forall \alpha,\, \beta \in \Lambda$ および $\forall (p,\, v) \in (U_\alpha \cap U_\beta) \times \mathbb{R}^n$ に対して
        \begin{align}
            \psi_\beta^{-1} (p,\, v) = \psi_\alpha^{-1} \bigl( p,\, t_{\alpha\beta}(p)(v) \bigr) 
        \end{align}
        が成り立つ.
    \end{description}
    このとき,集合 $E \coloneqq \coprod_{p \in M} E_p$ 上の\hyperref[lem:cinfty-chart]{$C^\infty$ 構造}が一意的に存在して,
    $\mathbb{R}^n \hookrightarrow E \xrightarrow{\pi} M$ が\hyperref[def:fiber-1]{局所自明化} $\Familyset[\big]{\psi_\lambda}{\lambda \in \Lambda}$ を持つ $C^\infty$ \hyperref[def:vect]{ベクトル束}になる.
\end{mylem}

\begin{proof}
    $\forall p \in M$ を1つとる.
    すると $\Familyset[\big]{U_\lambda}{\lambda \in \Lambda}$ は $M$ の開被覆なので,ある $\alpha_p \in \Lambda$ が存在して $p \in U_{\alpha_p}$ となる.
    さらに $U_{\alpha_p}$ は開集合なので,$C^\infty$ チャート $(V_p,\, \varphi_p)$ であって $p \in V_p \subset U_{\alpha_p}$ を充たすものが存在する.
    このとき,写像 $\tilde{\varphi}_p \colon \pi^{-1}(V_p) \lto \varphi_p(V_p) \times \mathbb{R}^n$ を
    \begin{align}
        \tilde{\varphi}_p \coloneqq (\varphi_p \times \mathrm{id}_{\mathbb{R}^n}) \circ \psi_{\alpha_p}|_{\pi^{-1}(V_p)}
    \end{align}
    と定義する.

    まず,$M$ が境界を持たない場合に
    \begin{itemize}
        \item 集合 $E$
        \item $E$ の部分集合族 $\Familyset[\big]{\pi^{-1}(V_p)}{p \in M}$
        \item 写像の族 $\Familyset[\big]{\tilde{\varphi}_p \colon \pi^{-1}(V_p) \lto \varphi_p(V_p) \times \mathbb{R}^n}{p \in M}$
    \end{itemize}
    の3つ組が補題\ref{lem:cinfty-chart}の5条件を充たすこと,i.e. $E$ が境界を持たない $C^\infty$ 多様体になることを示そう.
    \begin{description}
        \item[\textbf{(DS-1)}] $\forall p \in M$ に対して $\psi_p \bigl( \pi^{-1}(V_p) \bigr) = \varphi_p(V_p) \times \mathbb{R}^n \subset \mathbb{R}^{\dim M} \times \mathbb{R}^n$ であり,$\varphi_p(V_p) \subset \mathbb{R}^{\dim M}$ はチャートの定義から $\mathbb{R}^{\dim M}$ の開集合なので $\psi_p \bigl( \pi^{-1}(V_p) \bigr)$ は $\mathbb{R}^{\dim M} \times \mathbb{R}^n$ の開集合である.
        仮定より $\psi_{\alpha_p} \colon \pi^{-1}(V_p) \lto V_p \times \mathbb{R}^n$ は全単射であり,チャートの定義から $\varphi_p \times \mathrm{id}_{\mathbb{R}^n} \colon V_p \times \mathbb{R}^n \lto \varphi_p(V_p) \times \mathbb{R}^n$ は全単射なので $\psi_p = (\varphi_p \times \mathrm{id}_{\mathbb{R}^n}) \circ \tilde{\varphi}_{\alpha_p}$ も全単射である.
        \item[\textbf{(DS-2, 3)}] $\forall p,\, q \in M$ をとる.このとき
        \begin{align}
            \tilde{\varphi}_{p} \bigl( \pi^{-1}(V_p) \cap \pi^{-1}(V_q) \bigr) &= \varphi_p(V_p \cap V_q) \times \mathbb{R}^n \\
            \tilde{\varphi}_{q} \bigl( \pi^{-1}(V_p) \cap \pi^{-1}(V_q) \bigr) &= \varphi_q(V_p \cap V_q) \times \mathbb{R}^n
        \end{align}
        はどちらも $\mathbb{R}^{\dim M} \times \mathbb{R}^n$ の開集合である.
        さらに $\forall (x,\, v) \in \tilde{\varphi}_p^{-1} \bigl( \pi^{-1}(V_p) \cap \pi^{-1}(V_q) \bigr)$ に対して
        \begin{align}
            \tilde{\varphi}_q \circ \tilde{\varphi}_p^{-1} (x,\, v) 
            &= \tilde{\varphi}_q \circ \psi_{\alpha_p}^{-1} \bigl( \varphi_p^{-1}(x),\, v \bigr) \\
            &= \tilde{\varphi}_q \circ \psi_{\alpha_q}^{-1} \Bigl( \varphi_p^{-1}(x),\, t_{\alpha_q,\alpha_p} \bigl( \varphi_p^{-1}(x) \bigr)(v) \Bigr) \\
            &= \Bigl( \varphi_q \circ \varphi_p^{-1}(x),\,  t_{\alpha_q,\alpha_p} \bigl( \varphi_p^{-1}(x) \bigr)(v) \Bigr)
        \end{align}
        が成り立つが,$C^\infty$ チャートの定義から $\varphi_{\alpha_p},\, \varphi_{\alpha_q}$ は $C^\infty$ 写像で,かつ仮定より $t_{\alpha_q,\alpha_p}$ も $C^\infty$ 写像なので,最右辺は $C^\infty$ 写像の合成として書ける.よって $\tilde{\varphi}_q \circ \tilde{\varphi}_p^{-1}$ は $C^\infty$ 写像である.

        \item[\textbf{(DS-4)}] $\Familyset[\big]{(V_p,\, \varphi_p)}{p \in M}$ は $M$ のアトラスなので,高々可算濃度の部分集合 $I \subset M$ が存在して $\Familyset[\big]{V_i}{i \in I}$ が $M$ の開被覆になる.このとき
        \begin{align}
            E = \coprod_{p \in \bigcup_{i \in I} V_i} E_p = \bigcup_{i \in I} \coprod_{p_i \in V_i} E_{p_i} = \bigcup_{i \in I} \pi^{-1} (V_i)
        \end{align}
        が言える.
        \item[\textbf{(DS-5)}] 互いに相異なる $\xi = (p,\, v),\, \eta = (q,\, w) \in E$ をとる.もし $p = q$ ならば $\xi,\, \eta \in E_p \subset \pi^{-1}(V_p)$ である.
        $p \neq q$ ならば,$V_p,\, V_q \subset M$ を $V_p \cap V_q = \emptyset$ を充たすようにとれる.すると $\pi^{-1}(V_p) \cap \pi^{-1}(V_q) = \pi^{-1}(V_p \cap V_q) = \emptyset$ でかつ $\xi \in \pi^{-1}(V_p),\, \eta \in \pi^{-1}(V_q)$ が成り立つ.
    \end{description}
    
    次に,$M$ が境界付き多様体である場合を考える.
    座標を入れ替える写像
    \begin{align}
        \mathrm{swap} \colon \mathbb{R}^{\dim M} \times \mathbb{R}^n \lto \mathbb{R}^{n} \times \mathbb{R}^{\dim M},\; (x^1,\, \dots,\, x^{\dim M},\, v^1,\, \dots,\, v^n) \lmto (v^1,\, \dots,\, v^n,\, x^1,\, \dots,\, x^{\dim M})
    \end{align}
    は微分同相写像\footnote{$\mathbb{R}^{\dim M + n}$ には標準的な $C^\infty$ 構造を入れる.}であり,境界チャート $(V_p,\, \varphi_p)$ に関して
    \begin{align}
        \mathrm{swap} \circ \psi_p  \bigl( \pi^{-1}(V_p) \bigr) = \mathbb{R}^n \times \varphi_p (V_p) \subset \mathbb{R}^n \times \mathbb{H}^{\dim M} = \mathbb{H}^{\dim M + n}
    \end{align}
    が成り立つ.よって
    \begin{itemize}
        \item 集合 $E$
        \item $E$ の部分集合族 $\Familyset[\big]{\pi^{-1}(V_p)}{p \in M}$
        \item 写像の族 $\Familyset[\big]{\mathrm{swap} \circ \tilde{\varphi}_p \colon \pi^{-1}(V_p) \lto \mathbb{R}^n \times \varphi_p(V_p)}{p \in M}$
    \end{itemize}
    の3つ組が補題\ref{lem:cinfty-chart-b}の5条件を充たすことを示せば良いが,議論は $M$ が境界を持たない場合と全く同様である.
\end{proof}


% \section{余接束}

% 境界あり/なし $C^\infty$ 多様体 $M$ を与える.点 $p \in M$ における $M$ の接空間の双対ベクトル空間を\textbf{余接空間} (cotangent space) と呼び,$\bm{T_p^* M} \coloneqq (T_p M)^*$ と書く.
% $M$ の\textbf{余接束} (cotangent bundle) とは集合
% \begin{align}
%     \bm{T^* M} \coloneqq \coprod_{p \in M} T_p^* M
% \end{align}
% のことである.$T^* M$ の射影を
% \begin{align}
%     \pi \colon T^* M \lto M,\; (p,\, \omega) \lmto p
% \end{align}
% と定義する.



\end{document}