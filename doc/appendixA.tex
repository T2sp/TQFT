\documentclass[TQFT_main]{subfiles}

\begin{document}

% \setcounter{}{}
% \chapter{結び目理論入門}

% \begin{mydef}[label=def:knot]{結び目・絡み目}
%     \begin{itemize}
%         \item \textbf{結び目} (knot) とは,空間対 $(S^3,\, S^1)$ であって $S^1$ が $S^3$ の滑らかな部分多様体になっているもののこと.
%         \item $n$ 成分\textbf{絡み目} (link) とは,空間対 $(S^3,\, \underbrace{S^1 \amalg \cdots \amalg S^1}_n)$ であって $S^1 \amalg \cdots \amalg S^1$ が $S^3$ の滑らかな部分多様体になっているもののこと.$n$ のことを\textbf{成分数}と呼ぶ.
%     \end{itemize}
% \end{mydef}

% \begin{mydef}[label=def:knot-equiv]{結び目の同値}
%     結び目 $K_1,\, K_2$ を与える.以下の2つの定義は同値である.
%     \begin{itemize}
%         \item $S^3$ の向きを保つ自己同相写像 $\varphi$ が存在して $\varphi(K_1) = K_2$ を充たすとき,$K_1$ と $K_2$ は\textbf{同値}であるという.
%         \item $S^3$ の自己同相写像の族 $\{h_t \colon S^3 \lto S^3\}_{t \in [0,\, 1]}$ が存在するとき,$K_1$ と $K_2$ は\textbf{全同位同値} (ambient isotopic) であるという:
%         \begin{itemize}
%             \item $H \colon S^3 \times [0,\, 1] \lto S^3,\; (x,\, t) \lmto h_t(x)$ は連続写像
%             \item $h_0 = \mathrm{id}_{S^3}$
%             \item $h_1(K_1) = K_2$
%         \end{itemize}
%     \end{itemize}
% \end{mydef}

\chapter{コホモロジー}

$R$ を環とする.
\begin{mydef}[label=def:good-cover]{良い被覆}
    位相空間 $X$ の開被覆 $\Familyset[\big]{U_\alpha}{\alpha \in \Lambda}$ が\textbf{良い被覆} (good cover) であるとは,
    $\forall n \in \mathbb{N},\; \forall \alpha_1,\, \dots,\, \alpha_n \in \Lambda$ に対して
    \begin{align}
        U_{\alpha_1} \cap \cdots \cap U_{\alpha_n} \neq \emptyset \IMP U_{\alpha_1} \cap \cdots \cap U_{\alpha_n}\; \text{は可縮}
    \end{align}
    が成り立つこと.
\end{mydef}
以下では $\bm{U_{\alpha_1 \dots \alpha_n}} \coloneqq U_{\alpha_1} \cap \cdots \cap U_{\alpha_n}$ と略記する.

\section{導来関手}



\section{層係数コホモロジー}

位相空間 $X$ を1つ固定する.$X$ 上の\textbf{開集合の圏} $\bm{\mathbb{O}_X}$ を,
\begin{itemize}
    \item $X$ の開集合を対象とする
    \item $X$ の任意の開集合 $U,\, V \subset X$ に対して
    \begin{align}
        \Hom{\mathbb{O}_X} (U,\, V)
        \coloneqq \begin{cases}
            \{\, \text{包含写像}\; U \hookrightarrow V \,\}, &U \subset V \\
            \emptyset, &U \not\subset V
        \end{cases}
    \end{align}
\end{itemize}
と定義する.

\begin{mydef}[label=def:presheaf]{前層}
    位相空間 $X$ 上の,\underline{小圏} $\Cat{S}$ に値をとる\textbf{前層} (presheaf) とは,関手
    \begin{align}
        P \colon \OP{\OPEN{X}} \lto \Cat{S}
    \end{align}
    のこと\footnote{最も一般的には,関手 $P \colon \OP{\Cat{C}} \lto \Cat{S}$ のことを圏 $\Cat{C}$ 上の $\Cat{S}$ に値をとる前層と呼ぶ~\url{https://ncatlab.org/nlab/show/presheaf}}.
\end{mydef}

位相空間 $X$ 上の,圏 $\Cat{S}$ に値をとる\textbf{前層の圏} $\bm{\PSH{X}{\Cat{S}}}$ とは,
\begin{itemize}
    \item \hyperref[def:presheaf]{$\Cat{S}$ に値をとる前層}を対象とする.
    \item $\Cat{S}$ に値をとる前層 $P,\, Q \colon \OP{\OPEN{X}} \lto \Cat{S}$ に対して,\hyperref[def:nat]{自然変換}
    \begin{align}
        F \colon P \Longrightarrow Q
    \end{align}
    を射とする
\end{itemize}
圏のこと.

\begin{myexample}[label=ex:constant-presheaf]{定数前層}
    圏 $\Cat{S}$ の対象 $A$ に対して定まる\hyperref[def:presheaf]{前層}
    \begin{align}
        A_X \colon \OP{\OPEN{X}} &\lto \Cat{S}, \\
        U &\lmto A, \\
        (U \hookrightarrow V) &\lmto \mathrm{id}_{A}
    \end{align}
    のことを\textbf{定数前層} (constant presheaf) と呼ぶ.
\end{myexample}

\begin{myexample}[label=ex:presheaf-diff]{微分形式のなす前層}
    $X$ を $C^\infty$ 多様体とする.このとき
    \begin{align}
        C^\infty_X \colon \OP{\OPEN{X}} &\lto \MOD{\mathbb{R}}, \\
        U &\lmto C^\infty (U), \\
        (\iota \circ U \hookrightarrow V) &\lmto \bigl( f \lmto f \circ \iota \bigr) 
    \end{align}
    なる対応は\hyperref[def:presheaf]{前層}である.同様に,$\forall q \in \mathbb{Z}_{\ge 0}$ に対して
    \begin{align}
        \Omega^q_X \colon \OP{\OPEN{X}} &\lto \MOD{\mathbb{R}}, \\
        U &\lmto \Omega^q (U), \\
        (\iota \colon U \hookrightarrow V) &\lmto \bigl( \omega \lmto \iota^* \omega \bigr) 
    \end{align}
    なる対応は\hyperref[def:presheaf]{前層}である.
\end{myexample}
% 位相空間 $X$ 上の\hyperref[def:presheaf]{前層}から別の位相空間 $Y$ 上の前層を構成する方法を導入する:

% \begin{mydef}[label=def:direct-inverse]{順像・逆像}
%     連続写像 $f \colon X \lto Y$ および $\colim$ が常に存在する圏 $\Cat{S}$ を与える.
%     \begin{itemize}
%         \item \hyperref[def:presheaf]{前層} $P \in \Obj{\PSH{X}{\Cat{S}}}$ に対して,
%         \begin{itemize}
%             \item 任意の $Y$ の開集合 $U \subset Y$ に対して,$\Cat{S}$ の対象 $f_p (P)(U) \coloneqq P\bigl(f^{-1}(U)\bigr)$ を対応づける
%             \item $(U \hookrightarrow V) \in \Hom{\OPEN{Y}}(U,\, V)$ に対して,$\Cat{S}$ の射 $P \bigl( f^{-1}(U) \hookrightarrow f^{-1}(V) \bigr) \in \Hom{\Cat{S}} \bigl( f_p(P)(U),\, f_p(P)(V) \bigr)$  を対応付ける
%         \end{itemize}
%         \hyperref[def:presheaf]{前層}
%         \begin{align}
%             f_p(P) \colon \OP{\OPEN{Y}} \lto \Cat{S}
%         \end{align}
%         のことを $P$ の\textbf{順像} (direct image) と呼ぶ.
%         \item 
%     \end{itemize}
    
% \end{mydef}

\begin{mydef}[label=def:sheaf]{層}
    \hyperref[def:presheaf]{前層} $P \in \Obj{\PSH{X}{\Cat{S}}}$ が\textbf{層} (sheaf) であるとは,
    \begin{itemize}
        \item $X$ の任意の開集合 $U \subset X$ および $U$ の開被覆 $\Familyset[\big]{U_\alpha}{\alpha\in \Lambda}$ 
        \item 任意の族 $\Familyset[\big]{x_\alpha \in P(U_\alpha)}{\alpha \in \Lambda}$ であって,$\forall \alpha,\, \beta \in \Lambda$ に対して
        \begin{align}
            P(U_\alpha \cap U_\beta \hookrightarrow U_\alpha)(x_\alpha) = P(U_\alpha \cap U_\beta \hookrightarrow U_\beta)(x_\beta)
        \end{align}
        を充たすもの
    \end{itemize}
    に対して,$x \in P(U)$ が\underline{一意的に}存在して
    \begin{align}
        \forall \alpha \in \Lambda,\; P(U_\alpha \hookrightarrow U)(x) = x_\alpha
    \end{align}
    を充たすこと.
\end{mydef}

位相空間 $X$ 上の,圏 $\Cat{S}$ に値をとる\textbf{層の圏} $\bm{\SH{X}{\Cat{S}}}$ とは,
\begin{itemize}
    \item \hyperref[def:sheaf]{$\Cat{S}$ に値をとる層}を対象とする.
    \item $\Cat{S}$ に値をとる層 $P,\, Q \colon \OP{\OPEN{X}} \lto \Cat{S}$ に対して,自然変換
    \begin{align}
        F \colon P \Longrightarrow Q
    \end{align}
    を射とする
\end{itemize}
圏のこと.$\Cat{S}$ がアーベル圏のとき,圏 $\SH{X}{\Cat{S}}$ もまたアーベル圏である~\cite[p.298, 命題4.30]{Shiho2016}.

$X$ を位相空間とする.
加法的関手
\begin{align}
    \label{eq:additive-sheaf}
    A \colon \SH{X}{\MOD{R}} \lto \MOD{R}
\end{align}
を,
\begin{itemize}
    \item 任意の\hyperref[def:sheaf]{層} $P \in \Obj{\SH{X}{\MOD{R}}}$ に対して $R$-加群 $P(X) \in \Obj{\MOD{R}}$ を対応付ける
    \item 層 $P,\, Q \in \Obj{\SH{X}{\MOD{R}}}$ の間の任意の自然変換 $F \colon P \Longrightarrow Q$ に対して,$R$-加群の準同型 $F_X \colon P(X) \lto Q(X)$ を対応付ける
\end{itemize}
関手として定義する.

\begin{mydef}[label=def:sheaf-cohomology]{層係数コホモロジー}
    加法的関手\eqref{eq:additive-sheaf}の右導来関手を $\Dpmember[\big]{H^n(X,\, \mhyphen)}{n \in \mathbb{Z}_{\ge 0}}$ と書く.
    \hyperref[def:sheaf]{層} $P \in \Obj{\SH{X}{\MOD{R}}}$ を係数とする位相空間 $X$ の\textbf{層係数コホモロジー} (sheaf cohomology) とは,
    \begin{align}
        H^n(X,\, P)
    \end{align}
    のこと.
\end{mydef}

\section{\v{C}echコホモロジー}

位相空間 $X$ および\hyperref[def:presheaf]{前層} $P \in \Obj{\PSH{X}{\MOD{R}}}$ を与える.

$X$ の開被覆 $\mathcal{U} \coloneqq \Familyset[\big]{U_\alpha}{\alpha \in \Lambda}$ をとる.
$\forall n \in \mathbb{Z}_{\ge 0}$ に対して
\begin{align}
    \label{def:Cech-complex}
    \check{C}^n (\mathcal{U},\, P) \coloneqq \prod_{(\alpha_0,\, \dots,\, \alpha_n) \in \Lambda^{n+1}} P(U_{\alpha_0 \dots  \alpha_n})
\end{align}
と定義し,
\begin{align}
    \label{def:Cech-cobaundarymap}
    \delta^{n+1} \colon \check{C}^{n}(\mathcal{U},\, P) &\lto \check{C}^{n+1}(\mathcal{U},\, P),\\
    \Dpmember[\big]{x_{\alpha_0 \dots \alpha_n}}{(\alpha_0,\, \dots,\, \alpha_n)} 
    &\lmto \left(\sum_{j=0}^{n+1} (-1)^j P(U_{\alpha_0 \dots \alpha_{n+1}} \hookrightarrow U_{\alpha_0 \dots \hat{\alpha_j} \dots \alpha_{n+1}})(x_{\alpha_0 \dots \hat{\alpha_j} \dots \alpha_{n+1}})\right)_{(\alpha_0,\, \dots,\, \alpha_{n+1})}
\end{align}
と定義すると $\delta^{n+1} \circ \delta^n = 0$ であるから,$\MOD{R}$ の図式
\begin{align}
    \cdots \xrightarrow{\delta^{n-1}} \check{C}^{n-1} (\mathcal{U},\, P) \xrightarrow{\delta^n} \check{C}^n (\mathcal{U},\, P) \xrightarrow{\delta^{n+1}} \check{C}^{n+1}(\mathcal{U},\, P) \xrightarrow{\delta^{n+2}} \cdots
\end{align}
はコチェイン複体である.この複体を\textbf{\v{C}ech複体}と呼ぶ.

\begin{mydef}[label=def:cech-cohomology]{\v{C}echコホモロジー}
    \v{C}ech複体 $\bigl( \check{C}^\bullet (\mathcal{U},\, P),\; \delta^\bullet \bigr)$ のコホモロジーのことを
    $X$ の開被覆 $\mathcal{U}$ に関する $P$ 係数\textbf{\v{C}echコホモロジー}と呼び,
    $\bm{\Cech{\bullet}{\mathcal{U},\, P}}$ と書く.
\end{mydef}



\begin{myexample}[label=ex:Cech-deRham]{\v{C}ech-de Rham複体}
    $X$ を $C^\infty$ 多様体とし,$X$ の開被覆 $\mathcal{U}$ をとる.
    \exref{ex:presheaf-diff}において導入した前層 $\Omega^q_X \colon \OP{\OPEN{X}} \lto \MOD{\mathbb{R}}$ について,複体
    $\bigl( \check{C}^\bullet(\mathcal{U},\, \Omega^q_X),\; \delta^\bullet \bigr)$ のことを\textbf{\v{C}ech-de Rham複体}と呼ぶ.$\delta \colon \check{C}^{n}(\mathcal{U},\, \Omega^l_X)$ の定義\eqref{def:Cech-cobaundarymap}に \underline{$(-1)^{l}$ をつける}ことで,これは二重複体の構造\footnote{$(-1)^l$ の因子は $\dd{\delta} + \delta \dd = 0$ を成り立たせるために必要である.}を持つ:
    \begin{center}
        \begin{tikzcd}[row sep=large, column sep=large]
            &\check{C}^0(\mathcal{U},\, \Omega^0_X) \ar[r, "\dd"]\ar[d, "\delta"] &\check{C}^0(\mathcal{U},\, \Omega^1_X) \ar[r, "\dd"]\ar[d, "\delta"] &\cdots \ar[r, "\dd"] &\check{C}^0(\mathcal{U},\, \Omega^{\dim X}_X) \ar[r] \ar[d, "\delta"] &0 \\
            &\check{C}^1(\mathcal{U},\, \Omega^0_X) \ar[r, "\dd"]\ar[d, "\delta"] &\check{C}^1(\mathcal{U},\, \Omega^1_X) \ar[r, "\dd"]\ar[d, "\delta"] &\cdots \ar[r, "\dd"] &\check{C}^1(\mathcal{U},\, \Omega^{\dim X}_X) \ar[r] \ar[d, "\delta"] &0 \\
            &\check{C}^2(\mathcal{U},\, \Omega^0_X) \ar[r, "\dd"]\ar[d, "\delta"] &\check{C}^2(\mathcal{U},\, \Omega^1_X) \ar[r, "\dd"]\ar[d, "\delta"] &\cdots \ar[r, "\dd"] &\check{C}^2(\mathcal{U},\, \Omega^{\dim X}_X) \ar[r] \ar[d, "\delta"] &0 \\
            & \vdots &\vdots & &\vdots &
        \end{tikzcd}
    \end{center}
    
\end{myexample}

\begin{myprop}[label=prop:Cech-good-cover]{良い被覆に関する\v{C}echコホモロジー}
    \hyperref[ex:constant-presheaf]{定数前層} $\mathbb{R}_X \colon \OPEN{X} \lto \MOD{\mathbb{R}}$ について,
    もし $\mathcal{U}$ が\hyperref[def:good-cover]{良い被覆}ならば
    \begin{align}
        \dR{\bullet}{X;\, \mathbb{R}} \cong \Cech{\bullet}{\mathcal{U};\, \mathbb{R}_X}
    \end{align}
    が成り立つ.
\end{myprop}

\begin{proof}
    
\end{proof}


\section{Deligne-Beilinsonコホモロジー}

一般論には立ち入らず,\cite[p.21, Appendix A]{Bauer2005DB}を参考に,本文中で必要になる最小限だけDeligne-Beilinsonコホモロジーを導入する.

$C^\infty$ 多様体 $X$ とその開被覆 $\mathcal{U}$ を一つ固定する.
まず,\hyperref[ex:Cech-deRham]{\v{C}ech-de Rham複体}のde Rham複体成分を次数 $-1$ に拡張する:
\begin{align}
    \label{def:Cech-complex-1}
    \check{C}^{n} (\mathcal{U},\, \Omega^{-1}_X) \coloneqq \check{C}^n (\mathcal{U},\, \mathbb{Z}_X)
\end{align}
ただし $\mathbb{Z}_X$ は圏 $\MOD{\mathbb{R}}$ に値をとる\hyperref[ex:constant-presheaf]{定数前層}である.そして
\begin{align}
    \dd_{-1} \colon \check{C}^{n} (\mathcal{U},\, \Omega^{-1}_X) &\lto \check{C}^{n} (\mathcal{U},\, \Omega^{0}_X), \\
    \Dpmember[\big]{c_{\alpha_0 \dots \alpha_n}}{(\alpha_0,\, \dots ,\, \alpha_n)} &\lmto \Dpmember[\big]{(x \lmto c_{\alpha_0 \dots \alpha_n})}{(\alpha_0,\, \dots ,\, \alpha_n)}
\end{align}
と定義することで,二重複体
\begin{center}
    \begin{tikzcd}[row sep=large, column sep=large]
        &\check{C}^0(\mathcal{U},\, \Omega^{-1}_X) \ar[r, "\dd_{-1}"]\ar[d, "\delta"] &\check{C}^0(\mathcal{U},\, \Omega^0_X) \ar[r, "\dd"]\ar[d, "\delta"] &\check{C}^0(\mathcal{U},\, \Omega^1_X) \ar[r, "\dd"]\ar[d, "\delta"] &\cdots \ar[r, "\dd"] &\check{C}^0(\mathcal{U},\, \Omega^{\dim X}_X) \ar[r] \ar[d, "\delta"] &0 \\
        &\check{C}^1(\mathcal{U},\, \Omega^{-1}_X) \ar[r, "\dd_{-1}"]\ar[d, "\delta"] &\check{C}^1(\mathcal{U},\, \Omega^0_X) \ar[r, "\dd"]\ar[d, "\delta"] &\check{C}^1(\mathcal{U},\, \Omega^1_X) \ar[r, "\dd"]\ar[d, "\delta"] &\cdots \ar[r, "\dd"] &\check{C}^1(\mathcal{U},\, \Omega^{\dim X}_X) \ar[r] \ar[d, "\delta"] &0 \\
        &\check{C}^2(\mathcal{U},\, \Omega^{-1}_X) \ar[r, "\dd_{-1}"]\ar[d, "\delta"] &\check{C}^2(\mathcal{U},\, \Omega^0_X) \ar[r, "\dd"]\ar[d, "\delta"] &\check{C}^2(\mathcal{U},\, \Omega^1_X) \ar[r, "\dd"]\ar[d, "\delta"] &\cdots \ar[r, "\dd"] &\check{C}^2(\mathcal{U},\, \Omega^{\dim X}_X) \ar[r] \ar[d, "\delta"] &0 \\
        &\vdots & \vdots &\vdots & &\vdots &
    \end{tikzcd}
\end{center}
を得る.この2重複体を横方向に切り取り,かつ右斜め上方向に直和をとることで得られる複体を\textbf{Deligne-Beilinson複体}と呼ぶ.
あからさまには,ある $0 \le p \le \dim X + 1$ に対して
\begin{align}
    \irm{\mathbb{Z}(p)}{D}^q 
    &\coloneqq 
    \begin{cases}
        \bigoplus_{n+m = q-1} \check{C}^{n}(\mathcal{U},\, \Omega^{m}_X), &0 \le q \le p \\
        \bigoplus_{\substack{n+m = p-1,\\ m \le p-1}} \check{C}^{n}(\mathcal{U},\, \Omega^{m}_X), &q > p
    \end{cases}
\end{align}
と定義し $D \coloneqq \dd + (-1)^{\deg}\delta$ とおくと,図式
\begin{align}
    \cdots \xrightarrow{D} \irm{\mathbb{Z}(p)}{D}^q  \xrightarrow{D} \irm{\mathbb{Z}(p)}{D}^{q+1} \xrightarrow{D} \cdots
\end{align}
はコチェイン複体になる.

\begin{mydef}[label=def:Deligne-Beilinson]{Deligne-Beilinsonコホモロジー}
    $0 \le p \le \dim X + 1$ を与える.
    
    上述の複体 $\bigl(\mathcal{U},\; \irm{\mathbb{Z}(p)}{D}^\bullet,\; D \bigr)$ を開被覆 $\mathcal{U}$ に関する\textbf{Deligne-Beilinson複体},
    そのコホモロジーを\textbf{Deligne-Beilinsonコホモロジー}と呼ぶ.
    記号として $\bm{\DB{\bullet}{\mathcal{U};\, \irm{\mathbb{Z}(p)}{D}}}$ と書く.
\end{mydef}

標準的射影 $\pi^q \colon \irm{\mathbb{Z}(p)}{D}^q \lto \check{C}^q(\mathcal{U},\, \Omega_X^{-1})$ はチェイン写像
\begin{center}
    \begin{tikzcd}[row sep=large, column sep=large]
        &\cdots \ar[r, "D"] &\irm{\mathbb{Z}(p)}{D}^q \ar[r, "D"]\ar[d,"\pi"] &\irm{\mathbb{Z}(p)}{D}^{q+1} \ar[r, "D"]\ar[d, "\pi"] &\cdots \\
        &\cdots \ar[r, "\delta"] &{\check{C}^q (\mathcal{U};\, \mathbb{Z}_X)} \ar[r, "\delta"] &{\check{C}^{q+1} (\mathcal{U};\, \mathbb{Z}_X)} \ar[r, "\delta"] &\cdots
    \end{tikzcd}
\end{center}
となるので,誘導準同型
\begin{align}
    \DB{q}{\mathcal{U};\, \irm{\mathbb{Z}(p)}{D}} \lto \Cech{q}{\mathcal{U};\, \mathbb{Z}_X}
\end{align}
がある.

\begin{myprop}[label=prop:Deligne-Beilinson]{Deligne-Beilinsonコホモロジーと\v{C}echコホモロジー}
    $0 \le p \le \dim X + 1$ を与える.
    \begin{enumerate}
        \item $q < p$ ならば
        \begin{align}
            \DB{q}{\mathcal{U};\, \irm{\mathbb{Z}(p)}{D}} \cong \Cech{q-1}{\mathcal{U},\, \mathbb{R}/\mathbb{Z}}
        \end{align}
        \item $q > p$ ならば
        \begin{align}
            \DB{q}{\mathcal{U};\, \irm{\mathbb{Z}(p)}{D}} \cong \Cech{q-1}{\mathcal{U},\, \mathbb{Z}}
        \end{align}
        \item $q=p$ ならば,$\mathbb{Z}$ 加群の完全列
        \begin{align}
            0 \lto \Bigl\{ \substack{\text{closed global}\, (p-1)\text{-forms} \\ \text{with integral periods}} \Bigr\} \lto \Omega^{p-1}(X;\, \mathbb{R}) \lto \DB{p}{\mathcal{U};\, \irm{\mathbb{Z}(p)}{D}} \lto \Cech{p}{\mathcal{U},\, \mathbb{Z}} \quad (\text{exact})
        \end{align}
        が成り立つ.
    \end{enumerate}
\end{myprop}

\begin{proof}
    \begin{enumerate}
        \item 
    \end{enumerate}
\end{proof}

\begin{myprop}[label=prop:DB-good-cover]{良い被覆に関するDeligne-Beilinsonコホモロジー}
    $\DB{\bullet}{\mathcal{U};\, \irm{\mathbb{Z}(p)}{D}}$ は\hyperref[def:good-cover]{良い被覆} $\mathcal{U}$ の取り方によらない.
\end{myprop}

\begin{proof}
    
\end{proof}

\end{document}