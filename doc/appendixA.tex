\documentclass[TQFT_main]{subfiles}

\begin{document}

% \setcounter{}{}
% \chapter{結び目理論入門}

% \begin{mydef}[label=def:knot]{結び目・絡み目}
%     \begin{itemize}
%         \item \textbf{結び目} (knot) とは,空間対 $(S^3,\, S^1)$ であって $S^1$ が $S^3$ の滑らかな部分多様体になっているもののこと.
%         \item $n$ 成分\textbf{絡み目} (link) とは,空間対 $(S^3,\, \underbrace{S^1 \amalg \cdots \amalg S^1}_n)$ であって $S^1 \amalg \cdots \amalg S^1$ が $S^3$ の滑らかな部分多様体になっているもののこと.$n$ のことを\textbf{成分数}と呼ぶ.
%     \end{itemize}
% \end{mydef}

% \begin{mydef}[label=def:knot-equiv]{結び目の同値}
%     結び目 $K_1,\, K_2$ を与える.以下の2つの定義は同値である.
%     \begin{itemize}
%         \item $S^3$ の向きを保つ自己同相写像 $\varphi$ が存在して $\varphi(K_1) = K_2$ を充たすとき,$K_1$ と $K_2$ は\textbf{同値}であるという.
%         \item $S^3$ の自己同相写像の族 $\{h_t \colon S^3 \lto S^3\}_{t \in [0,\, 1]}$ が存在するとき,$K_1$ と $K_2$ は\textbf{全同位同値} (ambient isotopic) であるという:
%         \begin{itemize}
%             \item $H \colon S^3 \times [0,\, 1] \lto S^3,\; (x,\, t) \lmto h_t(x)$ は連続写像
%             \item $h_0 = \mathrm{id}_{S^3}$
%             \item $h_1(K_1) = K_2$
%         \end{itemize}
%     \end{itemize}
% \end{mydef}

\chapter{コホモロジー}

\section{層係数コホモロジー}

\begin{mydef}[label=def:good-cover]{良い被覆}
    位相空間 $X$ の開被覆 $\Familyset[\big]{U_\alpha}{\alpha \in \Lambda}$ が\textbf{良い被覆} (good cover) であるとは,
    $\forall n \in \mathbb{N},\; \forall \alpha_1,\, \dots,\, \alpha_n \in \Lambda$ に対して
    \begin{align}
        U_{\alpha_1} \cap \cdots \cap U_{\alpha_n} \neq \emptyset \IMP U_{\alpha_1} \cap \cdots \cap U_{\alpha_n}\; \text{は可縮}
    \end{align}
    が成り立つこと.
\end{mydef}
以下では $\bm{U_{\alpha_1 \dots \alpha_n}} \coloneqq U_{\alpha_1} \cap \cdots \cap U_{\alpha_n}$ と略記する.

\begin{mydef}[label=def:presheaf]{前層}
    $X$ を位相空間とする.$X$ の\textbf{前層} (presheaf) とは,
    
\end{mydef}

\section{Deligne-Beilinsonコホモロジー}



\end{document}