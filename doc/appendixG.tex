\documentclass[TQFT_main]{subfiles}

\begin{document}

% \setcounter{}{}
\chapter{層状化空間・因子化ホモロジー}

~\cite{AFT2014stratified}, ~\cite{AFT2014FH}のレビュー

\section{多様体の層状化}

\subsection{層状化空間}

\begin{mydef}[label=def:topo-poset]{半順序集合の位相}
    $(P,\, \le)$ を半順序集合とする.
    $P$ 上の位相 $\mathscr{O}_{\le} \subset 2^P$ を以下で定義する:
    \begin{align}
        U \in \mathscr{O}_{\le} \DEF \forall x \in U,\, \forall y \in P,\; \bigl[\, x \le y \IMP y \in U \,\bigr]
    \end{align}
\end{mydef}

実際,空集合の定義から $\emptyset \in \mathscr{O}_{\le}$ であり,$\forall U_1,\, U_2 \in \mathcal{O}_{\le}$ に対して $x \in U_1 \cap U_2$ であることは
\begin{align}
     \forall y \in P,\; x \le y \IMP y \in U_1 \AND y \in U_2
\end{align}
と同値なので $U_1 \cap U_2 \in \mathscr{O}_{\le}$ であり,
さらに勝手な開集合族 $\Familyset[big]{U_\lambda \in \mathscr{O}_{\le}}{\lambda \in \Lambda}$ に対して
$x \in \bigcup_{\lambda \in \Lambda} U_\lambda$ は
\begin{align}
    \exists \alpha \in \Lambda,\; \forall y \in P,\; x \le y \IMP y \in U_\alpha \subset \bigcup_{\lambda \in \Lambda} U_\lambda 
\end{align}
と同値であるから $\bigcup_{\lambda \in \Lambda} U_\lambda \in \mathscr{O}_{\le}$ であり,$\mathscr{O}_{\le}$ は集合 $P$ の位相である.

\begin{myexample}[label=ex:topo-poset]{$[n]$ の位相}
    半順序集合 $[2] \coloneqq \{0 \le 1 \le 2\}$ を考える.このとき,\hyperref[def:topo-poset]{位相 $\mathscr{O}_{\le}$}とは
    \begin{align}
        \mathscr{O}_{\le} = \bigl\{\, \emptyset,\, \{2\},\, \{1,\, 2\},\, \{0,\, 1,\, 2\} \,\bigr\} 
    \end{align}
    のことである.同様に,半順序集合 $[n] \coloneqq \{0 \le 1 \le \cdots \le n\}$ に対して
    \begin{align}
        \mathscr{O}_{\le} = \bigl\{\, \emptyset,\, \{n\},\, \{n-1,\, n\},\, \dots,\, \{0,\, \dots,\, n\}  \,\bigr\}
    \end{align}
    が成り立つ.
\end{myexample}

\begin{mydef}[label=def:stratified-space]{層状化空間・層状化写像}
    $(P,\, \le)$ を半順序集合とし,定義\ref{def:topo-poset}の位相を入れて位相空間にする.
    
    このとき,位相空間 $X$ が $\bm{P}$\textbf{-層状化}されている ($P$-stratified) とは,連続写像 $s \colon X \lto P$ が存在することを言う.
    組 $(X,\, s \colon X \lto P)$ のことを\textbf{$\bm{P}$-層状化空間} ($P$-stratified space) と呼ぶ.

    \tcblower

    層状化空間 $(X,\, s \colon X \lto P),\; (X',\, s' \colon X' \lto P')$ の間の\textbf{層状化写像} (stratified map) とは,連続写像の組み $(f \colon X \lto X',\; g \colon P \lto P')$ であって以下の図式を可換にするもののこと:
    \begin{center}
        % https://tikzcd.yichuanshen.de/#N4Igdg9gJgpgziAXAbVABwnAlgFyxMJZABgBpiBdUkANwEMAbAVxiRAA0QBfU9TXfIRQBGclVqMWbdgHJuvEBmx4CRMsPH1mrRCAAK8vssFFRG6lqm69cruJhQA5vCKgAZgCcIAWyRkQOBBIAEwWkjogAMog1Ax0AEYwDHr8KkIgHliOABY4hiCePkiiAUGIAMxh2myRtgqFvoj+gcVVVgX5DSHULRWxCUkpxqq6mTl5bRGO3BRcQA
\begin{tikzcd}
X \arrow[d, "s"'] \arrow[r, red,"f"] & X' \arrow[d, "s'"] \\
P \arrow[r, red,"g"']                & P'                
\end{tikzcd}
    \end{center}
\end{mydef}

\begin{myexample}[label=ex:str-CW]{CW複体}
    CW複体 $X$ を与える.$X_{\le k}$ を $X$ の $k$-骨格とするとき,$X_k \setminus X_{k-1}$ を $k \in \mathbb{Z}_{\ge 0}$ に写す写像 $s \colon X \lto \mathbb{Z}_{\ge 0}$ は $X$ の\hyperref[def:stratified-space]{層状化}を与える.
\end{myexample}


\begin{mydef}[label=def:strat-emb]{層状化埋め込み}
    \hyperref[def:stratified-space]{層状化写像} $(f,\, g) \colon (X,\, s \colon X \lto P) \lto (X',\, s' \colon X' \lto P')$ が\textbf{層状化開埋め込み} (stratified open embedding) であるとは,以下の2条件を充たすことを言う:
    \begin{enumerate}
        \item 連続写像 $f \colon X \lto X'$ は位相的埋め込みである\footnote{i.e. $f \colon X \lto f(X)$ が同相写像}
        \item $\forall p \in P$ に対して,$f$ の制限
        \begin{align}
            f|_{s^{-1}(\{p\})} \colon s^{-1} \bigl( \{p\} \bigr) \lto s'{}^{-1} \bigl( \{g(p)\} \bigr) 
        \end{align}
        は位相的埋め込みである.
    \end{enumerate}
\end{mydef}

\begin{marker}
    以下では混乱が生じにくい場合,\hyperref[def:stratified-space]{層状化写像} $(f,\, g) \colon (X,\, s \colon X \lto P) \lto (X',\, s' \colon X' \lto P')$ のことを $f \colon (X \to P) \lto (X' \to P')$ と略記し,連続写像 $g \colon P \lto P'$ のことも $f$ と書く.
\end{marker}

圏 $\StTOP$ を,
\begin{itemize}
    \item 第2可算なHausdorff空間の\hyperref[def:stratified-space]{層状化空間}を対象とする
    \item \hyperref[def:strat-emb]{層状化埋め込み}を射とする
\end{itemize}
ことで定義する.

\subsection{$C^0$ 級層状化空間}


\begin{mydef}[label=def:str-cone]{コーン}
    \hyperref[def:stratified-space]{層状化空間} $(X,\, s \colon X \to P)$ を与える.
    $X$ の\textbf{コーン} (cone) とは,以下のようにして構成される\hyperref[def:stratified-space]{層状化空間} $\bigl( \Cone{X},\, \Cone{s} \colon \Cone{X} \lto \Cone{P} \bigr)$ のこと:
    \begin{itemize}
        \item 位相空間 $\Cone{X}$ を,押し出し位相空間
        \begin{align}
            \Cone{X} \coloneqq \{\mathrm{pt}\} \amalg_{\{0\} \times X} (\mathbb{R}_{\ge 0} \times X)
        \end{align}
        と定義する:
        \begin{center}
            % https://tikzcd.yichuanshen.de/#N4Igdg9gJgpgziAXAbVABwnAlgFyxMJZABgBpiBdUkANwEMAbAVxiRAB13hjOBfTvAFt4AAgAaIXqXSZc+QigCM5KrUYs2nQXRwALAEb7gAJV4B9YJwDmMEcV4iBWYXHGTpIDNjwEiZRar0zKyIHFxaOroAToLAaDj87LzuMt7yRMoB1EEaoZyW7Np6MXEJfI7sdNoMVhb5PEkVQqJiDgAUEXqGJuYFNnYOTi7iAJSSqjBQNggooABmURCCSGQgOBBIAEzUDHT6MAwACrI+CiBRWFa6OCkgC0tIymsbiADMUvOLy4ir64-Z6hCYW45SGok60ViWCg5gkHzuXy21D+bx2ewOxzSvlCFyuN14FF4QA
        \begin{tikzcd}
\{0\}\times X \arrow[d] \arrow[r, hookrightarrow, "\{0\} \times \mathrm{id}_X"] & \mathbb{R}_{\ge 0} \times X \arrow[d,red]                                 \\
\{\mathrm{pt}\} \arrow[r,hookrightarrow,red]                                       & \textcolor{red}{\{\mathrm{pt}\} \amalg_{\{0\} \times X} (\mathbb{R}_{\ge 0} \times X)}
        \end{tikzcd}
        \end{center}
        \item 半順序集合 $\Cone{P}$ を,$P$ に最小の要素 $-\infty$ を付け足すことで定義する.これは半順序集合の圏における押し出し
        \begin{align}
            \Cone{P} \coloneqq \{-\infty\} \amalg_{\{0\} \times X} \bigl( [1] \times P \bigr) 
        \end{align}
        である.
        \item 連続写像
        \begin{align}
            \mathbb{R}_{\ge 0} \times X &\lto [1] \times P, \\
            (t,\, x) &\lmto 
            \begin{cases}
                \bigl( 0,\, s(x) \bigr), &t=0, \\
                \bigl( 1,\, s(x) \bigr), &t>0
            \end{cases}
        \end{align}
        が\hyperref[def:colim]{押し出しの普遍性}により誘導する連続写像 $\Cone{X} \lto \Cone{P}$ を $\Cone{s}$ と書く.
    \end{itemize}
\end{mydef}

\begin{mydef}[label=def:Snglr-C0]{$C^0$ 級層状化空間}
    以下を充たす $\StTOP$ の最小の\hyperref[def:faithful]{充満部分圏}を $\Snglr$ と書き,圏 $\Snglr$ の対象を\textbf{$\bm{C^0}$ 級層状化空間} ($C^0$ stratified space) と呼ぶ:
    \begin{description}
        \item[\textbf{(Snglr-1)}] $(\emptyset \to \emptyset) \in \Obj{\Snglr}$
        \item[\textbf{(Snglr-2)}] 
        
        $(X \to P) \in \Obj{\Snglr}$ かつ $X,\, P$ が位相空間としてコンパクト 
        
        $\IMP$ $\Cone{X \to P} \in \Obj{\Snglr}$
        
        \item[\textbf{(Snglr-3)}] 
        
        $(X \to P) \in \Obj{\Snglr}$ $\IMP$ $(X \times \mathbb{R} \to P) \in \Obj{\Snglr}$\footnote{$X \times \mathbb{R}$ の層状化は,連続写像 $X \times \mathbb{R} \lto X,\; (x,\, t) \lmto x$ を前もって合成することにより定める.}
        
        \item[\textbf{(Snglr-4)}] 
        
        $(X \to P) \in \Obj{\Snglr}$ かつ $\Hom{\StTOP} \bigl( (U \to P_U),\, (X \to P) \bigr) \neq \emptyset$ 
        
        $\IMP$ $(U \to P_U) \in \Obj{\Snglr}$
        
        \item[\textbf{(Snglr-5)}]  
        
        $(X \to P) \in \Obj{\StTOP}$ が開被覆 $\Familyset[big]{(U_\lambda \to P_\lambda) \lto (X \to P)}{\lambda \in \Lambda}$\footnote{i.e. $\Familyset[big]{U_\lambda}{\lambda \in \Lambda},\; \Familyset[big]{P_\lambda}{\lambda \in \Lambda}$ が,それぞれ位相空間 $X,\, P$ の開被覆を成す.} を持ち,かつ $\forall \lambda \in \Lambda$ に対して $(U_\lambda \to P_\lambda) \in \Obj{\Snglr}$ 
        
        $\IMP$ $(X \to P) \in \Obj{\Snglr}$
    \end{description}
\end{mydef}

\begin{myexample}[label=ex:topomfld]{位相多様体は $C^0$ 級層状化空間}
    \hyperref[def:Snglr-C0]{\textsf{\textbf{(Snglr-1)}}}より,$* \coloneqq \Cone{\emptyset \to \emptyset} \in \Obj{\Snglr}$ である.
    \hyperref[def:Snglr-C0]{\textsf{\textbf{(Snglr-3)}}}より,$\forall n \ge 0$ に対して $\mathbb{R}^n = (\mathbb{R}^n \to [0]) \in \Obj{\Snglr}$ であることが帰納的に分かる.
    $\mathbb{R}^n$ の任意の開集合 $U \hookrightarrow \mathbb{R}^n$ に対して,
    \begin{center}
            \begin{tikzcd}
        U \arrow[d] \arrow[r, hookrightarrow] & \mathbb{R}^n \arrow[d] \\
        {[0]} \arrow[r, "="']                & {[0]}               
    \end{tikzcd}
    \end{center}
    は\hyperref[def:strat-emb]{層状化埋め込み}であり,従って \hyperref[def:Snglr-C0]{\textsf{\textbf{(Snglr-4)}}}より $U \coloneqq (U \to [0]) \in \Obj{\Snglr}$ が分かる.
    以上の考察と\hyperref[def:Snglr-C0]{\textsf{\textbf{(Snglr-5)}}}を併せて,任意の位相多様体 $M$ は\footnote{より正確には,$M$ を\hyperref[def:stratified-space]{層状化空間} $(M \to [0])$ と同一視している.}圏 $\Snglr$ の対象である.
\end{myexample}




\end{document}