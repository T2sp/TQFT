\documentclass[TQFT_main]{subfiles}

\begin{document}

% \setcounter{}{}
\chapter{層状化空間・因子化ホモロジー}

~\cite{AFT2014stratified}, ~\cite{AFT2014FH}のレビュー

\section{conically smoothな層状化空間}

\subsection{層状化空間}

\begin{mydef}[label=def:topo-poset]{半順序集合の位相}
    $(P,\, \le)$ を半順序集合とする.
    $P$ 上の位相 $\mathscr{O}_{\le} \subset 2^P$ を以下で定義する:
    \begin{align}
        U \in \mathscr{O}_{\le} \DEF \forall x \in U,\, \forall y \in P,\; \bigl[\, x \le y \IMP y \in U \,\bigr]
    \end{align}
\end{mydef}

実際,空集合の定義から $\emptyset \in \mathscr{O}_{\le}$ であり,$\forall U_1,\, U_2 \in \mathcal{O}_{\le}$ に対して $x \in U_1 \cap U_2$ であることは
\begin{align}
     \forall y \in P,\; x \le y \IMP y \in U_1 \AND y \in U_2
\end{align}
と同値なので $U_1 \cap U_2 \in \mathscr{O}_{\le}$ であり,
さらに勝手な開集合族 $\Familyset[big]{U_\lambda \in \mathscr{O}_{\le}}{\lambda \in \Lambda}$ に対して
$x \in \bigcup_{\lambda \in \Lambda} U_\lambda$ は
\begin{align}
    \exists \alpha \in \Lambda,\; \forall y \in P,\; x \le y \IMP y \in U_\alpha \subset \bigcup_{\lambda \in \Lambda} U_\lambda 
\end{align}
と同値であるから $\bigcup_{\lambda \in \Lambda} U_\lambda \in \mathscr{O}_{\le}$ であり,$\mathscr{O}_{\le}$ は集合 $P$ の位相である.

\begin{myexample}[label=ex:topo-poset]{$[n]$ の位相}
    半順序集合 $[2] \coloneqq \{0 \le 1 \le 2\}$ を考える.このとき,\hyperref[def:topo-poset]{位相 $\mathscr{O}_{\le}$}とは
    \begin{align}
        \mathscr{O}_{\le} = \bigl\{\, \emptyset,\, \{2\},\, \{1,\, 2\},\, \{0,\, 1,\, 2\} \,\bigr\} 
    \end{align}
    のことである.同様に,半順序集合 $[n] \coloneqq \{0 \le 1 \le \cdots \le n\}$ に対して
    \begin{align}
        \mathscr{O}_{\le} = \bigl\{\, \emptyset,\, \{n\},\, \{n-1,\, n\},\, \dots,\, \{0,\, \dots,\, n\}  \,\bigr\}
    \end{align}
    が成り立つ.
\end{myexample}

\begin{mydef}[label=def:stratified-space]{層状化空間・層状化写像}
    $(P,\, \le)$ を半順序集合とし,定義\ref{def:topo-poset}の位相を入れて位相空間にする.
    
    このとき,位相空間 $X$ が $\bm{P}$\textbf{-層状化}されている ($P$-stratified) とは,連続写像 $s \colon X \lto P$ が存在することを言う.
    組 $(X,\, s \colon X \lto P)$ のことを\textbf{$\bm{P}$-層状化空間} ($P$-stratified space) と呼ぶ.
    また,$i \in P$ の逆像 $\bm{X_i} \coloneqq s^{-1}(\{i\}) \subset X$ のことを\textbf{$\bm{i}$-層} ($i$-strata) と呼ぶ.

    \tcblower

    層状化空間 $(X,\, s \colon X \lto P),\; (X',\, s' \colon X' \lto P')$ の間の\textbf{層状化写像} (stratified map) とは,連続写像の組み $(f \colon X \lto X',\; \tilde{f} \colon P \lto P')$ であって以下の図式を可換にするもののこと:
    \begin{center}
        % https://tikzcd.yichuanshen.de/#N4Igdg9gJgpgziAXAbVABwnAlgFyxMJZABgBpiBdUkANwEMAbAVxiRAA0QBfU9TXfIRQBGclVqMWbdgHJuvEBmx4CRMsPH1mrRCAAK8vssFFRG6lqm69cruJhQA5vCKgAZgCcIAWyRkQOBBIAEwWkjogAMog1Ax0AEYwDHr8KkIgHliOABY4hiCePkiiAUGIAMxh2myRtgqFvoj+gcVVVgX5DSHULRWxCUkpxqq6mTl5bRGO3BRcQA
\begin{tikzcd}
X \arrow[d, "s"'] \arrow[r, red,"f"] & X' \arrow[d, "s'"] \\
P \arrow[r, red,"\tilde{f}"']                & P'                
\end{tikzcd}
    \end{center}
\end{mydef}

\begin{myexample}[label=ex:strat-n]{$[n]$-層状化空間}
    半順序集合 $[n] \coloneqq \{0 \le \, \cdots \le n\}$ に対して\exref{ex:topo-poset}の位相を入れる.
    まず,
    \begin{align}
        X_0 = s^{-1} \bigl( [n] \setminus \{1,\, \dots,\, n\} \bigr) 
    \end{align}
    でかつ $\{1,\, \dots,\, n\}$ は $[n]$ の開集合であるから,$s$ の連続性から $X$ の部分空間 $X_0 \subset X$ は閉集合だとわかる.
    さらに
    \begin{align}
        X_0 \cup X_1 &= s^{-1} \bigl( [n] \setminus \{2,\, \dots,\, n\} \bigr), \\
        X_0 \cup X_1 \cup X_2 &= s^{-1} \bigl( [n] \setminus \{3,\, \dots,\, n\} \bigr), \\
        &\vdots \\
        X_0 \cup \cdots \cup X_n &= X
    \end{align}
    が成り立つことから,$s$ の連続性より $X$ の部分空間 $X_0 \cup \cdots \cup X_{m \le n}$ は閉集合だと分かる.
\end{myexample}

\begin{myexample}[label=ex:str-CW]{CW複体}
    CW複体 $X$ を与える.$X_{\le k}$ を $X$ の $k$-骨格とするとき,$X_k \setminus X_{k-1}$ を $k \in \mathbb{Z}_{\ge 0}$ に写す写像 $s \colon X \lto \mathbb{Z}_{\ge 0}$ は $X$ の\hyperref[def:stratified-space]{層状化}を与える.
\end{myexample}

直観的には,層状化空間とはdefect付き $C^\infty$ 多様体の一般化である.
特に $X$ を $C^\infty$ 多様体とするとき,$[n]$-層状化空間 $(X,\, s \colon X \lto [n])$ の $i$-層 $X_i$ とは,多様体 $X$ 上の余次元 $d-i$ のdefectを全て集めてきたものだと見做せる.

\begin{mydef}[label=def:strat-emb]{層状化開埋め込み}
    \hyperref[def:stratified-space]{層状化写像} $(f,\, \tilde{f}) \colon (X,\, s \colon X \lto P) \lto (X',\, s' \colon X' \lto P')$ が\textbf{層状化開埋め込み} (stratified open embedding) であるとは,以下の2条件を充たすことを言う:
    \begin{enumerate}
        \item 連続写像 $f \colon X \lto X'$ は位相的開埋め込みである\footnote{i.e. $f \colon X \lto f(X)$ が同相写像かつ $f(X) \subset Y$ が開集合}
        \item $\forall p \in P$ に対して,$f$ の \hyperref[def:stratified-space]{$p$-strata}への制限
        \footnote{\hyperref[def:stratified-space]{層状化写像}の定義に登場する図式の可換性より,$\forall x \in X_p$ に対して $s' \bigl( f(x) \bigr) = s' \circ f (x) = \tilde{f} \circ s (x) = \tilde{f}(p)$,i.e. $f(x) \in s'{}^{-1} \bigl( \{\tilde{f}(p)\} \bigr) = X'_{\tilde{f}(p)}$ が分かる.}
        \begin{align}
            f|_{X_p} \colon X_p \lto X'_{\tilde{f}(p)}
        \end{align}
        は位相的開埋め込みである.
    \end{enumerate}
\end{mydef}

\begin{marker}
    以下では混乱が生じにくい場合,\hyperref[def:stratified-space]{層状化空間} $(X,\, s \colon X \lto P)$ のことを $(X \xrightarrow{s} P)$ や $(X \to P)$ と略記する.
    さらに,\hyperref[def:stratified-space]{層状化写像} $(f,\, \tilde{f}) \colon (X,\, s \colon X \lto P) \lto (X',\, s' \colon X' \lto P')$ のことを $f \colon (X \to P) \lto (X' \to P')$ と略記し,連続写像 $\tilde{f} \colon P \lto P'$ のことも $f$ と書く場合がある.
\end{marker}

圏 $\StTOP$ を,
\begin{itemize}
    \item 第2可算なHausdorff空間であるような\hyperref[def:stratified-space]{層状化空間}を対象とする
    \item \hyperref[def:strat-emb]{層状化開埋め込み}を射とする
\end{itemize}
ことで定義する.

\subsection{$C^0$ 級層状化空間}


\begin{mydef}[label=def:str-cone,breakable]{コーン}
    \hyperref[def:stratified-space]{層状化空間} $(X \xrightarrow{s} P)$ を与える.
    $X$ の\textbf{コーン} (cone) とは,以下のようにして構成される\hyperref[def:stratified-space]{層状化空間} $\bigl( \Cone{X},\, \Cone{s} \colon \Cone{X} \lto \Cone{P} \bigr)$ のこと:
    \begin{itemize}
        \item 位相空間 $\Cone{X}$ を,\hyperref[def:pullback-pushout]{押し出し}位相空間
        \begin{align}
            \Cone{X} \coloneqq \{\mathrm{pt}\} \amalg_{\{0\} \times X} (\mathbb{R}_{\ge 0} \times X)
        \end{align}
        と定義する:
        \begin{center}
            % https://tikzcd.yichuanshen.de/#N4Igdg9gJgpgziAXAbVABwnAlgFyxMJZABgBpiBdUkANwEMAbAVxiRAB13hjOBfTvAFt4AAgAaIXqXSZc+QigCM5KrUYs2nQXRwALAEb7gAJV4B9YJwDmMEcV4iBWYXHGTpIDNjwEiZRar0zKyIHFxaOroAToLAaDj87LzuMt7yRMoB1EEaoZyW7Np6MXEJfI7sdNoMVhb5PEkVQqJiDgAUEXqGJuYFNnYOTi7iAJSSqjBQNggooABmURCCSGQgOBBIAEzUDHT6MAwACrI+CiBRWFa6OCkgC0tIymsbiADMUvOLy4ir64-Z6hCYW45SGok60ViWCg5gkHzuXy21D+bx2ewOxzSvlCFyuN14FF4QA
        \begin{tikzcd}
\{0\}\times X \arrow[d] \arrow[r, hookrightarrow, "\{0\} \times \mathrm{id}_X"] & \mathbb{R}_{\ge 0} \times X \arrow[d,red]                                 \\
\{\mathrm{pt}\} \arrow[r,hookrightarrow,red]                                       & \textcolor{red}{\{\mathrm{pt}\} \amalg_{\{0\} \times X} (\mathbb{R}_{\ge 0} \times X)}
        \end{tikzcd}
        \end{center}
        \item 半順序集合 $\Cone{P}$ を,$P$ に最小の要素 $-\infty$ を付け足すことで定義する.これは半順序集合の圏における押し出し
        \begin{align}
            \Cone{P} \coloneqq \{-\infty\} \amalg_{\{0\} \times P} \bigl( [1] \times P \bigr) 
        \end{align}
        である.
        \item 連続写像
        \begin{align}
            \mathbb{R}_{\ge 0} \times X &\lto [1] \times P, \\
            (t,\, x) &\lmto 
            \begin{cases}
                \bigl( 0,\, s(x) \bigr), &t=0, \\
                \bigl( 1,\, s(x) \bigr), &t>0
            \end{cases}
        \end{align}
        が誘導する連続写像 $\Cone{X} \lto \Cone{P}$ を $\Cone{s}$ と書く.
    \end{itemize}
\end{mydef}

位相空間の圏における押し出しの公式から,位相空間 $\Cone{X}$ とは
\begin{align}
    i_1 \colon \{0\} \times X &\lto \mathbb{R}_{\ge 0} \times X,\; x \lmto (0,\, x), \\
    i_2 \colon \{0\} \times X &\lto \{\mathrm{pt}\},\; x \lmto \mathrm{pt}
\end{align}
とおいたときのコイコライザ
\begin{center}
    % https://tikzcd.yichuanshen.de/#N4Igdg9gJgpgziAXAbVABwnAlgFyxMJZABgBpiBdUkANwEMAbAVxiRAB13hjOBfTvAFt4AAgAaIXqXSZc+QigCM5KrUYs2nYJ0F0cACwBOg4Ghz92FuroYBzEQAode-QCNXwAEq8A+tva2MCLEvCICWMJw4gCUktIgGNh4BEQATCrU9MysiBzsugZwAGbAAMK8DmKxvKowUIEIKKBFhhCCSGQgOBBIyiAMWGA5IHAQA1Ag1PowdBOIYEwMDNQ4dFgMbJBDkyP6WEU4SAC0fVkauVg+inHNre2Ind291HB7B8-9dK4wDAAKsskFCBDFhbPpDpl1MNLqkbiAWm0Pk9EKkarwgA
    \begin{tikzcd}
    \{0\}\times X \arrow[r, "i_1", shift left] \arrow[r, "i_2"', shift right] & \{\mathrm{pt}\}\sqcup (\mathbb{R}_{\ge 0} \times X) \arrow[r, "q"] & \mathsf{C}(X)
    \end{tikzcd}
\end{center}
である.i.e. 商位相空間
\begin{align}
    \frac{\mathbb{R}_{\ge 0} \times X}{i_1 (x) \sim i_2 (x)} = \frac{\mathbb{R}_{\ge 0} \times X}{\{0\} \times X}
\end{align}
のこと.従って $\Cone{s} \colon \Cone{X} \lto \Cone{P}$ とは,連続写像\footnote{$\Cone{P}$ の\hyperref[def:topo-poset]{位相 $\mathscr{O}_{\Cone{P}}$}は,$P$ の位相 $\mathscr{O}_{P}$ に1つの開集合 $\{-\infty\} \cup P$ を加えたものである.$\forall U \in \mathscr{O}_{P}$ に対して $\Cone{s}^{-1} (U) = \mathbb{R}_{\textcolor{red}{>0}} \times s^{-1}(U) \in \mathscr{O}_{\Cone{X}}$ で,かつ $\Cone{s}^{-1}(\{-\infty\} \cup P) = \Cone{X} \in \mathscr{O}_{\Cone{X}}$ なので $\Cone{s}$ は連続である.}
\begin{align}
    \frac{\mathbb{R}_{\ge 0} \times X}{\{0\} \times X} &\lto \Cone{P},\;
    [(t,\, x)] \lmto \begin{cases}
        -\infty, &t = 0 \\
        s(x), &t > 0
    \end{cases}
\end{align}
のことである.
また,コーンポイントのみからなる1点集合 $\{\mathrm{pt}\} \subset X$ は $q^{-1} \bigl( \{\mathrm{pt}\} \bigr) = \{0\} \times X$ を充たすが,$\{0\} \times X$ は 

\begin{marker}
    以下では,混乱の恐れがない限り\hyperref[def:stratified-space]{層状化空間} $(X \xrightarrow{s} P)$ の\hyperref[def:str-cone]{コーン}を $\Cone{X \xrightarrow{s} P}$ と略記する.
\end{marker}


\begin{mydef}[label=def:Snglr-C0]{$C^0$ 級層状化空間}
    以下を充たす $\StTOP$ の最小の\hyperref[def:faithful]{充満部分圏}を $\SnglrC$ と書き,圏 $\SnglrC$ の対象を\textbf{$\bm{C^0}$ 級層状化空間} ($C^0$ stratified space) と呼ぶ:
    \begin{description}
        \item[\textbf{(Snglr-1)}] $(\emptyset \to \emptyset) \in \Obj{\SnglrC}$
        \item[\textbf{(Snglr-2)}] 
        
        $(X \to P) \in \Obj{\SnglrC}$ かつ $X,\, P$ が位相空間としてコンパクト 
        
        $\IMP$ $\Cone{X \to P} \in \Obj{\SnglrC}$
        
        \item[\textbf{(Snglr-3)}] 
        
        $(X \to P) \in \Obj{\SnglrC}$ $\IMP$ $(X \times \mathbb{R} \to P) \in \Obj{\SnglrC}$\footnote{$X \times \mathbb{R}$ の層状化は,連続写像 $X \times \mathbb{R} \lto X,\; (x,\, t) \lmto x$ を前もって合成することにより定める.}
        
        \item[\textbf{(Snglr-4)}] 
        
        $(X \to P) \in \Obj{\SnglrC}$ かつ $\Hom{\StTOP} \bigl( (U \to P_U),\, (X \to P) \bigr) \neq \emptyset$ 
        
        $\IMP$ $(U \to P_U) \in \Obj{\SnglrC}$
        
        \item[\textbf{(Snglr-5)}]  
        
        $(X \to P) \in \Obj{\StTOP}$ が開被覆 $\Familyset[big]{(U_\lambda \to P_\lambda) \lto (X \to P)}{\lambda \in \Lambda}$\footnote{i.e. $\Familyset[big]{U_\lambda}{\lambda \in \Lambda},\; \Familyset[big]{P_\lambda}{\lambda \in \Lambda}$ が,それぞれ位相空間 $X,\, P$ の開被覆を成す.} を持ち,かつ $\forall \lambda \in \Lambda$ に対して $(U_\lambda \to P_\lambda) \in \Obj{\SnglrC}$ 
        
        $\IMP$ $(X \to P) \in \Obj{\SnglrC}$
    \end{description}
\end{mydef}

\begin{myexample}[label=ex:topomfld]{位相多様体は $C^0$ 級層状化空間}
    \hyperref[def:Snglr-C0]{\textsf{\textbf{(Snglr-1)}}}より,$* \coloneqq \Cone{\emptyset \to \emptyset} \in \Obj{\SnglrC}$ である.
    \hyperref[def:Snglr-C0]{\textsf{\textbf{(Snglr-3)}}}より,$\forall n \ge 0$ に対して $\mathbb{R}^n = (\mathbb{R}^n \to [0]) \in \Obj{\SnglrC}$ であることが帰納的に分かる.
    $\mathbb{R}^n$ の任意の開集合 $U \hookrightarrow \mathbb{R}^n$ に対して,
    \begin{center}
            \begin{tikzcd}
        U \arrow[d] \arrow[r, hookrightarrow] & \mathbb{R}^n \arrow[d] \\
        {[0]} \arrow[r, "="']                & {[0]}               
    \end{tikzcd}
    \end{center}
    は\hyperref[def:strat-emb]{層状化埋め込み}であり,従って \hyperref[def:Snglr-C0]{\textsf{\textbf{(Snglr-4)}}}より $U \coloneqq (U \to [0]) \in \Obj{\SnglrC}$ が分かる.
    以上の考察と\hyperref[def:Snglr-C0]{\textsf{\textbf{(Snglr-5)}}}を併せて,任意の位相多様体 $M$ は\footnote{より正確には,$M$ を\hyperref[def:stratified-space]{層状化空間} $(M \to [0])$ と同一視している.}圏 $\SnglrC$ の対象である.
\end{myexample}

\subsection{$C^0$ basic}

\begin{mydef}[label=def:C0-basic]{$C^0$ basic}
    \hyperref[def:Snglr-C0]{$\bm{C^0}$ 級層状化空間} $(X \to P) \in \Obj{\SnglrC}$ が $\bm{C^0}$\textbf{-basic}であるとは,ある $n \in \mathbb{Z}_{\ge 0}$ およびコンパクトな\hyperref[def:Snglr-C0]{$\bm{C^0}$ 級層状化空間} $(Z \to Q) \in \Obj{\SnglrC}$ が存在して
    $(X \to P) = \bigl(\mathbb{R}^n \to [0]\bigr) \times \Cone{Z \to Q}$ が成り立つことを言う.
\end{mydef}

いま,\hyperref[def:C0-basic]{$C^0$ basic}な $(U \to P_U) = (\mathbb{R}^n \to [0]) \times \Cone{Z \to P} \in \Obj{\SnglrC}$ を1つとる.
\hyperref[def:str-cone]{コーンの定義}から,$U$ の点を $(v,\, [t,\, z]) \in \mathbb{R}^n \times \frac{\mathbb{R}_{\ge 0} \times Z}{\{0\} \times Z}$ と表示することができる.
この表示の下で自己同相
\begin{align}
    \gamma \colon \mathbb{R}_{> 0} \times T \mathbb{R}^n \times \Cone{Z} &\lto \mathbb{R}_{> 0} \times T \mathbb{R}^n \times \Cone{Z}, \\
    \bigl( t,\, (v,\, p),\, [s,\, z] \bigr) &\lmto \bigl( t,\, (tv + p,\, p),\, [ts,\, z] \bigr)
\end{align}
を考える\footnote{接束 $T\mathbb{R}^n$ は $\mathbb{R}^{2n}$ と微分同相である.~\cite[p.23]{AFT2014stratified}の記法に合わせて底空間 $\mathbb{R}^n$ の点を $p$,$p$ 上のファイバーの元を $v$ としたとき $(v,\, p) \in T \mathbb{R}^n$ と書いた.命題\ref{prop:tangentbundle}の記法と順番が逆なので注意.}.

さらに,もう1つの\hyperref[def:C0-basic]{$C^0$ basic}な $(U' \to P_{U'}) = (\mathbb{R}^{n'} \to [0]) \times \Cone{Z' \to P'} \in \Obj{\SnglrC}$ および
$f \in \Hom{\SnglrC} \bigl( (U \to P_{U}),\, (U' \to P_{U'})  \bigr)$ をとる.ただし,$f$ はコーンポイントをコーンポイントへ写す,
i.e. $\forall u \in \mathbb{R}^n$ に対して $f (u,\, \mathrm{pt}) \in \mathbb{R}^{n'} \times \{\mathrm{pt}\}$ が成り立つことを仮定する.
$f|_{\mathbb{R}^n} \colon \mathbb{R}^n \times \{\mathrm{pt}\} \lto \mathbb{R}^{n'} \times \{\mathrm{pt}\}$ を $f$ のコーンポイントへの制限として,
\begin{align}
    f_{\Delta} \colon \mathbb{R}_{\textcolor{red}{>0}} \times T \mathbb{R}^n \times \Cone{Z} &\lto \mathbb{R}_{\textcolor{red}{>0}} \times T \mathbb{R}^{n'} \times \Cone{Z'}, \\
    \bigl( t,\, v,\, p,\, [s,\, z] \bigr) &\lmto \bigl( t,\, f|_{\mathbb{R}^n}(v),\, f(p,\, [ts,\, z]) \bigr)
\end{align}
とおこう.

\begin{myexample}[label=ex:cone-diff]{}
    $Z = Z' = \emptyset$ のとき,$f$ とは単に連続関数 $f \colon \mathbb{R}^n \lto \mathbb{R}^{n'}$ のことである.
    このとき,
    \begin{align}
        (\gamma^{-1} \circ f_{\Delta} \circ \gamma)(t,\, v,\, p)
        &= \gamma^{-1} \circ f_\Delta (t,\, tv+p,\, p) \\
        &= \gamma^{-1} \bigl(t,\, f(tv+p),\, f(p)\bigr) \\
        &= \left( t,\, \frac{f(tv+p) - f(p)}{t},\, f(p) \right)
    \end{align}
    と計算できるため,$f$ が $C^1$ 級であることと $\forall (v,\, p) \in T \mathbb{R}^n$ に対して $t \to +0$ の極限,i.e. $v$ に沿った片側方向微分が存在することは同値である.
\end{myexample}

\exref{ex:cone-diff}をもとに,\hyperref[def:C0-basic]{$C^0$ basic}な\hyperref[def:Snglr-C0]{$C^0$ 級層状化空間}の間の\hyperref[def:strat-emb]{層状化開埋め込み}の\textbf{conically smoothness}を定義する.
$C^\infty$ 多様体の $C^\infty$ 構造の定義においては,チャート $(U,\, \varphi \colon \mathbb{R}^n \to U),\, (V,\, \psi \colon \mathbb{R}^n \to V)$ の間の変換関数 $\psi^{-1} \circ \varphi \colon \mathbb{R}^n \lto \mathbb{R}^{n}$ が $C^\infty$ 級であることを要請した.
次の小節で\textbf{conically smooth structure}の定義を行うが,その際にチャートに対応するものは\textbf{basic} $U = \mathbb{R}^n \times \Cone{Z}$ から着目している\hyperref[def:Snglr-C0]{$C^0$-級層状化空間} $X$ への\hyperref[def:strat-emb]{層状化開埋め込み} $\varphi \colon U \hookrightarrow X$ であり,
概ね\footnote{コーンポイントをコーンポイントに写さない変換関数も存在しうるので,これだけではいけない.}2つのチャート $\varphi \colon U \hookrightarrow X,\; \psi \colon V \hookrightarrow X$ の間の変換関数 $\psi^{-1} \circ \varphi \colon U \lto V$ に対して\hyperref[def:c-smooth-along]{conically smooth (along $\mathbb{R}^n$)}であることを要請する.

\begin{mydef}[label=def:c-smooth-along]{$\mathbb{R}^n$ に沿ってconically smooth}
    \begin{itemize}
        \item \hyperref[def:C0-basic]{$C^0$ basic}な $(U \to P_{U}) = (\mathbb{R}^{n} \to [0]) \times \Cone{Z \to P} \in \Obj{\SnglrC}$
        \item \hyperref[def:C0-basic]{$C^0$ basic}な $(U' \to P_{U'}) = (\mathbb{R}^{n'} \to [0]) \times \Cone{Z' \to P'} \in \Obj{\SnglrC}$
        \item $f \in \Hom{\SnglrC} \bigl( (U \to P_{U}),\, (U' \to P_{U'})  \bigr)$ であって,コーンポイントを保存するもの
    \end{itemize}
    を与える.このとき,$f$ が\textbf{$\bm{\mathbb{R}^n}$ に沿って $C^1$ 級} ($C^1$ along $\mathbb{R}^n$) であるとは,以下の図式を可換にする連続写像
    \begin{align}
        \textcolor{red}{\tilde{D}f} \colon \mathbb{R}_{\textcolor{red}{\ge 0}} \times T \mathbb{R}^n \times \Cone{Z} &\lto \mathbb{R}_{\textcolor{red}{\ge 0}} \times T \mathbb{R}^{n'} \times \Cone{Z'}
    \end{align}
    が存在することを言う:
    \begin{center}
        % https://tikzcd.yichuanshen.de/#N4Igdg9gJgpgziAXAbVABwnAlgFyxMJZABgBpiBdUkANwEMAbAVxiRAB12BbOnACwBGA4ACUAvgH1gnHDAAeOAMYQGEAE7A1MKGOnsA5jAAExMWKMysXeEYAqF7r0HDxAPTAO81uA4DCBGGAALTEQMVJ0TFx8QhQARnIqWkYWNk4efiFRST1ZBWVVDS0dPUMTM08rG3t0pyy3YDAAcnNLbz8A4JawiJAMbDwCIjI4pPpmVkQOR0yXHIA+csr2mpnnbPdlm05-MECQnsiBmKIE0epx1Kna2eypRdMtn1WM9Ybm1vYvbfZd-e6xEltIYEChQAAzNQQLhIABM1BwECQAGZwhCoTDEPCQIikKZepDoSiEUjEHE0SBCZjsbjEMiLilJtN9HQuDxXMAALTkziKLBqRRGcFSTgAERgDBwdE+fIFDhZbLoIGoDDoAglAAUooNYiA1Fh9HwcIdKRi8SSkAlkhM0l8sAxYMBRWJwcqQAwsHs2FA6HA+NowhQxEA
        \begin{tikzcd}
        \mathbb{R}_{\textcolor{red}{\ge 0}} \times T \mathbb{R}^n \times \Cone{Z} \arrow[r, "\tilde{D}f", red, dashed]                         & \mathbb{R}_{\textcolor{red}{\ge 0}} \times T \mathbb{R}^{n'} \times \Cone{Z'} \\
        \mathbb{R}_{> 0} \times T \mathbb{R}^n \times \Cone{Z} \arrow[r] \arrow[u] \arrow[r, "\gamma^{-1}\circ f_{\Delta} \circ \gamma"'] & \mathbb{R}_{> 0} \times T \mathbb{R}^{n'} \times \Cone{Z'} \arrow[u]         
        \end{tikzcd}
    \end{center}
    \tcblower
    このような拡張が存在するとき,第一変数を $t=0$ に制限して得られる連続写像を
    \begin{align}
        \bm{Df} \colon T \mathbb{R}^n \times \Cone{Z} \lto T \mathbb{R}^{n'} \times \Cone{Z'}
    \end{align}
    と書く.$f$ が\textbf{$\bm{\mathbb{R}^n}$ に沿って $C^r$ 級} であるとは,$Df$ が $\mathbb{R}^n$ に沿って $C^{r-1}$ 級であることを言う.
    $f$ が\textbf{$\bm{\mathbb{R}^n}$ に沿ってconically smooth}であるとは,$\forall r \ge 1$ について $C^r$ 級であることを言う.
\end{mydef}

\subsection{conically smoothな層状化空間}

次に行うべきは,与えられた\hyperref[def:Snglr-C0]{$\bm{C^0}$ 級層状化空間} $(X \to P) \in \Obj{\SnglrC}$ の上の\textbf{conically smooth structure} i.e. 変換関数が\hyperref[def:c-smooth-along]{conically smooth}であるような\textbf{極大アトラス}を定義することである.
この手続きは,次で定義する次元と深さに関する帰納法によって構成される.

\begin{mydef}[label=def:covering-dim]{被覆次元}
    $X$ を位相空間とする.以下の条件を充たす最小の $d \in \mathbb{Z}_{\ge -1}$ のことを(存在すれば)$X$ の\textbf{被覆次元} (covering dimension) と呼ぶ:
    
    \begin{description}
        \item[\textbf{(covering)}] 
        
        $X$ の任意の開被覆 $\mathscr{U}$ に対して,十分細かい細分 $\mathscr{V}_{\mathscr{U}} \prec \mathscr{U}$ をとると,任意の互いに異なる $\forall m > d+1$ 個の開集合 $V_1,\, \dots,\, V_{m} \in \mathscr{V}_{\mathscr{U}}$ の共通部分が空になるようにできる.特に,$\emptyset$ の被覆次元は $-1$ と定義する.
    \end{description}
    
    \tcblower

    点 $x \in X$ における\textbf{被覆次元}を以下で定義する:
    \begin{align}
        \dim_x X \coloneqq \inf\, \bigl\{\, \dim U \ge -1 \bigm| x \in U \underset{\text{open}}{\subset} X \,\bigr\}
    \end{align}
\end{mydef}


\begin{mydef}[label=def:dim-depth,breakable]{次元と深さ}
    空でない\hyperref[def:Snglr-C0]{$\bm{C^0}$ 級層状化空間} $(X \to P) \in \Obj{\SnglrC}$ を与える.
    \begin{itemize}
        \item $(X \to P)$ の点 $x \in X$ における\textbf{局所的次元} (local dimension) とは,点 $x$ における $X$ の\hyperref[def:covering-dim]{被覆次元}
        % \footnote{以下の条件を充たす最小の $d \in \mathbb{Z}_{\ge -1}$ のことを(存在すれば)$X$ の\textbf{被覆次元} (covering dimension) と呼ぶ:$X$ の任意の開被覆 $\mathscr{U}$ に対して,十分細かい細分 $\mathscr{V}_{\mathscr{U}} \prec \mathscr{U}$ をとると,任意の互いに異なる $\forall m > d+1$ 個の開集合 $V_1,\, \dots,\, V_{m} \in \mathscr{V}_{\mathscr{U}}$ の共通部分が空になるようにできる.特に,$\emptyset$ の被覆次元は $-1$ と定義する.} 
        $\bm{\dim_x(X)}$ のことを言う.
        \item $(X \to P)$ の\textbf{次元} (dimension) とは
        \begin{align}
            \bm{\dim (X \to P)} \coloneqq \sup_{x \in X} \dim_x (X)
        \end{align}
        のこと.
        \item $(X \xrightarrow{s} P)$ の点 $x \in X$ における\textbf{局所的深さ} (local depth) とは,
        \begin{align}
            \bm{\depth_x(X \to P)} \coloneqq \dim_x (X) - \dim_x (X_{s(x)})
        \end{align}
        のこと.
        \item $(X \to P)$ の\textbf{深さ} (depth) とは,
        \begin{align}
            \bm{\depth(X \to P)} \coloneqq \sup_{x \in X} \depth_x(X \to P)
        \end{align}
        のこと.ただし,$\depth (\emptyset) \coloneqq -1$ と定義する.
    \end{itemize}
\end{mydef}

\begin{myexample}[label=ex:depth-Cone]{コーンの深さ}
    $n$ 次元位相多様体 $Z$ について,定義から $\forall x \in Z$ に対して $\dim_x(Z) = n$ が成り立つ.
    $Z$ を\exref{ex:topomfld}により\hyperref[def:Snglr-C0]{$C^0$ 級層状化空間} $(Z \xrightarrow{s} [0]) \in \Obj{\SnglrC}$ と見做すと,これの\hyperref[def:str-cone]{コーン} $\Cone{Z \xrightarrow{s} [0]}$ について
    \begin{align}
        \depth_x \bigl( \Cone{Z \xrightarrow{s} [0]} \bigr) 
        =
        \begin{cases}
            n+1, &x = \mathrm{pt}, \\
            0, &\text{otherwise}
        \end{cases}
    \end{align}
    であることがわかる.実際 $\Cone{Z}_{\Cone{s}(\mathrm{pt})} = \{\mathrm{pt}\}$ であるが,1点からなる位相空間の\hyperref[def:covering-dim]{被覆次元}は $0$ 次元なので $\dim_{\mathrm{pt}} (\Cone{Z}_{\Cone{s}(\mathrm{pt})}) = 0$ である.
        % ところで, $\dim_{\mathrm{pt}} \bigl( \Cone{Z} \bigr) = n+1$ である.
    一方,コーンポイント以外の点 $x \in \Cone{Z}$ に対して\hyperref[def:stratified-space]{$\Cone{s}(x)$-層}は
    $\Cone{Z}_{\Cone{s}(x)} = \mathbb{R}_{\textcolor{red}{> 0}} \times Z \approx \mathbb{R} \times Z$ であるから,
    $\dim_{x} (\Cone{Z}_{\Cone{s}(x)}) = n + 1$ と計算できる\footnote{さらに,$\forall x \in \Cone{Z}$ に対して $\dim_{x} \Cone{Z} = n+1$ である.}.
    
     また,$\forall (X \to P) \in \Obj{\SnglrC}$ に対して
    \begin{align}
        \dim \bigl( (\mathbb{R}^m \to [0]) \times (X \to P) \bigr) &= m + \dim \bigl( X \to P \bigr), \\
        \depth \bigl( (\mathbb{R}^m \to [0]) \times (X \to P) \bigr) &= \depth \bigl( X \to P \bigr) 
    \end{align}
    が成り立つ.従って,\hyperref[def:C0-basic]{$C^0$ basic}な $(U \to P_U) = (\mathbb{R}^n \to [0]) \times \Cone{Z \to P} \in \Obj{\SnglrC}$ に対して
    \begin{align}
        \depth (U \to P_U) = \depth (Z \to P) + 1
    \end{align}
    が成り立つ.
\end{myexample}

次元と深さに関する帰納法を実行する前に,構成したい圏を表す記号の整理をしておこう:
\begin{itemize}
    \item conically smoothチャートの素材となる,\textbf{basic}が成す圏
    \begin{align}
        \Bsc    
    \end{align}
    これは,$C^\infty$ 多様体の圏 $\Mfld$ において $\mathbb{R}^n\; (\forall n \ge -1)$ 全体が成す充満部分圏に相当するものである.
    
    \item 与えられた\hyperref[def:Snglr-C0]{$C^0$ 級層状化空間} $(X \to P) \in \Obj{\SnglrC}$ に対して,その上に入る\textbf{極大アトラス}\footnote{存在するか分からないし,存在したとして一意であるとは限らない.実際,例えば $C^\infty$ 多様体の段階においてさえ $\mathbb{R}^4$ の上の極大アトラス(i.e. $C^\infty$ 構造)は非可算無限個存在する~\cite{taubes1987gauge}.}全体が成す集合を返す\hyperref[def:presheaf]{前層}
    \begin{align}
        \Sm \colon \OP{(\SnglrC)} \lto \SETS
    \end{align}
    この対応が前層であることの直観は,\hyperref[def:strat-emb]{層状化開埋め込み} $f \in \Hom{\SnglrC} \bigl((X \to P),\, (Y \to Q)\bigr)$ が与えられると,$(X \to P)$ 上の極大アトラス $\Sm(X \to P)$ が $(Y \to Q)$ 上の極大アトラス $\Sm(Y \to Q)$ を「制限」する写像 $\Sm(f) \colon \Sm(Y \to Q) \lto \Sm(X \to P)$ によって得られるということである.
    
    \item \hyperref[def:dim-depth]{深さ}が $k$ 以下,かつ\hyperref[def:dim-depth]{次元}が $n$ 以下であるような\hyperref[def:Snglr-C0]{$C^0$ 級層状化空間}全体が成す $\SnglrC$ の充満部分圏を
    \begin{align}
        \SnglrC{}_{\underbrace{\le k}_{\text{depth}},\, \underbrace{\le n}_{\text{dimension}}}
    \end{align}
    と書く.同様に
    \begin{align}
        \Bsc_{\le k,\, \le n},\qquad \Sm_{\le k,\, \le n} \colon \OP{(\SnglrC_{\le k,\, \le \infty})} \lto \SETS
    \end{align}
    と書く.
    
    \item \textbf{conically smoothな層状化空間}の圏
    \begin{align}
        \Snglr
    \end{align}
    これを作ることが本小節の最終目標である.
\end{itemize}
帰納法により,$\forall k \ge -1$ に対して $\Bsc_{\le k,\, \le \infty}$ および $\Sm_{\le k,\, \le \infty} \colon \OP{(\SnglrC_{\le k,\, \le \infty})} \lto \SETS$ が構成される.

\begin{mydef}[label=def:induction-init]{帰納法の出発点}
    \hyperref[def:Snglr-C0]{\textsf{\textbf{(Snglr-1)}}}より $(\emptyset \to \emptyset) \in \Obj{\SnglrC_{\le -1,\, \le \infty}}$ である.
    \begin{enumerate}
        \item $\Bsc_{\le -1,\, \le \infty} \coloneqq \emptyset$
        \item $\Sm_{\le -1,\, \le \infty}(\emptyset) \coloneqq \{*\}$
    \end{enumerate}
    と定義する.
\end{mydef}

\begin{myassump}[label=assump:induction-Bsc-Sm]{帰納法の仮定}
    与えられた $k \ge -1$ に対して以下の構成が完了していると仮定する:
    \begin{enumerate}
        \item 圏 $\Bsc_{\le k,\, \le \infty}$
        \item 前層 $\Sm_{\le k,\, \le \infty} \colon \OP{(\SnglrC_{\le k,\, \le \infty})} \lto \SETS$
        \item 関手
        \begin{align}
            \mathbb{R} \times (\, \mhyphen \,) \colon \Bsc_{\le k,\, \le \infty} &\lto \Bsc_{\le k,\, \le \infty}, \\
            U &\lmto \mathbb{R} \times U, \\
            \bigl( U \xrightarrow{f} V \bigr) &\lmto \bigl( \mathbb{R} \times U \xrightarrow{\mathrm{id} \times f} \mathbb{R} \times V \bigr) 
        \end{align}
        およびそれが誘導する自然変換\footnote{$X$ の極大アトラス $\Familyset[big]{U_\alpha,\, \varphi_\alpha}{\alpha \in \Lambda}$ に対して,$\Familyset[big]{\mathbb{R} \times U_\alpha,\, \mathrm{id} \times \varphi_\alpha}{\alpha \in \Lambda}$ を対応づける.}
        % \begin{align}
        %     \OP{(\SnglrC_{\le k,\, \le \infty})} 
        % \end{align}
        \begin{center}
        \begin{tikzcd}[row sep=large, column sep=large]
            \OP{(\SnglrC_{\le k,\, \le \infty})}  \ar[bend left=50,r, "{\Sm_{\le k,\, \le  \infty}(\, \mhyphen \,)}"{name=U, above}] \ar[bend right=50,r, "{\Sm_{\le k,\, \le  \infty}(\mathbb{R} \times \, \mhyphen \,)}"{name=D, below}] &\SETS
            \ar[Rightarrow, from=U, to=D]
        \end{tikzcd}
        \end{center}
    \end{enumerate}
\end{myassump}

\begin{mydef}[label=def:Bsc-induction,breakable]{圏 $\Bsc_{\le k+1,\, \le \infty}$}
    帰納法の仮定\ref{assump:induction-Bsc-Sm}がある $k \ge -1$ において成立しているとする.
    また,\hyperref[def:C0-basic]{$C^0$ basic}を $U^n_Z \coloneqq (\mathbb{R}^n \to [0]) \times \Cone{Z \to P} \in \Obj{\SnglrC}$ と書く.
    このとき,圏 $\Bsc_{\le k+1,\, \le \infty}$ を以下で定義する:
    \begin{description}
        \item[(\textbf{対象})] 
        
        \hyperref[def:C0-basic]{$C^0$ basic}
        \footnote{\exref{ex:depth-Cone}より,$(Z \to P) \in \Obj{\SnglrC_{\le k,\, \le \infty}}$ であることが分かる.} $U^n_Z \in \Obj{\SnglrC_{\le k+1,\, \le \infty}}$ および,極大アトラス $\mathcal{A}_Z \in \Sm_{\textcolor{red}{\le k},\, \le \infty} (Z \to P)$ の組み
        \begin{align}
            (U^n_Z,\, \mathcal{A}_Z)
        \end{align}
        を対象とする.

        \item[(\textbf{射})] 
        
        任意の2つの対象 $(U^n_Z,\, \mathcal{A}_Z),\; (U^m_W,\, \mathcal{A}_W) \in \Obj{\Bsc_{\le k+1,\, \le \infty}}$ に対して,以下の条件を満たす\hyperref[def:strat-emb]{層状化開埋め込み} $f \in \Hom{\SnglrC_{\le k+1,\, \le \infty}} \bigl(U^n_Z,\, U^m_W \bigr)$ を対象とする:
        \begin{description}
            \item[\textbf{$\bm{f}$ がコーンポイントを保存しない場合}]   
            
            ある\hyperref[def:strat-emb]{層状化開埋め込み} $f_0 \in \Hom{\SnglrC_{\le k+1,\, \le \infty}} \bigl(U^n_Z,\; \mathbb{R}^m \times \mathbb{R}_{> 0} \times W\bigr)$ が存在して
            \begin{align}
                f \colon U^n_Z \xrightarrow{f_0} \mathbb{R}^m \times (\mathbb{R}_{> 0} \times W) \hookrightarrow U^m_W = \mathbb{R}^m \times \Cone{W}
            \end{align}
            と書けて,かつ $(U^n_Z,\, f_0) \in \mathcal{A}_{\mathbb{R}^m \times \mathbb{R}_{> 0} \times W} \in \Sm(\mathbb{R}^m \times \mathbb{R}_{> 0} \times W)$
            
            \item[\textbf{$\bm{f}$ がコーンポイントを保存する場合}]  
            
            $f$ は\hyperref[def:c-smooth-along]{$\mathbb{R}^n$ に沿ってconically smooth}であって,かつ $Df \colon \mathbb{R}^n \times U^n_Z \lto \mathbb{R}^m \times U^m_W$ が単射であり,
            かつ
            \begin{align}
                \mathcal{A}_{f^{-1}(U^m_W \setminus \mathbb{R}^m)} = \Sm_{\le k,\, \le \infty} \bigl( f|_{f^{-1}(U^m_W \setminus \mathbb{R}^m)} \bigr) \bigl(\mathcal{A}_{U^m_W \setminus \mathbb{R}^m}\bigr)
            \end{align}
            を充たす\footnote{ここで帰納法の仮定\ref{assump:induction-Bsc-Sm}-(3) を暗に使っている.}.ただし,$U^m_W \setminus \mathbb{R}^m \coloneqq U^m_W \setminus (\mathbb{R}^m \times \{\mathrm{pt}\}) = \mathbb{R}^{m+1} \times W$ と略記した.
        \end{description}
        
    \end{description}
\end{mydef}

\begin{mydef}[label=def:Sm-induction,breakable]{前層 $\Sm_{\le k+1,\, \le \infty}$}
    帰納法の仮定\ref{assump:induction-Bsc-Sm}がある $k \ge -1$ において成立しているとする.さらに定義\ref{def:Bsc-induction}が完成しているとする.
    \begin{itemize}
        \item \hyperref[def:Snglr-C0]{$C^0$ 級層状化空間} $\forall (X \to P) \in \Obj{\SnglrC_{\le k+1,\, \le \infty}}$ に対して,$X \to P$ の\textbf{アトラス} (atlas) を族
        \begin{align}
            \mathcal{A} \coloneqq \Familyset[big]{\bigl(U_\alpha \in \Obj{\Bsc_{\le k+1,\, \le \infty}},\, \varphi_\alpha \colon U_\alpha \hookrightarrow (X \to P)\bigr)}{\alpha \in \Lambda} \in \Sm_{\le k+1,\, \le \infty} (X \to P)
        \end{align}
        であって以下の条件を充たすものとして定義する:
        \begin{description}
            \item[\textbf{(Atlas-1)}] 
            
            $\mathcal{A}$ は $(X \to P)$ の開被覆である.
    
            \item[\textbf{(Atlas-2)}] 
            
            $\forall \alpha,\, \beta \in \Lambda$ および $\forall x \in \varphi_\alpha (U_\alpha) \cap \varphi_\beta (U_\beta)$ に対して,
            \hyperref[def:Bsc-induction]{圏 $\Bsc_{\le k+1,\, \le \infty}$}の可換図式
            \begin{center}
                % https://tikzcd.yichuanshen.de/#N4Igdg9gJgpgziAXAbVABwnAlgFyxMJZABgBpiBdUkANwEMAbAVxiRAHUQBfU9TXfIRRkAjFVqMWbAKoB9YAB0FjNAAs6XbrxAZseAkRGkx1es1aIQADS189gw+XFmpluYoUAjGDg3dxMFAA5vBEoABmAE4QALZIZCA4EEgAzKaSFiDh8krevpo8EdFxiAlJSEYS5mzZHirqmtQMdN4MAAr8+kIgkVhBqji2WcUV1OWIAExNLTDtnQ6Wvf2D6dWWSvSRalg5ygxqfoXDsaljyZPTrR32BpYMMOErVa4gG3RbqjseeYcUXEA
                \begin{tikzcd}
                \exists \textcolor{red}{W} \arrow[r, hookrightarrow, red,"f_{\beta}"] \arrow[d, hookrightarrow, red,"f_{\alpha}"'] & U_{\beta} \arrow[d, hookrightarrow, "\varphi_{\beta}"] \\
                U_{\alpha} \arrow[r, hookrightarrow, "\varphi_{\alpha}"']         & X                                     
                \end{tikzcd}
            \end{center}
            が存在して $x \in \varphi_\alpha \circ \textcolor{red}{f_\alpha} (\textcolor{red}{W}) = \varphi_\beta \circ \textcolor{red}{f_\beta} (\textcolor{red}{W})$ を充たす
        \end{description}
        アトラス $\mathcal{A}$ の元 $(U_\alpha,\, \varphi_\alpha) \in \mathcal{A}$ のことを\textbf{チャート} (chart) と呼ぶ.

        \item \hyperref[def:Snglr-C0]{$C^0$ 級層状化空間} $\forall (X \to P) \in \Obj{\SnglrC_{\le k+1,\, \le \infty}}$ の2つのアトラス $\mathcal{A},\, \mathcal{B}$ が\textbf{同値}であるとは,$\mathcal{A} \cup \mathcal{B}$ が $(X \to P)$ のアトラスであることを言う.
        これは $(X \to P)$ のアトラス全体の集合の上に同値関係を定める\footnote{同値関係であることの証明は~\cite[Lemma 3.2.11.]{AFT2014stratified}を参照.}.
        $(X \to P)$ の\textbf{極大アトラス} (maximal atlas) とは,この同値関係によるアトラス $\mathcal{A}$ の同値類 $[\mathcal{A}]$ のことを言う.

        \item 前層
        \begin{align}
            \Sm_{\le k+1,\, \le \infty} \colon \OP{(\SnglrC_{\le k+1,\, \le \infty})} \lto \SETS
        \end{align}
        を以下のように定義する:
        \begin{description}
            \item[\textbf{(対象)}] 
            
            任意の\hyperref[def:Snglr-C0]{$C^0$ 級層状化空間} $(X \to P) \in \Obj{\SnglrC_{\le k+1,\, \le \infty}}$ に対して
            \begin{align}
                \Sm_{\le k+1,\, \le \infty} (X \to P) \coloneqq \bigl\{\, [\mathcal{A}] \bigm| \mathcal{A}\; \text{is an atlas of}\; (X \to P) \,\bigr\} 
            \end{align}
            
            \item[\textbf{(射)}] 
            
            任意の\hyperref[def:strat-emb]{層状化開埋め込み} $f \in \Hom{\SnglrC_{\le k+1,\, \le \infty}}$ に対して,$f$ によるアトラスの引き戻しを対応付ける.
        \end{description}
    \end{itemize}
\end{mydef}

以上の帰納法をまとめて,conically smoothな層状化空間と\hyperref[def:strat-emb]{層状化開埋め込み}の圏 $\Snglr$ を得る.

\begin{mydef}[label=def:Snglr]{圏 $\Snglr$}
    \begin{itemize}
        \item \textbf{basic}のなす圏 $\Bsc$ を以下で定義する:
        \begin{align}
            \Bsc \coloneqq \bigcup_{k \ge -1} \Bsc_{\le k,\, \le \infty}
        \end{align}
        \item 極大アトラスの集合を与える関手 $\Sm \colon \OP{(\SnglrC)} \lto \SETS$ を以下の\hyperref[def:Kanext]{右Kan拡張}として定義する:
        \begin{center}
            % https://tikzcd.yichuanshen.de/#N4Igdg9gJgpgziAXAbVABwnAlgFyxMJZABgBpiBdUkANwEMAbAVxiRAB12B5ABWAApOAZTABzBgCcAwgH1gAHgAEnLGABmOAJ6lOpZewYx9qjZoC+ASjMgzpdJlz5CKMgEYqtRizadeA4WKSUlY2diAY2HgERK7kHvTMrIgc7EIAogAqQjYeMFCi8ESgahIQALZIZCA4EEgATNQJ3snCZXJKKupaOux6nIbGXeahxaUViFU1SK62o+XT1FOIDSAMqkkgUHRwABZ5II1eG60Hq3QARjAMPA5RziASWKI7ODlmQA
            \begin{tikzcd}[column sep=large]
            {\OP{(\SnglrC_{< \infty,\, \le \infty})}} \arrow[r, "{\Sm_{< \infty,\, \le \infty}}"] \arrow[d,hookrightarrow] & \SETS \\
            \OP{(\SnglrC)} \arrow[ru,red, "\Sm"', dashed]                                                       &      
            \end{tikzcd}
        \end{center}
        ただし,$\SnglrC_{< \infty,\, \le \infty} \coloneqq \bigcup_{k \ge -1} \SnglrC_{\le k,\, \le \infty}$ とおいた.
        \item \textbf{conically smoothな層状化空間} (conically smooth stratified space) と\hyperref[def:strat-emb]{層状化開埋め込み}の圏 $\Snglr$ を以下で定義する:
        
        \begin{description}
            \item[\textbf{(対象)}] 
            
            \hyperref[def:Snglr-C0]{$C^0$ 級層状化空間} $(X \to P) \in \Obj{\SnglrC}$ およびその\hyperref[def:Sm-induction]{極大アトラス} $\mathcal{A_X} \in \Sm (X \to P)$ の組み $\bigl((X\to P),\, \mathcal{A}_X\bigr)$ を対象とする.
        
            \item[\textbf{(射)}] 
            
            \hyperref[def:strat-emb]{層状化開埋め込み} $f \in \Hom{\SnglrC} \bigl((X \to P),\, (Y \to Q)\bigr)$ であって,$f^* \mathcal{A}_Y = \mathcal{A}_X$ を充たすものを射とする.
        \end{description}
    \end{itemize}
    
\end{mydef}

\subsection{conically smooth map}

ここまでは\hyperref[def:strat-emb]{層状化開埋め込み}のみを考えていたため,一般の\hyperref[def:stratified-space]{層状化写像}のconically smoothnessを定義しなくてはいけない.

\begin{mydef}[label=def:c-smooth-map]{conically smooth map}
    2つの\hyperref[def:Snglr]{\underline{\textbf{basic}}}\footnote{\hyperref[def:C0-basic]{$C^0$ basic}を $U^n_Z \coloneqq (\mathbb{R}^n \to [0]) \times \Cone{Z \to P} \in \Obj{\SnglrC}$ と書く.} $X = (U^n_Z,\, \mathcal{A}_Z),\, Y = (U^m_W,\, \mathcal{A}_W)\in \Obj{\Bsc}$ の間の\hyperref[def:stratified-space]{層状化写像} $f \colon U^n_Z \lto U^m_W$ が\textbf{conically smooth}であることを,
    \hyperref[def:dim-depth]{$\depth (Y)$} に関する帰納法によって定義する:
    \begin{enumerate}
        \item まず,$\depth (Y) = -1$ のときは $X = Y = \emptyset$ であり,一意的に定まる $X,\, Y$ 間の層状化写像がconically smoothであると定義する.
        \item 深さ $k \ge -1$ のbasicに対して定義が完了しているとする.
        $Y \in \Obj{\Bsc}$ の深さが高々 $k+1$ であるとき,層状化写像 $f \colon X \lto Y$ がconically smoothであることを以下で定義する:
        \begin{description}
            \item[\textbf{$\bm{f}$ がコーンポイントを保存しない場合}]   
            
            あるconically smoothな層状化写像 $f_0 \colon X \lto \mathbb{R}^m \times \mathbb{R}_{> 0} \times W$ が存在して
            \begin{align}
                f \colon X \xrightarrow{f_0} \mathbb{R}^m \times (\mathbb{R}_{> 0} \times W) \hookrightarrow Y = \mathbb{R}^m \times \Cone{W}
            \end{align}
            と書ける\footnote{\exref{ex:depth-Cone}より$\depth(W) < k+1$であり,帰納法の仮定が使える.}.
            
            \item[\textbf{$\bm{f}$ がコーンポイントを保存する場合}]  
            
            $f$ は\hyperref[def:c-smooth-along]{$\mathbb{R}^n$ に沿ってconically smooth}であって,かつ制限
            \begin{align}
                f|_{f^{-1}(Y \setminus \mathbb{R}^m)} \colon f^{-1}(Y \setminus \mathbb{R}^m) \lto Y \setminus \mathbb{R}^m
            \end{align}
            がconically smooth.ただし,$U^m_W \setminus \mathbb{R}^m \coloneqq U^m_W \setminus (\mathbb{R}^m \times \{\mathrm{pt}\}) = \mathbb{R}^{m+1} \times W$ と略記した.
        \end{description}
    \end{enumerate}
    
    \tcblower
    % \hyperref[def:C0-basic]{$C^0$ basic}を $U^n_Z \coloneqq (\mathbb{R}^n \to [0]) \times \Cone{Z \to P} \in \Obj{\SnglrC}$ と書く.
    \hyperref[def:Snglr]{conically smoothな層状化空間} $\bigl( (X \to P),\, \mathcal{A}_X \bigr),\, \bigl( (Y \to Q),\, \mathcal{A}_Y \bigr) \in \Obj{\Snglr}$ の間の\hyperref[def:stratified-space]{層状化写像} $f \colon (X \to P) \lto (Y \to Q)$ が\textbf{conically smooth}であるとは,
    任意のチャートの組み合わせ $(U,\, \varphi) \in \mathcal{A}_X,\, (V,\, \psi) \in \mathcal{A}_Y$ に対して
    \begin{align}
        \psi^{-1} \circ f \circ \varphi \colon U \lto V
    \end{align}
    がconically smooth (for \hyperref[def:Snglr]{basic}s) であることを言う.
\end{mydef}

\begin{myprop}[label=prop:c-smooth-map]{conically smooth mapの基本性質}
    2つの\hyperref[def:c-smooth-map]{conically smooth map}の合成もconically smoothである.
\end{myprop}

\begin{proof}
    ~\cite[Proposition 3.3.5]{AFT2014stratified}
\end{proof}

命題\ref{prop:c-smooth-map}より,\textbf{conically smoothな層状化空間}の圏を定義できる.

\begin{mydef}[label=def:c-smooth]{conically smoothな層状化空間の圏 $\Strat$ }
    \textbf{conically smoothな層状化空間}の圏 $\Strat$ を以下で定義する:
    \begin{description}
        \item[\textbf{(対象)}] 
        
        圏 \hyperref[def:Snglr]{$\Snglr$}と全く同じ対象を持つ:
        \begin{align}
            \Obj{\Strat} \coloneqq \Obj{\SnglrC}
        \end{align}
        \item[\textbf{(射)}] 
        
        \hyperref[def:c-smooth-map]{conically smooth map}を射とする.
    \end{description}
\end{mydef}

ここで,圏 $\Strat$ における特別な射に名前をつけておこう:

\begin{mydef}[label=def:cbl,breakable]{constructuble bundle}
    \begin{itemize}
        \item \hyperref[def:c-smooth-map]{conically smoothな層状化写像} $\pi \in \Hom{\Strat}\bigl( (E \to P),\, (B \to Q) \bigr)$ が\textbf{層状化ファイバー束} (conically smooth fiber bundle) であるとは,conically smoothな\hyperref[def:strat-emb]{層状化開埋め込み}の族 $\Familyset[big]{U_\alpha \hookrightarrow B}{\alpha \in \Lambda},\; \Familyset[big]{\varphi_\alpha \colon U_\alpha \times F_\alpha \hookrightarrow E}{\alpha \in \Lambda}$ が存在して以下を充たすことを言う:
        \begin{description}
            \item[\textbf{(Bun-1)}] 
            
            $\forall \alpha \in \Lambda$ に対して,圏 $\Strat$ における\hyperref[def:pullback-pushout]{引き戻し}の図式
            
            \begin{center}
                % https://tikzcd.yichuanshen.de/#N4Igdg9gJgpgziAXAbVABwnAlgFyxMJZABgBpiBdUkANwEMAbAVxiRAFUB9YAHR8bQALOgF8ABHzwBbeGIBi3PgOEiQI0uky58hFGQCMVWoxZsuvfgyGi1GkBmx4CRfaUPV6zVohAAhW5qOOi7kRp6mPgCiakYwUADm8ESgAGYAThBSSGQgOBBIAMzUOHRYDGyCEBAA1iAeJt4gfPRpQliKltaq6qkZWYiuufmIAEzFpeU+lTUBIOmZ2cXDg+GNfFJ0OIJpUsBoGQBWIpz6dSAMdABGMAwAClpOuiBpWPGCOLPz-UVDSGPGXjYfDQWBiIiAA
                \begin{tikzcd}
                U_{\alpha} \times F_{\alpha} \arrow[r, "\varphi_{\alpha}", hook] \arrow[d, "\mathrm{proj}_1"'] & E \arrow[d, "\pi"] \\
                U_{\alpha} \arrow[r, hook]                                                                     & B                 
                \end{tikzcd}
            \end{center}
            が成り立つ.

            \item[\textbf{(Bun-2)}] 
            
            族 $\Familyset[big]{U_\alpha}{\alpha \in \Lambda}$ は $B$ の開基である.
        \end{description}
        
        \item \hyperref[def:c-smooth-map]{conically smoothな層状化写像} $\pi \in \Hom{\Strat}\bigl( (E \to P),\, (B \to Q) \bigr)$ が\textbf{弱構成可能束} (weakly constructuble bundle) であるとは,$\forall q \in Q$ に対して,$\pi$ の\hyperref[def:stratified-space]{$q$-層}への制限
        \begin{align}
            \pi|_{\pi^{-1}(B_q)} \colon \pi^{-1}(B_q) \lto B_q
        \end{align}
        が層状化ファイバー束であることを言う.

        \item  \hyperref[def:c-smooth-map]{conically smoothな層状化写像} $\pi \in \Hom{\Strat}\bigl( (E \to P),\, (B \to Q) \bigr)$ が\textbf{構成可能束} (constructuble bundle) であることを,\hyperref[def:dim-depth]{$\depth (E)$} に関する帰納法によって定義する:
        \begin{enumerate}
            \item $\depth (E) = 0$ のとき,$\pi$ が構成可能束であるとは,$\pi$ が $C^\infty$ ファイバー束であることを言う.
            \item 深さ $k \ge 0$ までの定義が完了しているとする.$\depth(E) \le k+1$ のとき,$\pi$ が構成可能束であるとは,以下の2条件を充たすことを言う:
            \begin{description}
                \item[\textbf{(cBun-1)}] $f$ は弱構成可能束である.
                \item[\textbf{(cBun-2)}] $\forall q \in Q$ に対して,$\pi$ が誘導する層状化写像
                \begin{align}
                    \Link_{\pi^{-1}(B_q)} (E) \lto \pi^{-1}(B_q) \times_{B_q} \Link_{B_q} (B)
                \end{align}
                が構成可能束である.
            \end{description}
        \end{enumerate}
    \end{itemize}
\end{mydef}


\subsection{管状近傍・ハンドル分解}

\subsection{層状化空間の接構造}

\end{document}