\documentclass[TQFT_main]{subfiles}

\begin{document}

% \setcounter{}{}
\chapter{層状化空間・因子化ホモロジー}

~\cite{AFT2014stratified}, ~\cite{AFT2014FH}のレビュー

\section{conically smoothな層状化空間}

\subsection{層状化空間}

\begin{mydef}[label=def:topo-poset]{半順序集合の位相}
    $(P,\, \le)$ を半順序集合とする.
    $P$ 上の位相 $\mathscr{O}_{\le} \subset 2^P$ を以下で定義する:
    \begin{align}
        U \in \mathscr{O}_{\le} \DEF \forall x \in U,\, \forall y \in P,\; \bigl[\, x \le y \IMP y \in U \,\bigr]
    \end{align}
\end{mydef}

実際,空集合の定義から $\emptyset \in \mathscr{O}_{\le}$ であり,$\forall U_1,\, U_2 \in \mathcal{O}_{\le}$ に対して $x \in U_1 \cap U_2$ であることは
\begin{align}
     \forall y \in P,\; x \le y \IMP y \in U_1 \AND y \in U_2
\end{align}
と同値なので $U_1 \cap U_2 \in \mathscr{O}_{\le}$ であり,
さらに勝手な開集合族 $\Familyset[\big]{U_\lambda \in \mathscr{O}_{\le}}{\lambda \in \Lambda}$ に対して
$x \in \bigcup_{\lambda \in \Lambda} U_\lambda$ は
\begin{align}
    \exists \alpha \in \Lambda,\; \forall y \in P,\; x \le y \IMP y \in U_\alpha \subset \bigcup_{\lambda \in \Lambda} U_\lambda 
\end{align}
と同値であるから $\bigcup_{\lambda \in \Lambda} U_\lambda \in \mathscr{O}_{\le}$ であり,$\mathscr{O}_{\le}$ は集合 $P$ の位相である.

\begin{myexample}[label=ex:topo-poset]{{$[n]$} の位相}
    半順序集合 $[2] \coloneqq \{0 \le 1 \le 2\}$ を考える.このとき,\hyperref[def:topo-poset]{位相 $\mathscr{O}_{\le}$}とは
    \begin{align}
        \mathscr{O}_{\le} = \bigl\{\, \emptyset,\, \{2\},\, \{1,\, 2\},\, \{0,\, 1,\, 2\} \,\bigr\} 
    \end{align}
    のことである.同様に,半順序集合 $[n] \coloneqq \{0 \le 1 \le \cdots \le n\}$ に対して
    \begin{align}
        \mathscr{O}_{\le} = \bigl\{\, \emptyset,\, \{n\},\, \{n-1,\, n\},\, \dots,\, \{0,\, \dots,\, n\}  \,\bigr\}
    \end{align}
    が成り立つ.
\end{myexample}

\begin{mydef}[label=def:stratified-space]{層状化空間・層状化写像}
    $(P,\, \le)$ を半順序集合とし,定義\ref{def:topo-poset}の位相を入れて位相空間にする.
    
    このとき,位相空間 $X$ が $\bm{P}$\textbf{-層状化}されている ($P$-stratified) とは,連続写像 $s \colon X \lto P$ が存在することを言う.
    組 $(X,\, s \colon X \lto P)$ のことを\textbf{$\bm{P}$-層状化空間} ($P$-stratified space) と呼ぶ.
    また,$i \in P$ の逆像 $\bm{X_i} \coloneqq s^{-1}(\{i\}) \subset X$ のことを\textbf{$\bm{i}$-層} ($i$-strata) と呼ぶ.

    \tcblower

    層状化空間 $(X,\, s \colon X \lto P),\; (X',\, s' \colon X' \lto P')$ の間の\textbf{層状化写像} (stratified map) とは,連続写像の組み $(f \colon X \lto X',\; \tilde{f} \colon P \lto P')$ であって以下の図式を可換にするもののこと:
    \begin{center}
        % https://tikzcd.yichuanshen.de/#N4Igdg9gJgpgziAXAbVABwnAlgFyxMJZABgBpiBdUkANwEMAbAVxiRAA0QBfU9TXfIRQBGclVqMWbdgHJuvEBmx4CRMsPH1mrRCAAK8vssFFRG6lqm69cruJhQA5vCKgAZgCcIAWyRkQOBBIAEwWkjogAMog1Ax0AEYwDHr8KkIgHliOABY4hiCePkiiAUGIAMxh2myRtgqFvoj+gcVVVgX5DSHULRWxCUkpxqq6mTl5bRGO3BRcQA
\begin{tikzcd}
X \arrow[d, "s"'] \arrow[r, red,"f"] & X' \arrow[d, "s'"] \\
P \arrow[r, red,"\tilde{f}"']                & P'                
\end{tikzcd}
    \end{center}
\end{mydef}

\begin{myexample}[label=ex:strat-n]{{$[n]$}-層状化空間}
    半順序集合 $[n] \coloneqq \{0 \le \, \cdots \le n\}$ に対して\exref{ex:topo-poset}の位相を入れる.
    まず,
    \begin{align}
        X_0 = s^{-1} \bigl( [n] \setminus \{1,\, \dots,\, n\} \bigr) 
    \end{align}
    でかつ $\{1,\, \dots,\, n\}$ は $[n]$ の開集合であるから,$s$ の連続性から $X$ の部分空間 $X_0 \subset X$ は閉集合だとわかる.
    さらに
    \begin{align}
        X_0 \cup X_1 &= s^{-1} \bigl( [n] \setminus \{2,\, \dots,\, n\} \bigr), \\
        X_0 \cup X_1 \cup X_2 &= s^{-1} \bigl( [n] \setminus \{3,\, \dots,\, n\} \bigr), \\
        &\vdots \\
        X_0 \cup \cdots \cup X_n &= X
    \end{align}
    が成り立つことから,$s$ の連続性より $X$ の部分空間 $X_0 \cup \cdots \cup X_{m \le n}$ は閉集合だと分かる.
\end{myexample}

\begin{myexample}[label=ex:str-CW]{CW複体}
    CW複体 $X$ を与える.$X_{\le k}$ を $X$ の $k$-骨格とするとき,$X_k \setminus X_{k-1}$ を $k \in \mathbb{Z}_{\ge 0}$ に写す写像 $s \colon X \lto \mathbb{Z}_{\ge 0}$ は $X$ の\hyperref[def:stratified-space]{層状化}を与える.
\end{myexample}

直観的には,層状化空間とはdefect付き位相多様体の一般化である.
特に $X$ を位相多様体とするとき,$[n]$-層状化空間 $(X,\, s \colon X \lto [n])$ の $i$-層 $X_i$ とは,多様体 $X$ 上の余次元 $d-i$ のdefectを全て集めてきたものだと見做せる.

\begin{mydef}[label=def:strat-emb]{層状化開埋め込み}
    \hyperref[def:stratified-space]{層状化写像} $(f,\, \tilde{f}) \colon (X,\, s \colon X \lto P) \lto (X',\, s' \colon X' \lto P')$ が\textbf{層状化開埋め込み} (stratified open embedding) であるとは,以下の2条件を充たすことを言う:
    \begin{enumerate}
        \item 連続写像 $f \colon X \lto X'$ は位相的開埋め込みである\footnote{i.e. $f \colon X \lto f(X)$ が同相写像かつ $f(X) \subset Y$ が開集合}
        \item $\forall p \in P$ に対して,$f$ の \hyperref[def:stratified-space]{$p$-strata}への制限
        \footnote{\hyperref[def:stratified-space]{層状化写像}の定義に登場する図式の可換性より,$\forall x \in X_p$ に対して $s' \bigl( f(x) \bigr) = s' \circ f (x) = \tilde{f} \circ s (x) = \tilde{f}(p)$,i.e. $f(x) \in s'{}^{-1} \bigl( \{\tilde{f}(p)\} \bigr) = X'_{\tilde{f}(p)}$ が分かる.}
        \begin{align}
            f|_{X_p} \colon X_p \lto X'_{\tilde{f}(p)}
        \end{align}
        は位相的開埋め込みである.
    \end{enumerate}
\end{mydef}

\begin{marker}
    以下では混乱が生じにくい場合,\hyperref[def:stratified-space]{層状化空間} $(X,\, s \colon X \lto P)$ のことを $(X \xrightarrow{s} P)$ や $(X \to P)$ と略記する.
    さらに,\hyperref[def:stratified-space]{層状化写像} $(f,\, \tilde{f}) \colon (X,\, s \colon X \lto P) \lto (X',\, s' \colon X' \lto P')$ のことを $f \colon (X \to P) \lto (X' \to P')$ と略記し,連続写像 $\tilde{f} \colon P \lto P'$ のことも $f$ と書く場合がある.
\end{marker}

圏 $\StTOP$ を,
\begin{itemize}
    \item 第2可算なHausdorff空間であるような\hyperref[def:stratified-space]{層状化空間}を対象とする
    \item \hyperref[def:strat-emb]{層状化開埋め込み}を射とする
\end{itemize}
ことで定義する.

\subsection{$C^0$ 級層状化空間}


\begin{mydef}[label=def:str-cone,breakable]{コーン}
    \hyperref[def:stratified-space]{層状化空間} $(X \xrightarrow{s} P)$ を与える.
    $X$ の\textbf{コーン} (cone) とは,以下のようにして構成される\hyperref[def:stratified-space]{層状化空間} $\bigl( \Cone{X},\, \Cone{s} \colon \Cone{X} \lto \Cone{P} \bigr)$ のこと:
    \begin{itemize}
        \item 位相空間 $\Cone{X}$ を,\hyperref[def:pullback-pushout]{押し出し}位相空間
        \begin{align}
            \Cone{X} \coloneqq \{\mathrm{pt}\} \amalg_{\{0\} \times X} (\mathbb{R}_{\ge 0} \times X)
        \end{align}
        と定義する:
        \begin{center}
            % https://tikzcd.yichuanshen.de/#N4Igdg9gJgpgziAXAbVABwnAlgFyxMJZABgBpiBdUkANwEMAbAVxiRAB13hjOBfTvAFt4AAgAaIXqXSZc+QigCM5KrUYs2nQXRwALAEb7gAJV4B9YJwDmMEcV4iBWYXHGTpIDNjwEiZRar0zKyIHFxaOroAToLAaDj87LzuMt7yRMoB1EEaoZyW7Np6MXEJfI7sdNoMVhb5PEkVQqJiDgAUEXqGJuYFNnYOTi7iAJSSqjBQNggooABmURCCSGQgOBBIAEzUDHT6MAwACrI+CiBRWFa6OCkgC0tIymsbiADMUvOLy4ir64-Z6hCYW45SGok60ViWCg5gkHzuXy21D+bx2ewOxzSvlCFyuN14FF4QA
        \begin{tikzcd}
\{0\}\times X \arrow[d] \arrow[r, hookrightarrow, "\{0\} \times \mathrm{id}_X"] & \mathbb{R}_{\ge 0} \times X \arrow[d,red]                                 \\
\{\mathrm{pt}\} \arrow[r,hookrightarrow,red]                                       & \textcolor{red}{\{\mathrm{pt}\} \amalg_{\{0\} \times X} (\mathbb{R}_{\ge 0} \times X)}
        \end{tikzcd}
        \end{center}
        \item 半順序集合 $\Cone{P}$ を,$P$ に最小の要素 $-\infty$ を付け足すことで定義する.これは半順序集合の圏における押し出し
        \begin{align}
            \Cone{P} \coloneqq \{-\infty\} \amalg_{\{0\} \times P} \bigl( [1] \times P \bigr) 
        \end{align}
        である.
        \item 連続写像
        \begin{align}
            \mathbb{R}_{\ge 0} \times X &\lto [1] \times P, \\
            (t,\, x) &\lmto 
            \begin{cases}
                \bigl( 0,\, s(x) \bigr), &t=0, \\
                \bigl( 1,\, s(x) \bigr), &t>0
            \end{cases}
        \end{align}
        が誘導する連続写像 $\Cone{X} \lto \Cone{P}$ を $\Cone{s}$ と書く.
    \end{itemize}
\end{mydef}

位相空間の圏における押し出しの公式から,位相空間 $\Cone{X}$ とは
\begin{align}
    i_1 \colon \{0\} \times X &\lto \mathbb{R}_{\ge 0} \times X,\; x \lmto (0,\, x), \\
    i_2 \colon \{0\} \times X &\lto \{\mathrm{pt}\},\; x \lmto \mathrm{pt}
\end{align}
とおいたときの\hyperref[def:eq-coeq]{コイコライザ}
\begin{center}
    % https://tikzcd.yichuanshen.de/#N4Igdg9gJgpgziAXAbVABwnAlgFyxMJZABgBpiBdUkANwEMAbAVxiRAB13hjOBfTvAFt4AAgAaIXqXSZc+QigCM5KrUYs2nYJ0F0cACwBOg4Ghz92FuroYBzEQAode-QCNXwAEq8A+tva2MCLEvCICWMJw4gCUktIgGNh4BEQATCrU9MysiBzsugZwAGbAAMK8DmKxvKowUIEIKKBFhhCCSGQgOBBIyiAMWGA5IHAQA1Ag1PowdBOIYEwMDNQ4dFgMbJBDkyP6WEU4SAC0fVkauVg+inHNre2Ind291HB7B8-9dK4wDAAKsskFCBDFhbPpDpl1MNLqkbiAWm0Pk9EKkarwgA
    \begin{tikzcd}
    \{0\}\times X \arrow[r, "i_1", shift left] \arrow[r, "i_2"', shift right] & \{\mathrm{pt}\}\sqcup (\mathbb{R}_{\ge 0} \times X) \arrow[r, "q"] & \mathsf{C}(X),
    \end{tikzcd}
\end{center}
i.e. 商位相空間
\begin{align}
    \frac{\mathbb{R}_{\ge 0} \times X}{i_1 (x) \sim i_2 (x)} = \frac{\mathbb{R}_{\ge 0} \times X}{\{0\} \times X}
\end{align}
のこと.従って $\Cone{s} \colon \Cone{X} \lto \Cone{P}$ とは,連続写像\footnote{$\Cone{P}$ の\hyperref[def:topo-poset]{位相 $\mathscr{O}_{\Cone{P}}$}は,$P$ の位相 $\mathscr{O}_{P}$ に1つの開集合 $\{-\infty\} \cup P$ を加えたものである.$\forall U \in \mathscr{O}_{P}$ に対して $\Cone{s}^{-1} (U) = \mathbb{R}_{\textcolor{red}{>0}} \times s^{-1}(U) \in \mathscr{O}_{\Cone{X}}$ で,かつ $\Cone{s}^{-1}(\{-\infty\} \cup P) = \Cone{X} \in \mathscr{O}_{\Cone{X}}$ なので $\Cone{s}$ は連続である.}
\begin{align}
    \frac{\mathbb{R}_{\ge 0} \times X}{\{0\} \times X} &\lto \Cone{P},\;
    [(t,\, x)] \lmto \begin{cases}
        -\infty, &t = 0 \\
        s(x), &t > 0
    \end{cases}
\end{align}
のことである.
% また,コーンポイントのみからなる1点集合 $\{\mathrm{pt}\} \subset X$ は $q^{-1} \bigl( \{\mathrm{pt}\} \bigr) = \{0\} \times X$ を充たすが,$\{0\} \times X$ は 

\begin{marker}
    以下では,混乱の恐れがない限り\hyperref[def:stratified-space]{層状化空間} $(X \xrightarrow{s} P)$ の\hyperref[def:str-cone]{コーン}を $\Cone{X \xrightarrow{s} P}$ と略記する.
\end{marker}


\begin{mydef}[label=def:Snglr-C0]{$C^0$ 級層状化空間}
    以下を充たす $\StTOP$ の最小の\hyperref[def:faithful]{充満部分圏}を $\SnglrC$ と書き,圏 $\SnglrC$ の対象を\textbf{$\bm{C^0}$ 級層状化空間} ($C^0$ stratified space) と呼ぶ:
    \begin{description}
        \item[\textbf{(Snglr-1)}] $(\emptyset \to \emptyset) \in \Obj{\SnglrC}$
        \item[\textbf{(Snglr-2)}] 
        
        $(X \to P) \in \Obj{\SnglrC}$ かつ $X,\, P$ が位相空間としてコンパクト 
        
        $\IMP$ $\Cone{X \to P} \in \Obj{\SnglrC}$
        
        \item[\textbf{(Snglr-3)}] 
        
        $(X \to P) \in \Obj{\SnglrC}$ $\IMP$ $(X \times \mathbb{R} \to P) \in \Obj{\SnglrC}$\footnote{$X \times \mathbb{R}$ の層状化は,連続写像 $X \times \mathbb{R} \lto X,\; (x,\, t) \lmto x$ を前もって合成することにより定める.}
        
        \item[\textbf{(Snglr-4)}] 
        
        $(X \to P) \in \Obj{\SnglrC}$ かつ $\Hom{\StTOP} \bigl( (U \to P_U),\, (X \to P) \bigr) \neq \emptyset$ 
        
        $\IMP$ $(U \to P_U) \in \Obj{\SnglrC}$
        
        \item[\textbf{(Snglr-5)}]  
        
        $(X \to P) \in \Obj{\StTOP}$ が開被覆 $\Familyset[\big]{(U_\lambda \to P_\lambda) \lto (X \to P)}{\lambda \in \Lambda}$\footnote{i.e. $\Familyset[\big]{U_\lambda}{\lambda \in \Lambda},\; \Familyset[\big]{P_\lambda}{\lambda \in \Lambda}$ が,それぞれ位相空間 $X,\, P$ の開被覆を成す.} を持ち,かつ $\forall \lambda \in \Lambda$ に対して $(U_\lambda \to P_\lambda) \in \Obj{\SnglrC}$ 
        
        $\IMP$ $(X \to P) \in \Obj{\SnglrC}$
    \end{description}
\end{mydef}

\begin{myexample}[label=ex:topomfld]{位相多様体は $C^0$ 級層状化空間}
    \hyperref[def:Snglr-C0]{\textsf{\textbf{(Snglr-1)}}}より,$* \coloneqq \Cone{\emptyset \to \emptyset} \in \Obj{\SnglrC}$ である.
    \hyperref[def:Snglr-C0]{\textsf{\textbf{(Snglr-3)}}}より,$\forall n \ge 0$ に対して $\mathbb{R}^n = (\mathbb{R}^n \to [0]) \in \Obj{\SnglrC}$ であることが帰納的に分かる.
    $\mathbb{R}^n$ の任意の開集合 $U \hookrightarrow \mathbb{R}^n$ に対して,
    \begin{center}
            \begin{tikzcd}
        U \arrow[d] \arrow[r, hookrightarrow] & \mathbb{R}^n \arrow[d] \\
        {[0]} \arrow[r, "="']                & {[0]}               
    \end{tikzcd}
    \end{center}
    は\hyperref[def:strat-emb]{層状化埋め込み}であり,従って \hyperref[def:Snglr-C0]{\textsf{\textbf{(Snglr-4)}}}より $U \coloneqq (U \to [0]) \in \Obj{\SnglrC}$ が分かる.
    以上の考察と\hyperref[def:Snglr-C0]{\textsf{\textbf{(Snglr-5)}}}を併せて,任意の位相多様体 $M$ は\footnote{より正確には,$M$ を\hyperref[def:stratified-space]{層状化空間} $(M \to [0])$ と同一視している.}圏 $\SnglrC$ の対象である.
\end{myexample}

\exref{ex:topomfld}の意味で,\hyperref[def:Snglr-C0]{$C^0$ 級層状化空間}は位相多様体の一般化と見做せる.しかしまだそこには $C^\infty$ 構造を一般化した構造は入っておらず,$C^\infty$ 多様体の一般化とは見做せない.

\subsection{$C^0$ basic}

\begin{mydef}[label=def:C0-basic]{$C^0$ basic}
    \hyperref[def:Snglr-C0]{$\bm{C^0}$ 級層状化空間} $(X \to P) \in \Obj{\SnglrC}$ が $\bm{C^0}$\textbf{-basic}であるとは,ある $n \in \mathbb{Z}_{\ge 0}$ およびコンパクトな\hyperref[def:Snglr-C0]{$\bm{C^0}$ 級層状化空間} $(Z \to Q) \in \Obj{\SnglrC}$ が存在して
    $(X \to P) = \bigl(\mathbb{R}^n \to [0]\bigr) \times \Cone{Z \to Q}$ が成り立つことを言う.
\end{mydef}

いま,\hyperref[def:C0-basic]{$C^0$ basic}な $(U \to P_U) = (\mathbb{R}^n \to [0]) \times \Cone{Z \to P} \in \Obj{\SnglrC}$ を1つとる.
\hyperref[def:str-cone]{コーンの定義}から,$U$ の点を $(v,\, [t,\, z]) \in \mathbb{R}^n \times \frac{\mathbb{R}_{\ge 0} \times Z}{\{0\} \times Z}$ と表示することができる.
この表示の下で自己同相
\begin{align}
    \gamma \colon \mathbb{R}_{> 0} \times T \mathbb{R}^n \times \Cone{Z} &\lto \mathbb{R}_{> 0} \times T \mathbb{R}^n \times \Cone{Z}, \\
    \bigl( t,\, (v,\, p),\, [s,\, z] \bigr) &\lmto \bigl( t,\, (tv + p,\, p),\, [ts,\, z] \bigr)
\end{align}
を考える\footnote{接束 $T\mathbb{R}^n$ は $\mathbb{R}^{2n}$ と微分同相である.~\cite[p.23]{AFT2014stratified}の記法に合わせて底空間 $\mathbb{R}^n$ の点を $p$,$p$ 上のファイバーの元を $v$ としたとき $(v,\, p) \in T \mathbb{R}^n$ と書いた.命題\ref{prop:tangentbundle}の記法と順番が逆なので注意.}.

さらに,もう1つの\hyperref[def:C0-basic]{$C^0$ basic}な $(U' \to P_{U'}) = (\mathbb{R}^{n'} \to [0]) \times \Cone{Z' \to P'} \in \Obj{\SnglrC}$ および
$f \in \Hom{\SnglrC} \bigl( (U \to P_{U}),\, (U' \to P_{U'})  \bigr)$ をとる.ただし,$f$ はコーンポイントをコーンポイントへ写す,
i.e. $\forall u \in \mathbb{R}^n$ に対して $f (u,\, \mathrm{pt}) \in \mathbb{R}^{n'} \times \{\mathrm{pt}\}$ が成り立つことを仮定する.
$f|_{\mathbb{R}^n} \colon \mathbb{R}^n \times \{\mathrm{pt}\} \lto \mathbb{R}^{n'} \times \{\mathrm{pt}\}$ を $f$ のコーンポイントへの制限として,
\begin{align}
    f_{\Delta} \colon \mathbb{R}_{\textcolor{red}{>0}} \times T \mathbb{R}^n \times \Cone{Z} &\lto \mathbb{R}_{\textcolor{red}{>0}} \times T \mathbb{R}^{n'} \times \Cone{Z'}, \\
    \bigl( t,\, v,\, p,\, [s,\, z] \bigr) &\lmto \bigl( t,\, f|_{\mathbb{R}^n}(v),\, f(p,\, [ts,\, z]) \bigr)
\end{align}
とおこう.

\begin{myexample}[label=ex:cone-diff]{}
    $Z = Z' = \emptyset$ のとき,$f$ とは単に連続関数 $f \colon \mathbb{R}^n \lto \mathbb{R}^{n'}$ のことである.
    このとき,
    \begin{align}
        (\gamma^{-1} \circ f_{\Delta} \circ \gamma)(t,\, v,\, p)
        &= \gamma^{-1} \circ f_\Delta (t,\, tv+p,\, p) \\
        &= \gamma^{-1} \bigl(t,\, f(tv+p),\, f(p)\bigr) \\
        &= \left( t,\, \frac{f(tv+p) - f(p)}{t},\, f(p) \right)
    \end{align}
    と計算できるため,$f$ が $C^1$ 級であることと $\forall (v,\, p) \in T \mathbb{R}^n$ に対して $t \to +0$ の極限,i.e. $v$ に沿った片側方向微分が存在することは同値である.
\end{myexample}

\exref{ex:cone-diff}をもとに,\hyperref[def:C0-basic]{$C^0$ basic}な\hyperref[def:Snglr-C0]{$C^0$ 級層状化空間}の間の\hyperref[def:strat-emb]{層状化開埋め込み}の\textbf{conically smoothness}を定義する.
$C^\infty$ 多様体の $C^\infty$ 構造の定義においては,チャート $(U,\, \varphi \colon \mathbb{R}^n \to U),\, (V,\, \psi \colon \mathbb{R}^n \to V)$ の間の変換関数 $\psi^{-1} \circ \varphi \colon \mathbb{R}^n \lto \mathbb{R}^{n}$ が $C^\infty$ 級であることを要請した.
次の小節で\textbf{conically smooth structure}の定義を行うが,その際にチャートに対応するものは\textbf{basic} $U = \mathbb{R}^n \times \Cone{Z}$ から着目している\hyperref[def:Snglr-C0]{$C^0$-級層状化空間} $X$ への\hyperref[def:strat-emb]{層状化開埋め込み} $\varphi \colon U \hookrightarrow X$ であり,
概ね\footnote{コーンポイントをコーンポイントに写さない変換関数も存在しうるので,これだけではいけない.}2つのチャート $\varphi \colon U \hookrightarrow X,\; \psi \colon V \hookrightarrow X$ の間の変換関数 $\psi^{-1} \circ \varphi \colon U \lto V$ に対して\hyperref[def:c-smooth-along]{conically smooth (along $\mathbb{R}^n$)}であることを要請する.

\begin{mydef}[label=def:c-smooth-along]{$\mathbb{R}^n$ に沿ってconically smooth}
    \begin{itemize}
        \item \hyperref[def:C0-basic]{$C^0$ basic}な $(U \to P_{U}) = (\mathbb{R}^{n} \to [0]) \times \Cone{Z \to P} \in \Obj{\SnglrC}$
        \item \hyperref[def:C0-basic]{$C^0$ basic}な $(U' \to P_{U'}) = (\mathbb{R}^{n'} \to [0]) \times \Cone{Z' \to P'} \in \Obj{\SnglrC}$
        \item $f \in \Hom{\SnglrC} \bigl( (U \to P_{U}),\, (U' \to P_{U'})  \bigr)$ であって,コーンポイントを保存するもの
    \end{itemize}
    を与える.このとき,$f$ が\textbf{$\bm{\mathbb{R}^n}$ に沿って $C^1$ 級} ($C^1$ along $\mathbb{R}^n$) であるとは,以下の図式を可換にする連続写像
    \begin{align}
        \textcolor{red}{\tilde{D}f} \colon \mathbb{R}_{\textcolor{red}{\ge 0}} \times T \mathbb{R}^n \times \Cone{Z} &\lto \mathbb{R}_{\textcolor{red}{\ge 0}} \times T \mathbb{R}^{n'} \times \Cone{Z'}
    \end{align}
    が存在することを言う:
    \begin{center}
        % https://tikzcd.yichuanshen.de/#N4Igdg9gJgpgziAXAbVABwnAlgFyxMJZABgBpiBdUkANwEMAbAVxiRAB12BbOnACwBGA4ACUAvgH1gnHDAAeOAMYQGEAE7A1MKGOnsA5jAAExMWKMysXeEYAqF7r0HDxAPTAO81uA4DCBGGAALTEQMVJ0TFx8QhQARnIqWkYWNk4efiFRST1ZBWVVDS0dPUMTM08rG3t0pyy3YDAAcnNLbz8A4JawiJAMbDwCIjI4pPpmVkQOR0yXHIA+csr2mpnnbPdlm05-MECQnsiBmKIE0epx1Kna2eypRdMtn1WM9Ybm1vYvbfZd-e6xEltIYEChQAAzNQQLhIABM1BwECQAGZwhCoTDEPCQIikKZepDoSiEUjEHE0SBCZjsbjEMiLilJtN9HQuDxXMAALTkziKLBqRRGcFSTgAERgDBwdE+fIFDhZbLoIGoDDoAglAAUooNYiA1Fh9HwcIdKRi8SSkAlkhM0l8sAxYMBRWJwcqQAwsHs2FA6HA+NowhQxEA
        \begin{tikzcd}
        \mathbb{R}_{\textcolor{red}{\ge 0}} \times T \mathbb{R}^n \times \Cone{Z} \arrow[r, "\tilde{D}f", red, dashed]                         & \mathbb{R}_{\textcolor{red}{\ge 0}} \times T \mathbb{R}^{n'} \times \Cone{Z'} \\
        \mathbb{R}_{> 0} \times T \mathbb{R}^n \times \Cone{Z} \arrow[r] \arrow[u] \arrow[r, "\gamma^{-1}\circ f_{\Delta} \circ \gamma"'] & \mathbb{R}_{> 0} \times T \mathbb{R}^{n'} \times \Cone{Z'} \arrow[u]         
        \end{tikzcd}
    \end{center}
    \tcblower
    このような拡張が存在するとき,第一変数を $t=0$ に制限して得られる連続写像を
    \begin{align}
        \bm{Df} \colon T \mathbb{R}^n \times \Cone{Z} \lto T \mathbb{R}^{n'} \times \Cone{Z'}
    \end{align}
    と書く.$f$ が\textbf{$\bm{\mathbb{R}^n}$ に沿って $C^r$ 級} であるとは,$Df$ が $\mathbb{R}^n$ に沿って $C^{r-1}$ 級であることを言う.
    $f$ が\textbf{$\bm{\mathbb{R}^n}$ に沿ってconically smooth}であるとは,$\forall r \ge 1$ について $C^r$ 級であることを言う.
\end{mydef}

\subsection{conically smoothな層状化空間}

次に行うべきは,与えられた\hyperref[def:Snglr-C0]{$\bm{C^0}$ 級層状化空間} $(X \to P) \in \Obj{\SnglrC}$ の上の\textbf{conically smooth structure} i.e. 変換関数が\hyperref[def:c-smooth-along]{conically smooth}であるような\textbf{極大アトラス}を定義することである.
この手続きは,次で定義する次元と深さに関する帰納法によって構成される.

\begin{mydef}[label=def:covering-dim]{被覆次元}
    $X$ を位相空間とする.以下の条件を充たす最小の $d \in \mathbb{Z}_{\ge -1}$ のことを(存在すれば)$X$ の\textbf{被覆次元} (covering dimension) と呼び,$\bm{\dim X}$ と書く:
    
    \begin{description}
        \item[\textbf{(covering)}] 
        
        $X$ の任意の開被覆 $\mathscr{U}$ に対して,十分細かい細分 $\mathscr{V}_{\mathscr{U}} \prec \mathscr{U}$ が存在して,任意の互いに異なる $\forall m > d+1$ 個の開集合 $V_1,\, \dots,\, V_{m} \in \mathscr{V}_{\mathscr{U}}$ の共通部分が空になるようにできる.特に,$\emptyset$ の被覆次元は $-1$ と定義する.
    \end{description}
    
    \tcblower

    点 $x \in X$ における\textbf{被覆次元}を以下で定義する:
    \begin{align}
        \dim_x X \coloneqq \inf\, \bigl\{\, \dim U \ge -1 \bigm| x \in U \underset{\text{open}}{\subset} X \,\bigr\}
    \end{align}
\end{mydef}


\begin{mydef}[label=def:dim-depth,breakable]{次元と深さ}
    空でない\hyperref[def:Snglr-C0]{$\bm{C^0}$ 級層状化空間} $(X \to P) \in \Obj{\SnglrC}$ を与える.
    \begin{itemize}
        \item $(X \to P)$ の点 $x \in X$ における\textbf{局所的次元} (local dimension) とは,点 $x$ における $X$ の\hyperref[def:covering-dim]{被覆次元}
        % \footnote{以下の条件を充たす最小の $d \in \mathbb{Z}_{\ge -1}$ のことを(存在すれば)$X$ の\textbf{被覆次元} (covering dimension) と呼ぶ:$X$ の任意の開被覆 $\mathscr{U}$ に対して,十分細かい細分 $\mathscr{V}_{\mathscr{U}} \prec \mathscr{U}$ をとると,任意の互いに異なる $\forall m > d+1$ 個の開集合 $V_1,\, \dots,\, V_{m} \in \mathscr{V}_{\mathscr{U}}$ の共通部分が空になるようにできる.特に,$\emptyset$ の被覆次元は $-1$ と定義する.} 
        $\bm{\dim_x(X)}$ のことを言う.
        \item $(X \to P)$ の\textbf{次元} (dimension) とは
        \begin{align}
            \bm{\dim (X \to P)} \coloneqq \sup_{x \in X} \dim_x (X)
        \end{align}
        のこと.
        \item $(X \xrightarrow{s} P)$ の点 $x \in X$ における\textbf{局所的深さ} (local depth) とは,
        \begin{align}
            \bm{\depth_x(X \to P)} \coloneqq \dim_x (X) - \dim_x (X_{s(x)})
        \end{align}
        のこと.
        \item $(X \to P)$ の\textbf{深さ} (depth) とは,
        \begin{align}
            \bm{\depth(X \to P)} \coloneqq \sup_{x \in X} \depth_x(X \to P)
        \end{align}
        のこと.ただし,$\depth (\emptyset) \coloneqq -1$ と定義する.
    \end{itemize}
\end{mydef}

\begin{myexample}[label=ex:depth-Cone]{コーンの深さ}
    $n$ 次元位相多様体 $Z$ について,定義から $\forall x \in Z$ に対して $\dim_x(Z) = n$ が成り立つ.
    $Z$ を\exref{ex:topomfld}により\hyperref[def:Snglr-C0]{$C^0$ 級層状化空間} $(Z \xrightarrow{s} [0]) \in \Obj{\SnglrC}$ と見做すと,これの\hyperref[def:str-cone]{コーン} $\Cone{Z \xrightarrow{s} [0]}$ について
    \begin{align}
        \depth_x \bigl( \Cone{Z \xrightarrow{s} [0]} \bigr) 
        =
        \begin{cases}
            n+1, &x = \mathrm{pt}, \\
            0, &\text{otherwise}
        \end{cases}
    \end{align}
    であることがわかる.実際 $\Cone{Z}_{\Cone{s}(\mathrm{pt})} = \{\mathrm{pt}\}$ であるが,1点からなる位相空間の\hyperref[def:covering-dim]{被覆次元}は $0$ 次元なので $\dim_{\mathrm{pt}} (\Cone{Z}_{\Cone{s}(\mathrm{pt})}) = 0$ である.
        % ところで, $\dim_{\mathrm{pt}} \bigl( \Cone{Z} \bigr) = n+1$ である.
    一方,コーンポイント以外の点 $x \in \Cone{Z}$ に対して\hyperref[def:stratified-space]{$\Cone{s}(x)$-層}は
    $\Cone{Z}_{\Cone{s}(x)} = \mathbb{R}_{\textcolor{red}{> 0}} \times Z \approx \mathbb{R} \times Z$ であるから,
    $\dim_{x} (\Cone{Z}_{\Cone{s}(x)}) = n + 1$ と計算できる\footnote{さらに,$\forall x \in \Cone{Z}$ に対して $\dim_{x} \Cone{Z} = n+1$ である.}.
    
     また,$\forall (X \to P) \in \Obj{\SnglrC}$ に対して
    \begin{align}
        \dim \bigl( (\mathbb{R}^m \to [0]) \times (X \to P) \bigr) &= m + \dim \bigl( X \to P \bigr), \\
        \depth \bigl( (\mathbb{R}^m \to [0]) \times (X \to P) \bigr) &= \depth \bigl( X \to P \bigr) 
    \end{align}
    が成り立つ.従って,\hyperref[def:C0-basic]{$C^0$ basic}な $(U \to P_U) = (\mathbb{R}^n \to [0]) \times \Cone{Z \to P} \in \Obj{\SnglrC}$ に対して
    \begin{align}
        \depth (U \to P_U) 
        &= \depth \bigl( \Cone{Z \to P} \bigr)  \\
        &= \dim (Z \to P) + 1
    \end{align}
    が成り立つ.
\end{myexample}

次元と深さに関する帰納法を実行する前に,構成したい $(1,\, 1)$-圏を表す記号の整理をしておこう:
\begin{itemize}
    \item conically smoothチャートの素材となる,\textbf{basic}が成す圏
    \begin{align}
        \Bsc    
    \end{align}
    これは,$C^\infty$ 多様体の圏 $\Mfld$ において $\mathbb{R}^n\; (\forall n \ge -1)$ 全体が成す充満部分圏に相当するものである.
    
    \item 与えられた\hyperref[def:Snglr-C0]{$C^0$ 級層状化空間} $(X \to P) \in \Obj{\SnglrC}$ に対して,その上に入る\textbf{極大アトラス}\footnote{存在するか分からないし,存在したとして一意であるとは限らない.実際,例えば $C^\infty$ 多様体の段階においてさえ $\mathbb{R}^4$ の上の極大アトラス(i.e. $C^\infty$ 構造)は非可算無限個存在する~\cite{taubes1987gauge}.}全体が成す集合を返す\hyperref[def:presheaf]{前層}
    \begin{align}
        \Sm \colon \OP{(\SnglrC)} \lto \SETS
    \end{align}
    この対応が前層であることの直観は,\hyperref[def:strat-emb]{層状化開埋め込み} $f \in \Hom{\SnglrC} \bigl((X \to P),\, (Y \to Q)\bigr)$ が与えられると,$(X \to P)$ 上の極大アトラス $\Sm(X \to P)$ が $(Y \to Q)$ 上の極大アトラス $\Sm(Y \to Q)$ を「制限」する写像 $\Sm(f) \colon \Sm(Y \to Q) \lto \Sm(X \to P)$ によって得られるということである.
    
    \item \hyperref[def:dim-depth]{深さ}が $k$ 以下,かつ\hyperref[def:dim-depth]{次元}が $n$ 以下であるような\hyperref[def:Snglr-C0]{$C^0$ 級層状化空間}全体が成す $\SnglrC$ の充満部分圏を
    \begin{align}
        \SnglrC{}_{\underbrace{\le k}_{\text{depth}},\, \underbrace{\le n}_{\text{dimension}}}
    \end{align}
    と書く.同様に
    \begin{align}
        \Bsc_{\le k,\, \le n},\qquad \Sm_{\le k,\, \le n} \colon \OP{(\SnglrC_{\le k,\, \le n})} \lto \SETS
    \end{align}
    と書く.
    
    \item \textbf{conically smoothな層状化空間}の圏
    \begin{align}
        \Snglr
    \end{align}
    これを作ることが本小節の最終目標である.
\end{itemize}
帰納法により,$\forall k \ge -1$ に対して $\Bsc_{\le k,\, \le \infty}$ および $\Sm_{\le k,\, \le \infty} \colon \OP{(\SnglrC_{\le k,\, \le \infty})} \lto \SETS$ が構成される.

\begin{mydef}[label=def:induction-init]{帰納法の出発点}
    \hyperref[def:Snglr-C0]{\textsf{\textbf{(Snglr-1)}}}より $(\emptyset \to \emptyset) \in \Obj{\SnglrC_{\le -1,\, \le \infty}}$ である.
    \begin{enumerate}
        \item $\Bsc_{\le -1,\, \le \infty} \coloneqq \emptyset$
        \item $\Sm_{\le -1,\, \le \infty}(\emptyset) \coloneqq \{*\}$
    \end{enumerate}
    と定義する.
\end{mydef}

\begin{myassump}[label=assump:induction-Bsc-Sm]{帰納法の仮定}
    与えられた $k \ge -1$ に対して以下の構成が完了していると仮定する:
    \begin{enumerate}
        \item 圏 $\Bsc_{\le k,\, \le \infty}$
        \item 前層 $\Sm_{\le k,\, \le \infty} \colon \OP{(\SnglrC_{\le k,\, \le \infty})} \lto \SETS$
        \item 関手
        \begin{align}
            \mathbb{R} \times (\, \mhyphen \,) \colon \Bsc_{\le k,\, \le \infty} &\lto \Bsc_{\le k,\, \le \infty}, \\
            U &\lmto \mathbb{R} \times U, \\
            \bigl( U \xrightarrow{f} V \bigr) &\lmto \bigl( \mathbb{R} \times U \xrightarrow{\mathrm{id} \times f} \mathbb{R} \times V \bigr) 
        \end{align}
        およびそれが誘導する自然変換\footnote{$X$ の極大アトラス $\Familyset[\big]{U_\alpha,\, \varphi_\alpha}{\alpha \in \Lambda}$ に対して,$\Familyset[\big]{\mathbb{R} \times U_\alpha,\, \mathrm{id} \times \varphi_\alpha}{\alpha \in \Lambda}$ を対応づける.}
        % \begin{align}
        %     \OP{(\SnglrC_{\le k,\, \le \infty})} 
        % \end{align}
        \begin{center}
        \begin{tikzcd}[row sep=large, column sep=large]
            \OP{(\SnglrC_{\le k,\, \le \infty})}  \ar[bend left=50,r, "{\Sm_{\le k,\, \le  \infty}(\, \mhyphen \,)}"{name=U, above}] \ar[bend right=50,r, "{\Sm_{\le k,\, \le  \infty}(\mathbb{R} \times \, \mhyphen \,)}"{name=D, below}] &\SETS
            \ar[Rightarrow, from=U, to=D]
        \end{tikzcd}
        \end{center}
    \end{enumerate}
\end{myassump}

\begin{mydef}[label=def:Bsc-induction,breakable]{圏 $\Bsc_{\le k+1,\, \le \infty}$}
    帰納法の仮定\ref{assump:induction-Bsc-Sm}がある $k \ge -1$ において成立しているとする.
    また,\hyperref[def:C0-basic]{$C^0$ basic}を $U^n_Z \coloneqq (\mathbb{R}^n \to [0]) \times \Cone{Z \to P} \in \Obj{\SnglrC}$ と書く.
    このとき,圏 $\Bsc_{\le k+1,\, \le \infty}$ を以下で定義する:
    \begin{description}
        \item[(\textbf{対象})] 
        
        \hyperref[def:C0-basic]{$C^0$ basic}
        \footnote{\hyperref[def:dim-depth]{depthの定義}から $\depth (Z \to P) \le \dim (Z \to P)$ である.故に\exref{ex:depth-Cone}から,$\depth (Z \to P) \le \dim (Z \to P) = \depth U^n_Z - 1 \le k$ であること,i.e. $(Z \to P) \in \Obj{\SnglrC_{\le k,\, \le \infty}}$ が分かる.} $U^n_Z \in \Obj{\SnglrC_{\le k+1,\, \le \infty}}$ および,極大アトラス $\mathcal{A}_Z \in \Sm_{\textcolor{red}{\le k},\, \le \infty} (Z \to P)$ の組み
        \begin{align}
            (U^n_Z,\, \mathcal{A}_Z)
        \end{align}
        を対象とする.

        \item[(\textbf{射})] 
        
        任意の2つの対象 $(U^n_Z,\, \mathcal{A}_Z),\; (U^m_W,\, \mathcal{A}_W) \in \Obj{\Bsc_{\le k+1,\, \le \infty}}$ に対して,以下の条件を充たす\hyperref[def:strat-emb]{層状化開埋め込み} $f \in \Hom{\SnglrC_{\le k+1,\, \le \infty}} \bigl(U^n_Z,\, U^m_W \bigr)$ を射とする:
        \begin{description}
            \item[\textbf{$\bm{f}$ がコーンポイントを保存しない場合}]   
            
            ある\hyperref[def:strat-emb]{層状化開埋め込み} $f_0 \in \Hom{\SnglrC_{\le k+1,\, \le \infty}} \bigl(U^n_Z,\; \mathbb{R}^m \times \mathbb{R}_{> 0} \times W\bigr)$ が存在して
            \begin{align}
                f \colon U^n_Z \xrightarrow{f_0} \mathbb{R}^m \times (\mathbb{R}_{> 0} \times W) \hookrightarrow U^m_W = \mathbb{R}^m \times \Cone{W}
            \end{align}
            と書けて,かつ $(U^n_Z,\, f_0) \in \mathcal{A}_{\mathbb{R}^m \times \mathbb{R}_{> 0} \times W} \in \Sm(\mathbb{R}^m \times \mathbb{R}_{> 0} \times W)$
            
            \item[\textbf{$\bm{f}$ がコーンポイントを保存する場合}]  
            
            $f$ は\hyperref[def:c-smooth-along]{$\mathbb{R}^n$ に沿ってconically smooth}であって,かつ $Df \colon \mathbb{R}^n \times U^n_Z \lto \mathbb{R}^m \times U^m_W$ が単射であり,
            かつ
            \begin{align}
                \mathcal{A}_{f^{-1}(U^m_W \setminus \mathbb{R}^m)} = \Sm_{\le k,\, \le \infty} \bigl( f|_{f^{-1}(U^m_W \setminus \mathbb{R}^m)} \bigr) \bigl(\mathcal{A}_{U^m_W \setminus \mathbb{R}^m}\bigr)
            \end{align}
            を充たす\footnote{ここで帰納法の仮定\ref{assump:induction-Bsc-Sm}-(3) を暗に使っている.}.ただし,$U^m_W \setminus \mathbb{R}^m \coloneqq U^m_W \setminus (\mathbb{R}^m \times \{\mathrm{pt}\}) = \mathbb{R}^{m+1} \times W$ と略記した.
        \end{description}
        
    \end{description}
\end{mydef}

\begin{mydef}[label=def:Sm-induction,breakable]{前層 $\Sm_{\le k+1,\, \le \infty}$}
    帰納法の仮定\ref{assump:induction-Bsc-Sm}がある $k \ge -1$ において成立しているとする.さらに定義\ref{def:Bsc-induction}によって $\Bsc_{\le k+1,\, \le \infty}$ が完成しているとする.
    \begin{itemize}
        \item \hyperref[def:Snglr-C0]{$C^0$ 級層状化空間} $\forall (X \to P) \in \Obj{\SnglrC_{\le k+1,\, \le \infty}}$ に対して,$X \to P$ の\textbf{アトラス} (atlas) を族
        \begin{align}
            \mathcal{A} \coloneqq \Familyset[\Big]{\bigl(U_\alpha \in \Obj{\Bsc_{\le k+1,\, \le \infty}},\, \varphi_\alpha \colon U_\alpha \hookrightarrow (X \to P)\bigr)}{\alpha \in \Lambda} \in \Sm_{\le k+1,\, \le \infty} (X \to P)
        \end{align}
        であって以下の条件を充たすものとして定義する:
        \begin{description}
            \item[\textbf{(Atlas-1)}] 
            
            $\mathcal{A}$ は $(X \to P)$ の開被覆である.
    
            \item[\textbf{(Atlas-2)}] 
            
            $\forall \alpha,\, \beta \in \Lambda$ および $\forall x \in \varphi_\alpha (U_\alpha) \cap \varphi_\beta (U_\beta)$ に対して,
            圏 $\SnglrC_{\le k+1,\, \le \infty}$ の可換図式
            \begin{center}
                % https://tikzcd.yichuanshen.de/#N4Igdg9gJgpgziAXAbVABwnAlgFyxMJZABgBpiBdUkANwEMAbAVxiRAHUQBfU9TXfIRRkAjFVqMWbAKoB9YAB0FjNAAs6XbrxAZseAkRGkx1es1aIQADS189gw+XFmpluYoUAjGDg3dxMFAA5vBEoABmAE4QALZIZCA4EEgAzKaSFiDh8krevpo8EdFxiAlJSEYS5mzZHirqmtQMdN4MAAr8+kIgkVhBqji2WcUV1OWIAExNLTDtnQ6Wvf2D6dWWSvSRalg5ygxqfoXDsaljyZPTrR32BpYMMOErVa4gG3RbqjseeYcUXEA
                \begin{tikzcd}
                \exists \textcolor{red}{W} \arrow[r, hookrightarrow, red,"f_{\beta}"] \arrow[d, hookrightarrow, red,"f_{\alpha}"'] & U_{\beta} \arrow[d, hookrightarrow, "\varphi_{\beta}"] \\
                U_{\alpha} \arrow[r, hookrightarrow, "\varphi_{\alpha}"']         & X                                     
                \end{tikzcd}
            \end{center}
            が存在して $x \in \varphi_\alpha \circ \textcolor{red}{f_\alpha} (\textcolor{red}{W}) = \varphi_\beta \circ \textcolor{red}{f_\beta} (\textcolor{red}{W})$ を充たす.
            ただし,可換図式中の赤色の部分は全て\hyperref[def:Bsc-induction]{圏 $\Bsc_{\le k+1,\, \le \infty}$} の対象および射からなる.

        \end{description}
        アトラス $\mathcal{A}$ の元 $(U_\alpha,\, \varphi_\alpha) \in \mathcal{A}$ のことを\textbf{チャート} (chart) と呼ぶ.

        \item \hyperref[def:Snglr-C0]{$C^0$ 級層状化空間} $\forall (X \to P) \in \Obj{\SnglrC_{\le k+1,\, \le \infty}}$ の2つのアトラス $\mathcal{A},\, \mathcal{B}$ が\textbf{同値}であるとは,$\mathcal{A} \cup \mathcal{B}$ が $(X \to P)$ のアトラスであることを言う.
        これは $(X \to P)$ のアトラス全体の集合の上に同値関係を定める\footnote{同値関係であることの証明は~\cite[Lemma 3.2.11.]{AFT2014stratified}を参照.}.
        $(X \to P)$ の\textbf{極大アトラス} (maximal atlas) とは,この同値関係によるアトラス $\mathcal{A}$ の同値類 $[\mathcal{A}]$ のことを言う.

        \item 前層
        \begin{align}
            \Sm_{\le k+1,\, \le \infty} \colon \OP{(\SnglrC_{\le k+1,\, \le \infty})} \lto \SETS
        \end{align}
        を以下のように定義する:
        \begin{description}
            \item[\textbf{(対象)}] 
            
            任意の\hyperref[def:Snglr-C0]{$C^0$ 級層状化空間} $(X \to P) \in \Obj{\SnglrC_{\le k+1,\, \le \infty}}$ に対して
            \begin{align}
                \Sm_{\le k+1,\, \le \infty} (X \to P) \coloneqq \bigl\{\, [\mathcal{A}] \bigm| \mathcal{A}\; \text{is an atlas of}\; (X \to P) \,\bigr\} 
            \end{align}
            
            \item[\textbf{(射)}] 
            
            任意の\hyperref[def:strat-emb]{層状化開埋め込み} $f \in \Hom{\SnglrC_{\le k+1,\, \le \infty}}\bigl((X \to P),\, (Y \to Q)\bigr)$ に対して,$f$ によるアトラスの引き戻しを対応付ける.
        \end{description}
    \end{itemize}
\end{mydef}

以上の帰納法をまとめて,conically smoothな層状化空間と\hyperref[def:strat-emb]{層状化開埋め込み}の圏 $\Snglr$ を得る.

\begin{mydef}[label=def:Snglr]{圏 $\Snglr$}
    \begin{itemize}
        \item \textbf{basic}のなす圏 $\Bsc$ を以下で定義する:
        \begin{align}
            \Bsc \coloneqq \bigcup_{k \ge -1} \Bsc_{\le k,\, \le \infty}
        \end{align}
        \item 極大アトラスの集合を与える関手 $\Sm \colon \OP{(\SnglrC)} \lto \SETS$ を以下の\hyperref[def:Kanext]{右Kan拡張}として定義する:
        \begin{center}
            % https://tikzcd.yichuanshen.de/#N4Igdg9gJgpgziAXAbVABwnAlgFyxMJZABgBpiBdUkANwEMAbAVxiRAB12B5ABWAApOAZTABzBgCcAwgH1gAHgAEnLGABmOAJ6lOpZewYx9qjZoC+ASjMgzpdJlz5CKMgEYqtRizadeA4WKSUlY2diAY2HgERK7kHvTMrIgc7EIAogAqQjYeMFCi8ESgahIQALZIZCA4EEgATNQJ3snCZXJKKupaOux6nIbGXeahxaUViFU1SK62o+XT1FOIDSAMqkkgUHRwABZ5II1eG60Hq3QARjAMPA5RziASWKI7ODlmQA
            \begin{tikzcd}[column sep=large]
            {\OP{(\SnglrC_{< \infty,\, \le \infty})}} \arrow[r, "{\Sm_{< \infty,\, \le \infty}}"] \arrow[d,hookrightarrow] & \SETS \\
            \OP{(\SnglrC)} \arrow[ru,red, "\Sm"', dashed]                                                       &      
            \end{tikzcd}
        \end{center}
        ただし,$\SnglrC_{< \infty,\, \le \infty} \coloneqq \bigcup_{k \ge -1} \SnglrC_{\le k,\, \le \infty}$ とおいた.
        \item \textbf{conically smoothな層状化空間} (conically smooth stratified space) と\hyperref[def:strat-emb]{層状化開埋め込み}の圏 $\Snglr$ を以下で定義する:
        
        \begin{description}
            \item[\textbf{(対象)}] 
            
            \hyperref[def:Snglr-C0]{$C^0$ 級層状化空間} $(X \to P) \in \Obj{\SnglrC}$ およびその\hyperref[def:Sm-induction]{極大アトラス} $\mathcal{A_X} \in \Sm (X \to P)$ の組み $\bigl((X\to P),\, \mathcal{A}_X\bigr)$ を対象とする.
        
            \item[\textbf{(射)}] 
            
            \hyperref[def:strat-emb]{層状化開埋め込み} $f \in \Hom{\SnglrC} \bigl((X \to P),\, (Y \to Q)\bigr)$ であって,$f^* \mathcal{A}_Y = \mathcal{A}_X$ を充たすものを射とする.
        \end{description}
    \end{itemize}
    
\end{mydef}

\subsection{conically smooth map}

ここまでは\hyperref[def:strat-emb]{層状化開埋め込み}のみを考えていたため,一般の\hyperref[def:stratified-space]{層状化写像}のconically smoothnessを定義しなくてはいけない.

\begin{mydef}[label=def:c-smooth-map]{conically smooth map}
    2つの\hyperref[def:Snglr]{\underline{\textbf{basic}}}\footnote{\hyperref[def:C0-basic]{$C^0$ basic}を $U^n_Z \coloneqq (\mathbb{R}^n \to [0]) \times \Cone{Z \to P} \in \Obj{\SnglrC}$ と書く.} $X = (U^n_Z,\, \mathcal{A}_Z),\, Y = (U^m_W,\, \mathcal{A}_W)\in \Obj{\Bsc}$ の間の\hyperref[def:stratified-space]{層状化写像} $f \colon U^n_Z \lto U^m_W$ が\textbf{conically smooth}であることを,
    \hyperref[def:dim-depth]{$\depth (Y)$} に関する帰納法によって定義する:
    \begin{enumerate}
        \item まず,$\depth (Y) = -1$ のときは $X = Y = \emptyset$ であり,一意的に定まる $X,\, Y$ 間の層状化写像がconically smoothであると定義する.
        \item 深さ $k \ge -1$ のbasicに対して定義が完了しているとする.
        $Y \in \Obj{\Bsc}$ の深さが高々 $k+1$ であるとき,層状化写像 $f \colon X \lto Y$ がconically smoothであることを以下で定義する:
        \begin{description}
            \item[\textbf{$\bm{f}$ がコーンポイントを保存しない場合}]   
            
            あるconically smoothな層状化写像 $f_0 \colon X \lto \mathbb{R}^m \times \mathbb{R}_{> 0} \times W$ が存在して
            \begin{align}
                f \colon X \xrightarrow{f_0} \mathbb{R}^m \times (\mathbb{R}_{> 0} \times W) \hookrightarrow Y = \mathbb{R}^m \times \Cone{W}
            \end{align}
            と書ける\footnote{\exref{ex:depth-Cone}より$\depth(W) < k+1$であり,帰納法の仮定が使える.}.
            
            \item[\textbf{$\bm{f}$ がコーンポイントを保存する場合}]  
            
            $f$ は\hyperref[def:c-smooth-along]{$\mathbb{R}^n$ に沿ってconically smooth}であって,かつ制限
            \begin{align}
                f|_{f^{-1}(Y \setminus \mathbb{R}^m)} \colon f^{-1}(Y \setminus \mathbb{R}^m) \lto Y \setminus \mathbb{R}^m
            \end{align}
            がconically smooth.ただし,$U^m_W \setminus \mathbb{R}^m \coloneqq U^m_W \setminus (\mathbb{R}^m \times \{\mathrm{pt}\}) = \mathbb{R}^{m+1} \times W$ と略記した.
        \end{description}
    \end{enumerate}
    
    \tcblower
    % \hyperref[def:C0-basic]{$C^0$ basic}を $U^n_Z \coloneqq (\mathbb{R}^n \to [0]) \times \Cone{Z \to P} \in \Obj{\SnglrC}$ と書く.
    \hyperref[def:Snglr]{conically smoothな層状化空間} $\bigl( (X \to P),\, \mathcal{A}_X \bigr),\, \bigl( (Y \to Q),\, \mathcal{A}_Y \bigr) \in \Obj{\Snglr}$ の間の\hyperref[def:stratified-space]{層状化写像} $f \colon (X \to P) \lto (Y \to Q)$ が\textbf{conically smooth}であるとは,
    任意のチャートの組み合わせ $(U,\, \varphi) \in \mathcal{A}_X,\, (V,\, \psi) \in \mathcal{A}_Y$ に対して
    \begin{align}
        \psi^{-1} \circ f \circ \varphi \colon U \lto V
    \end{align}
    がconically smooth (for \hyperref[def:Snglr]{basic}s) であることを言う.
\end{mydef}

\begin{myprop}[label=prop:c-smooth-map]{conically smooth mapの基本性質}
    2つの\hyperref[def:c-smooth-map]{conically smooth map}の合成もconically smoothである.
\end{myprop}

\begin{proof}
    ~\cite[Proposition 3.3.5]{AFT2014stratified}
\end{proof}

命題\ref{prop:c-smooth-map}より,\textbf{conically smoothな層状化空間}の圏を定義できる.

\begin{mydef}[label=def:c-smooth]{conically smoothな層状化空間の圏 $\Strat$ }
    \textbf{conically smoothな層状化空間}の圏 $\Strat$ を以下で定義する:
    \begin{description}
        \item[\textbf{(対象)}] 
        
        圏 \hyperref[def:Snglr]{$\Snglr$}と全く同じ対象を持つ:
        \begin{align}
            \Obj{\Strat} \coloneqq \Obj{\Snglr}
        \end{align}
        \item[\textbf{(射)}] 
        
        \hyperref[def:c-smooth-map]{conically smooth map}を射とする.
    \end{description}
\end{mydef}

定義から明らかに $\Snglr \subset \Strat$ である.
ここで,圏 $\Strat$ における特別な射に名前をつけておこう:

\begin{mydef}[label=def:cbl,breakable]{constructuble bundle}
    \begin{itemize}
        \item \hyperref[def:c-smooth-map]{conically smoothな層状化写像} $\pi \in \Hom{\Strat}\bigl( (E \to P),\, (B \to Q) \bigr)$ が\textbf{層状化ファイバー束} (conically smooth fiber bundle) であるとは,conically smoothな\hyperref[def:strat-emb]{層状化開埋め込み}の族 $\Familyset[\big]{U_\alpha \hookrightarrow B}{\alpha \in \Lambda},\; \Familyset[\big]{\varphi_\alpha \colon U_\alpha \times F_\alpha \hookrightarrow E}{\alpha \in \Lambda}$ が存在して以下を充たすことを言う:
        \begin{description}
            \item[\textbf{(Bun-1)}] 
            
            $\forall \alpha \in \Lambda$ に対して,圏 $\Strat$ における\hyperref[def:pullback-pushout]{引き戻し}の図式
            
            \begin{center}
                % https://tikzcd.yichuanshen.de/#N4Igdg9gJgpgziAXAbVABwnAlgFyxMJZABgBpiBdUkANwEMAbAVxiRAFUB9YAHR8bQALOgF8ABHzwBbeGIBi3PgOEiQI0uky58hFGQCMVWoxZsuvfgyGi1GkBmx4CRfaUPV6zVohAAhW5qOOi7kRp6mPgCiakYwUADm8ESgAGYAThBSSGQgOBBIAMzUOHRYDGyCEBAA1iAeJt4gfPRpQliKltaq6qkZWYiuufmIAEzFpeU+lTUBIOmZ2cXDg+GNfFJ0OIJpUsBoGQBWIpz6dSAMdABGMAwAClpOuiBpWPGCOLPz-UVDSGPGXjYfDQWBiIiAA
                \begin{tikzcd}
                U_{\alpha} \times F_{\alpha} \arrow[r, "\varphi_{\alpha}", hook] \arrow[d, "\mathrm{proj}_1"'] & E \arrow[d, "\pi"] \\
                U_{\alpha} \arrow[r, hook]                                                                     & B                 
                \end{tikzcd}
            \end{center}
            が成り立つ.

            \item[\textbf{(Bun-2)}] 
            
            族 $\Familyset[\big]{U_\alpha}{\alpha \in \Lambda}$ は $B$ の開基である.
        \end{description}
        
        \item \hyperref[def:c-smooth-map]{conically smoothな層状化写像} $\pi \in \Hom{\Strat}\bigl( (E \to P),\, (B \to Q) \bigr)$ が\textbf{弱構成可能束} (weakly constructuble bundle) であるとは,$\forall q \in Q$ に対して,$\pi$ の\hyperref[def:stratified-space]{$q$-層}への制限
        \begin{align}
            \pi|_{\pi^{-1}(B_q)} \colon \pi^{-1}(B_q) \lto B_q
        \end{align}
        が層状化ファイバー束であることを言う.

        \item  \hyperref[def:c-smooth-map]{conically smoothな層状化写像} $\pi \in \Hom{\Strat}\bigl( (E \to P),\, (B \to Q) \bigr)$ が\textbf{構成可能束} (constructuble bundle) であることを,\hyperref[def:dim-depth]{$\depth (E)$} に関する帰納法によって定義する:
        \begin{enumerate}
            \item $\depth (E) = 0$ のとき,$\pi$ が構成可能束であるとは,$\pi$ が $C^\infty$ ファイバー束であることを言う.
            \item 深さ $k \ge 0$ までの定義が完了しているとする.$\depth(E) \le k+1$ のとき,$\pi$ が構成可能束であるとは,以下の2条件を充たすことを言う:
            \begin{description}
                \item[\textbf{(cBun-1)}] $\pi$ は弱構成可能束である.
                \item[\textbf{(cBun-2)}] $\forall q \in Q$ に対して,$\pi$ が誘導する層状化写像
                \begin{align}
                    \Link_{\pi^{-1}(B_q)} (E) \lto \pi^{-1}(B_q) \times_{B_q} \Link_{B_q} (B)
                \end{align}
                が構成可能束である.
            \end{description}
        \end{enumerate}
    \end{itemize}
\end{mydef}


\subsection{管状近傍・ハンドル分解}

\section{層状化空間の接構造}

\subsection{$\mathrm{Kan}$-豊穣化}

圏 $\Kan$ を,
\begin{itemize}
    \item \hyperref[def:KanCplx]{Kan複体}を対象に持つ
    \item Kan複体の間の自然変換を射に持つ
\end{itemize}
$(1,\, 1)$-圏とする.$\Kan$ は\hyperref[def:SimpSet]{単体的集合の圏 $\sSet$}の充満部分圏であり,直積\eqref{eq:sSet-tensor}をテンソル積とするモノイダル圏になる.

\begin{mydef}[label=def:standard-cosimplicial-mfd]{余単体的多様体}
    以下で定義する関手
    \begin{align}
        \bm{\Delta_e} \colon \Delta \lto \Strat
    \end{align}
    のことを\textbf{余単体的多様体} (standard cosimplicial manifold) と呼ぶ\footnote{\hyperref[def:simplicial-top]{幾何学的 $n$-単体}に似ているが,$x^i \ge 0$ の領域で切り取っていない.}:
    \begin{itemize}
        \item $[n] \in \Obj{\Delta}$ を,\hyperref[def:c-smooth]{conically smoothな層状化空間}
        \begin{align}
            \Delta_e^n \coloneqq \bigl\{\, (x^0,\, \dots,\, x^n) \lto \mathbb{R}^{n+1} \bigm| \sum_{i=0}^n x^i = 1 \,\bigr\} 
        \end{align}
        に対応付ける.
        \item $\alpha \in \Hom{\Delta}([m],\, [n])$ を,\hyperref[def:c-smooth-map]{conically smoothな層状化写像}
        \begin{align}
            \Delta_e(\alpha) \colon \Delta_e^m &\lto \Delta_e^n, \\
            (x^0,\, \dots,\, x^m) &\lmto \Bigl( \sum_{j,\, \alpha(j) = 0} x^j,\, \dots,\, \sum_{j,\, \alpha(j) = n} x^j \Bigr)
        \end{align}
        に対応付ける.
    \end{itemize}
\end{mydef}

$\PshSETS{\OP{\Strat}}$ から $\sSet$ への関手を
\begin{align}
    (\, \mhyphen \,)|_{\Delta} \colon \PshSETS{\Strat} &\lto \sSet, \\
    F &\lmto F \circ \Delta_e
\end{align}
で定義する.さらに,$\forall X,\, Y \in \Obj{\Strat}$ に対して前層 $\widetilde{\Hom{\STRAT}} (X,\, Y) \in \PshSETS{\Strat}$ を
\begin{align}
    \widetilde{\Hom{\STRAT}} (X,\, Y) \colon \OP{\Strat} &\lto \SETS, \\
    \textcolor{red}{Z} &\lmto \bigl\{\, f \in \Hom{\Strat}(\textcolor{red}{Z} \times X,\, \textcolor{red}{Z} \times Y) \bigm| \mathrm{proj}_Z \circ f = \mathrm{proj}_Z \,\bigr\}, \\
    (Z \xrightarrow{\alpha} W) &\lmto \Biggl(\substack{\widetilde{\Hom{\STRAT}(X,\, Y)}(Z) \lto \widetilde{\Hom{\STRAT}(X,\, Y)}(W), \\ f \lmto \biggl((w,\, x) \mapsto \bigl(w,\, \mathrm{proj}_Y \circ f (\alpha(w),\, x)\bigr)\biggr)}\Biggr)
\end{align}
で定義する.ただし,\hyperref[def:c-smooth-map]{conically smoothな層状化写像} $\mathrm{proj}_Z \in \Hom{\Strat}(Z \times X,\, Z)$ とは第一成分への射影のことである.
同様にして前層 $\widetilde{\Hom{\SNGLR}} (X,\, Y) \in \PshSETS{\Strat}$ を
\begin{align}
    \widetilde{\Hom{\SNGLR}} (X,\, Y) \colon \OP{\Strat} &\lto \SETS, \\
    \textcolor{red}{Z} &\lmto \bigl\{\, f \in \Hom{\Snglr}(\textcolor{red}{Z} \times X,\, \textcolor{red}{Z} \times Y) \bigm| \mathrm{proj}_Z \circ f = \mathrm{proj}_Z \,\bigr\}, \\
    (Z \xrightarrow{\alpha} W) &\lmto \Biggl(\substack{\widetilde{\Hom{\SNGLR}(X,\, Y)}(Z) \lto \widetilde{\Hom{\SNGLR}(X,\, Y)}(W), \\ f \lmto \biggl((w,\, x) \mapsto \bigl(w,\, \mathrm{proj}_Y \circ f (\alpha(w),\, x)\bigr)\biggr)}\Biggr)
\end{align}
で定義する.

\begin{mylem}[label=lem:Kan-enriched]{}
    $\forall X,\, Y \in \Obj{\Strat}$ に対して定まる\hyperref[def:SimpSet]{単体的集合}
    \begin{align}
        \Hom{\STRAT}(X,\, Y) &\coloneqq \eval{\widetilde{\Hom{\STRAT}}(X,\, Y)}_{\Delta}, \\
        \Hom{\SNGLR}(X,\, Y) &\coloneqq \eval{\widetilde{\Hom{\SNGLR}}(X,\, Y)}_{\Delta}
    \end{align}
    は\hyperref[def:KanCplx]{Kan複体}である.
\end{mylem}

\begin{proof}
    ~\cite[Lemma 4.1.4.]{AFT2014stratified}.
\end{proof}

\begin{mydef}[label=def:Strat-infty]{{$(\infty,\, 1)$}-圏 {$\STRAT,\, \SNGLR,\, \BSC$}}
    \hyperref[def:enriched]{$\Kan$-豊穣圏} $\STRAT$ を以下で定義する:
    \begin{itemize}
        \item $\Obj{\STRAT} \coloneqq \Obj{\Snglr}$
        \item 補題\ref{lem:Kan-enriched}で構成した $\Hom{\STRAT}(X,\, Y) \in \Obj{\Kan}$ をHom対象とする.
    \end{itemize}
    同様に,$\Kan$-豊穣圏 $\SNGLR$ を以下で定義する:
    \begin{itemize}
        \item $\Obj{\SNGLR} \coloneqq \Obj{\Snglr}$
        \item 補題\ref{lem:Kan-enriched}で構成した $\Hom{\SNGLR}(X,\, Y) \in \Obj{\Kan}$ をHom対象とする.
    \end{itemize}
    
    \tcblower

    $\Kan$-豊穣圏 $\SNGLR$ の対象を \hyperref[def:Snglr]{$\Obj{\Bsc}$} に制限して得られる充満部分圏を $\BSC$ と書く.
    % \tcblower

    % \hyperref[def:infinity-1]{$(\infty,\, 1)$-圏} $\STRAT,\, \SNGLR$ を以下で定義する:
    % \begin{align}
    %     \STRAT &\coloneqq \hcNer  (\STRAT), \\
    %     \SNGLR &\coloneqq \hcNer  (\SNGLR)
    % \end{align}
    
\end{mydef}

\begin{marker}
    % homotopy hypothesisより $\Kan$-豊穣圏は $(\infty,\, 1)$-圏のモデルと見做せる.故に
    $\Kan$-豊穣圏を\hyperref[def:nerve-hc]{homotopy coherent nerve functor} $\hcNer  \colon \sCat \lto \sSet$ で単体的集合の圏 $\sSet$ へ埋め込んだものは $(\infty,\, 1)$-圏である~\cite[Proposition 1.1.5.10.]{lurie2008higher}.
    故に以下では $\Kan$-豊穣圏 $\STRAT,\, \SNGLR$ と $(\infty,\, 1)$-圏 $\hcNer  (\STRAT),\, \hcNer  (\SNGLR)$ を区別しない.
\end{marker}

\subsection{$(\infty,\, 1)$-圏におけるファイブレーション}

\begin{mydef}[label=def:infty-fib]{{$(\infty,\, 1)$}-ファイブレーション}
    $p \colon \Cat{E} \lto \Cat{B}$ を\hyperref[def:infinity-1]{$(\infty,\, 1)$-圏の関手}とする.
    \begin{description}
        \item[\textbf{(lifting property)}] 
        
        包含 $\iota \in \Hom{\sSet}(\Lambda^n_j,\, \Delta^n)$ に対して $p \circ \textcolor{blue}{f_0} = \textcolor{blue}{f} \circ \iota$ を充たす任意の $(\textcolor{blue}{f_0},\, \textcolor{blue}{f}) \in \Hom{\sSet}(\Lambda^n_j,\, \Cat{E}) \times \Hom{\sSet}(\Delta^n_j,\, \Cat{B})$ に対して,以下の図式を可換にする $\textcolor{red}{\bar{f}} \in \Hom{\sSet}(\Delta^n,\, \Cat{E})$ が存在する:
        \begin{center}
            % https://tikzcd.yichuanshen.de/#N4Igdg9gJgpgziAXAbVABwnAlgFyxMJZABgBpiBdUkANwEMAbAVxiRAB12AZOgWwCModAHpoA+gCsQAX1LpMufIRRkAjFVqMWbTgBEYDHCLQy5IDNjwEiq0uur1mrRB3YBhOjmAAhaafmWSjbkGo7aLpweXgCiftIaMFAA5vBEoABmAE4QvEhkIDgQSADMDlrOIEn+IFk5SLYFRYgATGVObOkg1Ax0-AYACgpWyiCZWEkAFjjVtbmIpY1IrZrtLiayGdlz+YX11EZYDGwTEBAA1l0r4a74Rpc9fQyDgdYuY5PTGzVbe4vz3VgwBUhHAJolLmEKpx+HRMsB0nEKNIgA
            \begin{tikzcd}[column sep=large,row sep=large]
            \Lambda^n_j \arrow[r, blue, "\forall f_0"] \arrow[d, "\iota"', hook]   & \Cat{E} \arrow[d, "p"] \\
            \Delta^n \arrow[r, blue, "\forall f"'] \arrow[ru, red, "\exists \bar{f}", dashed] & \Cat{B}               
            \end{tikzcd}
        \end{center}
    \end{description}
    \tcblower
    \begin{itemize}
        \item $p$ が\textbf{内的ファイブレーション} (inner fibration) であるとは,$0 \,\textcolor{red}{<}\, \forall j \textcolor{red}{<} \forall n$ に対して \textsf{\textbf{(lifting property)}} を充たすことを言う.
        \item $p$ が\textbf{右ファイブレーション} (right fibration) であるとは,$0 \,\textcolor{red}{<}\, \forall j \le \forall n$ に対して \textsf{\textbf{(lifting property)}} を充たすことを言う.
        \item $p$ が\textbf{左ファイブレーション} (left fibration) であるとは,$0 \le \forall j \,\textcolor{red}{<}\, \forall n$ に対して \textsf{\textbf{(lifting property)}} を充たすことを言う.
        \item $p$ が\textbf{Kanファイブレーション} (Kan fibration) であるとは,$0 \le \forall j \le \forall n$ に対して \textsf{\textbf{(lifting property)}} を充たすことを言う.
    \end{itemize}
\end{mydef}

系\ref{col:horn-coeq}によると,\hyperref[def:infty-fib]{\textsf{\textbf{(lifting property)}}}は,\underline{$(\infty,\, 1)$-圏 $\Cat{B}$ における}\hyperref[def:horn]{角}の図式
$\bigl(p_{[n-1]}(\textcolor{blue}{f_0}{}_0),\, \dots,\, \underbrace{\bullet}_{j},\, \dots,\, p_{[n-1]}(\textcolor{blue}{f_0}{}_n)\bigr) \in (\Cat{B}_{n-1})^{\times n}$ を $n$-射 $\textcolor{blue}{f} \in \Cat{B}_n$ が埋めているならば,
\underline{$(\infty,\, 1)$-圏 $\Cat{E}$ における}\hyperref[def:horn]{角}の図式
$\bigl(\textcolor{blue}{f_{0}}{}_0,\, \dots,\, \underbrace{\bullet}_{j},\, \dots,\, \textcolor{blue}{f_0}{}_n\bigr) \in (\Cat{E}_{n-1})^{\times n}$ を埋める $n$-射 $\textcolor{red}{\bar{f}} \in \Cat{E}_n$ が存在することを主張している.

\begin{mydef}[label=def:fullsub-infty]{充満部分{$(\infty,\, 1)$-圏}}
    \begin{itemize}
        \item \hyperref[def:infinity-1]{$(\infty,\, 1)$-圏} $\Cat{C}$ の\textbf{部分 $\bm{(\infty,\, 1)}$-圏} (sub $(\infty,\, 1)$-category) とは,
        \hyperref[def:SimpSet]{単体的部分集合} $\Cat{S} \subset \Cat{C}$ であって,その\hyperref[def:SimpSet]{包含写像} $i \colon \Cat{S} \hookrightarrow \Cat{C}$ が\hyperref[def:infy-fib]{内的ファイブレーション}であるようなもののこと\footnote{このとき $\Cat{S}$ は $(\infty,\, 1)$-圏になる~\cite[\href{https://kerodon.net/tag/01CG}{Tag 01CG}]{kerodon}}.
        \item 部分 $(\infty,\, 1)$-圏 $\Cat{S} \subset \Cat{C}$ が\textbf{充満部分 $\bm{(\infty,\, 1)}$-圏} (full sub $(\infty,\, 1)$-category) であるとは,
        $\forall n \ge 0$ に対して以下の条件を充たすことを言う:
        \begin{description}
            \item[\textbf{(fullsub)}] 
            $\forall \sigma \in \Cat{C}_n \cong \Hom{\sSet} (\Delta^n,\, \Cat{C})$ に対して,$\sigma_{[0]}(\Delta^n_0) \subset \Cat{S}_0 \IMP \sigma \in \Cat{S}_n$ が成り立つ.
        \end{description}
    \end{itemize}
\end{mydef}

2つの\hyperref[def:infty-fib]{右ファイブレーション} $\mathcal{E} \xrightarrow{p} \mathcal{B},\; \mathcal{E}' \xrightarrow{p'} \mathcal{B}$ が与えられたとき,これらの間の射とは集合
\begin{align}
    \Hom{\Rfib_{\Cat{B}}} (\Cat{E},\, \Cat{E}') &\coloneqq 
    \bigl\{\, f \in \Hom{\sSet}(\mathcal{E},\, \mathcal{E}') \bigm| p' \circ f = p \,\bigr\}
\end{align}
のことである.$\Hom{\Rfib_{\Cat{B}}} (\Cat{E},\, \Cat{E}')$ の元を $\sSet$ における可換図式として表すと以下の通り:
\begin{center}
    \begin{tikzcd}
        \Cat{E} \arrow[rd, "p"'] \arrow[rr, "f"] &         & \Cat{E}' \arrow[ld, "p'"] \\
                                            & \Cat{B} &                          
    \end{tikzcd}
\end{center}
$\Hom{\Rfib_{\Cat{B}}} (\Cat{E},\, \Cat{E}') \subset \Hom{\sSet}(\mathcal{E},\, \mathcal{E}')$ を\exref{def:Kan}の方法で\hyperref[def:SimpSet]{単体的集合}と見做せる.
このようにして得られる単体的集合 $\Hom{\Rfib_{\Cat{B}}} (\Cat{E},\, \Cat{E}')$ の最大の\hyperref[def:fullsub-infty]{部分Kan複体}を $\bm{\Hom{\RFIB_{\Cat{B}}}(\Cat{E},\, \Cat{E}')}$ と書く.

\begin{mydef}[label=def:RFIB]{右ファイブレーションの成す {$(\infty,\, 1)$}-圏}
    $\Cat{B}$ を $(\infty,\, 1)$-圏とする.
    $\Kan$-豊穣圏 $\RFIB_{\Cat{B}}$ を
    \begin{itemize}
        \item \hyperref[def:infty-fib]{右ファイブレーション}を対象とする
        \item $\Hom{\RFIB_{\Cat{B}}}(\Cat{E},\, \Cat{E}')$ をHom対象とする
    \end{itemize}
    ことで定義する.
    以降では $(\infty,\, 1)$-圏 $\hcNer (\RFIB_{\Cat{B}}) \in \Obj{\sSet}$ のことも $\RFIB_{\Cat{B}}$ と書き,区別しない.
\end{mydef}

\subsection{{$(\infty,\, 1)$}-圏におけるスライス圏}

\begin{mydef}[label=def:Simp-Join]{単体的集合のjoin}
    2つの単体的集合 $S,\, T \in \Obj{\sSet}$ の\textbf{join}とは,単体的集合
    \begin{align}
        \bm{S \star T} \colon \OP{\Delta} &\lto \SETS, \\
        [n] &\lmto \coprod_{[i];\, -1 \le i \le n} (S_i \times T_{n-i-1}), \\
        \biggl( [m] \xrightarrow{\alpha} [n] \biggr) &\lmto \biggl( \Bigl( [i];\, (x,\, y) \Bigr) \mapsto \Bigl( \alpha^{-1}([i]);\, \bigl(S(\alpha|_{\alpha^{-1}([i])})(x),\, T(\alpha|_{\alpha^{-1}([m]\setminus[i])})(y) \bigr)  \Bigr)  \biggr) 
    \end{align}
    のこと.ただし $S_{-1} = T_{-1} \coloneqq \{*\},\; [-1] \coloneqq \emptyset$ とおいた.
\end{mydef}
$d^n_j \in \Hom{\OP{\Delta}}([n],\, [n-1])$ に対して
\begin{align}
    (d^n_j)^{-1} ([i]) &= 
    \begin{cases}
        [i], &-1 \le i < j \\
        [i-1], &j \le i \le n
    \end{cases} 
    \\
    (d^n_j)^{-1} ([n]\setminus [i]) &= \begin{cases}
        [n-1]\setminus [i], &-1 \le i < j \\
        [n-1]\setminus [i-1], &j \le i \le n
    \end{cases}
\end{align}
であるから,
$S \star T$ の\hyperref[def:SimpSet]{面写像}は $n \ge 1,\, 0 \le j \le n$ に対して
\begin{align}
    \label{eq:face-join}
    \partial^n_j \colon \coprod_{-1 \le i \le n} (S_i \times T_{n-i-1}) &\lto \coprod_{-1 \le i \le n-1} (S_i \times T_{n-i-2}), \\
    \bigl([i];\, (x,\, y) \bigr) &\lmto 
    \begin{cases}
        % \bigl(-1;\,y \bigr), &i=0 \\
        \bigl([-1];\, (*,\, \partial^n_j y) \bigr), &i=-1 \\
        \bigl([i];\, (x,\, \partial^{n-i-1}_{j-i-1} y) \bigr), &0 \le i < j,\; (i,\, j) \neq (n-1,\, n) \\
        \bigl([i-1];\, (\partial^i_j x,\, y) \bigr), &j \le i \le n-1,\; (i,\, j) \neq (0,\, 0) \\
        \bigl([n-1];\, (\partial^n_j x,\, *) \bigr), &i=n \\
        \bigl([n-1];\, (x,\, *) \bigr), &(i,\, j) = (n-1,\, n) \\
        \bigl([-1];\, (*,\, y) \bigr), &(i,\, j) = (0,\, 0)
        % \bigl(n;\, x\bigr), &j > i=0
        % \bigl(-1;\,\partial^n_j y \bigr), &i=-1
    \end{cases}
\end{align}
% であり,\hyperref[def:SimpSet]{縮退写像}は
% \begin{align}
%     \label{eq:degen-join}
%     \sigma^n_j \colon \coprod_{-1 \le i \le n} (S_i \times T_{n-i-1}) &\lto \coprod_{-1 \le i \le n+1} (S_i \times T_{n-i}), \\
%     \bigl( i;\, (x,\, y) \bigr) &\lmto 
%     \begin{cases}
%         % \bigl(n;\,\sigma^n_j x \bigr), &i=n \\
%         \bigl(i;\, (\sigma_j^i x,\, y) \bigr), &j < i,\; i\neq 0 \\
%         \bigl(i+1;\, (\sigma_{j}^i x,\, y) \bigr), &j = i \\
%         \bigl(i+1;\, (x,\, \sigma^{n-i-1}_{j-i-1} y) \bigr), &j > i,\; n-i-1\neq 0 \\
%         % \bigl(-1;\,\sigma^n_j y \bigr), &i=-1
%     \end{cases}
% \end{align}
となる.

\begin{myexample}[label=ex:join-0]{join {$\Delta^0 \star \Delta^0$}}
    $\textcolor{red}{\Delta^0} \star \textcolor{blue}{\Delta^0}$ を計算してみよう\footnote{左右の区別を付けるために色を付けた.}.まず対象は
    \begin{align}
        (\Delta^0 \star \Delta^0)_0 &= \textcolor{red}{\Delta^0}_0 \sqcup \textcolor{blue}{\Delta^0}_0 
        = \left\{ 
            \begin{tikzpicture}[baseline={([yshift=-.5ex]current bounding box.center)}]
                \path coordinate[bullet,blue,label=below:$\{0\}$] (x)
                ++(0,1) coordinate[bullet,red,label=above:$\{0\}$] (y)
                ;
            \end{tikzpicture}
        \right\}
    \end{align}
    である.1-射は
    \begin{align}
        (\Delta^0 \star \Delta^0)_1 &= \textcolor{red}{\Delta^0}_1 \sqcup (\textcolor{DarkGreen}{\Delta^0_0 \times \Delta^0_0}) \sqcup \textcolor{blue}{\Delta^0}_1
    \end{align}
    であるが,\eqref{eq:face-join}より始点関数は
    \begin{align}
        \partial^1_1|_{\textcolor{DarkGreen}{\Delta^0_0 \times \Delta^0_0}} \colon \textcolor{DarkGreen}{\Delta^0_0 \times \Delta^0_0} &\lto \textcolor{red}{\Delta^0}_0 \sqcup \textcolor{blue}{\Delta^0}_0, \\
        (\{0\},\, \{0\}) &\lmto \textcolor{red}{\{0\}}
    \end{align}
    終点関数は
    \begin{align}
        \partial^1_0|_{\textcolor{DarkGreen}{\Delta^0_0 \times \Delta^0_0}} \colon \textcolor{DarkGreen}{\Delta^0_0 \times \Delta^0_0} &\lto \textcolor{red}{\Delta^0}_0 \sqcup \textcolor{blue}{\Delta^0}_0, \\
        (\{0\},\, \{0\}) &\lmto \textcolor{blue}{\{0\}}
    \end{align}
    となるため,$\textcolor{red}{\Delta^0}_1,\, \textcolor{blue}{\Delta^0}_1$ が縮退していることを考慮すると
    \begin{align}
        (\Delta^0 \star \Delta^0)_1
        &= \left\{ 
            \begin{tikzpicture}[baseline={([yshift=-.5ex]current bounding box.center)}]
                \path coordinate[bullet,blue,label=below:$\{0\}$] (x)
                ++(0,1) coordinate[bullet,red,label=above:$\{0\}$] (y)
                ;
                \draw[->-=.5,DarkGreen] (y) -- (x);
            \end{tikzpicture}
        \right\}     
    \end{align}
    と図示できる.
\end{myexample}

\begin{myexample}[label=ex:join-1]{join {$\Delta^1 \star \Delta^0$}}
    $\Delta^1 \star \Delta^0$ を計算してみよう.まず対象は
    \begin{align}
        (\Delta^1 \star \Delta^0)_0 &= \textcolor{red}{\Delta^1}_0 \sqcup \textcolor{blue}{\Delta^0}_0 
        = \left\{ 
            \begin{tikzpicture}[baseline={([yshift=-.5ex]current bounding box.center)}]
                \path coordinate[bullet,blue,label=below:$\{0\}$] (x)
                +(120:1) coordinate[bullet,red,label=above:$\{0\}$] (y)
                +(60:1) coordinate[bullet,red,label=above:$\{1\}$] (z)
                ;
            \end{tikzpicture}
        \right\}
    \end{align}
    である.1-射は
    \begin{align}
        (\Delta^1 \star \Delta^0)_1 &= \textcolor{red}{\Delta^1}_1 \sqcup (\textcolor{DarkGreen}{\Delta^1_0 \times \Delta^0_0}) \sqcup \textcolor{blue}{\Delta^0}_1
    \end{align}
    であるが,\eqref{eq:face-join}より始点関数は
    \begin{align}
        \partial^1_1|_{\textcolor{DarkGreen}{\Delta^1_0 \times \Delta^0_0}} \colon \textcolor{DarkGreen}{\Delta^1_0 \times \Delta^0_0} &\lto \textcolor{red}{\Delta^1}_0 \sqcup \textcolor{blue}{\Delta^0}_0, \\
        (x,\, \{0\}) &\lmto \textcolor{red}{x}
    \end{align}
    終点関数は
    \begin{align}
        \partial^1_0|_{\textcolor{DarkGreen}{\Delta^1_0 \times \Delta^0_0}} \colon \textcolor{DarkGreen}{\Delta^1_0 \times \Delta^0_0} &\lto \textcolor{red}{\Delta^1}_0 \sqcup \textcolor{blue}{\Delta^0}_0, \\
        (x,\, \{0\}) &\lmto \textcolor{blue}{\{0\}}
    \end{align}
    となるため,
    \begin{align}
        \Delta^1 \star \Delta^0
        &= \left\{ 
            \begin{tikzpicture}[baseline={([yshift=-.5ex]current bounding box.center)}]
                \path coordinate[bullet,blue,label=below:$\{0\}$] (x)
                +(120:3) coordinate[bullet,red,label=above:$\{0\}$] (y)
                +(60:3) coordinate[bullet,red,label=above:$\{1\}$] (z)
                ;
                \draw[->-=.5,red] (y) -- (z);
                \draw[->-=.5,DarkGreen] (y) -- (x);
                \draw[->-=.5,DarkGreen] (z) -- (x);
                \node[gray] at (barycentric cs:x=1,y=1,z=1) {$(\Id_{[1]},\, \{0\})$};
                \begin{scope}[on background layer]
                    \fill[gray!20] (x.center) -- (y.center) -- (z.center);
                \end{scope}
            \end{tikzpicture}
        \right\}  
    \end{align}
    と図示できる.ただし,三角形の内部は2-射 $(\Id_{[1]},\, \{0\}) \in \Delta^1_1 \times \Delta^0_0 \subset (\Delta^1 \star \Delta^0)_2$ が埋めている.
    同様に
    \begin{align}
        (\Delta^0 \star \Delta^1)_1
        &= \left\{ 
            \begin{tikzpicture}[baseline={([yshift=-.5ex]current bounding box.center)}]
                \path coordinate[bullet,red,label=below:$\{0\}$] (x)
                +(120:3) coordinate[bullet,blue,label=above:$\{0\}$] (y)
                +(60:3) coordinate[bullet,blue,label=above:$\{1\}$] (z)
                ;
                \draw[->-=.5,blue] (y) -- (z);
                \draw[-<-=.5,DarkGreen] (y) -- (x);
                \draw[-<-=.5,DarkGreen] (z) -- (x);
                \node[gray] at (barycentric cs:x=1,y=1,z=1) {$(\{0\},\, \Id_{[1]})$};
                \begin{scope}[on background layer]
                    \fill[gray!20] (x.center) -- (y.center) -- (z.center);
                \end{scope}
            \end{tikzpicture}
        \right\}
    \end{align}
    であることが分かる.
\end{myexample}

\begin{myexample}[label=ex:join-2]{join {$\Delta^2 \star \Delta^0$}}
    $\Delta^2 \star \Delta^0$ を計算してみよう.まず
    \begin{align}
        (\Delta^2 \star \Delta^0)_0 
        &= \textcolor{red}{\Delta^2}_0 \sqcup \textcolor{blue}{\Delta^0}_0 
        = \left\{ 
            \begin{tikzpicture}[baseline={([yshift=-.5ex]current bounding box.center)}]
                \path coordinate[bullet,blue,label=below:$\{0\}$] (x)
                ++(0,1) coordinate[bullet,red,label=right:$\{2\}$] (z)
                +(150:1) coordinate[bullet,red,label=above left:$\{1\}$] (y)
                +(30:1) coordinate[bullet,red,label=above right:$\{0\}$] (w)
                ;
            \end{tikzpicture}
        \right\} \\
    \end{align}
    である.次に1射は
    \begin{align}
        (\Delta^2 \star \Delta^0)_1 
        &= \textcolor{red}{\Delta^2}_1 \sqcup (\textcolor{DarkGreen}{\Delta^2_0 \times \Delta^0_0}) \sqcup \textcolor{blue}{\Delta^0}_1
    \end{align}
    であるが,\eqref{eq:face-join}より始点関数は
    \begin{align}
        \partial^1_1|_{\textcolor{DarkGreen}{\Delta^2_0 \times \Delta^0_0}} \colon \textcolor{DarkGreen}{\Delta^2_0 \times \Delta^0_0} &\lto \textcolor{red}{\Delta^2}_0 \sqcup \textcolor{blue}{\Delta^0}_0,\\
        (x,\, \Id_{\{0\}}) &\lmto \textcolor{red}{x}
    \end{align}
    となり,終点関数は
    \begin{align}
        \partial^1_0|_{\textcolor{DarkGreen}{\Delta^2_0 \times \Delta^0_0}} \colon \textcolor{DarkGreen}{\Delta^2_0 \times \Delta^0_0} &\lto \textcolor{red}{\Delta^2}_0 \sqcup \textcolor{blue}{\Delta^0}_0,\\
        (x,\, \Id_{\{0\}}) &\lmto \textcolor{blue}{\{0\}}
    \end{align}
    となる.従って図式\ref{fig:1-simp}に倣うと
    \begin{align}
        (\Delta^2 \star \Delta^0)_1 
        &= \textcolor{red}{\Delta^2}_1 \sqcup (\textcolor{DarkGreen}{\Delta^2_0 \times \Delta^0_0}) \sqcup \textcolor{blue}{\Delta^0}_1 =
        \left\{ 
            \begin{tikzpicture}[baseline={([yshift=-.5ex]current bounding box.center)}]
                \path coordinate[bullet,blue,label=below:$\{0\}$] (x)
                ++(0,1.5) coordinate[bullet,red,label=right:$\{1\}$] (y)
                +(150:1) coordinate[bullet,red,label=above left:$\{0\}$] (z)
                +(30:1) coordinate[bullet,red,label=above right:$\{2\}$] (w)
                ;
                \draw[->-=.5,DarkGreen] (y) -- (x);
                \draw[->-=.5,DarkGreen] (z) -- (x);
                \draw[->-=.5,DarkGreen] (w) -- (x);
                \draw[->-=.5,red] (z) -- (y);
                \draw[->-=.5,red] (y) -- (w);
                \draw[->-=.5,red] (z) -- (w);
            \end{tikzpicture}
        \right\} 
    \end{align}
    と図示できる.ただし,四面体の内部は3-射 $(\Id_{[2]},\, \textcolor{blue}{\{0\}}) \in \Delta^2_2 \times \Delta^0_0 \subset (\Delta^1 \star \Delta^0)_3$ が埋めている.
    同様に,$\Delta^0 \star \Delta^2$ の1-射を図示すると
    \begin{align}
        (\Delta^0 \star \Delta^2)_1 
        &= \textcolor{red}{\Delta^0}_1 \sqcup (\textcolor{DarkGreen}{\Delta^0_0 \times \Delta^2_0}) \sqcup \textcolor{blue}{\Delta^2}_1 =
        \left\{ 
            \begin{tikzpicture}[baseline={([yshift=-.5ex]current bounding box.center)}]
                \path coordinate[bullet,blue,label=below left:$\{0\}$] (x)
                ++(30:1) coordinate[bullet,blue,label=above right:$\{1\}$] (y)
                +(-30:1) coordinate[bullet,blue,label=below right:$\{2\}$] (z)
                +(0,1.5) coordinate[bullet,red,label=above:$\{0\}$] (w)
                ;
                \draw[->-=.5,blue] (x) -- (y);
                \draw[->-=.5,blue] (x) -- (z);
                \draw[->-=.5,blue] (y) -- (z);
                \draw[-<-=.5,DarkGreen] (x) -- (w);
                \draw[-<-=.5,DarkGreen] (y) -- (w);
                \draw[-<-=.5,DarkGreen] (z) -- (w);
            \end{tikzpicture}
        \right\} 
    \end{align}
    のようになる.
\end{myexample}

\begin{mylem}[label=lem:join-infty]{{$(\infty,\, 1)$}-圏同士のjoinは{$(\infty,\, 1)$}-圏}
    \hyperref[def:infinity-1]{$(\infty,\, 1)$-圏}同士の\hyperref[def:Simp-Join]{join}は $(\infty,\, 1)$-圏である.
\end{mylem}

\begin{proof}
    ~\cite[Proposition 1.2.8.3]{lurie2008higher}
\end{proof}

\begin{mydef}[label=def:overcat-infty,breakable]{スライス {$(\infty,\, 1)$}-圏}
    \hyperref[def:infinity-1]{$(\infty,\, 1)$-圏} $\Cat{D},\, \Cat{C}$ および\hyperref[def:infinity-1]{$(\infty,\, 1)$-圏の関手} $p \in \Hom{\sSet} (\Cat{D},\, \Cat{C})$ を与える.
    $p$ に沿った $\Cat{C}$ の\textbf{スライス圏} (overcategory) 
    \begin{align}
        \bm{\Cat{C}_{/p}} \colon \OP{\Delta} &\lto \SETS
    \end{align}
    を以下で定義する:
    \begin{itemize}
        \item $\forall [n] \in \Obj{\OP{\Delta}}$ に対して,集合
        \begin{align}
            \Hom{p} (\Delta^n \star \Cat{D},\, \Cat{C})
            \coloneqq \bigl\{\, f \in \Hom{\sSet} (\Delta^n \star \Cat{D},\, \Cat{C}) \bigm| f|_{\Cat{D}} = p \,\bigr\} 
        \end{align}
        を対応付ける\footnote{$f|_{\Cat{D}}$ というのは,\hyperref[def:Simp-Join]{joinの定義}における $(\Delta^n \star \Cat{D})_k$ のdisjoint unionのうち,添字 $i=0$ が振られている成分への制限を意味する.}.
        \item $\forall \alpha \in \Hom{\OP{\Delta}} ([m],\, [n])$ に対して,写像
        \begin{align}
            \Cat{C}_{/p} (\alpha) \colon \Hom{p} (\Delta^m \star \Cat{D},\, \Cat{C}) &\lto \Hom{p} (\Delta^n \star \Cat{D},\, \Cat{C}), \\
            f &\lmto f \circ (\alpha_* \star \Id_{\Cat{D}})
        \end{align}
        を対応付ける.
    \end{itemize}
    実際,単体的集合 $\Cat{C}_{/p}$ は $(\infty,\, 1)$-圏である~\cite[\href{https://kerodon.net/tag/018F}{Tag 018F}]{kerodon}.
    
    \tcblower

    $p$ に沿った $\Cat{C}$ の\textbf{コスライス圏} (undercategory) 
    \begin{align}
        \bm{\Cat{C}_{p/}} \colon \OP{\Delta} &\lto \SETS
    \end{align}
    を以下で定義する:
    \begin{itemize}
        \item $\forall [n] \in \Obj{\OP{\Delta}}$ に対して,集合
        \begin{align}
            \Hom{p} (\Cat{D} \star \Delta^n,\, \Cat{C})
            \coloneqq \bigl\{\, f \in \Hom{\sSet} (\Cat{D} \star \Delta^n,\, \Cat{C}) \bigm| f|_{\Cat{D}} = p \,\bigr\} 
        \end{align}
        を対応付ける\footnote{$f|_{\Cat{D}}$ というのは,\hyperref[def:Simp-Join]{joinの定義}における $(\Cat{D} \star \Delta^n)_k$ のdisjoint unionのうち,添字 $i=n$ が振られている成分への制限を意味する.}.
        \item $\forall \alpha \in \Hom{\OP{\Delta}} ([m],\, [n])$ に対して,写像
        \begin{align}
            \Cat{C}_{p/} (\alpha) \colon \Hom{p} (\Cat{D} \star \Delta^m,\, \Cat{C}) &\lto \Hom{p} (\Cat{D} \star \Delta^n,\, \Cat{C}), \\
            f &\lmto f \circ (\Id_{\Cat{D}} \star \alpha_*)
        \end{align}
        を対応付ける.
    \end{itemize}
\end{mydef}

特に注目すべきは,\hyperref[def:infinity-1]{$(\infty,\, 1)$-圏の関手} $p \colon \Delta^0 \lto \Cat{C}$ をとった場合である.
このとき $X \coloneqq p_{[0]} (\{0\}) \in \Cat{C}_0$ とおいて $\bm{\Cat{C}_{/X}},\; \bm{\Cat{C}_{X/}}$ などと書く.

まず,$(\infty,\, 1)$-圏 $\Cat{C}_{/X}$ の対象 $\varphi \in (\Cat{C}_{/X})_0 = \Hom{p}(\Delta^0 \star \textcolor{blue}{\Delta^0},\, \Cat{C})$ をとる.
すると\exref{ex:join-0}および $\varphi|_{\textcolor{blue}{\Delta^0}} = p$ の条件から,$\varphi_{[1]} \colon (\Delta^0 \star \textcolor{blue}{\Delta^0})_1 \lto \Cat{C}_1$ とは図式
\begin{align}
    \varphi =  
    \begin{tikzpicture}[baseline={([yshift=-.5ex]current bounding box.center)}]
        \path coordinate[bullet,blue,label=below:$X$] (x)
        ++(0,1) coordinate[bullet,label=above:$\varphi_{[0]}|_{\Delta^0_0}(\{0\})$] (y)
        ;
        \draw[->-=.5] (y) --node[midway,right]{$\varphi_{[1]}|_{\textcolor{DarkGreen}{\Delta^0_0 \times \Delta^0_0}} (\{0\}\to \textcolor{blue}{\{0\}})$} (x);
    \end{tikzpicture}
\end{align}
である.$n \ge 2$ 射に相当する $\varphi_{[n]} \colon (\Delta^0 \star \textcolor{blue}{\Delta^0})_n \lto \Cat{C}_n$ のデータは縮退していて自明である.従って,$\varphi$ は \hyperref[def:slice-category]{$(1,\, 1)$-圏における $X$ 上のスライス圏}の対象と等価なデータを与える.

同様に,$(\infty,\, 1)$-圏 $\Cat{C}_{/X}$ の1-射 $f \in (\Cat{C}_{/X})_1 = \Hom{p}(\Delta^1 \star \textcolor{blue}{\Delta^0},\, \Cat{C})$ とは,\exref{ex:join-1}より
\begin{align}
    f =     
    \begin{tikzpicture}[baseline={([yshift=-.5ex]current bounding box.center)}]
        \path coordinate[bullet,blue,label=below:$X$] (x)
        +(120:3) coordinate[bullet,label=left:$f_{[0]}|_{\Delta^1_0}(\{0\})$] (y)
        +(60:3) coordinate[bullet,label=right:$f_{[0]}|_{\Delta^1_0}(\{1\})$] (z)
        ;
        \draw[->-=.5] (y) -- (z);
        \draw[->-=.5] (y) -- (x);
        \draw[->-=.5] (z) -- (x);
         \node[gray] at (barycentric cs:x=1,y=1,z=1) {$f_{[2]}|_{\Delta^1_1 \times \Delta^0_0}(\Id_{[1]},\, \{0\})$};
                \begin{scope}[on background layer]
                    \fill[gray!20] (x.center) -- (y.center) -- (z.center);
                \end{scope}
        % \fill[red!20] (x) -- (y) -- (z) -- cycle;
    \end{tikzpicture}
\end{align}
のことである.ただし三角形の内部は2-射 $f_{[2]}|_{\Delta^1_1 \times \Delta^0_0}(\Id_{[2]},\, \textcolor{blue}{\{0\}}) \in \Cat{C}_2$ が埋めている.これは\hyperref[def:slice-category]{$(1,\, 1)$-圏における $X$ 上のスライス圏}の射のデータに対応しているが,横向きの矢印を決めるだけでは $f$ がupto \hyperref[def:infty-homotopy-morphism]{homotopy}でしか定まらないという点で異なっている.

$(\infty,\, 1)$-圏 $\Cat{C}_{/X}$ の $n$-射も同様に図示できる.

\begin{myexample}[label=def:infty-forget]{スライス圏からのforgetful functor}
    \hyperref[def:infinity-1]{$(\infty,\, 1)$-圏} $\Cat{C}$ の,$X \colon \Delta^0 \lto \Cat{C}$ に沿った\hyperref[def:overcat-infty]{スライス圏}に対して,\textbf{忘却関手} (forgetful functor)
    \begin{align}
        \text{forget} \colon \Cat{C}_{/X} \lto \Cat{C}
    \end{align}
    を次のように定義する:
    \begin{align}
        \label{eq:forget}
        \text{forget}_{[n]} \colon (\Cat{C}_{/X})_n = \Hom{X} (\Delta^n \star \Delta^0,\, \Cat{C}) &\lto \Cat{C}_n = \Hom{\sSet}(\Delta^n,\, \Cat{C}), \\
        f &\lmto f|_{\Delta^n}
    \end{align}
    $n = 0,\, 1,\, 2$ の場合,i.e. \exref{ex:join-0}, \exref{ex:join-1}, \exref{ex:join-2}の図式においては,ちょうど $X \in \Cat{C}_0$ に対応する青色の頂点(コーンポイント)を除去する操作に対応している.
    \eqref{eq:forget}の定義は\hyperref[def:infty-fib]{右ファイブレーション}になっている.
\end{myexample}


\subsection{{$(\infty,\, 1)$}-圏のlimit/colimit}

後の議論のため,先取りして $(\infty,\, 1)$-圏におけるlimit/colimitを定義しておこう.
$(\infty,\, 1)$-圏のモデルとして\hyperref[def:infinity-1]{quasi-category}を採用する場合,これは\textbf{homotopy limit/colimit}と呼ばれることもある.

\begin{mydef}[label=def:initial-final-infty]{{$(\infty,\, 1)$}-圏における始対象と終対象}
    \begin{itemize}
        \item \hyperref[def:infinity-1]{$(\infty,\, 1)$-圏} $\Cat{C}$ における対象 $x \in \Cat{C}_0$ が\textbf{始対象} (initial object) であるとは,\hyperref[def:hcat-infty]{ホモトピー圏} $\htpy{\Cat{C}}$ における始対象\footnote{$(1,\, 1)$-圏の\textbf{始対象}とは,空の図式における\hyperref[def:colim]{余極限}のこと.}であること.
        \item \hyperref[def:infinity-1]{$(\infty,\, 1)$-圏} $\Cat{C}$ における対象 $x \in \Cat{C}_0$ が\textbf{終対象} (final object) であるとは,\hyperref[def:hcat-infty]{ホモトピー圏} $\htpy{\Cat{C}}$ における終対象\footnote{$(1,\, 1)$-圏の\textbf{終対象}とは,空の図式における\hyperref[def:lim]{極限}のこと.}であること.
    \end{itemize}
\end{mydef}

\begin{mydef}[label=def:lim-colim-infty]{{$(\infty,\, 1)$}-圏のlimit/colimit}
    \hyperref[def:infinity-1]{$(\infty,\, 1)$-圏の関手} $D \colon \Cat{I} \lto \Cat{C}$ を与える\footnote{\hyperref[def:diagram]{$(1,\, 1)$-圏の場合}からのアナロジーで,$D$ を図式と見做す.}.
    \begin{itemize}
        \item $D$ の\textbf{limit}とは,\hyperref[def:overcat-infty]{スライス $(\infty,\, 1)$-圏} $\Cat{C}_{/D}$ における\hyperref[def:initial-final-infty]{終対象}のこと.$\bm{\lim D} \in \Cat{C}_0$ と書く.
        \item $D$ の\textbf{colimit}とは,\hyperref[def:overcat-infty]{スライス $(\infty,\, 1)$-圏} $\Cat{C}_{D/}$ における\hyperref[def:initial-final-infty]{始対象}のこと.$\bm{\colim D} \in \Cat{C}_0$ と書く.
    \end{itemize}
\end{mydef}

\begin{myexample}[label=def:pullback-infty]{pullback}
    単体的集合の積 $S \times T \colon \OP{\Delta} \lto \SETS$ における面写像とは 
    \begin{align}
        \partial^n_j \colon S_n \times T_n &\lto S_{n-1} \times T_{n-1}, \\
        (x,\, y) &\lmto (\partial^n_j x,\, \partial^n_j y)
    \end{align}
    のことであった.故に,単体的集合 $\Delta^1 \times \Delta^1$ は,$\Delta^1_0 \eqqcolon \{\bullet_0,\, \bullet_1\}$ とおくと
    \begin{align}
        \Delta^1 \times \Delta^1 =
        \begin{tikzpicture}[baseline={([yshift=-.5ex]current bounding box.center)}]
            \path  coordinate[bullet,label=below left:{$(0,\,1)$}] (x)
            +(1,0) coordinate[bullet,label=below right:{$(1,\,1)$}] (y)
            +(0,1) coordinate[bullet,label=above left:{$(0,\,0)$}] (z)
            +(1,1) coordinate[bullet,label=above right:{$(1,\,0)$}] (w)
            ;
            \draw[->-=.5] (z) -- (x);
            \draw[->-=.5] (z) -- (y);
            \draw[->-=.5] (z) -- (w);
            \draw[->-=.5] (x) -- (y);
            \draw[->-=.5] (w) -- (y);
        \end{tikzpicture}
    \end{align}
    と図示できる.ただし,2-射以上は縮退していて見えない.
    
     \hyperref[def:infinity-1]{$(\infty,\, 1)$-圏の関手} $D \colon \Delta^1 \times \Delta^1 \lto \Cat{C}$ の\hyperref[def:lim-colim-infty]{limit}のことを(存在すれば)\textbf{pullback}と呼び,
    \begin{align}
        \bm{D(0,\, 1)} \bm{\times}_{\bm{D(1,\, 1)}} \bm{D(1,\, 0)} \coloneqq \lim D \in \Cat{C}_0
    \end{align}
    と書く.
\end{myexample}

\subsection{Unstraightening construction}

\begin{mytheo}[label=thm:unstraightening]{unstraightening construction}
    \hyperref[def:equiv-infty]{$(\infty,\, 1)$-圏同値}
    \begin{align}
        \inftyPsh{\Cat{B}} \xrightarrow{\simeq} \RFIB_{\Cat{B}}
    \end{align}
    が存在する.
\end{mytheo}

\begin{proof}
    \hyperref[def:equiv-infty]{$(\infty,\, 1)$-圏同値}
    \begin{align}
        \Un\colon \inftyPsh{\Cat{B}} \lto \RFIB_{\Cat{B}}
    \end{align}
    は,次のようにして構成される (\textbf{unstraightening construction}):
    \begin{description}
        \item[\textbf{対象}] 
        $F \in \inftyPsh{\Cat{B}}_0$ に対して,\hyperref[def:pullback-infty]{$(\infty,\, 1)$-圏のpullback}
        \begin{center}
            % https://tikzcd.yichuanshen.de/#N4Igdg9gJgpgziAXAbVABwnAlgFyxMJZABgBpiBdUkANwEMAbAVxiRAB12BVMACgDEAlCAC+pdJlz5CKAIzkqtRizacAygAUAggGEAomoD6wAPQAqEaPEgM2PASLzZi+s1aIO7TboNWJd6SIyZ2pXFQ9OHTocYAAhSxFFGCgAc3giUAAzACcIAFskAGZqHAgkACZqBjoAIxgGDUl7GRBsrBSACxwQUOV3EH4-EBz8pHkQUoret1V2HBgADxjMiGy0nATrEYLEMgmyxEKxLNydvcnEWUSRIA
            \begin{tikzcd}
            \textcolor{red}{\Un(F)} \arrow[red,d] \arrow[r] & \SPACES_{/*} \arrow[d, "\text{forget}"] \\
            \Cat{B} \arrow[r, "F"']    & \SPACES                                
            \end{tikzcd}
        \end{center}
        により得られる\hyperref[def:infty-fib]{右ファイブレーション} $\textcolor{red}{\Un(F) \lto}\, \Cat{B} \in (\RFIB_{\Cat{B}}){}_0$ を対応付ける.
        \item[\textbf{$\bm{n}$-射}] 
    \end{description}
    逆向きの\hyperref[def:equiv-infty]{$(\infty,\, 1)$-圏同値} (\textbf{straightning construction})
    \begin{align}
        \St\colon \RFIB_{\Cat{B}} \lto \inftyPsh{\Cat{B}}
    \end{align}
    は難しい.詳細は~\cite[Proposition 2.2.3.11]{lurie2008higher}を参照.
\end{proof}

\subsection{{$(\infty,\, 1)$}-圏における米田埋め込み}

\begin{mydef}[label=def:Tw]{twisted arrow category}
    \hyperref[def:infinity-1]{$(\infty,\, 1)$-圏} $\Cat{C}$ を与える.
    このとき,$\Cat{C}$ の\textbf{twisted arrow category}と呼ばれる $(\infty,\, 1)$-圏を以下で定義する:
    \begin{align}
        \Tw(\Cat{C}) \colon \OP{\Delta} &\lto \SETS, \\
        [n] &\lmto \Hom{\sSet} \bigl( \OP{(\Delta^n)} \star \Delta^n,\, \Cat{C} \bigr), \\
        \bigl( [m] \xrightarrow{\alpha} [n] \bigr) &\lmto \bigl( f \mapsto f \circ (\alpha_* \times \alpha_*) \bigr) 
    \end{align}
\end{mydef}

\hyperref[def:infinity-1]{$(\infty,\, 1)$-圏の関手} 
\begin{align}
    \mathrm{pr} \colon \Tw(\Cat{C}) \lto \OP{\Cat{C}} \times \Cat{C}
\end{align}
を以下で定義すると,これは\hyperref[def:infty-fib]{左ファイブレーション}になる~\cite[\href{https://kerodon.net/tag/03JQ}{Tag 03JQ}]{kerodon}:
\begin{align}
    \mathrm{pr}_{[n]} \colon \Tw(\Cat{C})_n &\lto \OP{\Cat{C}}_n \times \Cat{C}_n = \Hom{\sSet}(\Delta^n,\, \OP{\Cat{C}}) \times \Hom{\sSet}(\Delta^n,\, \Cat{C}) \\
    f &\lmto \bigl( f|_{\OP{(\Delta^n)}},\, f|_{\Delta^n} \bigr) 
\end{align}
\hyperref[def:infinity-1]{$(\infty,\, 1)$-圏}におけるHom関手
\begin{align}
    \label{def:infty-Hom}
    \inftyMap{\Cat{C}} \colon \OP{\Cat{C}} \times \Cat{C} \lto \SPACES
\end{align}
の自然な構成は,\hyperref[thm:unstraightening]{straightning construction}を用いた
\begin{center}
    % https://tikzcd.yichuanshen.de/#N4Igdg9gJgpgziAXAbVABwnAlgFyxMJZABgBpiBdUkANwEMAbAVxiRAB12BVMACgDEAlCAC+pdJlz5CKAIzkqtRizacAygAUAggGEAomoD6wAPQAqEaPEgM2PASLzZi+s1aIO7TboNWJd6SIyZ2pXFQ9OHTocYAAhSxFFGCgAc3giUAAzACcIAFskAGZqHAgkACZqBjoAIxgGDUl7GRBsrBSACxwQUOV3EH4-EBz8pHkQUoret1V2HBgADxjMiGy0nATrEYLEMgmyxEKxLNydvcnEWUSRIA
    \begin{tikzcd}[column sep=large]
    \Tw(\Cat{C}) \arrow[d, "\mathrm{pr}"'] \arrow[r] & \SPACES_{*/} \arrow[d, "\text{forget}"] \\
    \OP{\Cat{C}} \times \Cat{C} \arrow[r, red, "\inftyMap{\Cat{C}} \coloneqq \St(\mathrm{pr})"']    & \SPACES                                
    \end{tikzcd}
\end{center}
である~\cite[I.26., p.19]{Hebestreit2020K-theory}.

\begin{mydef}[label=def:Yoneda-infty]{{$(\infty,\, 1)$}-圏の米田埋め込み (informal)}
    $\Cat{C}$ を\hyperref[def:infinity-1]{$(\infty,\, 1)$-圏}とする.$(\infty,\, 1)$-圏の\textbf{米田埋め込み} (Yoneda embedding)
    \begin{align}
        \yo \colon \Cat{C} \lto \inftyPsh{\Cat{C}}
    \end{align}
    とは,$\Cat{C}$ から\hyperref[def:infinity-presheaf]{$(\infty,\, 1)$-前層の成す $(\infty,\, 1)$-圏} $\inftyPsh{\Cat{C}}$ への\hyperref[def:infinity-1]{$(\infty,\, 1)$-圏の関手}であって,
    対象 $x \in \Cat{C}_0$ に対して
    \begin{align}
        \yo_{[0]}(x)_{[0]} \colon \OP{\Cat{C}}_0 &\lto \SPACES_0, \\
        y &\lmto \inftyMap{\Cat{C}} (x,\, y)
    \end{align}
    を充たす\footnote{$(\infty,\, 1)$-圏 $\Cat{C}$ において,対象 $x,\, y \in \Cat{C}_0$ の間の\hyperref[def:Map]{射の空間} $\inftyMap{\Cat{C}}(x,\, y)$ はKan複体を成すのだった~\cite[\href{https://kerodon.net/tag/01JC}{Tag 01JC}]{kerodon}.}ような $(\infty,\, 1)$-圏の関手 $\yo_{[0]}(x) \in \inftyPsh{\Cat{C}}_0 = \Hom{\sSet}(\OP{\Cat{C}},\, \SPACES)$ を対応付けるもののこと\footnote{厳密な構成については~\cite[\href{https://kerodon.net/tag/03NF}{Tag 03NF}]{kerodon}を参照.}.
\end{mydef}
% \subsection{$\mathcal{B}\mathrm{sc}$ の構造}



\subsection{層状化空間の接構造}

\begin{mydef}[label=def:Enter]{enter-path category}
    \hyperref[def:c-smooth]{conically smoothな層状化空間} $X \in \Obj{\Strat}$ の\textbf{enter-path $(\infty,\, 1)$-category}とは,
    $(\infty,\, 1)$-圏 \hyperref[def:Strat-infty]{$\BSC$} の\hyperref[def:overcat-infty]{スライス圏}
    \begin{align}
        \ENTR{X} \coloneqq \BSC_{/X}
    \end{align}
    のこと.
\end{mydef}

\begin{mydef}[label=def:tangent-classifier]{tangent classifier}
    $\iota \colon \BSC \hookrightarrow \SNGLR$ を包含とする.
    \textbf{tangent classifier}とは,\hyperref[def:infinity-1]{$(\infty,\, 1)$-圏の関手}
    \begin{align}
        \bm{\tau} \colon \SNGLR \xrightarrow{\yo} \inftyPsh{\SNGLR} \xrightarrow{\iota^*} \inftyPsh{\BSC}
    \end{align}
    のこと.
\end{mydef}

定義\ref{def:Yoneda-infty}より,
\hyperref[def:tangent-classifier]{tangent classifier}は\hyperref[def:c-smooth]{conically smoothな層状化空間} $X \in \SNGLR_0$ に対して $(\infty,\, 1)$-圏の表現可能前層
\begin{align}
    \label{eq:representable-infty}
    \tau_{[0]}(X) = \inftyMap{\BSC}(\, \mhyphen\, ,\, X) \in \inftyPsh{\BSC}_0
\end{align}
を対応付ける.

\begin{marker}
    定理\ref{thm:unstraightening}により,\hyperref[def:tangent-classifier]{tangent classifier} $\tau$ のことを
    \begin{align}
        \tau \colon \SNGLR \xrightarrow{\yo} \inftyPsh{\SNGLR} \xrightarrow{\iota^*} \inftyPsh{\BSC} \xrightarrow{\simeq} \RFIB_{\BSC}
    \end{align}
    と見做すこともできる.このとき,\hyperref[def:RFIB]{$\RFIB_{\BSC}$} の構成および $(\infty,\, 1)$-前層\eqref{eq:representable-infty}に対する定理\ref{thm:unstraightening}の具体的構成から,
    \hyperref[def:c-smooth]{conically smoothな層状化空間} $X \in \SNGLR_0$ に対して定まる $(\infty,\, 1)$-圏の\hyperref[def:infty-fib]{右ファイブレーション} $\tau_{[0]}(X) \in (\RFIB_{\BSC})_0$ とは
    忘却関手
    \begin{align}
        \tau_{[0]}(X) \colon \ENTR{X} \lto \BSC
    \end{align}
    のことである.この忘却関手を以下では $\bm{\tau_X} \coloneqq \tau_{[0]}(X)$ と書く.
\end{marker}

\begin{mydef}[label=def:B-mfld,breakable]{{$\Cat{B}$}-多様体}
    \begin{itemize}
        \item \textbf{$\bm{(\Cat{B},\, f)}$ 構造}\footnote{~\cite[Definition 1.1.6]{AFT2014stratified}では\textbf{$(\infty,\, 1)$-category of basics}と呼ばれている.}とは,$(\infty,\, 1)$-圏の\hyperref[def:infty-fib]{右ファイブレーション} $(\Cat{B} \xrightarrow{f} \BSC) \in (\RFIB_{\BSC})_0$ のこと.
        \item $(\Cat{B},\, f)$ 構造 $\Cat{B} \xrightarrow{f} \BSC$ を1つ固定する.このとき,\textbf{$\bm{\Cat{B}}$-多様体} ($\Cat{B}$-manifold) の成す $(\infty,\, 1)$-圏 $\MFLDB{\Cat{B}}$ とは,\hyperref[def:pullback-infty]{$(\infty,\, 1)$-圏のpullback}
        \begin{center}
            % https://tikzcd.yichuanshen.de/#N4Igdg9gJgpgziAXAbVABwnAlgFyxMJZABgBpiBdUkANwEMAbAVxiRAB12BZAMQBkAIgCFgnAMJ0cwIQF8ZIGaXSZc+QigCM5KrUYs2ACk4AlHgEkhAfVHshAZTEyAlNfGTpMgAScAHgCcsAHMACxw6Pz8IAHdgNC9Oe0cFJRAMbDwCIi0NHXpmVkQOdlMLV1sHeUVldLUiMhzqPP1CzjsAOQBxPmMFHRgoQPgiUAAzSIBbJABmahwIJAAmRr0CorCmEGoGOgAjGAYABRUM9RAAkJxk0YmkMhA56aqQMYhJxC17+cQl3Xy2TnGkmCfnGwBGED8gxwlRSLzedwe7xkFBkQA
            \begin{tikzcd}
            \MFLDB{\Cat{B}} \arrow[d] \arrow[r] & (\RFIB_{\BSC})_{/f} \arrow[d, "\mathrm{forget}"] \\
            \SNGLR \arrow[r, "\tau"']           & \RFIB_{\BSC}                                                              
            \end{tikzcd}
        \end{center}
        のこと.特に,$\MFLDB{\Cat{B}}_0$ の元は以下の2つのデータから成り,\textbf{$\bm{\Cat{B}}$-多様体}と呼ばれる:
        \begin{itemize}
            \item \hyperref[def:c-smooth]{conically smoothな層状化空間} $X \in \SNGLR_0$
            \item \hyperref[def:infinity-1]{$(\infty,\, 1)$-圏の関手} $g \colon \ENTR{X} \lto \Cat{B}$
        \end{itemize}
        これらは以下の条件を充たさねばならない:
        \begin{description}
            \item[\textbf{(lift of tangent classifier)}] 
            
            $\sSet$ における以下の図式は可換である:
            \begin{center}
                % https://tikzcd.yichuanshen.de/#N4Igdg9gJgpgziAXAbVABwnAlgFyxMJZABgBoBGAXVJADcBDAGwFcYkQAdDgUQDkAVAErAAGgF8QY0uky58hFOQrU6TVuy4AhAMoBhSdJAZseAkSXEVDFm0ScOu+jmCaJYlTCgBzeEVAAzACcIAFskMhAcCCQAJhprdTsvAwDgsMQIqKQlVRsNDhx6ZgB9ERAaRnoAIxhGAAVZUwUQQKwvAAscFJAg0NiaLMQchNseyUoxIA
                \begin{tikzcd}
                                                            & \Cat{B} \arrow[d, "f"] \\
                \ENTR{X} \arrow[ru, "g"] \arrow[r, "\tau_X"'] & \BSC                  
                \end{tikzcd}
            \end{center}
        \end{description}
    \end{itemize}
\end{mydef}

\subsection{$C^\infty$-多様体の接構造との比較}

\section{Disk algebras}

\subsection{{$(\infty,\, 1)$}-オペラッド}

本資料では,$(\infty,\, 1)$-圏を\hyperref[def:infinity-1]{quasi-category}として定義した.
この小節では,quai-categoryにおけるcolored operadを定義する.

\begin{mydef}[label=def:Fin]{{$(1,\, 1)$}-圏 {$\FIN,\; \FIN_*$}}
    $(1,\, 1)$-圏 $\FIN$ を以下で定義する:
    \begin{itemize}
        \item 有限集合および空集合を対象に持つ.
        \item $I,\, J \in \Obj{\FIN}$ の間の写像を射とする.
    \end{itemize}
    \tcblower
    $(1,\, 1)$-圏 $\FIN_*$ を以下で定義する:
    \begin{itemize}
        \item 基点付き有限集合
        \begin{align}
            \expval{n} \coloneqq \{*,\, 1,\, \dots,\, n\}
        \end{align}
        を対象に持つ.i.e.
        \begin{align}
            \Obj{\FIN_*} \coloneqq \bigl\{\, \expval{n} \bigm| n \in \mathbb{Z}_{\ge 0} \,\bigr\}.
        \end{align}
        \item $\forall \expval{m},\, \expval{n} \in \Obj{\FIN_*}$ に対して,それらの間の基点を保つ写像を射とする.i.e.
        \begin{align}
            \Hom{\FIN_*}\bigl( \expval{m},\, \expval{n} \bigr)  \coloneqq \bigl\{\, f \in \Hom{\SETS} \bigl( \expval{m},\, \expval{n} \bigr)  \bigm| f(*) = * \,\bigr\}.
        \end{align}
    \end{itemize}
\end{mydef}

\begin{mydef}[label=def:inert-active]{inert/active morphism}
    \begin{itemize}
        \item \hyperref[def:Fin]{圏 $\FIN_*$} における射 $f \in \Hom{\FIN_*} (\expval{m},\, \expval{n})$ が\textbf{inert}であるとは,$\forall i \in \expval{n} \setminus \{*\}$ に対して $f^{-1} \bigl( \{i\} \bigr) \subset \expval{m}$ が1点集合であることを言う.
        \item \hyperref[def:Fin]{圏 $\FIN_*$} における射 $f \in \Hom{\FIN_*} (\expval{m},\, \expval{n})$ が\textbf{active}であるとは,$f^{-1} \bigl( \{*\} \bigr) = \{*\} \subset \expval{m}$ であることを言う.
    \end{itemize}
\end{mydef}

\begin{myexample}[label=def:rho-i]{inertな射 {$\rho^i$}}
    $1 \le \forall i \le \forall n$ を1つ固定する.このとき,写像
    \begin{align}
        \rho^i \colon \expval{n} &\lto \expval{1}, \\
        j &\lmto 
        \begin{cases}
            1, &j=i \\
            *, &j\neq i
        \end{cases}
    \end{align}
    は圏 $\FIN_*$ における\hyperref[def:inert-active]{inertな射}である.
\end{myexample}

\begin{myexample}[label=def:alpha-n]{activeな射 {$\alpha_n$}}
    $\forall n \ge 1$ を1つ固定する.このとき,写像
    \begin{align}
        \alpha_n \colon \expval{n} &\lto \expval{1}, \\
        j &\lmto 
        \begin{cases}
            1, &j\neq * \\
            *, &j = *
        \end{cases}
    \end{align}
    は圏 $\FIN_*$ における\hyperref[def:inert-active]{activeな射}である.なお,$\alpha_n$ は射の集合 $\Hom{\FIN_*} \bigl( \expval{n},\, \expval{1} \bigr)$ の元のうち,唯一のactiveな射である.
\end{myexample}

\hyperref[def:nerve]{脈体の定義}において $[n] \in \Obj{\Delta}$ を $(1,\, 1)$-圏と見做した方法と同様にして,半順序集合 $\{n-1 \le n\}$ を $(1,\, 1)$-圏と見做す.
このとき,
\begin{align}
    \Ner \bigl( \{n-1 \le n\} \bigr) = 
    \left(
        \begin{tikzpicture}[baseline={([yshift=-.5ex]current bounding box.center)}]
            \path coordinate[bullet,label=below:$n-1$] (x)
            +(2,0) coordinate[bullet,label=below:$n$] (y)
            ;
            \draw[->-=.5] (x) -- (y);
        \end{tikzpicture}
     \right) \cong \Delta^1
\end{align}
と図示できる.同様に,
\begin{align}
    \Ner \bigl( \{0 \le 1\} \bigr) = 
    \left(
        \begin{tikzpicture}[baseline={([yshift=-.5ex]current bounding box.center)}]
            \path coordinate[bullet,label=below:$0$] (x)
            +(2,0) coordinate[bullet,label=below:$1$] (y)
            ;
            \draw[->-=.5] (x) -- (y);
        \end{tikzpicture}
     \right) \cong \Delta^1
\end{align}
である.

\begin{mydef}[label=def:p-Cart,breakable]{$p$-Cartesian morphism}
    $p \colon \Cat{E} \lto \Cat{B}$ を\hyperref[def:infty-fib]{内的ファイブレーション}とする.
    \begin{itemize}
        \item $\Cat{E}$ の1-射 $(x \xrightarrow{f} y) \in \Cat{E}_1$ が以下の条件を充たすとき,$f$ は\textbf{$\bm{p}$-Cartesian}であると言う:
        \begin{description}
            \item[\textbf{(Cartesian)}]
            
            $\forall n \ge 2$ に対して,以下の $\sSet$ の図式を可換にする $\Cat{E}$ の $n$-射 $\textcolor{red}{\bar{\varphi}} \in \Hom{\sSet} (\Delta^n,\, \Cat{E}) \cong \Cat{E}_n$ が存在する:
            \begin{center}
                % https://tikzcd.yichuanshen.de/#N4Igdg9gJgpgziAXAbVABwnAlgFyxMJZABgBpiBdUkANwEMAbAVxiRAB12BbOnACwBOXYADkAvpwBGWAOYMAFJ2BgAtAEYABJwYwNYThPbSZAgJRb2AYwIyLAERgMcdAHpqQY0uky58hFGqkalS0jCxsnADCvMAAomIeXiAY2HgERIEATCH0zKyIHOzROMAAQgme3ql+RGTZ1LnhBZwOTq6Elck+af4kQTlh+YUAMnRcklDtAPodITBQMvBEoABmAhBcSGQgOBBIACzUzlgMbHwQEADWiavrm4jbu0iBoXkRRnQCwCsVSWsbByOe0Qak6-3uAGYgUhMg1Bu8VhABIwGBZ6AI0HwsCBqAw6JJHAAFbo1AoCWR8HA3EDgwE7YFQnZ0E5nC7XMF3Z7QxCw15NZLU2mIRlPEG4rBgIaTOB8eY4vlDTgwAAeWDgODgFkkn2AnHRmKwFQoYiAA
                \begin{tikzcd}
                \Ner \bigl(\{n-1 \le n\}\bigr) \cong \Delta^1 \arrow[d, hook] \arrow[rd, "f"] &                        \\
                \Lambda^n_n \arrow[r,blue, "\forall \varphi_0"] \arrow[d, hook]                                                    & \Cat{E} \arrow[d, "p"] \\
                \Delta^n \arrow[r, blue,"\forall \varphi"'] \arrow[ru, red, "\exists \bar{\varphi}", dashed]       & \Cat{B}               
                \end{tikzcd}
            \end{center}
        \end{description}
        ただし,$\sSet$ の射 $f \colon \Ner \bigl(\{n-1 \le n\}\bigr) \lto \Cat{E}$ とは
        \begin{align}
            f_{[1]}
                \left( 
                   \begin{tikzpicture}[baseline={([yshift=-.5ex]current bounding box.center)}]
                       \path coordinate[bullet,label=below:$n-1$] (x)
                       +(1,0) coordinate[bullet,label=below:$n$] (y)
                       ;
                       \draw[->-=.5] (x) -- (y);
                   \end{tikzpicture}
                \right)
            =
                   \begin{tikzpicture}[baseline={([yshift=-.5ex]current bounding box.center)}]
                       \path coordinate[bullet,label=below:$x$] (x)
                       +(1,0) coordinate[bullet,label=below:$y$] (y)
                       ;
                       \draw[->-=.5] (x) --node[midway,above] {$f$} (y);
                   \end{tikzpicture}
        \end{align}
        を充たす唯一の自然変換である.
        \item $\Cat{E}$ の1-射 $(x \xrightarrow{f} y) \in \Cat{E}_1$ が以下の条件を充たすとき,$f$ は\textbf{$\bm{p}$-coCartesian}であると言う:
        \begin{description}
            \item[\textbf{(coCartesian)}] 
            
            $\forall n \ge 2$ に対して,以下の $\sSet$ の図式を可換にする $\Cat{E}$ の $n$-射 $\textcolor{red}{\bar{\varphi}} \in \Hom{\sSet} (\Delta^n,\, \Cat{E}) \cong \Cat{E}_n$ が存在する:
            \begin{center}
                % https://tikzcd.yichuanshen.de/#N4Igdg9gJgpgziAXAbVABwnAlgFyxMJZABgBpiBdUkANwEMAbAVxiRAB12BbOnACwBOXYADkAvpwBGWAOYMAFJ2BgAtAEYABJwYwNYThPbSZAgJRb2AYwIyLAERgMcdAHpqQY0uky58hFGqkalS0jCxsnADCvMAAomIeXiAY2HgERIEATCH0zKyIHOzROMAAQgme3ql+RGTZ1LnhBZwOTq6Elck+af4kQTlh+YUAMnRcklDtAPodITBQMvBEoABmAhBcSGQgOBBIACzUzlgMbHwQEADWiavrm4jbu0iBoXkRRnQCwCsVSWsbByOe0Qak6-3uAGYgUhMg1Bu8VhABIwGBZ6AI0HwsCBqAw6JJHAAFbo1AoCWR8HA3EDgwE7YFQnZ0E5nC7XMF3Z7QxCw15NZLU2mIRlPEG4rBgIaTOB8eY4vlDTgwAAeWDgODgFkkn2AnHRmKwFQoYiAA
                \begin{tikzcd}
                \Ner \bigl(\{0 \le 1\}\bigr) \cong \Delta^1 \arrow[d, hook] \arrow[rd, "f"] &                        \\
                \Lambda^n_0 \arrow[r,blue,"\forall \varphi_0"] \arrow[d, hook]                                                    & \Cat{E} \arrow[d, "p"] \\
                \Delta^n \arrow[r, blue,"\forall \varphi"'] \arrow[ru, red, "\exists \bar{\varphi}", dashed]       & \Cat{B}               
                \end{tikzcd}
            \end{center}
            ただし,$\sSet$ の射 $f \colon \Ner \bigl(\{0 \le 1\}\bigr) \lto \Cat{E}$ とは
            \begin{align}
                f_{[1]}
                    \left( 
                       \begin{tikzpicture}[baseline={([yshift=-.5ex]current bounding box.center)}]
                           \path coordinate[bullet,label=below:$n-1$] (x)
                           +(1,0) coordinate[bullet,label=below:$n$] (y)
                           ;
                           \draw[->-=.5] (x) -- (y);
                       \end{tikzpicture}
                    \right)
                =
                       \begin{tikzpicture}[baseline={([yshift=-.5ex]current bounding box.center)}]
                           \path coordinate[bullet,label=below:$x$] (x)
                           +(1,0) coordinate[bullet,label=below:$y$] (y)
                           ;
                           \draw[->-=.5] (x) --node[midway,above] {$f$} (y);
                       \end{tikzpicture}
            \end{align}
            を充たす唯一の自然変換である.
        \end{description}
    \end{itemize}
\end{mydef}

$n=2$ の場合の\hyperref[def:p-Cart]{\textsf{\textbf{(coCartesian)}}}の可換図式の意味を,系\ref{col:horn-coeq}を用いて解読しよう.
まず,包含 $\Ner \bigl( \{0 \le 1\} \bigr) \hookrightarrow \Lambda^2_0$ というのは,系\ref{col:horn-coeq}による角 $\Lambda^2_0$ の図示
\begin{center}
    \begin{tikzpicture} 
        \path coordinate[bullet,label=below left:$\{0\}$] (v0)
        +(2,0) coordinate[bullet,label=below right:$\{2\}$] (v2)
        +(60:2) coordinate[bullet,label=above:$\{1\}$] (v1)
        ;
        \draw[->-=.5] (v0) -- (v1);
        \draw[->-=.5] (v0) -- (v2);
        % \draw[->-=.5] (v1) -- (v2);
    \end{tikzpicture}   
\end{center}
のうち辺 $\{0\} \lto \{1\}$ への埋め込みであるから,可換図式の
\begin{center}
    % https://tikzcd.yichuanshen.de/#N4Igdg9gJgpgziAXAbVABwnAlgFyxMJZABgBpiBdUkANwEMAbAVxiRAB12BbOnACwBOXYADkAvpwBGWAOYMAFJ2BgAtAEYABJwYwNYThPbSZAgJRb2AYwIyLAERgMcdAHpqQY0uky58hFGqkalS0jCxsnADCvMAAomIeXiAY2HgERIEATCH0zKyIHOzROMAAQgme3ql+RGTZ1LnhBZwOTq6Elck+af4kQTlh+YUAMnRcklDtAPodITBQMvBEoABmAhBcSGQgOBBIACzUzlgMbHwQEADWiavrm4jbu0iBoXkRRnQCwCsVSWsbByOe0Qak6-3uAGYgUhMg1Bu8VhABIwGBZ6AI0HwsCBqAw6JJHAAFbo1AoCWR8HA3EDgwE7YFQnZ0E5nC7XMF3Z7QxCw15NZLU2mIRlPEG4rBgIaTOB8eY4vlDTgwAAeWDgODgFkkn2AnHRmKwFQoYiAA
    \begin{tikzcd}
    \Ner \bigl(\{0 \le 1\}\bigr) \cong \Delta^1 \arrow[d, hook] \arrow[rd, "f"] &                        \\
    \Lambda^2_0 \arrow[r,blue,"\forall \varphi_0"]                                                  & \Cat{E}
    \end{tikzcd}
\end{center}
の部分は勝手な角の図式 $\textcolor{blue}{\varphi_0} = (\bullet,\, f,\, \textcolor{blue}{\varphi_0{}_2}) \in \Cat{E}_1^{\times 2}$ を与えることに対応する.
図示すると
\begin{align}
    \label{eq:horn-coCart}
    \textcolor{blue}{\varphi_0}
    &= \begin{tikzpicture}[baseline={([yshift=-.5ex]current bounding box.center)}]
        \path coordinate[bullet,label=below left:$x$] (v0)
        +(2,0) coordinate[bullet,blue] (v2)
        +(60:2) coordinate[bullet,label=above:$y$] (v1)
        ;
        \draw[->-=.5] (v0) --node[midway,above left]{$f$} (v1);
        \draw[->-=.5,blue] (v0) --node[midway,below,blue]{$\varphi_0{}_2$} (v2);
        % \draw[->-=.5] (v1) -- (v2);
    \end{tikzpicture}
\end{align}
となる.従って,\hyperref[def:Cart-coCart]{\textsf{\textbf{(coCart-2)}}}の主張は次のような意味を持つ:

$(\infty,\, 1)$-圏 \underline{$\Cat{B}$ において}角の図式が2-射 $\textcolor{blue}{\varphi} \in \Hom{\sSet}(\Delta^2,\, \Cat{B}) \cong \Cat{B}_2$ によって
\begin{center}
    \begin{tikzpicture} 
        \path coordinate[bullet,label=below left:$p_{[0]}(x)$] (v0)
        +(2,0) coordinate[bullet,blue] (v2)
        +(60:2) coordinate[bullet,label=above:$p_{[0]}(y)$] (v1)
        ;
        \draw[->-=.5] (v0) --node[midway,above left]{$p_{[1]}(f)$} (v1);
        \draw[->-=.5,blue] (v0) --node[midway,below,blue]{$p_{[1]}(\varphi_0{}_2)$} (v2);
        \draw[->-=.5,blue] (v1) --node[midway,above right,blue]{$\partial^2_0(\varphi)$} (v2);

        \node[blue] at (barycentric cs:v0=1,v1=1,v2=1) {$\varphi$};

        \begin{scope}[on background layer]
            \fill[blue!20] (v0.center) -- (v1.center) -- (v2.center) --cycle;
        \end{scope}
        % \draw[->-=.5] (v1) -- (v2);
    \end{tikzpicture}   
\end{center}
と埋められているならば,\underline{$\Cat{E}$ において}角の図式\eqref{eq:horn-coCart}を
\begin{center}
    \begin{tikzpicture} 
        \path coordinate[bullet,label=below left:$x$] (v0)
        +(2,0) coordinate[bullet,blue] (v2)
        +(60:2) coordinate[bullet,label=above:$y$] (v1)
        ;
        \draw[->-=.5] (v0) --node[midway,above left]{$f$} (v1);
        \draw[->-=.5,blue] (v0) --node[midway,below,blue]{$\varphi_0{}_2$} (v2);
        \draw[->-=.5,red] (v1) --node[midway,above right,red]{$\partial^2_0(\bar{\varphi})$} (v2);
        \node[red] at (barycentric cs:v0=1,v1=1,v2=1) {$\bar{\varphi}$};

        \begin{scope}[on background layer]
            \fill[red!20] (v0.center) -- (v1.center) -- (v2.center) --cycle;
        \end{scope}
        % \draw[->-=.5] (v1) -- (v2);
    \end{tikzpicture}   
\end{center}
のように埋める2-射 $\textcolor{red}{\bar{\varphi}} \in \Hom{\sSet}(\Delta^2,\, \Cat{E}) \cong \Cat{E}_2$ が存在する.

\begin{mydef}[label=def:Cart-coCart,breakable]{デカルトファイブレーション}
    $p \colon \Cat{E} \lto \Cat{B}$ を\hyperref[def:infty-fib]{内的ファイブレーション}とする.
    \begin{itemize}
        \item $p$ が\textbf{デカルトファイブレーション} (Cartesian fibration) であるとは,
        \begin{itemize}
            \item $\Cat{B}$ の任意の1射 $(x \xrightarrow{f} y) \in \Cat{B}_1$ 
            \item $p_{[0]}(\bar{y}) = y$ を充たす $\Cat{E}$ の任意の対象 $\bar{y} \in \Cat{E}_0$
        \end{itemize}
        に対して,以下の条件を充たす $\Cat{E}$ の1-射 $(z \xrightarrow{\bar{f}} \bar{y}) \in \Cat{E}_1$ が存在することを言う:
        \begin{description}
            \item[\textbf{(Cart-1)}] 
            
            $\bar{f}$ は $f$ の持ち上げである.i.e. $p_{[1]}(\bar{f}) = f$ が成り立つ.

            \item[\textbf{(Cart-2)}] 
            
            $\bar{f}$ は\hyperref[def:p-Cart]{$p$-Cartesian}である.
        \end{description}
        \item $p$ が\textbf{余デカルトファイブレーション} (coCartesian fibration) であるとは,
        \begin{itemize}
            \item $\Cat{B}$ の任意の1射 $(x \xrightarrow{f} y) \in \Cat{B}_1$ 
            \item $p_{[0]}(\bar{x}) = x$ を充たす $\Cat{E}$ の任意の対象 $\bar{x} \in \Cat{E}_0$
        \end{itemize}
        に対して,以下の条件を充たす $\Cat{E}$ の1-射 $(\bar{x} \xrightarrow{\bar{f}} z) \in \Cat{E}_1$ が存在することを言う:
        \begin{description}
            \item[\textbf{(coCart-1)}] 
            
            $\bar{f}$ は $f$ の持ち上げである.i.e. $p_{[1]}(\bar{f}) = f$ が成り立つ.

            \item[\textbf{(coCart-2)}] 
            
            $\bar{f}$ は\hyperref[def:p-Cart]{$p$-coCartesian}である.
        \end{description}
    \end{itemize}
\end{mydef}

\begin{mydef}[label=def:infty-operad,breakable]{{$(\infty,\, 1)$}-オペラッド}
    $\bm{(\infty,\, 1)}$\textbf{-オペラッド} ($(\infty,\, 1)$-operad)\footnote{$\infty$-operadとも呼ばれる.} とは,
    \hyperref[def:infinity-1]{$(\infty,\, 1)$-圏の関手}
    \begin{align}
        p \colon \Cat{O}^{\otimes} \lto \Ner (\FIN_*)
    \end{align}
    であって以下の条件を充たすもののこと~\cite[Definition 2.1.1.10.]{Lurie2017HA}:
    \begin{description}
        \item[\textbf{(Op-1)}] 
        
        任意の\hyperref[def:inert-active]{inertな}射 $f \in \Hom{\FIN_*} \bigl( \expval{m},\, \expval{n} \bigr)$ および $\forall c \in (\Cat{O}^\otimes_{\expval{m}})_0$ に対して,
        $\Cat{O}^\otimes$ における\hyperref[def:Cart-coCart]{$p$-coCartesian}な1-射 $\bar{f} \colon c \lto c'$ が存在して $p_{[1]}(\bar{f}) = f$ を充たす.
        
        \item[\textbf{(Op-2)}] 
        
        $\forall f \in \Hom{\FIN_*} \bigl( \expval{m},\, \expval{n} \bigr)$ および $\forall c \in (\Cat{O}^\otimes_{\expval{m}})_0,\; \forall c' \in (\Cat{O}^\otimes_{\expval{n}})_0$ に対して,
        \exref{def:rho-i}のinertな射の族 $\Familyset[\big]{\rho^i \in  \Hom{\FIN_*}\bigl( \expval{n},\, \expval{1} \bigr)}{1 \le i \le n}$ に \textsf{\textbf{(Op-1)}} を適用して得られる\hyperref[def:Cart-coCart]{$p$-coCartesian}な1-射の族 $\Familyset[\big]{c' \xrightarrow{\bar{\rho}^i} c'_i \in (\Cat{O}^\otimes)_1}{1 \le i \le n}$ が誘導する\hyperref[def:infinity-1]{Kan複体の関手}
        \footnote{対象の対応としては $\bar{f} \lmto (\bar{\rho}^1 \circ \bar{f},\, \dots,\, \bar{\rho}^n \circ \bar{f})$ であるが,右辺の\hyperref[def:infinity-1]{$(\infty,\, 1)$-圏} $\Cat{O}^\otimes$ における1-射の合成はup to homotopyでしか定まらないため,この関手はup to homotopyでしか決まらない.}
        \begin{align}
            \inftyMap{\Cat{O}^\otimes} (c,\, c')_f \lto \prod_{i=1}^n \inftyMap{\Cat{O}^\otimes} (c,\, c'_i)_{\rho^i \circ f}
        \end{align}
        は,$(\infty,\, 1)$-圏\hyperref[def:Spaces]{$\SPACES$}における\hyperref[def:isom-infty]{同型射}である.

        \item[\textbf{(Op-3)}] 
        
        $\forall c_1,\, \dots,\, c_n \in (\Cat{O}^\otimes_{\expval{1}})_0$ に対して,
        ある $c \in (\Cat{O}^\otimes_{\expval{n}})_0$ および \hyperref[def:Cart-coCart]{$p$-coCartesian}な1-射 $\widehat{\rho_i} \in \bigl(\inftyMap{\Cat{O}^\otimes} (c,\, c_i)_{\rho^i}\bigr)_0$ が存在する.
    \end{description}

    ただし,以下の記法を採用した:
    \begin{itemize}
        \item 点 $\expval{n} \in \Ner (\FIN_*)_0$ における\hyperref[def:infinity-1]{$(\infty,\, 1)$-圏の関手} $p \colon \Cat{O}^\otimes \lto \Ner (\FIN_*)$ の\textbf{ファイバー}\footnote{これ自体が $(\infty,\, 1)$-圏である.} $\Cat{O}^{\otimes}_{\expval{n}}$ を,\hyperref[def:pullback-infty]{$(\infty,\, 1)$-圏のpullback}
        \begin{center}
            % https://tikzcd.yichuanshen.de/#N4Igdg9gJgpgziAXAbVABwnAlgFyxMJZABgBpiBdUkANwEMAbAVxiRAB12BhOnYAeQC+APWCcIeALbxBAfTHsYADzT0GwMIMEhBpdJlz5CKMgEYqtRizacefIcPFT4OvSAzY8BIqdLnqataIHOySvAAWAE6SwAByggAUnABiAJKxsgBUAJSu+p5GPuQWgazBnAAiMAw4dMLEOhYwUADmLiigAGaREJJIZCA4EEgAzLpdPX2IA0NIpuMg3b2j1LOIAEyrdFgMbOEQEADWIAFWZSHKqowa2gtLU76DwxunzOdoJyAMdABG1QAKBi8xhAkSwLXCOEagiAA
            \begin{tikzcd}
            \textcolor{red}{\Cat{O}^{\otimes}_{\expval{n}}} \arrow[red, r] \arrow[red,d] & \Delta^0 \arrow[d, "\expval{n}", hook] \\
            \Cat{O}^\otimes \arrow[r, "p"']                    & \Ner (\FIN_*)                    
            \end{tikzcd}
        \end{center}
        と定義した.
        \item $f \in \Hom{\FIN_*} \bigl( \expval{m},\, \expval{n} \bigr)$ および $c \in (\Cat{O}^\otimes_{\expval{m}})_0,\; c' \in (\Cat{O}^\otimes_{\expval{n}})_0$ に対して,
        $\bar{f} \in \inftyMap{\Cat{O}^\otimes}(c,\, c')_{0}$ であって $p_{[1]}(\bar{f}) = f$ を充たすもの全体が生成する,$\inftyMap{\Cat{O}^\otimes} (c,\, c') \in \SPACES_0$
        の
        % \footnote{\hyperref[def:infty-1]{Kan複体} $\Hom{\Cat{O}^\otimes} (c,\, c') \in \SPACES_0$ は,\hyperref[def:infty-Hom]{$(\infty,\, 1)$-圏におけるHom関手} $\inftyMap{\Cat{O}^\otimes} \colon \OP{(\Cat{O}^\otimes)} \times \Cat{O}^\otimes \lto \SPACES$ による対象 $(c,\, c') \in (\OP{(\Cat{O}^\otimes)} \times \Cat{O}^\otimes)_0$ の像である.}
        \hyperref[def:fullsub-infty]{充満部分Kan複体}を
        \begin{align}
            \bm{\inftyMap{\Cat{O}^\otimes} (c,\, c')_f} \hookrightarrow \inftyMap{\Cat{O}^\otimes} (c,\, c')
        \end{align}
        と書いた.
    \end{itemize}
\end{mydef}

\begin{mylem}[label=lem:Segal-cond]{Segal条件}
    \hyperref[def:infinity-1]{$(\infty,\, 1)$-圏の関手}
    \begin{align}
        p \colon \Cat{O}^{\otimes} \lto \Ner (\FIN_*)
    \end{align}
    であって条件\hyperref[def:infty-operad]{\textsf{\textbf{(Op-1)}}, \textsf{\textbf{(Op-2)}}}を充たすものを与える.
    このとき,条件\hyperref[def:infty-operad]{\textsf{\textbf{(Op-3)}}}は以下と同値である:
    \begin{description}
        \item[\textbf{(Segal)}] 
        
        $\forall n \ge 0$ に対して,\exref{def:rho-i}の\hyperref[def:inert-active]{inertな射}の族 $\Familyset[\big]{\rho^i \in \Hom{\FIN_*} \bigl( \expval{n},\, \expval{1} \bigr)}{1 \le i \le n}$ が条件\hyperref[def:infty-operad]{\textsf{\textbf{(Op-1)}}}により誘導する\hyperref[def:infinity-1]{$(\infty,\, 1)$-圏の関手}の族 $\Familyset[\big]{\rho^i_{!} \colon \Cat{O}^\otimes_{\expval{n}} \lto \Cat{O}^\otimes_{\expval{1}}}{1 \le i \le n}$ は,\hyperref[def:equiv-infty]{$(\infty,\, 1)$-圏同値}
        \begin{align}
            \label{eq:Segalmap}
            (\rho^1_{!},\, \dots,\, \rho^n_{!}) \colon \Cat{O}^\otimes_{\expval{n}} \xrightarrow{\simeq} \bigl( \Cat{O}^\otimes_{\expval{1}} \bigr)^{\times n}
        \end{align}
        を与える.
    \end{description}
\end{mylem}

\begin{proof}
    \begin{description}
        \item[\textbf{(OP-3)} $\Longrightarrow$ \textbf{(Segal)}] 
        
         \textsf{\textbf{(OP-3)}}を仮定する.命題\ref{prop:equiv-FF-ES}より,$(\infty,\, 1)$-圏の関手\eqref{eq:Segalmap}が\hyperref[def:FF-ES]{忠実充満かつ本質的全射}であることを示せば良い.

         $\forall n \ge 0$ および $\forall c,\, c' \in (\Cat{O}^\otimes_{\expval{n}})_0$ を固定する.このとき $\Id_{\expval{n}} \in \Hom{\FIN_*} \bigl( \expval{n},\, \expval{n} \bigr)$ に対して\textsf{\textbf{(OP-2)}}を用いることで,
        ホモトピー同値
        \begin{align}
            \inftyMap{\Cat{O}^\otimes_{\expval{n}}}(c,\, c') \simeq \inftyMap{\Cat{O}^\otimes} (c,\, c')_{\Id_{\expval{n}}} \lto \prod_{i=1}^n \inftyMap{\Cat{O}^\otimes} \bigl(c,\, c'_i\bigr)_{\rho^i}
        \end{align}
        が得られる.さらに,\hyperref[def:Cart-coCart]{$p$-coCartesianな射の定義}からホモトピー同値
        \begin{align}
            \prod_{i=1}^n \inftyMap{\Cat{O}^\otimes_{\expval{1}}} \bigl(\rho^i_!{}_{[0]}(c),\, \rho^i_!{}_{[0]}(c')\bigr) \simeq \prod_{i=1}^n \inftyMap{\Cat{O}^\otimes} \bigl(c_i,\, c'_i\bigr)_{\Id_{\expval{1}}} \lto \prod_{i=1}^n \inftyMap{\Cat{O}^\otimes} \bigl(c,\, c'_i\bigr)_{\rho^i}
        \end{align}
        が得られる\footnote{\textsf{\textbf{(OP-2)}}における $c'_i \in (\Cat{O}^\otimes_{\expval{1}})_0$ とは,ちょうど $\rho^i_!{}_{[0]}(c')$ のことである.}
        ため,\eqref{eq:Segalmap}が\hyperref[def:FF-ES-infty]{忠実充満}だと分かった.
        \hyperref[def:FF-ES-infty]{本質的全射}であることは\textsf{\textbf{(OP-3)}}より従う.

        \item[\textbf{(OP-3)} $\Longleftarrow$ \textbf{(Segal)}] 
        明らか.
    \end{description}
\end{proof}


\begin{mydef}[label=def:infty-operad-map]{{$(\infty,\, 1)$}-オペラッドの射}
    2つの\hyperref[def:infty-operad]{$(\infty,\, 1)$-オペラッド} $p \colon \Cat{O}^{\otimes} \lto \Ner (\FIN_*)$ を与える.
    このとき,$(\infty,\, 1)$-圏 $\Cat{O}^\otimes$ の1-射 $f \in (\Cat{O}^\otimes)_1$ が\textbf{inert}であるとは,以下の2条件を充たすことを言う:
    \begin{description}
        \item[\textbf{(inert-1)}] $p_{[1]}(f)$ は $(1,\, 1)$-圏 $\FIN_*$ における\hyperref[def:inert-active]{innerな射}である.
        \item[\textbf{(inert-2)}] $f$ は\hyperref[def:Cart-coCart]{$p$-coCartesian}な1-射である.
    \end{description}
    
    \tcblower
    
    2つの\hyperref[def:infty-operad]{$(\infty,\, 1)$-オペラッド} $p \colon \Cat{O}^{\otimes} \lto \Ner (\FIN_*),\; q \colon \Cat{O}'{}^\otimes \lto \Ner (\FIN_*)$ を与える.
    このとき,\hyperref[def:infinity-1]{$(\infty,\, 1)$-圏の関手} $f \colon \Cat{O}^{\otimes} \lto \Cat{O}'{}^\otimes$ が\textbf{$\bm{(\infty,\, 1)}$-オペラッドの射} ($\infty$-operad map) であるとは,以下の2つの条件を充たすことを言う:
    \begin{description}
        \item[\textbf{(Opmap-1)}] $(1,\, 1)$-圏 $\sSet$ の図式
        \begin{center}
            % https://tikzcd.yichuanshen.de/#N4Igdg9gJgpgziAXAbVABwnAlgFyxMJZABgBpiBdUkANwEMAbAVxiRAB12BhOnYAeQC+APU4Q8AW3ghBpdJlz5CKAEzkqtRizacefIQHJgIsZOmz52PASIBGUrY31mrRB3YA5GACcABAApOADEASQ8AfQAqAEoZDRgoAHNpFFAAM28ICSQyEBwIJHtNFzY0mTkQDKyc6nykNWLtNzQQagY6ACMYBgAFBWtlEG8sRIALHHL0zOzEIrrEBucmkABHOMEgA
            \begin{tikzcd}
            \Cat{O}^\otimes \arrow[rr, "f"] \arrow[rd, "p"'] &               & \Cat{O}'{}^\otimes \arrow[ld, "q"] \\
                                                             & \Ner (\FIN_*) &                                   
            \end{tikzcd}
        \end{center}
        は可換である.

        \item[\textbf{(Opmap-2)}] 1-射の間の写像 $f_{[1]} \colon (\Cat{O}^{\otimes})_1 \lto (\Cat{O}'{}^{\otimes})_1$ により,innertな1-射が保存される\footnote{条件 \textsf{\textbf{(Opmap-1)}} より,inertな1-射の $p_{[1]}$ による像が条件 \textsf{\textbf{(inert-1)}} を充たすことは明らかである.}.
    \end{description}
\end{mydef}

定義\ref{def:infty-operad}がオペラッドと呼ぶにふさわしいことを示すために,次の小節では $(1,\, 1)$-圏の文脈で対応物を考えよう.

\subsection{色付きオペラッドと {$(1,\, 1)$}-圏のcoCartesian fibration}

\begin{mydef}[label=def:c-operad,breakable]{colored operad}
    \textbf{色付きオペラッド}\footnote{いわゆる\textbf{対称色付きオペラッド} (symmetric colored operad) である.} (colored operad) $\Cat{O}$ は,以下の4つのデータから成る:
    \begin{itemize}
        \item \textbf{対象}\footnote{\textbf{色} (color) と呼ばれることもある.} (object) の集まり
        \begin{align}
            \bm{\Obj{\Cat{O}}}
        \end{align}
        \item $\forall I \in \Obj{\FIN},\; \forall \Familyset{x_i \in \Obj{\Cat{O}}}{i \in I},\;  \forall y \in \Obj{\Cat{O}}$ の3つ組に対して定まっている,
        $\Familyset{x_i}{i \in I}$ から $y$ へ向かう\textbf{多射} (multimorphism) の集合
        \begin{align}
            \bm{\Mul{\Cat{O}}} \bigl( \Familyset{x_i}{i \in I},\, y \bigr) \in \Obj{\SETS}
        \end{align}
        \item $\forall \alpha \in \Hom{\FIN} (I,\, J),\; \forall \Familyset[\big]{x_i \in \Obj{\Cat{O}}}{i \in I},\; \Familyset[\big]{y_j \in \Obj{\Cat{O}}}{j \in J},\; \forall z \in \Obj{\Cat{O}}$ の4つ組に対して定まっている,
        多射の\textbf{合成} (composition map) と呼ばれる写像
        \begin{align}
            \bm{\circ}_{\bm{\alpha}} \colon \Mul{\Cat{O}} \bigl( \Familyset{y_j}{j \in J},\, z \bigr) \times \prod_{j \in J} \Mul{\Cat{O}} \bigl( \Familyset{x_i}{i \in \alpha^{-1}(\{j\})},\, y_j \bigr) &\lto \Mul{\Cat{O}} \bigl( \Familyset{x_i}{i \in I},\, z \bigr), \\
            \bigl( G,\, (F_j)_{j \in J} \bigr) &\lmto G \circ_{\alpha} (F_j)_{j \in J}
        \end{align}
        \item \textbf{恒等射} (identitiy) と呼ばれる多射の族 $\Familyset[\Big]{\Id_x \in \Mul{\Cat{O}} \bigl(\{x\},\, x \bigr)}{x \in \Obj{\Cat{O}}}$
    \end{itemize}
    これらは以下の条件を充たさねばならない:
    \begin{description}
        \item[\textbf{(cOp-1)}] 
        
        恒等射は合成に関して単位元として振る舞う.
        
        \item[\textbf{(cOp-2)}] 
        
        多射の合成は結合則を充たす.i.e.
        $\forall \alpha \in \Hom{\FIN}(I,\, J),\; \forall \beta \in \Hom{\FIN}(J,\, K),\; \forall \Familyset[\big]{x_i \in \Obj{\Cat{O}}}{i \in I},\; \forall \Familyset[\big]{y_j \in \Obj{\Cat{O}}}{j \in J},\; \forall \Familyset[\big]{z_k \in \Obj{\Cat{O}}}{k \in K},\; \forall w \in \Obj{\Cat{O}}$ に対して,
        $(1,\, 1)$-圏 $\SETS$ の図式
        \small
        \begin{center}
            % https://tikzcd.yichuanshen.de/#N4Igdg9gJgpgziAXAbVABwnAlgFyxMJZARgBoAGAXVJADcBDAGwFcYkQAdDgWWceC4BhejmAB5AL4SuAIywBzRgAouAMXoBbLIwCecGKIBeAfQDWE4Ka5YwAAgDSE0l1K2A7rIUAnAJS2ueBrw-hxoXtDGliE2DhIhvPxCIuJSIXKKKhzqWrr6ojrGAFYWhdF2sgb0AHrAALTEEpmWXBI+Ti62JlYc6b4hgcFcYRHApdZ2AFJxXAkCHMKiktM9Cspqmtp6BsAAHsZYFlhlIUxoABbVdQ1NhS1tzhyuBbcr8r4gTuiYuPiEKABMpGI1DoTFY7BmfDmCxS0leayyG1y2y6Fm6MUcD1cHlefQCWCCcBCwygkVMx0c8ShSUWqU8GXWOS2oj2B2AR3GtkyMkqIQAxlgvHyToxzvQfDV6o0uM0OK12o9OmZ6e9PiAMNg8AQiOQgSCGCw2IhODxqfNkkt6Qjsps8sBniVOVMse4VX58YTieFSaNjlMqYlzbTlulrUjmbt9odjlxThdJdcZS95S7nm6PqQvprfkQyP99WCjSbZjTYVbMjbkSyo+zOQBJBWuQzpiQgmBQeTwIigABm4Q0SF1IBwECQgNBhohHAFQsiFRw9GWAyJXDrUBANEY9B5jAACt8tX8QF4FGccBne-3BzQR0gyBPwcbV1B+gTgsBMjO+XOOHHFz4yQpKQNxALcd33bNtWNRgYB7c81T7CAB0Qcdb0QABmGgDUfE0vx-P8JBAsCYD3A8c2NE95DPC8QEQ5D7zQzCHyLLg8LmHkF35QVhVjUULkI1sJCAA
            \begin{tikzcd}[column sep={10em,between origins},row sep=large]
                                                                                                                                                                                                    & {\Mul{\Cat{O}}\bigl(\Familyset{z_k}{k\in K},\, w\bigr) \times \prod_{k \in K} \Mul{\Cat{O}} \bigl(\Familyset{y_j}{j \in \beta^{-1}(\{k\})},\, z_k\bigr) \times \prod_{j \in J} \Mul{\Cat{O}} \bigl(\Familyset{x_i}{i \in \alpha^{-1}(\{j\})},\, y_j\bigr)} \arrow[ld, "\circ_{\beta} \times \Id"'] \arrow[ddr, "\Id \times {\bigl(\circ_{\alpha|_{\alpha^{-1}(\beta^{-1}(\{k\}))}}\bigr)_{k \in K}}"] &                                                                                                                                                                                                                  \\
            {\Mul{\Cat{O}}\bigl(\Familyset{y_j}{j\in J},\, w\bigr) \times \prod_{j \in J} \Mul{\Cat{O}} \bigl(\Familyset{x_i}{i \in \alpha^{-1}(\{j\})},\, y_j\bigr)} \arrow[ddr, "\circ_{\alpha}"'] &                                                                                                                                                                                                                                                                                                                                                          &                                                                                                                                                                                                                  \\
                                                                                                                                                                                                    &                                                                                                                                                                                                                                                                                                                                                          &{\Mul{\Cat{O}}\bigl(\Familyset{z_k}{k\in K},\, w\bigr) \times \prod_{k \in K} \Mul{\Cat{O}} \bigl(\Familyset{x_i}{i \in (\beta \circ \alpha)^{-1}(\{k\})},\, z_k\bigr)} \arrow[ld, "\circ_{\beta \circ \alpha}"]  \\
                                                                                                                                                                                                    & {\Mul{\Cat{O}}\bigl(\Familyset{x_i}{i\in I},\, z\bigr)}                                                                                                                                                                                                                                                                                                  &                                                                                                                                                                                                                 
            \end{tikzcd}
        \end{center}
        \normalsize
        は可換である.
    \end{description}
\end{mydef}

\hyperref[def:c-operad]{色付きオペラッド}の定義において,写像 $\alpha \in \Hom{\FIN} (I,\, J)$ が多射の合成の「型」を規定している.


\begin{mydef}[label=def:p-coCart-ord,breakable]{{$(1,\, 1)$}-圏におけるcoCartesian fibration}
    $p \colon \Cat{E} \lto \Cat{B}$ を $(1,\, 1)$-圏の関手とする.
    \begin{itemize}
        \item $\Cat{E}$ の射 $\bar{f} \in \Hom{\Cat{E}} (\bar{x},\, \bar{y})$ が以下の条件を充たすとき,$\bar{f}$ は\textbf{$\bm{p}$-coCartesian}であると言う:
        \begin{description}
            \item[\textbf{(coCart-ord)}] 
            $(1,\, 1)$-圏 $\Cat{B}$ の図式
            \begin{center}
                % https://tikzcd.yichuanshen.de/#N4Igdg9gJgpgziAXAbVABwnAlgFyxMJZABgBoBGAXVJADcBDAGwFcYkQ0AKAHW4CN6AJ2AAPAL4BKEGNLpMufIRTlSxanSat2XXgOEBPSdNkd5eAkQBMFdQxZtEHHvyHAAXkbHqYUAObwiUAAzQQgAWyQyEBwIJBUNe21nPWAgzxMQ8MiaGKRrBK1HHW4giEEmRgACXgZBNAALLAB9YikaRno+GEYABTNFdkEsX3qcY2DQiMR43MR8u0KQXlLyxiqaoQasaUoxIA
                \begin{tikzcd}
                                                                                        & p(\bar{y}) \arrow[rd, blue, "\forall \varphi"] &            \\
                p(\bar{x}) \arrow[ru, "p(\bar{f})"] \arrow[rr, blue, "p(\forall \varphi_0)"'] &                                          & p(\forall \textcolor{blue}{\bar{z}})
                \end{tikzcd}
            \end{center}
            を可換にする勝手な2つの射 $\textcolor{blue}{\varphi_0} \in \Hom{\Cat{E}} (\bar{x},\, \textcolor{blue}{\bar{z}}),\; \textcolor{blue}{\varphi} \in \Hom{\Cat{B}} \bigl( p(\bar{y}),\, p(\textcolor{blue}{\bar{z}}) \bigr)$ に対して,
            射 $\textcolor{red}{\bar{\varphi}} \in \Hom{\Cat{E}} (\bar{y},\, \textcolor{blue}{\bar{z}})$ が\underline{一意的に}存在して $(1,\, 1)$-圏 $\Cat{E}$ の図式
            \begin{center}
                % https://tikzcd.yichuanshen.de/#N4Igdg9gJgpgziAXAbVABwnAlgFyxMJZABgBoBGAXVJADcBDAGwFcYkQAdDgI3oCdgADwC+IYaXSZc+QinKli1Ok1bsuvAQE9R4ydjwEiAJgpKGLNok49+wAF46lMKAHN4RUADM+EALZIyEBwIJHllCzUbAU8dCRBvPwCaYKQTcNUrLgY+NAALLAB9YhAaRnpuGEYABSkDWRA+LBdcnDE4hP9EMJTENPMM6w1gLP48rEdhIA
                \begin{tikzcd}
                                                                       & \bar{y} \arrow[rd, red,"\exists ! \bar{\varphi}"] &         \\
                \bar{x} \arrow[ru, "\bar{f}"] \arrow[rr, blue, "\varphi_0"'] &                                     & \textcolor{blue}{\bar{z}}
                \end{tikzcd}
            \end{center}
            を可換にする.
        \end{description}
        \item $p$ が\textbf{coCartesian fibration}であるとは,
        \begin{itemize}
            \item $(1,\, 1)$-圏 $\Cat{B}$ の任意の射 $f \in \Hom{\Cat{B}} (x,\, y)$
            \item $p(\bar{x}) = x$ を充たす $(1,\, 1)$-圏 $\Cat{E}$ の対象 $\bar{x} \in \Obj{\Cat{E}}$
        \end{itemize}
        に対して,以下の条件を充たす $(1,\, 1)$-圏 $\Cat{E}$ の射 $\bar{f} \in \Hom{\Cat{E}} (\bar{x},\, z)$ が存在することを言う:
        \begin{description}
            \item[\textbf{(coCart-ord-1)}] 
            
            $\bar{f}$ は $f$ の持ち上げである.i.e. $p(\bar{f}) = f$ が成り立つ.

            \item[\textbf{(coCart-ord-2)}] 
            
            $\bar{f}$ は $p$-coCartesianである.
        \end{description}
        % \item $p$ が\textbf{isofibration}であるとは,
        % \begin{itemize}
        %     \item $(1,\, 1)$-圏 $\Cat{B}$ の任意の\underline{同型射} $f \in \Hom{\Cat{B}} (x,\, y)$
        %     \item $p(\bar{x}) = x$ を充たす $(1,\, 1)$-圏 $\Cat{E}$ の対象 $\bar{x} \in \Obj{\Cat{E}}$
        % \end{itemize}
        % に対して,条件 \textsf{\textbf{(coCart-ord-1)}}, \textsf{\textbf{(cocCart-ord-2)}} を充たす $(1,\, 1)$-圏 $\Cat{E}$ の射 $\bar{f} \in \Hom{\Cat{E}} (\bar{x},\, z)$ が存在することを言う.
    \end{itemize}
\end{mydef}

定義\ref{def:infty-operad}に合わせて,$(1,\, 1)$-圏の関手
\begin{align}
    \label{eq:operad-ord}
    p \colon \Cat{O}^\otimes \lto \FIN_*
\end{align}
であって以下の3条件を充たすものを考えてみる:
\begin{description}
    \item[\textbf{(OP-ord-1)}] 
    \label{cond:Op-ord-1}

    任意の\hyperref[def:inert-active]{inertな}射 $f \in \Hom{\FIN_*} \bigl( \expval{m},\, \expval{n} \bigr)$ および $\forall c \in \Obj{\Cat{O^\otimes}_{\expval{m}}}$ に対して,
    $\Cat{O}^\otimes$ における\hyperref[def:p-coCart-ord]{$p$-coCartesian}な射 $\bar{f} \in \Hom{\Cat{O}^\otimes} (c,\, c')$ が存在して $p(\bar{f}) = f$ を充たす.
        
    \item[\textbf{(OP-ord-2)}]  
    \label{cond:Op-ord-2}
            
    $\forall f \in \Hom{\FIN_*} \bigl( \expval{m},\, \expval{n} \bigr)$ および $\forall c \in \Obj{\Cat{O}^\otimes_{\expval{m}}},\, \forall c' \in \Obj{\Cat{O}^\otimes_{\expval{n}}}$ に対して,
    inertな射 $\rho^i \in  \Hom{\FIN_*}\bigl( \expval{n},\, \expval{1} \bigr)$ に \hyperref[cond:Op-ord-1]{\textsf{\textbf{(Op-ord-1)}}} を適用して得られる\hyperref[def:p-coCart-ord]{$p$-coCartesian}な射の族 $\Familyset[\big]{\bar{\rho}^i \in \Hom{\Cat{O}^\otimes} (c',\, c'_i)}{1 \le i \le n}$ が誘導する写像
    \begin{align}
        \Hom{\Cat{O}^\otimes} (c,\, c')_f &\lto \prod_{i=1}^n \Hom{\Cat{O}^\otimes} (c,\, c'_i)_{\rho^i \circ f}, \\
        \varphi &\lmto (\bar{\rho}^1 \circ \varphi,\, \dots,\, \bar{\rho}^n \circ \varphi)
    \end{align}
    は,$(1,\, 1)$-圏 $\SETS$ における\hyperref[def:iso]{同型射}(i.e. 全単射)である.

    \item[\textbf{(OP-ord-3)}]  
    \label{cond:Op-ord-3}

    $\forall c_1,\, \dots,\, c_n \in \Obj{\Cat{O}^\otimes_{\expval{1}}}$ に対して,
    ある $c \in \Obj{\Cat{O}^\otimes_{\expval{n}}}_0$ および \hyperref[def:Cart-coCart]{$p$-coCartesian}な射 $\widehat{\rho_i} \in \Hom{\Cat{O}^\otimes} (c,\, c_i)_{\rho^i}$ が存在する.
\end{description}
ここで,$\CAT$ における\hyperref[def:pullback-pushout]{引き戻し}
\begin{center}
    % https://tikzcd.yichuanshen.de/#N4Igdg9gJgpgziAXAbVABwnAlgFyxMJZABgBpiBdUkANwEMAbAVxiRAB12BhOnYAeQC+APWCcIeALbxBAfTHsYADzT0GwMIMEhBpdJlz5CKMgEYqtRizacefIcPFT4OvSAzY8BIqdLnqataIHOySvAAWAE6SwAByggAUnABiAJKxsgBUAJSu+p5GPuQWgazBnAAiMAw4dMLEOhYwUADmLiigAGaREJJIZCA4EEgAzLpdPX2IA0NIpuMg3b2j1LOIAEyrdFgMbOEQEADWIAFWZSHKqowa2gtLU76DwxunzOdoJyAMdABG1QAKBi8xhAkSwLXCOEagiAA
    \begin{tikzcd}
    \textcolor{red}{\Cat{O}^{\otimes}_{\expval{n}}} \arrow[red, r] \arrow[red,d] &* \arrow[d, "\expval{n}", hook] \\
    \Cat{O}^\otimes \arrow[r, "p"']                    & \FIN_*
    \end{tikzcd}
\end{center}
により $(1,\, 1)$-圏 $\Cat{O}^\otimes$ を定義している.具体的には
\begin{align}
    \Obj{\Cat{O}^\otimes} &= \bigl\{\, c \in \Obj{\Cat{O}^\otimes} \bigm| p(c) = \expval{n} \,\bigr\}
\end{align}
である.さらに,$\forall f \in \Hom{\FIN_*} \bigl( \expval{m},\, \expval{n} \bigr)$ に対して
\begin{align}
    \Hom{\Cat{O}^\otimes} (c,\, c')_f \coloneqq \bigl\{\, \varphi \in \Hom{\Cat{O}^\otimes} (c,\, c') \bigm| p(\varphi) = f \,\bigr\} 
\end{align}
と定義した.

\begin{myprop}[label=prop:operad-ord,breakable]{色付きオペラッドの再構成}
    条件\hyperref[cond:Op-ord-1]{\textsf{\textbf{(Op-ord-1)}}}-\hyperref[cond:Op-ord-1]{\textsf{\textbf{(Op-ord-3)}}}を充たす $(1,\, 1)$-圏の関手\eqref{eq:operad-ord}から次のように構成されたデータの組み $\Cat{O}$ は\hyperref[def:c-operad]{色付きオペラッド}を成す:
    \begin{itemize}
        \item 対象の集まりを $\Obj{\Cat{O}} \coloneqq \Obj{\Cat{O}^\otimes_{\expval{1}}}$ と定義する.
        \item $\forall x_1,\, \dots,\, x_n,\, y \in \Obj{\Cat{O}}$ に対して,以下の3つ組全体が成す集合を $\Mul{\Cat{O}} \bigl( (x_1,\, \dots,\, x_n),\, y \bigr)$ と定義する.
        \begin{enumerate}
            \item $x_1,\, \dots,\, x_n$ に対して\hyperref[cond:Op-ord-1]{\textsf{\textbf{(Op-ord-3)}}}を適用することにより定まる $X \in \Obj{\Cat{O}^\otimes_{\expval{n}}}$ 
            \item $x_1,\, \dots,\, x_n$ に対して\hyperref[cond:Op-ord-1]{\textsf{\textbf{(Op-ord-3)}}}を適用することにより定まる\hyperref[def:Cart-coCart]{$p$-coCartesian}な射の族 $\Familyset[\big]{\widehat{\rho_i} \in \Hom{\Cat{O}^\otimes}(X,\, x_i)_{\rho^i}}{1 \le i \le n}$
            \item \exref{def:alpha-n}の\hyperref[def:inert-active]{active}な射
            $\alpha_n \in \Hom{\FIN_*} \bigl( \expval{n},\, \expval{1} \bigr)$ およびその上の $\Cat{O}^\otimes$ の射 $F \in \Hom{\Cat{O}^\otimes} (X,\, y)_{\alpha_n}$
        \end{enumerate}
        \item $n$ 個の多射
        \begin{align}
            (X^1,\, \Familyset{\widehat{\rho_i}^1}{1 \le i \le m_1},\, F^1) &\in \Mul{\Cat{O}} \bigl( (x^1_1,\, \dots,\, x^1_{m_1}),\, y_1 \bigr),\\ 
            &\vdots \\ 
            (X^n,\, \Familyset{\widehat{\rho_i}^n}{1 \le i \le m_n},\, F^n) &\in \Mul{\Cat{O}} \bigl( (x^n_1,\, \dots,\, x^n_{m_n}),\, y_n \bigr)
        \end{align}
        と1つの多射 $(Y,\, \Familyset{\widehat{\rho_j}}{1 \le j \le n},\, G) \in \Mul{\Cat{O}} \bigl( (y_1,\, \dots,\, y_n),\, z\bigr)$ の合成
        \begin{align}
            \bigl(X,\, \Familyset{\widehat{\rho_i}}{1 \le i \le m_1+\cdots +m_n},\, G \circ (F^1;\, \dots \, ;\, F^n)\bigr) \in \Mul{\Cat{O}} \bigl( (x^1_1,\, \dots,\, x^1_{m_1};\, \dots\, ;\, x^n_1,\, \dots,\, x^n_{m_n}),\, z \bigr) 
        \end{align}
        を次のように定義する:
        \begin{enumerate}
            \item $X \in \Obj{\Cat{O}^\otimes_{\expval{m_1 + \cdots + m_n}}}$ は,$x^1_1,\, \dots,\, x^n_{m_n} \in \Obj{\Cat{O}}$ に対して\hyperref[cond:Op-ord-1]{\textsf{\textbf{(Op-ord-3)}}}を適用することにより定める.
            \item \hyperref[def:Cart-coCart]{$p$-coCartesian}な射の族 $\Familyset[\big]{\widehat{\rho^j_{i}} \in \Hom{\Cat{O}^\otimes}(X,\, x^j_i)_{\rho^{i+m_1+\cdots+m_{j-1}}}}{\substack{1 \le j \le n \\ 1 \le i \le m_j}}$ は $x^1_1,\, \dots,\, x^n_{m_n} \in \Obj{\Cat{O}}$ に対して\hyperref[cond:Op-ord-1]{\textsf{\textbf{(Op-ord-3)}}}を適用することにより定める.
            \item $\Cat{O}^\otimes$ の射 $G \circ (F^1;\, \dots \, ;\, F^n) \in \Hom{\Cat{O}^\otimes} (X,\, z)_{\alpha_{m_1 + \cdots + m_n}}$ は以下の手順に従って構成する:
            \begin{description}
                \item[\textbf{(STEP-1)}] 
                
                    まず,$1 \le \forall j \le n$ に対して\hyperref[def:inert-active]{inertな射} $\pi_j \in \Hom{\FIN_*} \bigl( \expval{m_1 + \cdots + m_n},\, \expval{m_j} \bigr)$ を
                    \begin{align}
                        \pi_j \colon \expval{m_1 + \cdots + m_n} &\lto \expval{m_j}, \\
                        k &\lmto \begin{cases}
                            k - (m_1 + \cdots + m_{j-1}), &1 \le k - (m_1 + \cdots + m_{j-1}) \le m_j \\
                            *, &\text{otherwise}
                        \end{cases}
                    \end{align}
                    で定義する.
                \item[\textbf{(STEP-2)}] 
                
                    \hyperref[def:inert-active]{inertな射} $\pi_j \in \Hom{\FIN_*} \bigl( \expval{m_1 + \cdots + m_n},\, \expval{m_j} \bigr)$ に対して\hyperref[cond:Op-ord-1]{\textsf{\textbf{(Op-ord-1)}}}を適用することにより,\hyperref[def:p-coCart-ord]{$p$-coCartesianな射} 
                    $\bar{\pi}_j \in \Hom{\Cat{O}^\otimes} (X,\, X^j)_{\pi_j}$ を取得する.

                \item[\textbf{(STEP-3)}] 
                
                    \begin{align}
                        (F^1 \circ \bar{\pi}_1,\, \dots,\, F^n \circ \bar{\pi}_n) \in \prod_{j=1}^n \Hom{\Cat{O}^\otimes} (X,\, y_j)_{\alpha_{m_j} \circ \pi_j}
                    \end{align}
                    に対して\hyperref[cond:Op-ord-1]{\textsf{\textbf{(Op-ord-2)}}}を適用することで,対応する
                    \begin{align}
                        F \in \Hom{\Cat{O}^\otimes} (X,\, Y)_\pi
                    \end{align}
                    が一意的に定まる.これに $G \in \Hom{\Cat{O}^\otimes}(Y,\, z)_{\alpha_n}$ を合成して
                    \begin{align}
                        G \circ_{\pi} (F^1;\, \dots \, ;\, F^n) \coloneqq G \circ F \in \Hom{\Cat{O}^\otimes} (X,\, z)_{\alpha_{m_1 + \cdots + m_n}}
                    \end{align}
                    と定義する.
                    ただし,$\pi \in \Hom{\FIN_*} \bigl( \expval{m_1 + \cdots + m_n},\, \expval{n} \bigr)$ は
                    \begin{align}
                        \pi(k) \coloneqq 
                        \begin{cases}
                            j, &m_{j-1} < k \le m_j \\
                            *, &k=*
                        \end{cases}
                    \end{align}
                    と定義される\hyperref[def:inert-active]{activeな射}である\footnote{$1 \le \forall j \le n$ に対して $\rho^j \circ \pi = \alpha_{m_j} \circ \pi_j$ が成り立つ.}.
            \end{description}
        \end{enumerate}
        
    \end{itemize}
\end{myprop}

\begin{proof}
    
\end{proof}

逆の対応を作ることもできる.

\begin{mydef}[label=def:cat-of-operators]{Category of operators}
    いま,\hyperref[def:c-operad]{colored operad} $\Cat{O}$ が与えられたとする.このとき\textbf{category of operators}と呼ばれる $(1,\, 1)$-圏 $\Cat{O}^{\otimes}$ を次のように定義する:
    \begin{itemize}
        \item $\Cat{O}$ の対象の有限列 $x_1,\, \dots,\, x_n \in \Obj{\Cat{O}}$ を対象に持つ.
        \item $\forall (x_1,\, \dots,\, x_m),\, (y_1,\, \dots,\, y_n) \in \Obj{\Cat{O}^\otimes}$ に対して,
        以下の2つ組を全て集めて得られる集合を $\Hom{\Cat{O}^\otimes} \bigl( \Familyset{x_i}{1 \le i \le m},\, \Familyset{y_j}{1 \le j \le n} \bigr)$ とする.
        \begin{enumerate}
            \item $\FIN_*$ の射
            \begin{align}
                \alpha \in \Hom{\FIN_*} \bigl( \expval{m},\, \expval{n} \bigr)
            \end{align}
            \item 多射の族
            \begin{align}
                \Familyset[\Big]{\phi_j \in \Mul{\Cat{O}} \bigl( \Familyset{x_i}{i \in \alpha^{-1}(\{j\})},\, y_j \bigr) }{1 \le j \le n}
            \end{align}
        \end{enumerate}
        より具体的には,
        \begin{align}
            \Hom{\Cat{O}^\otimes} \bigl( \Familyset{x_i}{1 \le i \le m},\, \Familyset{y_j}{1 \le j \le n} \bigr) \coloneqq \coprod_{\alpha \in \Hom{\FIN_*}\bigl( \expval{m},\, \expval{n} \bigr)} \prod_{j=1}^n \Mul{\Cat{O}} \bigl( \Familyset{x_i}{i \in \alpha^{-1}(\{j\})},\, y_j \bigr)
        \end{align}
        である.
        \item 射の合成は,$\FIN_*$ における射の合成および $\Cat{O}$ における多射の合成によって定める.
    \end{itemize}
    \tcblower
    $\Cat{O}^\otimes$ から $\FIN_*$ への忘却関手を
    \begin{align}
        p \colon \Cat{O}^{\otimes} &\lto \FIN_*, \\
        \Familyset{x_i}{1 \le i \le n} &\lmto \expval{n}, \\
        \bigl( \alpha,\, \Familyset{\phi_j}{1 \le j \le n} \bigr) &\lmto \alpha
    \end{align}
    と定義する.
\end{mydef}

\begin{myprop}[label=prop:c-operad]{色付きオペラッドとcategory of operators}
    $(1,\, 1)$-圏の関手
    \begin{align}
        p \colon \Cat{O}^\otimes \lto \FIN_*
    \end{align}
    において,$\Cat{O}^\otimes$ がある\hyperref[def:c-operad]{色付きオペラッド} $\Cat{O}$ の\hyperref[def:cat-of-operators]{category of operators}と圏同値になる必要十分条件は,
    $p$ が条件\hyperref[cond:Op-ord-1]{\textsf{\textbf{(Op-ord-1)}}}-\hyperref[cond:Op-ord-1]{\textsf{\textbf{(Op-ord-3)}}}を充たすことである.
\end{myprop}

\begin{proof}
    ~\cite[Proposition 2.2.II.]{Haugseng2023iop}
\end{proof}

\subsection{{$(\infty,\, 1)$}-圏における代数}

話を\hyperref[def:infty-operad]{$(\infty,\, 1)$-オペラッド}に戻そう.




\end{document}