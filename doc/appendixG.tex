\documentclass[TQFT_main]{subfiles}

\begin{document}

% \setcounter{}{}
\chapter{層状化空間・因子化ホモロジー}

~\cite{AFT2014stratified}, ~\cite{AFT2014FH}のレビュー

\section{多様体の層状化}

\subsection{層状化空間}

\begin{mydef}[label=def:topo-poset]{半順序集合の位相}
    $(P,\, \le)$ を半順序集合とする.
    $P$ 上の位相 $\mathscr{O}_{\le} \subset 2^P$ を以下で定義する:
    \begin{align}
        U \in \mathscr{O}_{\le} \DEF \forall x \in U,\, \forall y \in P,\; \bigl[\, x \le y \IMP y \in U \,\bigr]
    \end{align}
\end{mydef}

実際,空集合の定義から $\emptyset \in \mathscr{O}_{\le}$ であり,$\forall U_1,\, U_2 \in \mathcal{O}_{\le}$ に対して $x \in U_1 \cap U_2$ であることは
\begin{align}
     \forall y \in P,\; x \le y \IMP y \in U_1 \AND y \in U_2
\end{align}
と同値なので $U_1 \cap U_2 \in \mathscr{O}_{\le}$ であり,
さらに勝手な開集合族 $\Familyset[big]{U_\lambda \in \mathscr{O}_{\le}}{\lambda \in \Lambda}$ に対して
$x \in \bigcup_{\lambda \in \Lambda} U_\lambda$ は
\begin{align}
    \exists \alpha \in \Lambda,\; \forall y \in P,\; x \le y \IMP y \in U_\alpha \subset \bigcup_{\lambda \in \Lambda} U_\lambda 
\end{align}
と同値であるから $\bigcup_{\lambda \in \Lambda} U_\lambda \in \mathscr{O}_{\le}$ であり,$\mathscr{O}_{\le}$ は集合 $P$ の位相である.

\begin{myexample}[label=ex:topo-poset]{$[n]$ の位相}
    半順序集合 $[2] \coloneqq \{0 \le 1 \le 2\}$ を考える.このとき,\hyperref[def:topo-poset]{位相 $\mathscr{O}_{\le}$}とは
    \begin{align}
        \mathscr{O}_{\le} = \bigl\{\, \emptyset,\, \{2\},\, \{1,\, 2\},\, \{0,\, 1,\, 2\} \,\bigr\} 
    \end{align}
    のことである.同様に,半順序集合 $[n] \coloneqq \{0 \le 1 \le \cdots \le n\}$ に対して
    \begin{align}
        \mathscr{O}_{\le} = \bigl\{\, \emptyset,\, \{n\},\, \{n-1,\, n\},\, \dots,\, \{0,\, \dots,\, n\}  \,\bigr\}
    \end{align}
    が成り立つ.
\end{myexample}

\begin{mydef}[label=def:stratified-space]{層状化空間・層状化写像}
    $(P,\, \le)$ を半順序集合とし,定義\ref{def:topo-poset}の位相を入れて位相空間にする.
    
    このとき,位相空間 $X$ が $\bm{P}$\textbf{-層状化}されている ($P$-stratified) とは,連続写像 $s \colon X \lto P$ が存在することを言う.
    組 $(X,\, s \colon X \lto P)$ のことを\textbf{$\bm{P}$-層状化空間} ($P$-stratified space) と呼ぶ.

    \tcblower

    層状化空間 $(X,\, s \colon X \lto P),\; (X',\, s' \colon X' \lto P')$ の間の\textbf{層状化写像} (stratified map) とは,連続写像の組み $(f \colon X \lto X',\; g \colon P \lto P')$ であって以下の図式を可換にするもののこと:
    \begin{center}
        % https://tikzcd.yichuanshen.de/#N4Igdg9gJgpgziAXAbVABwnAlgFyxMJZABgBpiBdUkANwEMAbAVxiRAA0QBfU9TXfIRQBGclVqMWbdgHJuvEBmx4CRMsPH1mrRCAAK8vssFFRG6lqm69cruJhQA5vCKgAZgCcIAWyRkQOBBIAEwWkjogAMog1Ax0AEYwDHr8KkIgHliOABY4hiCePkiiAUGIAMxh2myRtgqFvoj+gcVVVgX5DSHULRWxCUkpxqq6mTl5bRGO3BRcQA
\begin{tikzcd}
X \arrow[d, "s"'] \arrow[r, red,"f"] & X' \arrow[d, "s'"] \\
P \arrow[r, red,"g"']                & P'                
\end{tikzcd}
    \end{center}
\end{mydef}

\begin{myexample}[label=ex:str-CW]{CW複体}
    CW複体 $X$ を与える.$X_{\le k}$ を $X$ の $k$-骨格とするとき,$X_k \setminus X_{k-1}$ を $k \in \mathbb{Z}_{\ge 0}$ に写す写像 $s \colon X \lto \mathbb{Z}_{\ge 0}$ は $X$ の\hyperref[def:stratified-space]{層状化}を与える.
\end{myexample}


\begin{mydef}[label=def:strat-emb]{層状化埋め込み}
    \hyperref[def:stratified-space]{層状化写像} $(f,\, g) \colon (X,\, s \colon X \lto P) \lto (X',\, s' \colon X' \lto P')$ が\textbf{層状化開埋め込み} (stratified open embedding) であるとは,以下の2条件を充たすことを言う:
    \begin{enumerate}
        \item 連続写像 $f \colon X \lto X'$ は位相的埋め込みである\footnote{i.e. $f \colon X \lto f(X)$ が同相写像}
        \item $\forall p \in P$ に対して,$f$ の制限
        \begin{align}
            f|_{s^{-1}(\{p\})} \colon s^{-1} \bigl( \{p\} \bigr) \lto s'{}^{-1} \bigl( \{g(p)\} \bigr) 
        \end{align}
        は位相的埋め込みである.
    \end{enumerate}
\end{mydef}

\begin{marker}
    以下では混乱が生じにくい場合,\hyperref[def:stratified-space]{層状化写像} $(f,\, g) \colon (X,\, s \colon X \lto P) \lto (X',\, s' \colon X' \lto P')$ のことを $f \colon (X \to P) \lto (X' \to P')$ と略記し,連続写像 $g \colon P \lto P'$ のことも $f$ と書く.
\end{marker}

圏 $\StTOP$ を,
\begin{itemize}
    \item 第2可算なHausdorff空間の\hyperref[def:stratified-space]{層状化空間}を対象とする
    \item \hyperref[def:strat-emb]{層状化埋め込み}を射とする
\end{itemize}
ことで定義する.

\subsection{$C^0$ 級層状化空間}


\begin{mydef}[label=def:str-cone]{コーン}
    \hyperref[def:stratified-space]{層状化空間} $(X,\, s \colon X \to P)$ を与える.
    $X$ の\textbf{コーン} (cone) とは,以下のようにして構成される\hyperref[def:stratified-space]{層状化空間} $\bigl( \Cone{X},\, \Cone{s} \colon \Cone{X} \lto \Cone{P} \bigr)$ のこと:
    \begin{itemize}
        \item 位相空間 $\Cone{X}$ を,\hyperref[def:pullback-pushout]{押し出し}位相空間
        \begin{align}
            \Cone{X} \coloneqq \{\mathrm{pt}\} \amalg_{\{0\} \times X} (\mathbb{R}_{\ge 0} \times X)
        \end{align}
        と定義する:
        \begin{center}
            % https://tikzcd.yichuanshen.de/#N4Igdg9gJgpgziAXAbVABwnAlgFyxMJZABgBpiBdUkANwEMAbAVxiRAB13hjOBfTvAFt4AAgAaIXqXSZc+QigCM5KrUYs2nQXRwALAEb7gAJV4B9YJwDmMEcV4iBWYXHGTpIDNjwEiZRar0zKyIHFxaOroAToLAaDj87LzuMt7yRMoB1EEaoZyW7Np6MXEJfI7sdNoMVhb5PEkVQqJiDgAUEXqGJuYFNnYOTi7iAJSSqjBQNggooABmURCCSGQgOBBIAEzUDHT6MAwACrI+CiBRWFa6OCkgC0tIymsbiADMUvOLy4ir64-Z6hCYW45SGok60ViWCg5gkHzuXy21D+bx2ewOxzSvlCFyuN14FF4QA
        \begin{tikzcd}
\{0\}\times X \arrow[d] \arrow[r, hookrightarrow, "\{0\} \times \mathrm{id}_X"] & \mathbb{R}_{\ge 0} \times X \arrow[d,red]                                 \\
\{\mathrm{pt}\} \arrow[r,hookrightarrow,red]                                       & \textcolor{red}{\{\mathrm{pt}\} \amalg_{\{0\} \times X} (\mathbb{R}_{\ge 0} \times X)}
        \end{tikzcd}
        \end{center}
        \item 半順序集合 $\Cone{P}$ を,$P$ に最小の要素 $-\infty$ を付け足すことで定義する.これは半順序集合の圏における押し出し
        \begin{align}
            \Cone{P} \coloneqq \{-\infty\} \amalg_{\{0\} \times P} \bigl( [1] \times P \bigr) 
        \end{align}
        である.
        \item 連続写像
        \begin{align}
            \mathbb{R}_{\ge 0} \times X &\lto [1] \times P, \\
            (t,\, x) &\lmto 
            \begin{cases}
                \bigl( 0,\, s(x) \bigr), &t=0, \\
                \bigl( 1,\, s(x) \bigr), &t>0
            \end{cases}
        \end{align}
        が誘導する連続写像 $\Cone{X} \lto \Cone{P}$ を $\Cone{s}$ と書く.
    \end{itemize}
\end{mydef}

位相空間の圏における押し出しの公式から,位相空間 $\Cone{X}$ とは
\begin{align}
    i_1 \colon \{0\} \times \mathbb{R} &\lto \mathbb{R}_{\ge 0} \times X,\; x \lmto (0,\, x), \\
    i_2 \colon \{0\} \times \mathbb{R} &\lto \{\mathrm{pt}\},\; x \lmto \mathrm{pt}
\end{align}
とおいたときのコイコライザ
\begin{center}
    % https://tikzcd.yichuanshen.de/#N4Igdg9gJgpgziAXAbVABwnAlgFyxMJZABgBpiBdUkANwEMAbAVxiRAB13hjOBfTvAFt4AAgAaIXqXSZc+QigCM5KrUYs2nYJ0F0cACwBOg4Ghz92FuroYBzEQAode-QCNXwAEq8A+tva2MCLEvCICWMJw4gCUktIgGNh4BEQATCrU9MysiBzsugZwAGbAAMK8DmKxvKowUIEIKKBFhhCCSGQgOBBIyiAMWGA5IHAQA1Ag1PowdBOIYEwMDNQ4dFgMbJBDkyP6WEU4SAC0fVkauVg+inHNre2Ind291HB7B8-9dK4wDAAKsskFCBDFhbPpDpl1MNLqkbiAWm0Pk9EKkarwgA
    \begin{tikzcd}
    \{0\}\times X \arrow[r, "i_1", shift left] \arrow[r, "i_2"', shift right] & \{\mathrm{pt}\}\sqcup (\mathbb{R}_{\ge 0} \times X) \arrow[r] & \mathsf{C}(X)
    \end{tikzcd}
\end{center}
である.i.e. 商位相空間
\begin{align}
    \frac{\mathbb{R}_{\ge 0} \times X}{i_1 (x) \sim i_2 (x)} = \frac{\mathbb{R}_{\ge 0} \times X}{\{0\} \times X}
\end{align}
のこと.従って $\Cone{s} \colon \Cone{X} \lto \Cone{P}$ とは,連続写像\footnote{$\Cone{P}$ の\hyperref[def:topo-poset]{位相 $\mathscr{O}_{\Cone{P}}$}は,$P$ の位相 $\mathscr{O}_{P}$ に1つの開集合 $\{-\infty\} \cup P$ を加えたものである.$\forall U \in \mathscr{O}_{P}$ に対して $\Cone{s}^{-1} (U) = \mathbb{R}_{\textcolor{red}{>0}} \times s^{-1}(U) \in \mathscr{O}_{\Cone{X}}$ で,かつ $\Cone{s}^{-1}(\{-\infty\} \cup P) = \Cone{X} \in \mathscr{O}_{\Cone{X}}$ なので $\Cone{s}$ は連続である.}
\begin{align}
    \frac{\mathbb{R}_{\ge 0} \times X}{\{0\} \times X} &\lto \Cone{P},\;
    [(t,\, x)] \lmto \begin{cases}
        -\infty, &t = 0 \\
        s(x), &t > 0
    \end{cases}
\end{align}
のことである.


\begin{mydef}[label=def:Snglr-C0]{$C^0$ 級層状化空間}
    以下を充たす $\StTOP$ の最小の\hyperref[def:faithful]{充満部分圏}を $\Snglr$ と書き,圏 $\Snglr$ の対象を\textbf{$\bm{C^0}$ 級層状化空間} ($C^0$ stratified space) と呼ぶ:
    \begin{description}
        \item[\textbf{(Snglr-1)}] $(\emptyset \to \emptyset) \in \Obj{\Snglr}$
        \item[\textbf{(Snglr-2)}] 
        
        $(X \to P) \in \Obj{\Snglr}$ かつ $X,\, P$ が位相空間としてコンパクト 
        
        $\IMP$ $\Cone{X \to P} \in \Obj{\Snglr}$
        
        \item[\textbf{(Snglr-3)}] 
        
        $(X \to P) \in \Obj{\Snglr}$ $\IMP$ $(X \times \mathbb{R} \to P) \in \Obj{\Snglr}$\footnote{$X \times \mathbb{R}$ の層状化は,連続写像 $X \times \mathbb{R} \lto X,\; (x,\, t) \lmto x$ を前もって合成することにより定める.}
        
        \item[\textbf{(Snglr-4)}] 
        
        $(X \to P) \in \Obj{\Snglr}$ かつ $\Hom{\StTOP} \bigl( (U \to P_U),\, (X \to P) \bigr) \neq \emptyset$ 
        
        $\IMP$ $(U \to P_U) \in \Obj{\Snglr}$
        
        \item[\textbf{(Snglr-5)}]  
        
        $(X \to P) \in \Obj{\StTOP}$ が開被覆 $\Familyset[big]{(U_\lambda \to P_\lambda) \lto (X \to P)}{\lambda \in \Lambda}$\footnote{i.e. $\Familyset[big]{U_\lambda}{\lambda \in \Lambda},\; \Familyset[big]{P_\lambda}{\lambda \in \Lambda}$ が,それぞれ位相空間 $X,\, P$ の開被覆を成す.} を持ち,かつ $\forall \lambda \in \Lambda$ に対して $(U_\lambda \to P_\lambda) \in \Obj{\Snglr}$ 
        
        $\IMP$ $(X \to P) \in \Obj{\Snglr}$
    \end{description}
\end{mydef}

\begin{myexample}[label=ex:topomfld]{位相多様体は $C^0$ 級層状化空間}
    \hyperref[def:Snglr-C0]{\textsf{\textbf{(Snglr-1)}}}より,$* \coloneqq \Cone{\emptyset \to \emptyset} \in \Obj{\Snglr}$ である.
    \hyperref[def:Snglr-C0]{\textsf{\textbf{(Snglr-3)}}}より,$\forall n \ge 0$ に対して $\mathbb{R}^n = (\mathbb{R}^n \to [0]) \in \Obj{\Snglr}$ であることが帰納的に分かる.
    $\mathbb{R}^n$ の任意の開集合 $U \hookrightarrow \mathbb{R}^n$ に対して,
    \begin{center}
            \begin{tikzcd}
        U \arrow[d] \arrow[r, hookrightarrow] & \mathbb{R}^n \arrow[d] \\
        {[0]} \arrow[r, "="']                & {[0]}               
    \end{tikzcd}
    \end{center}
    は\hyperref[def:strat-emb]{層状化埋め込み}であり,従って \hyperref[def:Snglr-C0]{\textsf{\textbf{(Snglr-4)}}}より $U \coloneqq (U \to [0]) \in \Obj{\Snglr}$ が分かる.
    以上の考察と\hyperref[def:Snglr-C0]{\textsf{\textbf{(Snglr-5)}}}を併せて,任意の位相多様体 $M$ は\footnote{より正確には,$M$ を\hyperref[def:stratified-space]{層状化空間} $(M \to [0])$ と同一視している.}圏 $\Snglr$ の対象である.
\end{myexample}

\subsection{conically smoothness}

\begin{mydef}[label=def:C0-basic]{$C^0$ basic}
    \hyperref[def:Snglr-C0]{$\bm{C^0}$ 級層状化空間} $(X \to P) \in \Obj{\Snglr}$ が $\bm{C^0}$\textbf{-basic}であるとは,ある $n \in \mathbb{Z}_{\ge 0}$ およびコンパクトな\hyperref[def:Snglr-C0]{$\bm{C^0}$ 級層状化空間} $(Z \to Q) \in \Obj{\Snglr}$ が存在して
    $(X \to P) = (\mathbb{R}^n \to *) \times \Cone{Z \to Q}$ が成り立つことを言う.
\end{mydef}

いま,\hyperref[def:C0-basic]{$C^0$ basic}な $(U \to P_U) = (\mathbb{R}^n \to [0]) \times \Cone{Z \to P} \in \Obj{\Snglr}$ を1つとる.
\hyperref[def:str-cone]{コーンの定義}から,$U$ の点を $(v,\, [t,\, z]) \in \mathbb{R}^n \times \frac{\mathbb{R}_{\ge 0} \times Z}{\{0\} \times Z}$ と表示することができる.
自己同相
\begin{align}
    \gamma \colon \mathbb{R}_{> 0} \times T \mathbb{R}^n \times \Cone{Z} &\lto \mathbb{R}_{> 0} \times T \mathbb{R}^n \times \Cone{Z}, \\
    \bigl( t,\, (v,\, p),\, [s,\, z] \bigr) &\lmto \bigl( t,\, (tv + p,\, p),\, [ts,\, z] \bigr)
\end{align}
を考える\footnote{接束 $T\mathbb{R}^n$ は $\mathbb{R}^{2n}$ と微分同相である.~\cite[p.23]{AFT2014stratified}の記法に合わせて底空間 $\mathbb{R}^n$ の点を $p$,$p$ 上のファイバーの元を $v$ としたとき $(v,\, p) \in T \mathbb{R}^n$ と書いた.命題\ref{prop:tangentbundle}の記法と順番が逆なので注意.}.

さらに,もう1つの\hyperref[def:C0-basic]{$C^0$ basic}な $(U' \to P_{U'}) = (\mathbb{R}^{n'} \to [0]) \times \Cone{Z' \to P'} \in \Obj{\Snglr}$ および
$f \in \Hom{\Snglr} \bigl( (U \to P_{U}),\, (U' \to P_{U'})  \bigr)$ をとる.ただし,$\forall u \in \mathbb{R}^n$ に対して $f (u,\, \mathrm{pt}) = f (u,\, \mathrm{pt})$ が成り立つことを仮定する.
$f|_{\mathbb{R}^n} \colon \mathbb{R}^n \times \{\mathrm{pt}\} \lto \mathbb{R}^n \times \{\mathrm{pt}\}$ を $f$ のコーンポイントへの制限として,
\begin{align}
    f_{\Delta} \colon \mathbb{R}_{\textcolor{red}{>0}} \times T \mathbb{R}^n \times \Cone{Z} &\lto \mathbb{R}_{\textcolor{red}{>0}} \times T \mathbb{R}^{n'} \times \Cone{Z'}, \\
    \bigl( t,\, v,\, p,\, [s,\, z] \bigr) &\lmto \bigl( t,\, f|_{\mathbb{R}^n}(v),\, f(p,\, [ts,\, z]) \bigr)
\end{align}
とおこう.

\begin{myexample}[label=ex:cone-diff]{}
    $Z = Z' = \emptyset$ のとき,$f$ とは単に連続関数 $f \colon \mathbb{R}^n \lto \mathbb{R}^{n'}$ のことである.
    このとき,
    \begin{align}
        (\gamma^{-1} \circ f_{\Delta} \circ \gamma)(t,\, v,\, p)
        &= \gamma^{-1} \circ f_\Delta (t,\, tv+p,\, p) \\
        &= \gamma^{-1} \bigl(t,\, f(tv+p),\, f(p)\bigr) \\
        &= \left( t,\, \frac{f(tv+p) - f(p)}{t},\, f(p) \right)
    \end{align}
    と計算できるため,$f$ が $C^1$ 級であることと $\forall (v,\, p) \in T \mathbb{R}^n$ に対して $t \to +0$ の極限,i.e. $v$ に沿った片側方向微分が存在することは同値である.
\end{myexample}

\begin{mydef}[label=def:c-smooth-along]{$\mathbb{R}^n$ に沿ってconically smooth}
    \begin{itemize}
        \item \hyperref[def:C0-basic]{$C^0$ basic}な $(U \to P_{U}) = (\mathbb{R}^{n} \to [0]) \times \Cone{Z \to P} \in \Obj{\Snglr}$
        \item \hyperref[def:C0-basic]{$C^0$ basic}な $(U' \to P_{U'}) = (\mathbb{R}^{n'} \to [0]) \times \Cone{Z' \to P'} \in \Obj{\Snglr}$
        \item $f \in \Hom{\Snglr} \bigl( (U \to P_{U}),\, (U' \to P_{U'})  \bigr)$ であって,コーンポイントを保存するもの
    \end{itemize}
    を与える.このとき,$f$ が\textbf{$\bm{\mathbb{R}^n}$ に沿って $C^1$ 級} ($C^1$ along $\mathbb{R}^n$) であるとは,以下の図式を可換にする連続写像
    \begin{align}
        \textcolor{red}{\tilde{D}f} \colon \mathbb{R}_{\textcolor{red}{\ge 0}} \times T \mathbb{R}^n \times \Cone{Z} &\lto \mathbb{R}_{\textcolor{red}{\ge 0}} \times T \mathbb{R}^{n'} \times \Cone{Z'}
    \end{align}
    が存在することを言う:
    \begin{center}
        % https://tikzcd.yichuanshen.de/#N4Igdg9gJgpgziAXAbVABwnAlgFyxMJZABgBpiBdUkANwEMAbAVxiRAB12BbOnACwBGA4ACUAvgH1gnHDAAeOAMYQGEAE7A1MKGOnsA5jAAExMWKMysXeEYAqF7r0HDxAPTAO81uA4DCBGGAALTEQMVJ0TFx8QhQARnIqWkYWNk4efiFRST1ZBWVVDS0dPUMTM08rG3t0pyy3YDAAcnNLbz8A4JawiJAMbDwCIjI4pPpmVkQOR0yXHIA+csr2mpnnbPdlm05-MECQnsiBmKIE0epx1Kna2eypRdMtn1WM9Ybm1vYvbfZd-e6xEltIYEChQAAzNQQLhIABM1BwECQAGZwhCoTDEPCQIikKZepDoSiEUjEHE0SBCZjsbjEMiLilJtN9HQuDxXMAALTkziKLBqRRGcFSTgAERgDBwdE+fIFDhZbLoIGoDDoAglAAUooNYiA1Fh9HwcIdKRi8SSkAlkhM0l8sAxYMBRWJwcqQAwsHs2FA6HA+NowhQxEA
        \begin{tikzcd}
        \mathbb{R}_{\textcolor{red}{\ge 0}} \times T \mathbb{R}^n \times \Cone{Z} \arrow[r, "\tilde{D}f", red, dashed]                         & \mathbb{R}_{\textcolor{red}{\ge 0}} \times T \mathbb{R}^{n'} \times \Cone{Z'} \\
        \mathbb{R}_{> 0} \times T \mathbb{R}^n \times \Cone{Z} \arrow[r] \arrow[u] \arrow[r, "\gamma^{-1}\circ f_{\Delta} \circ \gamma"'] & \mathbb{R}_{> 0} \times T \mathbb{R}^{n'} \times \Cone{Z'} \arrow[u]         
        \end{tikzcd}
    \end{center}
    \tcblower
    このような拡張が存在するとき,第一変数を $t=0$ に制限して得られる連続写像を
    \begin{align}
        \bm{Df} \colon T \mathbb{R}^n \times \Cone{Z} \lto T \mathbb{R}^{n'} \times \Cone{Z'}
    \end{align}
    と書く.$f$ が\textbf{$\bm{\mathbb{R}^n}$ に沿って $C^r$ 級} であるとは,$Df$ が $\mathbb{R}^n$ に沿って $C^{r-1}$ 級であることを言う.
    $f$ が\textbf{$\bm{\mathbb{R}^n}$ に沿ってconically smooth}であるとは,$\forall r \ge 1$ について $C^r$ 級であることを言う.
\end{mydef}

次に行うべきは,与えられた\hyperref[def:Snglr-C0]{$\bm{C^0}$ 級層状化空間} $(X \to P) \in \Obj{\Snglr}$ の上の\textbf{conically smooth structure} i.e. 変換関数が\hyperref[def:c-smooth-along]{conically smooth}であるような\textbf{極大アトラス}を定義することである.
この手続きは,次で定義する次元と深さに関する帰納法によって構成される.

\begin{mydef}[label=def:dim-depth]{次元と深さ}
    空でない\hyperref[def:Snglr-C0]{$\bm{C^0}$ 級層状化空間} $(X \to P) \in \Obj{\Snglr}$ を与える.
    \begin{itemize}
        \item $(X \to P)$ の点 $x \in X$ における\textbf{局所的次元} (local dimension) とは,点 $x$ における $X$ の被覆次元\footnote{以下の条件を充たす最小の $d \in \mathbb{Z}_{\ge -1}$ のことを(存在すれば)$X$ の\textbf{被覆次元} (covering dimension) と呼ぶ:$X$ の任意の開被覆 $\mathscr{U}$ に対して,十分細かい細分 $\mathscr{V}_{\mathscr{U}} \prec \mathscr{U}$ をとると,任意の互いに異なる $\forall m > d+1$ 個の開集合 $V_1,\, \dots,\, V_{m} \in \mathscr{V}_{\mathscr{U}}$ の共通部分が空になるようにできる.特に,$\emptyset$ の被覆次元は $-1$ と定義する.} $\bm{\dim_x(X)}$ のことを言う.
        \item $(X \to P)$ の\textbf{次元} (dimension) とは
        \begin{align}
            \bm{\dim (X \to P)} \coloneqq \sup_{x \in X} \dim_x (X)
        \end{align}
        \item $(X \xrightarrow{s} P)$ の点 $x \in X$ における\textbf{局所的深さ} (local depth) とは,
        \begin{align}
            \bm{\depth_x(X \to P)} \coloneqq \dim_x (X) - \dim_x \bigl( s^{-1}(\{s(x)\}) \bigr)
        \end{align}
        のこと.
        \item $(X \xrightarrow{s} P)$ の\textbf{深さ} (depth) とは,
        \begin{align}
            \bm{\depth(X)} \coloneqq \sup_{x \in X} \depth_x(X)
        \end{align}
        のこと.
    \end{itemize}
\end{mydef}

\begin{myexample}[label=ex:depth-Cone]{コーンの深さ}
    $n$ 次元位相多様体 $Z$ について,定義から $\dim_x(Z) = n$ が成り立つ.
    これを\exref{ex:topomfld}により\hyperref[def:Snglr-C0]{$C^0$ 級層状化空間} $(Z \to [0]) \in \Obj{\Snglr}$ と見做すと,これの\hyperref[def:str-cone]{コーン} $\Cone{Z \to [0]}$ について
    \begin{align}
        \depth_x \bigl( \Cone{Z \to [0]} \bigr) 
        =
        \begin{cases}
            n+1, &x = \mathrm{pt}, \\
            0, &\text{otherwise}
        \end{cases}
    \end{align}
    であることがわかる.実際,コーンポイントの逆像は1点なので $0$ 次元であり,かつ $\dim_x \bigl( \Cone{Z} \bigr) = n+1$ である.
    一方,コーンポイント以外の点の逆像は潰れておらず,$n+1$ 次元である.

     また,$\forall (X \to P) \in \Obj{\Snglr}$ に対して
    \begin{align}
        \dim \bigl( (\mathbb{R}^m \to [0]) \times (X \to P) \bigr) &= m + \dim \bigl( X \to P \bigr), \\
        \depth \bigl( (\mathbb{R}^m \to [0]) \times (X \to P) \bigr) &= \depth \bigl( X \to P \bigr) 
    \end{align}
    が成り立つ.従って,\hyperref[def:C0-basic]{$C^0$ basic}な $(U \to P_U) = (\mathbb{R}^n \to [0]) \times \Cone{Z \to P} \in \Obj{\Snglr}$ に対して
    \begin{align}
        \depth (U \to P_U) = \depth (Z \to P) + 1
    \end{align}
    が成り立つ.
\end{myexample}


\end{document}