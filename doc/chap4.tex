\documentclass[TQFT_main]{subfiles}

\begin{document}

\setcounter{chapter}{3}

\chapter{圏論とトポロジカル秩序}

この節で登場する多様体は特に断らない限り常に $C^\infty$ 多様体である.また,体 $\mathbb{K}$ と言ったら $\mathbb{K} = \mathbb{R},\, \mathbb{C},\, \mathbb{H}$ のいずれかを指すことにする.

\section{トポロジカル秩序のミクロな定義}

% \section{SPT相の定義}

この節では常に $d \coloneqq D+1$ 次元時空 $\mfdcal{M}{d} = \mfd{\Sigma}{D} \times \mathbb{R}$ または $\mfd{\Sigma}{D} \times S^1$ を考える\footnote{i.e. 時間方向は必要に応じてコンパクト化する}.
混乱が生じない時は時空点を $x \coloneqq (\bm{x},\, t) \in \mfd{\Sigma}{D} \times \mathbb{R}$ と書く.
境界を持たない $D$ 次元多様体\footnote{コンパクト性は仮定しない.} $\mfd{\Sigma}{D}$ はノルム $\norm*{\cdot}$ を持つ距離空間であるとする.

\begin{itemize}
    \item $\mfd{\Sigma}{D}$ 上の $D$ 次元\textbf{格子} (lattice) $\Lambda \subset \mfd{\Sigma}{D}$ とは,$\mfd{\Sigma}{D}$ の有限な\footnote{空間多様体 $\mfd{\Sigma}{D}$ や局所Hilbert空間に適当な境界条件を課して有限にする.}離散部分集合のことである.
    \item 格子点 $\bm{x} \in \Lambda$ 上のHilbert空間を $\mathcal{H}_{\bm{x}}$ と書く.
    \item 全系のHilbert空間とは,合成系 $\irm{\mathcal{H}}{tot} \coloneqq \bigotimes_{\bm{x} \in \Lambda} \mathcal{H}_{\bm{x}}$ のことである.
\end{itemize}

\begin{mydefph}[label=def:bosonic-lattice-model,breakable]{bosonicな格子模型}
    $D$ 次元格子 $\Lambda \subset \mfd{\Sigma}{D}$ を1つ固定する.
    \begin{itemize}
        \item $\forall \bm{x} \in \Lambda$ を1つとる.別の格子点 $\bm{y} \in \Lambda$ が $\bm{x}$ についてレンジ $R > 0$ であるとは,$\norm*{\bm{x} - \bm{y}} \le R$ が成り立つことを言う.$\bm{x}$ についてレンジ $R$ な格子点全体の集合を $N_R (\bm{x}) \subset \Lambda$ と書く.
        \item \underline{格子 $\Lambda$ 上の}\textbf{bosonicな格子模型} (bosonic lattice model) とは,エルミート演算子 $\hat{H}_{\Lambda} \in \Hom{\HILB} (\irm{\mathcal{H}}{tot},\, \irm{\mathcal{H}}{tot})$ のこと.
        bosonicな格子模型 $\hat{H}_\Lambda$ が\textbf{局所的} (local) であるとは,
        ある有限の $R > 0$ が存在して以下の条件を充たすもののことを言う:
        \begin{description}
            \item[\textbf{(locality)}] 
            
            $\forall \bm{x} \in \Lambda$ に対して,$\forall \bm{y} \in N_R(\bm{x})$ における局所的Hilbert空間 $\mathcal{H}_{\bm{y}}$ にのみ非自明に作用するエルミート演算子 $\hat{h}_{\bm{x}} \in \Hom{\HILB} (\irm{\mathcal{H}}{tot},\, \irm{\mathcal{H}}{tot})$ が存在して,
            \begin{align}
                \hat{H}_{\Lambda} = \sum_{\bm{x} \in \Lambda} \hat{h}_{\bm{x}}
            \end{align}
            と書ける.
        \end{description}
        
    \end{itemize}
    
    \tcblower

    \begin{itemize}
        \item $D$ 次元の\textbf{bosonicな量子系} (bosonic quantum system) とは,
        \begin{itemize}
            \item 格子の増大列\footnote{i.e. $\Lambda_1 \subsetneq \Lambda_2 \subsetneq \cdots$ が成り立つ.} $\{\Lambda_i\}_{i=1}^\infty$
            \item bosonicな格子模型の列 $\{\hat{H}_{\Lambda_i}\}_{i=1}^\infty$
        \end{itemize}
        の組のこと.
        \item bosonicな量子系 $\bigl( \{\Lambda_i\}_i,\, \{\hat{H}_{\Lambda_i}\}_i \bigr)$ の\textbf{熱力学極限} (thermodynamic limit) とは\footnote{厳密に言うと,極限 $\lim_{i \to \infty}$ は,格子の形状などの追加のデータを与えない限りill-definedである.},
        $D$ 次元格子 $\Lambda_\infty \coloneqq \lim_{i \to \infty} \Lambda_i \subset \mfd{\Sigma}{D}$ 上のbosonicな格子模型
        $\hat{H}_{\Lambda_\infty} \coloneqq \lim_{i \to \infty} \hat{H}_{\Lambda_i}$ のこと.
    \end{itemize}
    
\end{mydefph}

\subsection{ミクロな視点から量子相を定義する試み}

この小節では,\hyperref[def:bosonic-lattice-model]{格子模型の列}を直接用いて量子相を定義する試みを,~\cite{ChenGuWen2010}に倣って簡単に紹介する\footnote{ミクロな視点というのは,格子模型を用いるという意味である.}.
ここで紹介する定義は物理学者の直観に基づくものであり,数学的には大部分が未完成であることを先に断っておく.

\begin{mydefph}[label=def:gapped,breakable]{gappedな量子系}
    \hyperref[def:bosonic-lattice-model]{bosonicな量子系} $\bigl( \{\Lambda_i\}_i,\, \{\hat{H}_{\Lambda_i}\}_i \bigr)$ が\textbf{gapped}であるとは,
    ある $\Delta > 0$ および $E_0$ が存在して以下の条件を充たすことを言う(図\ref{fig:gapped}):
    \begin{description}
        \item[\textbf{(gap-1)}] $\forall E \in (E_0,\, E_0+\Delta)$ に対してある $N_E \in \mathbb{N}$ が存在して,
        \begin{align}
            i > N_E \IMP \mathrm{Spec}(\hat{H}_{\Lambda_i}) \cap (E_0,\, E_0+\Delta) = \emptyset
        \end{align}
        が成り立つ.
        \item[\textbf{(gap-2)}] $\forall \varepsilon > 0$ に対してある $N_\varepsilon \in \mathbb{N}$ が存在して,
        \begin{align}
            i > N_\varepsilon \IMP \mathrm{diam}\bigl(\Spec (\hat{H}_{\Lambda_i}) \cap (-\infty,\, E_0] \bigr) < \varepsilon
        \end{align}
        が成り立つ.
    \end{description}
    特に,十分大きな $i \in \mathbb{N}$ について定まる
    \begin{align}
        \mathrm{GSD}_{\Lambda_i}\bigl(\{\hat{H}_{\Lambda_i}\}\bigr) \coloneqq \abs{\Spec (\hat{H}_{\Lambda_i}) \cap (-\infty,\, E_0]}
    \end{align}
    のことを\textbf{基底状態の縮退度} (ground state degeneracy) と呼ぶ.
\end{mydefph}

\begin{figure}[H]
    \centering
    \begin{tikzpicture}
        \path coordinate (E_0)
        ++(0,0.1) coordinate (E_1)
        +(1.2,0.05) coordinate (g)
        % +(3.2,0.05) coordinate (lim)
        ++(0,0.1) coordinate (E_2)
        ++(0,0.1) coordinate (E_3)
        +(3.2,0) coordinate (lim)
        ++(0,1) coordinate (E_4)
        +(3.2,0) coordinate (lim_exc)
        ++(0,0.2) coordinate (E_5)
        ++(0,0.1) coordinate (E_6)
        ++(0,0.3) coordinate (E_7)
        ++(0,0.1) coordinate (E_8)
        ++(0,0.1) coordinate (E_9)
        ;
        \foreach \i in {0,...,9} {
            \draw (E_\i) -- ($(E_\i) + (1,0)$);
        }
        \node at ($(E_9)+(0.5,0.5)$) {$\vdots$};
        \draw[<->] ($(E_3)+(0.5,0)$) -- node[midway,fill=white] {$\Delta$} ($(E_4)+(0.5,0)$);
        \node at (g) {$\}$};
        \draw[->] ($(g)+(0.1,0)$) -- node[midway,above] {$L \to \infty$} ($(lim)-(0.2,0)$);
        \draw[red,very thick] (lim) -- ++(1,0);
        \draw[<->] ($(lim)+(0.5,0)$) -- node[midway,fill=white] {$\Delta$} ($(lim_exc)+(0.5,0)$);
        \draw[] (lim_exc) -- ++(1,0);
        \fill[gray!20] (lim_exc) rectangle +(1,1.8);
    \end{tikzpicture}
    \caption{gappedな量子系のエネルギースペクトル $\mathrm{Spec}(\hat{H}_{\Lambda_i})$}
    \label{fig:gapped}
\end{figure}%


\begin{mydefph}[label=def:gappedQL]{gapped quantum liquid}
    \hyperref[def:gapped]{gapped}かつ\hyperref[def:bosonic-lattice-model]{bosonicな量子系} $\bigl( \{\Lambda_i\}_i,\, \{\hat{H}_{\Lambda_i}\}_i \bigr)$ が\textbf{gappedな量子液体} (gapped quantum liquid) であるとは,
    ある $N \in \mathbb{N}$ が存在して
    \begin{align}
        \mathrm{GSD}_{\Lambda_{N}}\bigl( \{\hat{H}_{\Lambda_i}\} \bigr) = \mathrm{GSD}_{\Lambda_{N+1}}\bigl( \{\hat{H}_{\Lambda_i}\} \bigr) = \mathrm{GSD}_{\Lambda_{N+2}}\bigl( \{\hat{H}_{\Lambda_i}\} \bigr) = \cdots < \infty
    \end{align}
    が成り立つことを言う.
\end{mydefph}

\begin{marker}
    gapless quantum liquidの厳密な定義は,2025年現在でもこれといったものがない.
\end{marker}

\begin{myexample}[label=ex:Haah]{Haahコード}
    Haahコード~\cite{Haah2011}の基底状態の縮退度は
    \begin{align}
        \ln(\mathrm{GSD}_{\Lambda_i}) \sim \abs{\Lambda_i}
    \end{align}
    と振る舞うことが知られており,\hyperref[def:gapped]{gapped}だが\hyperref[def:gappedQL]{gappedな量子液体}でない\hyperref[def:bosonic-lattice-model]{bosonicな量子系}の例である.
\end{myexample}

% 熱力学極限 $\abs{\Lambda} \to \infty$ をとった $\mfd{\Sigma}{D}$ 上の\hyperref[def:bosonic-lattice-model]{bosonicな格子模型}全体の集合を $\bm{\Lat{\mfd{\Sigma}{D}}} \subset \Hom{\HILB} (\irm{\mathcal{H}}{tot},\, \irm{\mathcal{H}}{tot})$ と書き,\footnote{$\irm{\mathcal{H}}{tot}$ は一般には熱力学極限の取り方に依存する.}
% $\Lat{\mfd{\Sigma}{D}}$ の元の基底状態全体の集合を $\bm{\Gnd{\mfd{\Sigma}{D}}} \subset \irm{\mathcal{H}}{tot}$ と書く.
% $\Lat{\mfd{\Sigma}{D}}$ にコンパクト開位相\footnote{operator normによる距離位相としても良い.}を入れて位相空間にする.

\textbf{gappedな量子相} (gapped quantum phase) とは,大雑把には\underline{$\Sigma^{(D)} = \mathbb{R}^{D}$ としたときの}\footnote{最初から\hyperref[def:bosonic-lattice-model]{熱力学極限}により得られる無限系を念頭においている.} \hyperref[def:gappedQL]{gappedな量子液体}の同値類のことである.特に,同値関係の定義で対称性を考慮しないもののことを\textbf{トポロジカル秩序} (topological order) と呼ぶ.
この同値類の,物理的に妥当かつ数学的にも正確な定義を与える仕事は大変困難で,2025年時点で未達成である.
そのため,実際の量子相の研究の文脈では,物理的考察から等価であることが期待される別の(より扱いやすい)定義を用いて議論することが常である.

トポロジカル秩序を与える同値類の定義は,~\cite[p.3]{ChenGuWen2010}に倣うと以下のようになる:

\begin{mydefph}[label=def:quantum-phase,breakable]{bosonicかつgappedなトポロジカル秩序}
    空間多様体 $\Sigma^{(D)}$ 上の2つの\hyperref[def:gappedQL]{gappedな量子液体} $\bigl( \{\Lambda_i\}_i,\, \{\hat{H}^{(0)}_{\Lambda_i}\}_i \bigr),\; \bigl( \{\Lambda_i\}_i,\, \{\hat{H}^{(1)}_{\Lambda_i}\}_i \bigr)$ を与える.
    \hyperref[def:bosonic-lattice-model]{熱力学極限}をとった\hyperref[def:gappedQL]{gappedな量子液体}全体がなす集合を $\mathrm{gQL}(\Sigma^{(D)})$ とおく.$\mathrm{gQL}(\Sigma^{(D)})$ には適切な位相を入れて位相空間にする.

     このとき,$\hat{H}^{(0)}_{\Lambda_\infty}$ と $\hat{H}^{(1)}_{\Lambda_\infty}$ が同じ\textbf{gappedなトポロジカル秩序} (gapped topological order) にあるとは,連続曲線 $\hat{H} \colon [0,\, 1] \lto \mathrm{gQL}(\mfd{\Sigma}{D})$ が存在して
    $\hat{H}(0) = \hat{H}^{(0)}_{\Lambda_\infty} \AND \hat{H}(1) = \hat{H}^{(0)}_{\Lambda_\infty}$ を充たすことを言う.

    \tcblower

    空間次元 $D$ のトポロジカル秩序全体の集まりを $\bm{\TO{D}}$ と書くことにする.
\end{mydefph}

\begin{myexample}[label=def:trivialTO]{自明相}
    $\mathcal{H}$ を有限次元Hilber空間,$\hat{P} \in \End(\mathcal{H})$ を,唯一の基底状態 $\ket{\mathrm{gnd}} \in \mathcal{H}$ を持つエルミート演算子とする.
    このとき任意の空間次元 $D$ と格子 $\Lambda \subset \mfd{\Sigma}{D}$ に対して,\hyperref[def:bosonic-lattice-model]{bosonicな格子模型} $\hat{H}_\Lambda = \sum_{x \in \Lambda} \hat{h}_x$ を
    \begin{itemize}
        \item $\mathcal{H}_x \coloneqq V$ とする.
        \item $\hat{h}_x \coloneqq \unity \otimes \cdots \otimes \unity \otimes \underbrace{\hat{P}}_{x} \otimes \unity \otimes \cdots \otimes \unity$ を局所的なハミルトニアンとする.
    \end{itemize}
    ことで定義する.この格子模型は一意的な基底状態 $\bigotimes_{x \in \Lambda} \ket{\mathrm{gnd}}$ を持ち,\hyperref[def:gappedQL]{gappedな量子液体}を成し,\textbf{自明相} (trivial phase) と呼ばれる\hyperref[def:quantum-phase]{bosonicなトポロジカル秩序}を定める.
    空間次元 $D$ の自明相を $\bm{1_D}$ と書く.
\end{myexample}

\begin{myexample}[label=def:stacking]{トポロジカル秩序の積層}
    空間次元 $D$ と格子 $\Lambda \subset \mfd{\Sigma}{D}$ を与え,その上の2つの\hyperref[def:bosonic-lattice-model]{bosonicな格子模型} $\hat{H}^{(0)}_\Lambda \in \End (\irm{\mathcal{H}}{tot}^{(0)}),\; \hat{H}^{(1)}_\Lambda \in \End (\irm{\mathcal{H}}{tot}^{(1)})$ を考える.
    このとき,
    \begin{itemize}
        \item $\irm{\mathcal{H}}{tot} \coloneqq \irm{\mathcal{H}}{tot}^{(0)} \otimes \irm{\mathcal{H}}{tot}^{(1)}$
        \item $\hat{H}_\Lambda \coloneqq \hat{H}_\Lambda^{(0)} \otimes \unity + \unity \otimes \hat{H}_\Lambda^{(1)}$
    \end{itemize}
    と定義することで,新たな\hyperref[def:bosonic-lattice-model]{bosonicな格子模型} $\hat{H}_\Lambda \in \End (\irm{\mathcal{H}}{tot})$ を得る.この操作を格子模型の\textbf{積層} (stacking) と呼ぶ.
    格子模型の積層を用いて,\hyperref[def:gappedQL]{gappedな量子液体}及び\hyperref[def:quantum-phase]{bosonicなトポロジカル秩序}の積層を定義することができる.
    特に,空間次元 $D$ の2つの量子相 $\mathsf{C}_D,\, \mathsf{D}_D \in \TO{D}$ に対して,その積層によって得られる新たな量子相を $\bm{\mathsf{C}_D \boxtimes \mathsf{D}_D} \in \TO{D}$ と書く.
\end{myexample}

\begin{mypropph}[label=prop:monoidalTO]{トポロジカル秩序のモノイド構造}
    任意の空間次元 $D$ において,以下のデータの3つ組みは\underline{可換}モノイドを成す:
    \begin{itemize}
        \item \hyperref[def:quantum-phase]{トポロジカル秩序}全体の集合 $\TO{D}$
        \item トポロジカル秩序の\hyperref[def:stacking]{積層} $\boxtimes \colon \TO{D} \times \TO{D} \lto \TO{D}$
        \item \hyperref[def:trivialTO]{自明相} $1_D \in \TO{D}$
    \end{itemize}
\end{mypropph}

\begin{proof}
    
\end{proof}

\subsection{重力アノマリー}

トポロジカル相の文脈で\textbf{重力アノマリー} (gravitational anomaly) と呼ばれるものを導入しておこう~\cite{KongWen2014braidedfusioncategoriesgravitational}.
これは文字通り量子重力を考えていると言うわけではなく,\hyperref[label=hypo:anomaly-inflow]{アノマリー流入}に類似の概念である.

\begin{mydefph}[label=def:anomalousQP]{gappedな量子相のアノマリー}
    空間 $D$ 次元の\hyperref[def:quantum-phase]{量子相}が\textbf{アノマリーを持たない} (anomaly-free) とは,それが\hyperref[def:bosonic-lattice-model]{局所的な格子模型}として実現できることを言う.
\end{mydefph}

\begin{myconjph}[label=conj:gravitational-anomaly]{重力アノマリーの仮説}
    空間次元 $D$ の任意の\hyperref[def:quantum-phase]{gappedな量子相} $\mathsf{A}_D$ に対して,
    ある空間次元 $D+1$ の\hyperref[def:anomalousQP]{アノマリー}を持たないgappedな量子相 $\Bulk{\mathsf{A}_D}$ が一意的に存在して,$\Bulk{\mathsf{A}_D}$ の境界として $\mathsf{A}_D$ が実現される.
\end{myconjph}

予想\ref{conj:gravitational-anomaly}は,$\mathsf{A}_D$ が\hyperref[def:quantum-phase]{トポロジカル秩序}の場合には物理的な証明があるらしい~\cite[Lemma 2, p.19]{KongWen2014braidedfusioncategoriesgravitational}.


\section{トポロジカル秩序のマクロな特徴付け}

零温度における\hyperref[def:quantum-phase]{量子相}を特徴付けるデータとは,くりこみ群のフローのIR側においても生き残っているような物理量だと考えられる.
もしくは,同じことだが,低エネルギー有効理論の長距離の振る舞いが零温度における量子相を特徴付けるという物理学者の期待がある.

もし量子相がgaplessならば,相関関数は典型的には長距離の振る舞い (algebraic decay) を示し,量子相を特徴付けるデータの一部であると考えらえる.
ところが,\hyperref[def:quantum-phase]{gappedな量子相}に関してはそうはいかない.

\begin{mytheo}[label=thm:gapped-decay]{gappedな格子模型における相関関数の振る舞い}
    $D$ 次元格子 $\Lambda \subset \mfd{\Sigma}{D}$ 及びその上の\hyperref[def:gapped]{gappedな量子系} $\hat{H}_\Lambda$ を考える.
    このとき,有限集合 $X,\, Y \subset \Lambda$ に台を持つ任意の演算子 $\hat{A}_X,\, \hat{B}_Y$ について,ある定数 $C,\, D,\, \xi_0$ が存在して以下が成り立つ:
    \begin{align}
        \mel{\psi_0^a}{\hat{A}_X \hat{B}_Y}{\psi_0^a} - \mel{\psi_0^a}{\hat{A}_X \hat{P}_0 \hat{B}_Y}{\psi_0^a} 
        &\le C  \norm*{\hat{A}_X} \norm*{\hat{B}_Y} \Bigl\{ e^{- \frac{\mathrm{dist} (X,\, Y)}{\xi_0}} + \min (\abs{X},\, \abs{Y}) g \bigl(\mathrm{dist} (X,\, Y) \bigr) \Bigr\} \\
        &\qquad + D\, \mathrm{diam}\bigl(\Spec (\hat{H}_{\Lambda_i}) \cap (-\infty,\, E_0] \bigr)
    \end{align}
    ただし,$\ket{\psi_0^a} \in \Spec (\hat{H}_{\Lambda_i}) \cap (-\infty,\, E_0]$ であり,$\hat{P}_0 \coloneqq \sum_a \ketbra{\psi_0^a}{\psi_0^a}$ は基底状態が成す部分空間への射影演算子である.
\end{mytheo}

\begin{proof}
    Lieb-Robinson boundを用いる.詳細は~\cite[Theorem 2, p.7]{Hastings2010localityquantumsystems}を参照.
\end{proof}

定理\ref{thm:gapped-decay}により,gappedな量子系の相関関数は熱力学極限において指数減衰するため,量子相を特徴付けるデータとなり得ない.
gappedな量子系を特徴付けるデータは,\textbf{トポロジカル欠陥}のデータだと考えられる.

\begin{mydefph}[label=def:TD]{$p$-次元のトポロジカル欠陥}
    空間多様体 $\Sigma^{(D)}$ 上の\hyperref[def:gappedQL]{gappedな量子液体} $\bigl( \{\Lambda_i\}_i,\, \{\hat{H}_{\Lambda_i}\}_i \bigr)$ を与える.
    $p < D$ について,\textbf{$p$ 次元の励起} ($p$-dimensional excitation) とは,
    $\hat{H}_{\infty} + \delta \hat{H}(\mfd{M}{p})$ の\hyperref[def:gapped]{gapped}な基底状態が成す $\irm{\mathcal{H}}{tot}$ の部分空間のこと~\cite[Definition 5., p.10]{KongWen2014braidedfusioncategoriesgravitational}.
    ただし, $\delta \hat{H}(\mfd{M}{p})$ は $\mfd{\Sigma}{D}$ の $p$ 次元部分多様体 $\mfd{M}{p} \subset \mfd{\Sigma}{D}$ の上に台を持つエルミート演算子のことである.

    \tcblower

    2つの $p$ 次元の励起 $\hat{H}_{\infty} + \delta \hat{H}(\mfd{M}{p}),\; \hat{H}_{\infty} + \delta \hat{H}(\mfd{N}{p})$ が同値であるとは,ある可逆な\footnote{\hyperref[def:stacking]{stacking}に関して可逆という意味.}\hyperref[def:quantum-phase]{トポロジカル秩序} $\mathsf{C}_p \in \TO{p}$ が存在して,$\mathsf{C}_p$ 及び $\mathsf{C}_p^{-1}$ を\hyperref[def:stacking]{積層}することによって互いに行き来できることを言う~\cite[Definition 6., p.10]{KongWen2014braidedfusioncategoriesgravitational}.
    この同値関係による同値類を\textbf{トポロジカル欠陥} (topological defect) と呼ぶ.
\end{mydefph}

\begin{marker}
    $p$ 次元の励起を特徴付けるトラップハミルトニアン $\delta \hat{H}(\mfd{M}{p})$ は,\hyperref[def:bosonic-lattice-model]{格子模型}の $p$ 次元部分多様体 $\mfd{M}{p}$ 上における境界条件を定めていると見做すことができる.
    この意味で,$p$ 次元のトポロジカル欠陥は\textbf{トポロジカルな境界条件} (topological boundary condition) と見做すことができる.
\end{marker}

定義\ref{def:TD}より,空間次元 $D$ の\hyperref[def:quantum-phase]{トポロジカル秩序} $\mathsf{C}_D \in \TO{D}$ 内部における $p$ 次元のトポロジカル欠陥はそれ自身が $p$ 次元の\hyperref[def:quantum-phase]{トポロジカル秩序}を成す.

\begin{myexample}[label=ex:toric-emf]{toric codeのトポロジカル欠陥-1}
    2次元正方格子 $\Lambda = \bigl( \mathsf{V}(\Lambda),\, \mathsf{E}(\Lambda) \bigr)$ 上の\textbf{toric code模型}~\cite{Kitaev1997TC}は,以下のように構成されるスピン $1/2$ 模型である:
    \begin{itemize}
        \item $\forall e \in \mathsf{E}(\Lambda)$ の上にはHilbert空間 $\mathcal{H}_e \coloneqq \mathbb{C}^2$ をアサインする.
        \item $\forall v \in  \mathsf{V}(\Lambda)$ に対して,$\hat{A}_v$ を次のように定義する:
        \begin{align}
            \hat{A}_v \coloneqq 
            \begin{tikzpicture}[baseline={([yshift=-.5ex]current bounding box.center)}]
                \path coordinate[bullet,label=above right:$v$] (v)
                +(1,0) coordinate (r)
                +(-1,0) coordinate (l)
                +(0,1) coordinate (u)
                +(0,-1) coordinate (d)
                ;
                \foreach \i in {r,l,u,d} {
                    \draw (v) -- node[midway] {$\hat{X}$} (\i);
                }
            \end{tikzpicture}
        \end{align}
        \item 任意の面 $p$ に対して,$\hat{B}_p$ を次のように定義する:
        \begin{align}
            \hat{B}_p \coloneqq 
            \begin{tikzpicture}[baseline={([yshift=-.5ex]current bounding box.center)}]
                \path 
                coordinate (dl)
                +(1,0) coordinate (dr)
                +(1,1) coordinate (ur)
                +(0,1) coordinate (ul)
                ;
                \draw[spath/save=pl] (dl) -- node[midway] {$\hat{Z}$} (dr) -- node[midway] {$\hat{Z}$} (ur) -- node[midway] {$\hat{Z}$} (ul) -- node[midway] {$\hat{Z}$} cycle;
                \begin{scope}[on background layer]
                    \fill[blue!10,spath/use=pl];
                \end{scope}
                \node at (barycentric cs:dl=1,dr=1,ul=1,ur=1) {$p$};
            \end{tikzpicture}
        \end{align}
        \item ハミルトニアンを
        \begin{align}
            \label{def:TC}
            \hat{H}_\Lambda \coloneqq \sum_{v}(1 - \hat{A}_v) - \sum_p (1 - \hat{B}_p)
        \end{align}
        で定義する.
    \end{itemize}
    ただし
    \begin{align}
        \hat{X} \coloneqq \mqty[0&1\\1&0],\quad \hat{Z} \coloneqq \mqty[1&0\\0&-1]
    \end{align}
    とおいた.全ての $\hat{A}_v,\, \hat{B}_p$ は可換なので同時対角化可能である.
    そのうえ $(\hat{A}_v)^2 = (\hat{B}_p)^2 = \unity$ が成り立つ(このような模型を\textbf{commuting projector hamiltonian}と呼ぶ)ので,$\hat{A}_v,\, \hat{B}_p$ の固有値は \underline{システムサイズ $\abs{\Lambda}$ に関係なく}どちらも $\pm 1$ である.
    故にこの模型は\hyperref[def:quantum-phase]{gappedなトポロジカル秩序} $\mathsf{TC}$ を示す.
    
     簡単のため,2次元Euclid空間上の無限系 $\Lambda = \mathbb{Z}^2 \subset \mathbb{R}^2$ を考える.toric code模型の持つ $0$ 次元の\hyperref[def:TD]{トポロジカル欠陥}を考えよう.
    $\irm{\mathcal{H}}{tot}$ は $\hat{A}_v,\, \hat{B}_p$ による固有空間分解を持つので,任意の状態は $\hat{A}_v,\, \hat{B}_p$ の同時固有状態で展開できる.
    \begin{enumerate}
        \item 基底状態が生成する\hyperref[def:TD]{トポロジカル欠陥} $\bm{1}$ が存在する.
        \item 頂点 $v_0 \in \mathsf{V}(\Lambda)$ 上に局在した励起を特徴付けるハミルトニアンとして $\delta \hat{H}(v_0) = 2 \hat{A}_{v_0}$ を選ぶと,
        \begin{align}
            \hat{A}_v \ket{\psi_e} &=
            \begin{cases}
                -\ket{\psi_e}, &v = v_0, \\
                \ket{\psi_e}, &v \neq v_0
            \end{cases},
            &
            \hat{B}_p \ket{\psi_e} &= \ket{\psi_e}
        \end{align}
        なる状態 $\ket{\psi_e}$ が張る1次元部分空間 $\mathbb{C} \ket{\psi_e} \subset \irm{\mathcal{H}}{tot}$ が\hyperref[def:TD]{0次元の励起}だと分かる.
        ここで,局所演算子
        \begin{align}
            \hat{Z}_1 \coloneqq 
            \begin{tikzpicture}[baseline={([yshift=-.5ex]current bounding box.center)}]
                \path coordinate[bullet,label=left:$v_0$] (v0)
                ++(1,0) coordinate[bullet,label=right:$v_1$] (v1)
                ;
                \draw[red,thick] (v0) -- node[midway] {$\hat{Z}$} (v1);
            \end{tikzpicture}
        \end{align}
        を $\ket{\psi_e}$ に作用させると,$\hat{A}_{v_0},\, \hat{A}_{v_1}$ のみが $\hat{Z}_1$ と反交換することから
        \begin{align}
            \hat{A}_v \hat{Z}_1\ket{\psi_e} &=
            \begin{cases}
                -\hat{Z}_1\ket{\psi_e}, &v = v_1, \\
                 \hat{Z}_1\ket{\psi_e}, &v \neq v_1
            \end{cases}
            &
            \hat{B}_p \hat{Z}_1\ket{\psi_e} &= \hat{Z}_1\ket{\psi_e}
        \end{align}
        が成り立つ.i.e. $\delta \hat{H}(v_0),\; \delta \hat{H}(v_1)$ が特徴付ける0次元の励起は互いに\hyperref[def:TD]{同値}である.
        同様にして任意の頂点上に局在した0次元の励起が互いに同値になるので,これらは1つの\hyperref[def:TD]{トポロジカル欠陥} $\bm{e}$ を成す.
        \item 任意の面 $p_0$ に局在した励起を特徴付けるハミルトニアンとして $\delta \hat{H}(p_0) = 2 \hat{B}_{p_0}$ を選ぶと,
        \begin{align}
            \hat{A}_v \ket{\psi_m} &= \ket{\psi_m}
            &
            \hat{B}_p \ket{\psi_m} &= 
            \begin{cases}
                -\ket{\psi_m}, &p=p_0 \\
                \ket{\psi_m}, &p\neq p_0
            \end{cases}
        \end{align}
        なる状態 $\ket{\psi_m}$ が張る1次元部分空間 $\mathbb{C} \ket{\psi_m} \subset \irm{\mathcal{H}}{tot}$ が\hyperref[def:TD]{0次元の励起}だと分かる.
        今度は
        \begin{align}
            \hat{X}_1 \coloneqq 
            \begin{tikzpicture}[baseline={([yshift=-.5ex]current bounding box.center)}]
                \path coordinate (dl)
                ++(1,0) coordinate (a)
                ++(1,0) coordinate (dr)
                ++(0,1) coordinate (ur)
                ++(-1,0) coordinate (b)
                ++(-1,0) coordinate (ul)
                ;
                \draw[spath/save=pl1] (dl) -- (a) -- (b) -- (ul) -- cycle;
                \draw[spath/save=pl2] (dr) -- (a) -- (b) -- (ur) -- cycle;
                \begin{scope}[on background layer]
                    \fill[blue!10,spath/use=pl1];
                    \fill[red!10,spath/use=pl2];
                \end{scope}
                \node at (barycentric cs:dl=1,a=1,b=1,ul=1) {$p_0$};
                \node at (barycentric cs:dr=1,a=1,b=1,ur=1) {$p_1$};
                \draw[red,thick] (a) -- node[midway] {$\hat{X}$} (b);
            \end{tikzpicture}
        \end{align}
        なる局所演算子を $\ket{\psi_m}$ に作用させると,$\hat{X}_1$ と反交換するのは $\hat{B}_{p_0},\, \hat{B}_{p_1}$ のみだから,
        \begin{align}
            \hat{A}_v \hat{X}_1\ket{\psi_m} &= \hat{X}_1\ket{\psi_m}
            &
            \hat{B}_p \hat{X}_1\ket{\psi_m} &= 
            \begin{cases}
                -\hat{X}_1\ket{\psi_m}, &p=p_1 \\
                 \hat{X}_1\ket{\psi_m}, &p\neq p_1
            \end{cases}
        \end{align}
        が成り立つ.同様にして任意の面上に局在した0次元の励起は互いに同値なので,これらは1つの\hyperref[def:TD]{トポロジカル欠陥} $\bm{m}$ を成す.
        \item 任意の点と面の組み $(v_0,\, p_0)$ 上に局在した励起を特徴付けるハミルトニアンとして $\delta \hat{H}\bigl(\{v_0,\, p_0\}\bigr) = 2\hat{A}_{v_0} + 2 \hat{B}_{p_0}$ を選ぶと,
        \begin{align}
            \hat{A}_v \ket{\psi_f} &=
            \begin{cases}
                -\ket{\psi_f}, &v = v_0, \\
                \ket{\psi_f}, &v \neq v_0
            \end{cases},
            &
            \hat{B}_p \ket{\psi_f} &= 
            \begin{cases}
                -\ket{\psi_f}, &p=p_0 \\
                \ket{\psi_f}, &p\neq p_0
            \end{cases}
        \end{align}
        なる状態 $\ket{\psi_f}$ が張る1次元部分空間 $\mathbb{C} \ket{\psi_f} \subset \irm{\mathcal{H}}{tot}$ が\hyperref[def:TD]{トポロジカル欠陥} $\bm{f}$ を成すことがわかる.
    \end{enumerate}
    このようにして,互いに異なる4つのトポロジカル欠陥 $\{1,\, e,\, m,\, f\}$ が構成された.
\end{myexample}

可逆なトポロジカル秩序の積層を同一視すると言うことは,任意の局所演算子の作用に関して不変であることと等価である.
従って,局所演算子が合成に関して成す結合代数 (\textbf{local operator algebra}) を $\irm{\mathcal{A}}{loc}$ と書くと,\hyperref[def:excitation]{トポロジカル欠陥}を次のように特徴付けることもできる:

\begin{marker}
    演算子の局所性の概念は,系のスケールに依存する.逆に言うと,トポロジカル欠陥は系のスケールに依存しない.
\end{marker}


\begin{myprop}[label=prop:TD]{トポロジカル欠陥の特徴付け}
    \hyperref[def:excitation]{トポロジカル欠陥}とは,$\irm{\mathcal{A}}{loc}$-加群のこと.
\end{myprop}

\begin{myexample}[label=ex:toric-loc]{toric codeのトポロジカル欠陥-2}
    \exref{ex:toric-emf}で得た4つのトポロジカル欠陥 $\{1,\, e,\, m,\, f\}$ を,$\irm{\mathcal{A}}{loc}$-加群として導出してみよう.
    まず,演算子の局所性を次の図で定義する:
    \begin{figure}[H]
        \centering
        \begin{tikzpicture}
            \path coordinate (dl)
            +(1,0) coordinate (dr)
            +(1,1) coordinate (ur)
            +(0,1) coordinate[bullet,label=above left:$v$] (ul)
            ;
            \draw[spath/save=pl] (dl) -- (dr) -- (ur) -- (ul.center) -- cycle;
            \draw (dl) -- ++(-0.5,0);
            \draw (dl) -- ++(0,-0.5);
            \draw (dr) -- ++(0.5,0);
            \draw (dr) -- ++(0,-0.5);
            \draw (ul) -- ++(-0.5,0);
            \draw (ul) -- ++(0,0.5);
            \draw (ur) -- ++(0.5,0);
            \draw (ur) -- ++(0,0.5);
            \begin{scope}[on background layer]
                \fill[blue!10,spath/use=pl];
            \end{scope}
            \node at (barycentric cs:dl=1,dr=1,ul=1,ur=1) {$p$};
            \draw[dashed] ($(ul)+(0.2,-0.2)$) circle (0.5);
        \end{tikzpicture}
    \end{figure}%
    すると,local operator algebra $\irm{\mathcal{A}}{loc}$ は生成元 $\hat{A}_v,\, \hat{B}_p$ と関係式
    \begin{align}
        (\hat{A}_v)^2 = (\hat{B}_p)^2 = 1,\quad \hat{A}_v \hat{B}_p = \hat{B}_p \hat{A}_v
    \end{align}
    で表現される.故に $\irm{\mathcal{A}}{loc} \cong \mathbb{C}^4$ であり,$\irm{\mathcal{A}}{loc}$-\hyperref[def:moduleobj]{加群が成す圏} $\MODC{\VEC{\mathbb{C}}}{\irm{\mathcal{A}}{loc}}$ は
    \begin{align}
        \hat{P}_{\pm\pm} \coloneqq \frac{1 \pm \hat{A}_v}{2} \frac{1 \pm \hat{B}_p}{2}
    \end{align}
    の4つの単純対象を持つが,これらはそれぞれ
    \begin{enumerate}
        \item $P_{++}$ は\hyperref[def:TD]{トポロジカル欠陥} $1$ への射影演算子
        \item $P_{-+}$ は点 $v$ における\hyperref[def:TD]{トポロジカル欠陥} $e$ への射影演算子
        \item $P_{+-}$ は面 $p$ における\hyperref[def:TD]{トポロジカル欠陥} $m$ への射影演算子
        \item $P_{--}$ は組 $(v,\, p)$ における\hyperref[def:TD]{トポロジカル欠陥} $f$ への射影演算子
    \end{enumerate}
    に対応している.

     系のスケールを大きくするにつれて $\mathcal{A}_{\text{loc}}$ の生成元は複雑になるが,異なるスケールの $\mathcal{A}_{\text{loc}}$ 同士は\hyperref[def:Morita-equiv]{森田同値}であることが知られている~\cite{Kitaev1997TC}.
    つまり,\hyperref[ex:toric-emf]{toric code模型}の\hyperref[def:TD]{トポロジカル欠陥}は系のスケールに依存しない.
\end{myexample}

$\mathsf{C}_D$ 内部の全ての\hyperref[def:TD]{トポロジカル欠陥}(i.e. トポロジカル秩序)の集まりを\textbf{topological skelton}と呼び,$\sk (\mathsf{C}_D)$ と書くことにする~\cite{KongZhang2022TO}.
次に考えるべきなのは,topologial skeltonのどの構造が,親となるトポロジカル秩序 $\mathsf{C}_D$ を完全に特徴付けるのか,と言う問題である.
2025年現在では次のように予想されている~\cite[Conjecture 2, p.11]{KongWen2014braidedfusioncategoriesgravitational}:

\begin{myconjph}[label=conj:TO-BFC]{トポロジカル秩序の特徴付け}
    \hyperref[def:quantum-phase]{トポロジカル秩序} $\mathsf{C}_D$ を完全に特徴付けるのは,$\sk (\mathsf{C}_D)$ の組紐付き(高次)フュージョン圏としての構造,及びchiral central chargeである.
\end{myconjph}

以下では,特に $\sk (\mathsf{C}_D)$ の要素のうち空間次元が $0$ であるものに焦点を当てて予想\eqref{conj:TO-BFC}を解読する.
便宜上,$\sk (\mathsf{C}_D)$ の要素のうち $p+1$ 次元の\hyperref[def:TD]{トポロジカル欠陥}であるもの全体を $\sk_p (\mathsf{C}_D)$ と書くことにする.

\begin{mydef}[label=def:anyon]{エニオン}
    $\sk (\mathsf{C}_2)$ の要素のうち空間次元が $0$ であるものを\textbf{エニオン} (anyon) と呼ぶ.
\end{mydef}

\section{$0+1$ 次元のトポロジカル欠陥}

\subsection{アーベル圏としての構造}

点 $\xi \in \mfd{\Sigma}{D}$ に局在した $0+1$ 次元の\hyperref[def:TD]{トポロジカル欠陥} $\forall x,\, y \in \sk_0(\mathsf{C}_D)$ をとる.
定義\ref{def:TD}よりこれらの代表元は全系のHilbert空間の部分空間であり,その間に作用する線形作用素 $f \colon x \lto y$ を考えることができる.
$f$ がwell-definedであるためには,系の任意のスケールにおいて局所演算子と可換でなくてはいけない.よって $f$ は $0+0$ 次元の\hyperref[def:TD]{トポロジカル欠陥}と見做すべきである.
このような全ての $f$ が成す集合を $\bm{\Hom{\sk_0(\mathsf{C}_D)}(x,\, y)}$ と書く.

\begin{mypropph}[label=prop:sk0-category]{$\sk_0(\mathsf{C}_D)$ の圏としての構造}
    $\sk_0(\mathsf{C}_D)$ は,
    \begin{itemize}
        \item $0+1$ 次元の\hyperref[def:TD]{トポロジカル欠陥}を対象
        \item $\Hom{\sk_0(\mathsf{C}_D)}(x,\, y)$ をHom集合
    \end{itemize}
    とする\hyperref[def:additive-cat]{$\mathbb{C}$-線形}な\hyperref[def:category]{圏}を成す.特に,$\Id_x \in \Hom{\sk_0(\mathsf{C}_D)} (x,\, x)$ は恒等作用素に対応する\hyperref[def:trivialTO]{自明}な $0+0$ 次元の\hyperref[def:TD]{トポロジカル欠陥}である.
\end{mypropph}

\begin{proof}
    Hom集合の $\mathbb{C}$-線形性は,$f \in \Hom{\sk_0(\mathsf{C}_D)}(x,\, y)$ がミクロには線形作用素であったことによる.
\end{proof}

命題\ref{prop:sk0-category}に合わせて,以下では $0+1$ 次元の\hyperref[def:TD]{トポロジカル欠陥}全体が成す集まりを $\Obj{\sk_0(\mathsf{C}_D)}$ と書く.
$\forall x,\, y \in \Obj{\sk_0(\mathsf{C}_D)}$ の代表元は全系のHilbert空間の部分空間であるから,その直和 $x \oplus y \in \Obj{\sk_0(\mathsf{C}_D)}$ を考えることができる.

\begin{mypropph}[label=prop:sk0-category]{$\sk_0(\mathsf{C}_D)$ は半単純}
    $\sk_0(\mathsf{C}_D)$ は,\hyperref[def:semisimple-cat]{半単純}かつ\hyperref[def:finite-abcat]{有限}な\hyperref[def:additive-cat]{$\mathbb{C}$-線形アーベル圏}である.
\end{mypropph}

\begin{myexample}[label=ex:toric-semisimple]{toric codeのトポロジカル欠陥の直和}
    \hyperref[ex:toric-emf]{toric code}の\hyperref[def:quantum-phase]{トポロジカル秩序} $\mathsf{TC}$ を考える.
    プラケット $p_0$ に局在したトラップハミルトニアン $\delta \hat{H}(p_0) \coloneqq \hat{B}_{p_0}$ によって特徴付けられる\hyperref[def:TD]{トポロジカル欠陥}は
    \begin{align}
        \hat{A}_v \ket{\psi_{\pm}} &= \ket{\psi_{\pm}}
        &
        \hat{B}_p \ket{\psi_{\pm}} &= 
        \begin{cases}
            \pm\ket{\psi_{\pm}}, &p=p_0 \\
            \ket{\psi_{\pm}}, &p\neq p_0
        \end{cases}
    \end{align}
    を充たす,元のハミルトニアン $\hat{H}$ の2つの固有状態 $\ket{\psi_{\pm}}$ で貼られる2次元の部分空間 $\mathbb{C} \ket{\psi_+} \oplus \mathbb{C} \ket{\psi_-}$ である.特に\exref{ex:toric-emf}で得た $1,\, e,\, m,\, f \in \Obj{\sk_0(\mathsf{TC})}$ を用いると $\mathbb{C} \ket{\psi_+} = 1,\; \mathbb{C} \ket{\psi_-} = m$ と書ける.
    同様に,トラップハミルトニアンを
    \begin{align}
        \delta \hat{H} \bigl( \{v_1,\, p_1\} \bigr) \coloneqq \hat{A}_{v_0} + \hat{B}_{p_0} + \hat{A}_{v_0} \hat{B}_{p_0}
    \end{align}
    で定義すると,対応する\hyperref[def:TD]{トポロジカル欠陥}は $v_0$ と $p_0$ において符号が逆にならねばならないので
    \begin{align}
        e \oplus m
    \end{align}
    になる.トラップハミルトニアンを
    \begin{align}
        \delta \hat{H} \bigl( \{v_1,\, p_1\} \bigr) \coloneqq \hat{A}_{v_0} + \hat{B}_{p_0} - \hat{A}_{v_0} \hat{B}_{p_0}
    \end{align}
    で定義すると,対応する\hyperref[def:TD]{トポロジカル欠陥}は $v_0$ と $p_0$ において符号が同じにならねばならないので
    \begin{align}
        1 \oplus f
    \end{align}
    になる.

     一番最初のトラップハミルトニアンに摂動を加えて $\delta \hat{H}_{\varepsilon} (p_0) \coloneqq (1+\varepsilon) \hat{B}_{p_0}$ にしてみよう.
    このとき,実現するトポロジカル欠陥は $\varepsilon = 0$ ならば $1 \oplus m$ だが,$\varepsilon < 0$ ならば $1$ であり,$\varepsilon > 0$ ならば $m$ である.
    このように,直和の形で書けるトポロジカル欠陥は摂動に対して不安定である.
\end{myexample}

\begin{mydefph}[label=def:stableTO]{安定なトポロジカル秩序}
    $D+1$ 次元の\hyperref[def:quantum-phase]{トポロジカル秩序} $\mathsf{C}_D \in \TO{D}$ が\textbf{安定} (stable) であるとは,$\mfd{\Sigma}{D} = \mathbb{R}^D$ のときに\hyperref[def:gapped]{基底状態の縮退度}が $1$ になることを言う.
    もしくは,同じことだが,$\mfd{\Sigma}{D} = \mathbb{R}^D$ のときに\hyperref[def:trivialTO]{自明なトポロジカル欠陥} $1_p \in \Obj{\sk_p (\mathsf{C}_D)}$ が\hyperref[def:semisimple-cat]{単純対象}であることを言う.
\end{mydefph}

\subsection{モノイダル圏としての構造}

$\forall x,\, y \in \Obj{\sk_0(\mathsf{C}_D)}$ の代表元は全系のHilbert空間の部分空間であるから,そのテンソル積 $x \otimes y \in \Obj{\sk_0(\mathsf{C}_D)}$ を考えることができる.
直観的には,系のスケールを変えて2つの\hyperref[def:TD]{トポロジカル欠陥}を1つの\hyperref[def:TD]{トポロジカル欠陥}と見做すと言うことである.トポロジカル欠陥を\hyperref[def:quantum-phase]{トポロジカル秩序}と見做すと,\hyperref[def:stacking]{積層}を行っていると考えても良い.
このような操作を\textbf{フュージョン} (fusion) と呼ぶ.

\begin{mypropph}[label=prop:sk0-category]{$\sk_0(\mathsf{C}_D)$ はモノイダル圏}
    $\sk_0(\mathsf{C}_D)$ は\hyperref[redef:monoidal-category]{モノイダル圏}である.
    % \hyperref[def:semisimple-cat]{半単純}かつ\hyperref[def:finite-abcat]{有限}な\hyperref[def:additive-cat]{$\mathbb{C}$-線形アーベル圏}であり
\end{mypropph}

ところが,実はトポロジカル欠陥のフュージョンの厳密な定義には様々な困難がつきまとう.
$x,\, y$ がそれぞれ互いに異なる点 $\xi,\, \eta \in \mfd{\Sigma}{D}$ に局在しているとしよう.このとき,代表元の素朴なテンソル積は,トポロジカル欠陥のレベルでのテンソル積 $x \otimes_{(\xi,\, \eta)} y$ を定める.
i.e. $x,\, y \in \Obj{\sk_0(\mathsf{C}_D)}$ のテンソル積の定義には,集合 $\bigl\{\, (\xi,\, \eta) \in \mfd{\Sigma}{D} \bigm| \xi \neq \eta \,\bigr\}$ だけの不定性が存在し得る.
定義\ref{def:TD}直下の注の方法で $\sk_0(\mathsf{C}_D)$ を,点 $\xi \in \mfd{\Sigma}{D}$ におけるトポロジカルな境界条件が成す圏 $\sk_0(\mathsf{C}_D)_\xi$ と見做すと,$\otimes_{(\xi,\, \eta)}$ は関手
\begin{align}
    \otimes_{(\xi,\, \eta)} \colon \sk_0(\mathsf{C}_D)_\xi \times \sk_0(\mathsf{C}_D)_\eta \lto \sk_0(\mathsf{C}_D)
\end{align}
を定める.

ところで,$(\xi,\, \eta) \in $ を連続曲線 $\gamma \colon [0,\, 1] \lto \Conf (\mfd{\Sigma}{D})$ に沿って $(\xi',\, \eta')$ まで断熱的に\footnote{i.e. 途中経過のトラップハミルトニアンの\hyperref[def:gapped]{ギャップ}を閉じずに}移動することで,
いつでも同型写像 $T^\gamma_{x,\, y} \colon x \otimes_{(\xi,\, \eta)} y \lto x \otimes_{(\xi',\, \eta')} y$ を作ることができる.この同型写像を $\forall x,\, y \in \Obj{\sk_0(\mathsf{C}_D)}$ に関して集めたものは\hyperref[def:nat]{自然変換}
\begin{align}
    T^\gamma \colon \otimes_{(\xi,\, \eta)} \Longrightarrow \otimes_{(\xi',\, \eta')}
\end{align}
を定める.状況を簡略化するためにしばしば次の仮定をおく:

\begin{myhypo}[label=hypo:homotopic-equal]{}
    2つの互いにホモトピックな道 $\gamma_1,\, \gamma_2$ に対して,$T^{\gamma_1} = T^{\gamma_2}$ が成り立つ.
\end{myhypo}

然るに,2つの互いにホモトピックな道 $\gamma_1,\, \gamma_2$ をとったとしても,ギャップが有限だと $T^{\gamma_1},\, T^{\gamma_2}$ が一致してくれる保証はない.
このようなときには,テンソル積 $\otimes_{(\xi,\, \eta)}$ は高次のホモトピーに依存し,$\sk_0(\mathsf{C}_D)$ の上に\textbf{$A_\infty$-圏}の構造を与える.

このように,$\sk_0(\mathsf{C}_D)$ のテンソル積の構造は大きな不定性を持つ.\hyperref[def:quantum-phase]{トポロジカル秩序のミクロな定義}から出発し,断熱変形を利用して構成することもできるが,そうするとモノイダル圏の同型類しか定まらない~\cite{KawagoeLevin2020anyon}.
この意味で,$\sk_0(\mathsf{C}_D)$ のモノイダル構造は「ゲージ不変でない」などと言うことがある.

\subsection{ユニタリティ}

$\sk_0(\mathsf{C}_D)$ における射とは,任意の局所演算子と可換な線形作用素のことであった.故に,そのエルミート共役が自然に定まる.ストリング図式で書く場合は次のようにする:
\begin{align}
    \begin{tikzpicture}[baseline={([yshift=-.5ex]current bounding box.center)}]
        \path coordinate (x)
        ++(0,1) coordinate[bullet,label=right:$f$] (f)
        ++(0,1) coordinate (y)
        ;
        \draw[->-=.25,->-=.75] (x) -- node[midway,left] {$x$} (f.center) -- node[midway,left] {$y$} (y);
    \end{tikzpicture}
    \lmto
    \begin{tikzpicture}[baseline={([yshift=-.5ex]current bounding box.center)}]
        \path coordinate (x)
        ++(0,1) coordinate[bullet,label=right:$f^\dagger$] (f)
        ++(0,1) coordinate (y)
        ;
        \draw[->-=.25,->-=.75] (x) -- node[midway,left] {$y$} (f.center) -- node[midway,left] {$x$} (y);
    \end{tikzpicture}
\end{align}

\begin{mydef}[label=def:unitary]{ユニタリ構造}
    $\Cat{C}$ を $\mathbb{C}$-線形な\hyperref[def:additivecat]{アーベル圏}とする.
    $\Cat{C}$ の\textbf{ダガー構造} (dagger structure) とは,反線形な関手
    \begin{align}
        \dagger \colon \OP{\Cat{C}} \lto \Cat{C}
    \end{align}
    であって以下を充たすもののこと:
    \begin{description}
        \item[\textbf{(dag-1)}] $\forall x \in \Obj{\Cat{C}}$ に対して $x^\dagger = x$
        \item[\textbf{(dag-2)}] $\forall x,\, y \in \Obj{\Cat{C}}$,及び $\forall f \in \Hom{\Cat{C}}(x,\, y)$ に対して $(f^\dagger)^\dagger = f$
    \end{description}
    さらに,以下の条件を充たすダガー構造は\textbf{ユニタリ構造} (unitary structure) と呼ばれる:
    \begin{description}
        \item[\textbf{(dag-3)}] $f^\dagger \circ f = 0 \IFF f = 0$
    \end{description}
    
    \tcblower

    ユニタリ圏 $\Cat{C}$ の射 $f \in \Hom{\Cat{C}}(x,\, y)$ が\textbf{ユニタリ} (unitary) であるとは,$f^\dagger \circ f = \Id_x \AND f \circ f^\dagger = \Id_y$ が成り立つことを言う.
\end{mydef}

\begin{myprop}[label=prop:unitary-Cstar]{ユニタリ構造と $C^*$ 構造}
    $\mathbb{C}$-線形な\hyperref[def:additivecat]{アーベル圏} $\Cat{C}$ が\hyperref[def:starcat]{$C^*$-圏}であることと,\hyperref[def:unitary]{ユニタリ圏}であることは同値である.
\end{myprop}

\begin{proof}
    ~\cite[Proposition 2.1, p.5]{Mueger1998}
\end{proof}

\begin{mydef}[label=def:unitary-monoidal]{ユニタリモノイダル圏}
    $\mathbb{C}$-線形なアーベル圏でもあるモノイダル圏 $(\Cat{C},\, \otimes ,\, a,\, l,\, r)$ が\textbf{ユニタリモノイダル圏}であるとは,
    \hyperref[def:unitary]{ユニタリ構造} $\dagger \colon \OP{\Cat{C}} \lto \Cat{C}$ が\hyperref[redef:monidal-functor]{モノイダル関手}であることを言う.
\end{mydef}

\subsection{rigidity}

非自明な\hyperref[def:TD]{トポロジカル欠陥}の単体を,局所演算子によって生成・消滅させることはできない.
トポロジカル欠陥の生成・消滅はその世界線を曲げることによって成される(図\ref{fig:creanh}).
この過程を, $0+0$-次元のトポロジカル欠陥を用いてそれぞれ
\begin{align}
    \lcoev_x &\colon 1 \lto x \otimes x^*, \\
    \lev_x &\colon x^* \otimes x \lto 1
\end{align}
と表現する.
ミクロには,このようなプロセスは\hyperref[def:trivialTO]{自明相} $1_0 \in \Obj{\sk_0(\mathsf{C}_D)}$ にストリング状の演算子を作用させることによって成される.
\begin{figure}[H]
    \centering
    \begin{subfigure}{0.4\columnwidth}
        \centering
        \begin{tikzpicture}
            \path coordinate[label=below left:$x$] (x)
            ++(1,0) coordinate[label=below right:$x^*$] (y);
            \LCOEV{x}{y}
        \end{tikzpicture}
        \caption{$\lcoev_x$ による生成}
    \end{subfigure}
    \hspace{5mm}
    \begin{subfigure}{0.4\columnwidth}
        \centering
        \begin{tikzpicture}
            \path coordinate[label=below left:$x$] (x)
            ++(1,0) coordinate[label=below right:$x^*$] (y);
            \LEV{x}{y}
        \end{tikzpicture}
        \caption{$\lev_x$ による消滅}
    \end{subfigure}
    \caption{トポロジカル欠陥 $x \in \Obj{\sk_0(\mathsf{C}_D)}$ の生成と消滅}
    \label{fig:creanh}
\end{figure}%

\begin{mypropph}[label=prop:sk0-unitaryfusion]{$\sk_0(\mathsf{C}_D)$ はユニタリフュージョン圏}
    $\sk_0(\mathsf{C}_D)$ は\hyperref[def:unitary-monoidal]{ユニタリ}\hyperref[def:tensorfusion-cat]{多重フュージョン圏}である.
    特に,$\mathsf{C}_D$ が\hyperref[def:stableTO]{安定}ならばユニタリフュージョン圏である.
\end{mypropph}

\hyperref[def:unitary-monoidal]{ユニタリ構造}と\hyperref[def:pivotal]{旋回構造}の間には密接な関係がある.

\begin{mylem}[label=lem:rdual-unitary]{ユニタリモノイダル圏における右双対}
    \hyperref[def:unitary-monoidal]{ユニタリモノイダル圏} $(\Cat{C},\, \otimes,\, 1,\, a,\, l,\, r,\, \dagger)$ の対象 $x \in \Obj{\Cat{C}}$ が\hyperref[def:dual]{左双対} $(x^*,\, \lcoev_x,\, \lev_x)$ を持つとする.
    このとき,$(x^*,\, \lev_x{}^\dagger,\, \lcoev_x{}^\dagger)$ は $x$ の右双対である.
\end{mylem}

\begin{proof}
    \hyperref[def:dual]{\textsf{\textbf{(zig-zag equations)}}}を示す.実際,
    \begin{align}
        (\lcoev_x{}^\dagger \otimes \Id_x) \circ (\Id_x \otimes \lev_x{}^\dagger)
        &= (\lcoev_x{}^\dagger \otimes \Id_x^\dagger) \circ (\Id_x^\dagger \otimes \lev_x{}^\dagger) \\
        &= (\lcoev_x \otimes \Id_x)^\dagger \circ (\Id_x \otimes \lev_x)^\dagger \\
        &= \bigl( (\Id_x \otimes \lev_x) \circ (\lcoev_x \otimes \Id_x)\bigr)^\dagger \\
        &= \Id_x^\dagger = \Id_x, \\
        (\Id_{x^*} \otimes \lcoev_x{}^\dagger) \circ (\lev_x{}^\dagger \otimes \Id_{x^*})
        &= \bigl( (\lev_x\otimes \Id_{x^*}) \circ (\Id_{x^*} \otimes \lcoev_x) \bigr)^\dagger \\
        &= \Id_{x^*}
    \end{align}
    が成り立つ.
\end{proof}

\begin{myprop}[label=prop:spherical-unitary]{ユニタリテンソル圏における球状構造}
    \hyperref[def:unitary-monoidal]{ユニタリ}\hyperref[def:tensorfusion-cat]{テンソル圏} $(\Cat{C},\, \otimes,\, 1,\, a,\, l,\, r,\, \lcoev,\, \lev,\, \dagger)$ における\hyperref[def:pivotal]{旋回構造}
    \begin{align}
        p_x \coloneqq (\lcoev_x{}^\dagger \otimes \Id_{x^{**}}) \circ (\Id_x \otimes \lcoev_{x^*}) \colon x \lto x^{**}
    \end{align}
    が\hyperref[def:spherical]{球状構造}になる必要十分条件は,$\forall x \in \Obj{\Cat{C}}$ に対して
    \begin{align}
        \lcoev_x{}^\dagger \circ \lcoev_x = \lev_x \circ \lev_x{}^\dagger
    \end{align}
    が成り立つことである.
\end{myprop}

\begin{proof}
    \begin{align}
        \dim_p(x)
        &= \lev_{x^*} \circ (p_x \otimes \Id_{x^*}) \circ \lcoev_x \\
        &= \lcoev_x{}^\dagger \circ \lcoev_x
    \end{align}
    ところで,
    \begin{align}
        (p_x^*)^{-1}
        &= \quad \begin{tikzpicture}[baseline={([yshift=-.5ex]current bounding box.center)}]
            \path coordinate (x_1)
            \foreach \i in {2,...,5} {
                ++(1,0) coordinate (x_\i)
            }
            ;
            \LEV{x_1}{x_4}
            \REV{x_2}{x_3}
            \RCOEV{x_3}{x_4}
            \LCOEV{x_2}{x_5}
            \draw[->-=.5] (x_1) -- +(0,-2);
            \draw[-<-=.5] (x_5) -- +(0,2);
            \node at ($0.5*(x_1)+0.5*(x_4)+(0,2.2)$) {$\lev_x$};
            \node at ($0.5*(x_2)+0.5*(x_3)+(0,1)$) {$\lev_{x^*}$};
            \node at ($0.5*(x_3)+0.5*(x_4)+(0,-1)$) {$\lev_{x}{}^\dagger$};
            \node at ($0.5*(x_2)+0.5*(x_5)+(0,-2.2)$) {$\lcoev_{x^{**}}$};
        \end{tikzpicture}
    \end{align}
    であるから,
    \begin{align}
        \dim_p(x^*)
        &= \lqTr \bigl( (p^*_x)^{-1} \bigr) &&\because \quad \text{補題\ref{lem:pivotal}-(1)} \\
        &= \lev_x \circ \lev_x{}^\dagger
    \end{align}
    と計算できる.よって
    \begin{align}
        \dim_p (x) = \dim_p (x^*) \IFF \lcoev_x{}^\dagger \circ \lcoev_x = \lev_x \circ \lev_x{}^\dagger
    \end{align}
\end{proof}

\hyperref[redef:rigid]{coevaluation/evaluation}を適切に選ぶことで,いつでも命題\ref{prop:spherical-unitary}の条件を充たすようにできる.
実際,$\lcoev_x,\, \lev_x$ が $x,\, x^* \in \Obj{\Cat{C}}$ に関する\hyperref[redef:dual]{(co)evaluation}ならば,$\forall \lambda \in \mathbb{C}^{\times}$ に対して $\lambda\, \lcoev_x,\; \lambda^{-1}\, \lev_x$ もまた\hyperref[redef:dual]{\textsf{\textbf{(zig-zag equations)}}}を充たすので $x,\, x^*$ に関する\hyperref[redef:dual]{(co)evaluation}である.
よって
\begin{align}
    \abs{\lambda}^2 = \sqrt{\frac{\lev_x \circ \lev_x{}^\dagger}{\lcoev_x{}^\dagger \circ \lcoev_x}}
\end{align}
を充たすように $\lambda$ を選べば良い~\cite[LEMMA 3.9., p.9]{yamagami2004frobenius}.

\begin{mydef}[label=def:ev-coev-balanced]{balancedなユニタリテンソル圏}
    \hyperref[def:unitary-monoidal]{ユニタリ}\hyperref[def:tensorfusion-cat]{テンソル圏} $(\Cat{C},\, \otimes,\, 1,\, a,\, l,\, r,\, \lcoev,\, \lev,\, \dagger)$ 
    が\textbf{balanced}であるとは,$\forall x \in \Obj{\Cat{C}}$ に対して
    \begin{align}
        \lcoev_x{}^\dagger \circ \lcoev_x = \lev_x \circ \lev_x{}^\dagger
    \end{align}
    が成り立つこと.このとき,命題\ref{prop:spherical-unitary}の構成によって $\Cat{C}$ は\hyperref[def:spherical]{球状圏}になる.
\end{mydef}


\subsection{組紐}

\hyperref[def:TD]{トポロジカル欠陥} $x,\, y \in \sk_0(\mathsf{C}_2)$ をとる.
$x$ が $y$ の周りを断熱的に1周する過程は同型射 $x \otimes y \xrightarrow{\cong} x \otimes y$ で記述される(図\ref{fig:TD-braiding-full}).
特に,反時計方向に交換する過程は\textbf{組紐} (braiding) と呼ばれる同型射 $b_{x,\, y} \colon x \otimes y \xrightarrow{\cong} y \otimes x$ で記述され,
時計方向の過程はその逆射 (anti-braiding) で与えられる(図\ref{fig:TD-braiding-half}).組紐を全ての $x,\, y$ に関して集めたものは\hyperref[def:nat]{自然同型} $b \colon \otimes \Longrightarrow \otimes \circ \tau$ を成す.
ただし $\tau \colon \Cat{C} \times \Cat{C} \lto \Cat{C} \times \Cat{C}$ は成分の互換である.

\begin{figure}[H]
    \centering
    \begin{subfigure}{0.4\columnwidth}
        \centering
        \begin{tikzpicture}[use Hobby shortcut]
            \path coordinate[bullet, label=left:$x$] (dx)
                  ++(1,0) coordinate[bullet, label=right:$y$] (dy)
                  ++(0,2) coordinate (my)
                  ++(1,0) coordinate (mx)
                  ++(-1,2) coordinate (uy)
                  ++(-1,0) coordinate (ux)
            ;
            \draw[->-=.5,name path=y,spath/save=y] (dy) -- (uy);
            \draw[->-=.5,name path=x,spath/save=x] (dx.center) .. controls +(0,1) and +(0,-1) .. ++(2,2)
                .. controls +(0,1) and +(0,-1) .. (ux);
            \begin{scope}
                \clip[name intersections={of=x and y}] (intersection-1) circle (5pt);
                \draw[spath/use=x,white,ultra thick,double=black,double distance=0.4pt];
            \end{scope}
            \begin{scope}
                \clip[name intersections={of=x and y}] (intersection-2) circle (5pt);
                \draw[spath/use=y,white,ultra thick,double=black,double distance=0.4pt];
            \end{scope}
        \end{tikzpicture}
        \caption{トポロジカル欠陥のfull braiding}
        \label{fig:TD-braiding-full}
    \end{subfigure}
    \hspace{5mm}
    \begin{subfigure}{0.4\columnwidth}
        \centering
        \begin{tikzpicture}
            \path coordinate[bullet,label=left:$x$] (dx) at (0,0)
                +(2,0) coordinate[bullet,label=right:$y$] (dy)
                +(0,4) coordinate (uy)
                +(2,4) coordinate (ux)
            ;
            \coordinate (CENTER_1) at ($0.5*(dx) + 0.5*(ux)$);
            \coordinate (CENTER_2) at ($0.5*(dy) + 0.5*(uy)$);
        
            \draw[->-=.75,name path=BRAID_1,spath/save=BRAID_1] (dx) .. controls (dx |- CENTER_1) and (ux |- CENTER_1) .. (ux);
            \draw[->-=.75,name path=BRAID_2,spath/save=BRAID_2] (dy) .. controls (dy |- CENTER_2) and (uy |- CENTER_2) .. (uy);
            \begin{scope}
                \clip[name intersections={of=BRAID_1 and BRAID_2}] (intersection-1) circle (5pt);
                \draw[spath/use=BRAID_1,white,ultra thick,double=black,double distance=0.4pt];
            \end{scope}
        \end{tikzpicture}
        \caption{braiding morphism}
        \label{fig:TD-braiding-half}
    \end{subfigure}
    \caption{組紐}
    \label{fig:TD-braiding}
\end{figure}%


\begin{mypropph}[label=prop:sk0-braided]{}
    $\sk_0(\mathsf{C}_D)$ は\hyperref[def:braided-monoidal]{組紐付き}ユニタリ多重フュージョン圏である.
    特に,$\mathsf{C}_D$ が\hyperref[def:stableTO]{安定}ならば組紐付きユニタリフュージョン圏である.
\end{mypropph}

% \begin{mydef}[label=def:S-matrix]{$S$-行列}
%     $(\Cat{C},\, \otimes,\, 1,\, a,\, l,\, r,\, \lcoev,\, \lev,\, \dagger,\,  b)$ を組紐付きユニタリフュージョン圏とする.
%     $\forall x,\, y \in \Simp (\Cat{C})$ に対して,$S$-行列を
%     \begin{align}
%         S_{xy} \coloneqq \Tr (b_{y^*,\, x} \circ b_{x^*,\, y})
%     \end{align}
%     で定義する.
% \end{mydef}

\subsection{リボン構造}



\subsection{UMTC}

トポロジカル秩序のアノマリーの有無と $\mathsf{C}_D$ の\hyperref[redef:braided-monoidal]{組紐構造}の間には重要な関係があると考えられている~\cite{KongWen2014braidedfusioncategoriesgravitational}:

\begin{myconjph}[label=conj:remote-detectable]{remote-detectable}
    トポロジカル秩序が\hyperref[def:anomalousQP]{アノマリーを持たない}ことと,
    余次元 $2$ 以上のトポロジカル欠陥が\hyperref[redef:braided-monoidal]{braiding}によって検出できることは等価である.
\end{myconjph}

簡単のため $D=2$ として考える.$D=2$ 次元における余次元2のトポロジカル欠陥とは,$D-2+1 = 1$ 次元の線状欠陥のことである.
非自明なトポロジカル欠陥 $x \in \Obj{\sk_0(\mathsf{C}_D)}$ が検出できるというのは,他の全てのトポロジカル欠陥 $y \in \Obj{\sk_0(\mathsf{C}_D)}$ を $x$ の周りに1周させたときに,非自明な位相を吐き出すような $y$ が少なくとも1つは存在するという意味である.

\begin{mydef}[label=def:Muger-center]{M\"{u}ger中心}
    $(\Cat{C},\, \otimes,\, 1,\, a,\, l,\, r,\, \lev,\, \lcoev,\, \rev,\, \rcoev,\, b)$ を\hyperref[redef:braided-monoidal]{組紐}付き\hyperref[def:tensorfusion-cat]{フュージョン圏}とする.
    $\Cat{C}$ の\textbf{M\"{u}ger中心} (M\"{u}ger center) とは,
    \begin{align}
        \mathfrak{Z}_2 (\Cat{C}) \coloneqq \bigl\{\, x \in \Obj{\Cat{C}} \bigm| \forall y \in \Obj{\Cat{C}},\; b_{y,\, x} \circ b_{x,\, y} = \Id_{x \otimes y} \,\bigr\} 
    \end{align}
    を対象とする $\Cat{C}$ の充満部分圏 $\mathfrak{Z}_2 (\Cat{C})$ のこと.
\end{mydef}
$\mathfrak{Z}_2 (\Cat{C})$ は $\Cat{C}$ から\hyperref[redef:braided-monoidal]{組紐}付き\hyperref[def:tensorfusion-cat]{フュージョン圏}の構造を引き継ぐが,定義から組紐構造は\hyperref[redef:braided-monoidal]{対称}になる.

\begin{mydef}[label=def:nondegen-BFC]{非退化な組紐付きフュージョン圏}
    \hyperref[redef:braided-monoidal]{組紐}付き\hyperref[def:tensorfusion-cat]{フュージョン圏}が\textbf{非退化} (non-degenerate) であるとは,
    \hyperref[def:Muger-center]{M\"{u}ger中心}の単純対象が唯一であることを言う.
\end{mydef}

\begin{mypropph}[label=prop:remote-detectable]{}
    \hyperref[def:stable]{安定な} $2+1$ 次元のトポロジカル秩序 $\mathsf{C}_2$ が\hyperref[def:anomalyQP]{アノマリー}を持たないことと,
    組紐付きユニタリフュージョン圏 $\sk_0(\mathsf{C}_2)$ が\hyperref[def:nondegen-BFC]{非退化}であることは同値である.
\end{mypropph}

\begin{mydef}[label=def:MTC]{MTC}
    \textbf{モジュラーテンソル圏} (modular tensor category; MTC) とは,\hyperref[def:nondegen-BFC]{非退化}な\hyperref[def:premodular-cat]{前モジュラー圏}のこと.
\end{mydef}

\begin{mypropph}[label=prop:sk0-braided]{$\sk_0(\mathsf{C}_2)$ はMTC}
    空間多様体 $\mathbb{R}^2$ 上に置かれた\hyperref[def:stable]{安定}かつ\hyperref[def:anomalyQP]{アノマリーを持たない} 2+1次元のトポロジカル秩序 $\mathsf{C}_2$ が持つ点状のトポロジカル欠陥全体
    $\sk_0(\mathsf{C}_2)$ はユニタリ\hyperref[def:MTC]{モジュラーテンソル圏}を成す.
\end{mypropph}

なお,$2+1$ 次元のトポロジカル欠陥 $\mathsf{C}_2$ が\hyperref[def:anomaly-QP]{アノマリー}を持たないことと,$\mathsf{C}_2$ の低エネルギー有効理論であるTQFTがアノマリーを持たないことは等価\underline{ではない}.
両者の間には\textbf{フレーミングアノマリー} (framing anomaly) の分だけ差異がある~\cite{turaev2010quantum}.

\begin{myprop}[label=prop:MTC]{MTCの性質}
    $(\Cat{C},\, \otimes,\, 1,\, a,\, l,\, r,\, \lev,\, \lcoev,\, \rev,\, \rcoev,\, b,\, \theta)$ が\hyperref[def:MTC]{MTC}ならば以下が成り立つ:
    \begin{enumerate}
        \item $S_{xy} = S_{yx} = S_{x^*y^*} = \bar{S}_{xy^*},\quad S_{1x} = S_{x1} = \dim (x)$
        \item $S_{xy} S_{xz} = \dim(x) \sum_{w \in \Simp(\Cat{C})} N^w_{yz} S_{xw}$
        \item 行列 $C$ を $C_{xy} = \delta_{x^*,y}$ で定義すると,$S^2 = \dim(\Cat{C}) C$
        \item \textbf{Verlinde formula}
        \begin{align}
            \sum_{x \in \Simp(\Cat{C})} \frac{S_{xy} S_{xz} S_{xw^*}}{S_{x1}} = \dim(\Cat{C}) N^w_{yz}
        \end{align}
        \item \textbf{Gauss sum}を
        \begin{align}
            \tau^{\pm} (\Cat{C}) \coloneqq \sum_{x \in \Simp(\Cat{C})} \theta_x^{\pm 1} \dim (x)^2
        \end{align}
        で定義すると,$\tau^+(\Cat{C}) \tau^-(\Cat{C}) = \dim (\Cat{C})$ が成り立つ.
        \item \textbf{加法的中心電荷} (additive central charge) を
        \begin{align}
            e^{2\pi \iunit c(\Cat{C})/8} \coloneqq \frac{\tau^+(\Cat{C})}{\sqrt{\dim (\Cat{C})}}
        \end{align}
        で定義すると,$c(\Cat{C}) \in \mathbb{Q} / 8 \mathbb{Z}$ が成り立つ.
        \item \textbf{$\bm{T}$-行列} ($T$-matrix) を
        \begin{align}
            T_{xy} \coloneqq \theta_x \delta_{x,\, y}
        \end{align}
        で定義すると,$(ST)^3 = \tau^+(\Cat{C}) S^2$ が成り立つ.
        従って $\LSL(2,\, \mathbb{Z})$ の生成元と関係式
        \begin{align}
            \langle\, \mathfrak{s},\, \mathfrak{t} \mid (\mathfrak{s}\mathfrak{t})^3 = \mathfrak{s}^2,\; \mathfrak{s}^4 = 1 \,\rangle
        \end{align}
        を思い出すと,群準同型
        \begin{align}
            \mathfrak{s} \lmto \frac{S}{\sqrt{\dim (\Cat{C})}},\; \mathfrak{t} \lmto T
        \end{align}
        は $\LSL(2,\, \mathbb{Z})$ のユニタリな射影表現を与える.特に,この射影表現が表現になるための障害は $c(\Cat{C}) \notin 8 \mathbb{Z}$ となることである.
    \end{enumerate}
    
\end{myprop}

\begin{proof}
    \begin{enumerate}
        \item 
    \end{enumerate}
    
\end{proof}


\section{Levin-Wen模型}


\end{document}