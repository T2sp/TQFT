\documentclass[TQFT_main]{subfiles}

\begin{document}

\chapter{モノイダル圏・フュージョン圏・高次群}

この章では\underline{標数 $0$ の}体 $\mathbb{K}$ のみを考える.

\section{加法圏}

\begin{mydef}[label=def:additive-cat]{加法圏・アーベル圏}
    圏 $\Cat{C}$ が体 $\mathbb{K}$ 上の\textbf{加法圏} (additive category) であるとは,以下を充たすこと:
    \begin{description}
        \item[\textbf{(add-1)}] 
        
        任意のHom集合 $\Hom{\Cat{C}}(X,\, Y)$ が $\fk$-ベクトル空間の構造をもち,かつ射の合成
        \begin{align}
            \circ \colon \Hom{\Cat{C}}(Y,\, Z) \times \Hom{\Cat{C}}(X,\, Y) \lto \Hom{\Cat{C}}(X,\, Z)
        \end{align}
        が $\mathbb{K}$-双線形写像である.

        \item[\textbf{(add-2)}] 
        
        \textbf{零対象}\footnote{始対象かつ終対象} (zero object) $\bm{0} \in \Obj{\Cat{C}}$ が存在し,$\forall X \in \Obj{\Cat{C}}$ に対して $\Hom{\Cat{C}}(\bm{0},\, X) = \Hom{\Cat{C}}(\bm{X},\, 0) = 0$ を充たす\footnote{最右辺は \textsf{\textbf{(add-1)}} の意味で零ベクトル空間.}.

        \item[\textbf{(add-3)}] 
        
        有限の\hyperref[def:product-coproduct]{余積}が常に存在する.
    \end{description}
    
    \tcblower

    加法圏 $\Cat{C}$ は,以下の条件を充たすとき\textbf{アーベル圏} (abelian category) と呼ばれる:
    \begin{description}
        \item[\textbf{(Ab-1)}] 
        
        任意の射 $f \in \Hom{\Cat{C}}(X,\, Y)$ が核 $\ker f \colon \Ker f \lto X$ および余核 $\coker f \colon Y \lto \Coker f$ を持つ.

        \item[\textbf{(Ab-2)}] 
        
        $\Ker f = \bm{0}$ ならば $f = \ker (\coker f)$,かつ $\Coker f = \bm{0}$ ならば $f = \coker (\ker f)$
    \end{description}
    
\end{mydef}

以下では\hyperref[def:additive-cat]{加法圏} $\Cat{C},\, \Cat{D}$ の間の関手 $F \colon \Cat{C} \lto \Cat{D}$ には,$F_{X,\, Y} \colon \Hom{\Cat{C}}(X,\, Y) \lto \Hom{\Cat{D}} \bigl( F(X),\, F(Y) \bigr),\; f \lmto F(f)$ が $\mathbb{K}$-線型写像となることを常に要請する.

\begin{myexample}[label=def:Rep]{表現の圏}
    $G$ を群とする.このとき
    \begin{itemize}
        \item $G$ の表現 $(\rho,\, V)$ を対象とする
        \item $G$-同変な $\mathbb{K}$-線型写像 $(\rho,\, V) \xrightarrow{f} (\rho',\, V')$ を射とする
    \end{itemize}
    圏を $\REP{G}$ と書く.$\REP{G}$ は\hyperref[def:additive-cat]{アーベル圏}である.
\end{myexample}

\begin{mydef}[label=def:semisimple-cat]{単純・半単純}
    \begin{itemize}
        \item \hyperref[def:additive-cat]{アーベル圏} $\Cat{C}$ の対象 $X \in \Cat{C}$ が\textbf{単純} (simple) であるとは,
        任意のモノ射 $\textcolor{blue}{i} \colon \textcolor{blue}{U} \hookrightarrow X$ が $0$ であるか同型射であることを言う.
        \item アーベル圏 $\Cat{C}$ が\textbf{半単純} (semisimple) であるとは,$\forall X \in \Cat{C}$ が単純対象の有限余積と同型であることを言う.i.e.
        単純対象の族 $\Familyset[\big]{V_i \in \Obj{\Cat{C}}}{i \in I}$ および有限個を除いて $0$ であるような非負整数の族 $\Familyset[\big]{N_i \in \mathbb{Z}_{\ge 0}}{i \in I}$ が存在して
        \begin{align}
            X \cong \bigoplus_{i \in I} N_i X_i
        \end{align}
        が成り立つこと.
    \end{itemize}
    
\end{mydef}

\section{モノイダル圏}

これまでも何回か登場したが,モノイダル圏についてまとめておく:

\begin{mydef}[label=redef:monoidal-category,breakable]{モノイダル圏}
    \textbf{モノイダル圏} (monidal category) は,以下の5つのデータからなる:
    \begin{itemize}
        \item 圏 $\mathcal{C}$
        \item \textbf{テンソル積} (tensor product) と呼ばれる関手 $\otimes \colon \mathcal{C} \times \mathcal{C} \lto \mathcal{C}$
        \item \textbf{単位対象} (unit object) $I \in \Obj{\mathcal{C}}$
        \item \textbf{associator}と呼ばれる\hyperref[def:nat]{自然同値}
        \begin{align}
            \Familyset[\big]{a_{X,\, Y,\, Z} \colon (X \otimes Y) \otimes Z \xrightarrow{\cong} X \otimes (Y \otimes Z)}{X,\, Y,\, Z \in \Obj{\mathcal{C}}}
        \end{align}
        \item \textbf{left/right unitors}と呼ばれる\hyperref[def:nat]{自然同値}
        \begin{align}
            &\Familyset[\big]{l_X \colon I \otimes X \xrightarrow{\cong} X}{X \in \Obj{\mathcal{C}}}, \\
            &\Familyset[\big]{r_X \colon X \otimes I \xrightarrow{\cong} X}{X \in \Obj{\mathcal{C}}}
        \end{align}
        
    \end{itemize}
    これらは $\forall X,\, Y,\, Z,\, W \in \Obj{\mathcal{C}}$ について以下の2つの図式を可換にする:
    \begin{description}
        \item[\textbf{(triangle diagram)}] 
        
        \begin{center}
            \begin{tikzcd}[row sep=large, column sep=large]
                &(X \otimes I) \otimes Y \ar[rr, "a_{X,\, I,\, Y}"]\ar[dr, "r_X \otimes \mathrm{Id}_Y"'] & &X \otimes (I \otimes Y) \ar[dl, "\mathrm{Id}_X \otimes l_Y"] \\
                & &X \otimes Y &
            \end{tikzcd}
        \end{center}
        
        \item[\textbf{(pentagon diagram)}] 
        
        \begin{center}
            \begin{tikzcd}[row sep=large, column sep=large]
                & &((W \otimes X) \otimes Y) \otimes Z \ar[ddl, "a_{W \otimes X,\, Y,\, Z}"']\ar[dr, "a_{W,\, X,\, Y} \otimes \mathrm{Id}_Z"] & \\
                & & &(W \otimes (X \otimes Y)) \otimes Z \ar[dd, "a_{W,\, X \otimes Y,\, Z}"] \\
                &(W \otimes X) \otimes (Y \otimes Z) \ar[ddr, "a_{W,\, X,\, Y \otimes Z}"'] & & \\
                & & &W \otimes ((X \otimes Y) \otimes Z) \ar[dl, "\mathrm{Id}_Z \otimes a_{X,\, Y,\, Z}"]\\
                & &W \otimes (X \otimes (Y \otimes Z)) &
            \end{tikzcd}
        \end{center}
        
    \end{description}
    
    \tcblower

    モノイダル圏 $\Cat{C}$ が\textbf{厳密} (strict) であるとは,$\forall X,\, Y,\, Z \in \Obj{\Cat{C}}$ に対して
    \begin{align}
        &(X \otimes Y) \otimes Z = X \otimes (Y \otimes Z), \\
        &I \otimes X = X,\quad X \otimes I = X
    \end{align}
    が成り立ち,かつ $a_{X,\, Y,\, Z},\; l_{X},\, r_X$ が恒等射であることを言う.
\end{mydef}

\begin{marker}
    定義\ref{redef:monoidal-category}で言うモノイダル圏を,\textbf{弱いモノイダル圏} (weak monoidal category) と呼ぶこともある.
\end{marker}


\begin{mydef}[label=redef:braided-monoidal]{組紐付きモノイダル圏}
    \textbf{組紐付きモノイダル圏} (braided monoidal category) とは,以下の2つからなる:
    \begin{itemize}
        \item \hyperref[def:monoidal-category]{モノイダル圏} $\mathcal{C}$
        \item \textbf{組紐} (braiding) と呼ばれる自然同型
        \begin{align}
            \Familyset[\big]{b_{X,\, Y} \colon X \otimes Y \xrightarrow{\cong} Y \otimes X}{X,\, Y \in \Obj{\mathcal{C}}}
        \end{align}
    \end{itemize}
    これらは $\forall X,\, Y,\, Z \in \Obj{\mathcal{C}}$ について以下の図式を可換にする:
    \begin{description}
        \item[\textbf{(hexagon diagrams)}] 
        
        \begin{center}
            \begin{tikzcd}[row sep=large, column sep=large]
                &X \otimes (Y \otimes Z) \ar[r, "a_{X,\, Y,\, Z}^{-1}"]\ar[d, "b_{X,\, Y \otimes Z}"] &(X \otimes Y) \otimes Z \ar[r, "b_{X,\, Y} \otimes \mathrm{Id}_Z"] &(Y \otimes X) \otimes Z \ar[d, "a_{Y,\, X,\, Z}"] \\
                &(Y \otimes Z) \otimes X &Y \otimes (Z \otimes X) \ar[l, "a_{Y,\, Z,\, X}^{-1}"] &Y \otimes (X \otimes Z) \ar[l, "\mathrm{Id}_X \otimes b_{X,\, Z}"]
            \end{tikzcd}
        \end{center}
        
        \begin{center}
            \begin{tikzcd}[row sep=large, column sep=large]
                &(X \otimes Y) \otimes Z \ar[r, "a_{X,\, Y,\, Z}"]\ar[d, "b_{X \otimes Y ,\, Z}"] &X \otimes (Y \otimes Z) \ar[r, "\mathrm{Id}_X \otimes b_{Y,\, Z}"] &X \otimes (Z \otimes Y) \ar[d, "a_{X,\, Z,\, Y}^{-1}"] \\
                &Z \otimes (X \otimes Y) &(Z \otimes X) \otimes Y \ar[l, "a_{Z,\, X,\, Y}"] &(X \otimes Z) \otimes Y \ar[l, "b_{X,\, Z} \otimes \mathrm{Id}_Y"]
            \end{tikzcd}
        \end{center}
    \end{description}
    \tcblower
    組紐付きモノイダル圏 $\mathcal{C}$ であって,$\mathcal{C}$ の組紐が $b_{X,\, Y} = b_{Y,\, X}^{-1}$ を充たすもののことを\textbf{対称モノイダル圏} (symmetric monoidal category) と呼ぶ.
\end{mydef}


\begin{mydef}[label=redef:dual,breakable]{双対}
    \hyperref[def:monoidal-category]{モノイダル圏} $\mathcal{C}$ およびその任意の対象 $X,\, X^* \in \Obj{\mathcal{C}}$ を与える.
    $X^*$ が $X$ の\textbf{右双対} (right dual) であり,かつ $X$ が $X^*$ の\textbf{左双対} (left dual) であるとは,
    \begin{itemize}
        \item \textbf{unit}と呼ばれる射
        \begin{align}
            i_X \colon I \lto X^* \otimes X
        \end{align}
        \item \textbf{counit}と呼ばれる射
        \begin{align}
            e_X \colon X \otimes X^* \lto I
        \end{align}
    \end{itemize}
    が存在して以下の図式を可換にすることを言う:
    \begin{description}
        \item[\textbf{(zig-zag equations)}] 
        
        \begin{center}
            \begin{tikzcd}[row sep=large, column sep=large]
                &X \otimes I \ar[dd, "r_X"']\ar[rr, "\mathrm{Id}_X \otimes i_X"] & &X \otimes (X^* \otimes X) \ar[d, "a_{X,\, X^*,\, X}^{-1}"] \\
                & & &(X \otimes X^*) \otimes X \ar[d, "e_X \otimes \mathrm{Id}_X"] \\
                &X & &I \otimes X \ar[ll, "l_X"] \\
            \end{tikzcd}
        \end{center}
        

        \begin{center}
            \begin{tikzcd}[row sep=large, column sep=large]
                &I \otimes X^* \ar[dd, "l_{X^*}"']\ar[rr, "i_X \otimes \mathrm{Id}_{X^*}"] & &(X^* \otimes X) \otimes X^* \ar[d, "a_{X^*,\, X,\, X^*}"] \\
                & & &X^* \otimes (X \otimes X^*) \ar[d, "\mathrm{Id}_{X^*} \otimes e_X"] \\
                &X^* & &X^* \otimes I \ar[ll, "r_{X^*}"] \\
            \end{tikzcd}
        \end{center}
    \end{description}
\end{mydef}

\begin{mydef}[label=redef:rigid]{rigidなモノイダル圏}
    \hyperref[def:monoidal-category]{モノイダル圏} $\mathcal{C}$ が\textbf{rigid}であるとは,$\forall X \in \Obj{\mathcal{C}}$ が\hyperref[def:dual]{左・右双対}を持つことを言う.
\end{mydef}

\begin{marker}
    これまでは圏 $\Cat{C}$ の対象を大文字で書いてきたが,以下では文脈によっては小文字で書くことがある.
\end{marker}

\begin{mydef}[label=redef:monidal-functor,breakable]{モノイダル関手}
    2つの\hyperref[def:monoidal-category]{モノイダル圏} $\mathcal{C},\, \mathcal{D}$ の間の関手
    \begin{align}
        F \colon \mathcal{C} \lto \mathcal{D}
    \end{align}
    が\textbf{弱いモノイダル関手} (lax monoidal functor) であるとは,
    \begin{itemize}
        \item 射
        \begin{align}
            \varepsilon \colon I_{\mathcal{D}} \lto F(I_{\mathcal{C}})
        \end{align}
        
        \item 自然変換
        \begin{align}
            \Familyset[\big]{\mu_{X,\, Y} \colon F(X) \otimes_{\mathcal{D}} F(Y) \lto F(X \otimes_{\mathcal{C}} Y)}{X,\, Y \in \Obj{\mathcal{C}}}
        \end{align}
    \end{itemize}
    があって,$\forall X,\, Y,\, Z \in \Obj{\mathcal{C}}$ に対して以下の図式が可換になること:
    \begin{description}
        \item[\textbf{(associatibity)}] 
        
        \begin{center}
            \begin{tikzcd}[row sep=large, column sep=huge]
                &(F(X) \otimes_{\mathcal{D}} F(Y)) \otimes_{\mathcal{D}} F(Z) \ar[d, "\mu_{X,\, Y} \otimes \mathrm{Id}_{F(Z)}"']\ar[r, "a^{\mathcal{D}}_{F(X),\, F(Y),\, F(Z)}"'] &F(X) \otimes_{\mathcal{D}} (F(Y) \otimes_{\mathcal{D}} F(Z)) \ar[d, "\mathrm{Id}_{F(X)} \otimes \mu_{Y,\, Z}"]\\
                &F(X \otimes_{\mathcal{C}} Y) \otimes_{\mathcal{D}} F(Z) \ar[d, "\mu_{X \otimes_{\mathcal{C}} Y,\, Z}"'] & F(X) \otimes_{\mathcal{D}} F(Y \otimes_{\mathcal{C}} Z) \ar[d, "\mu_{X,\, Y \otimes_{\mathcal{C}} Z}"] \\
                &F((X \otimes_{\mathcal{C}} Y) \otimes_{\mathcal{C}} Z) \ar[r, "F(a_{X,\, Y,\, Z}^{\mathcal{C}})"'] &F(X \otimes_{\mathcal{C}} (Y \otimes_{\mathcal{C}} Z))
            \end{tikzcd}
        \end{center}
        
        \item[\textbf{(unitality)}] 
        
        \begin{center}
            \begin{tikzcd}[row sep=large, column sep=huge]
                &I_{\mathcal{D}} \otimes_{\mathcal{D}} F(X) \ar[d, "l^{\mathcal{D}}_{F(X)}"']\ar[r, "\varepsilon \otimes \mathrm{Id}_{F(X)}"] &F(I_{\mathcal{C}}) \otimes_{\mathcal{D}} F(X) \ar[d, "\mu_{I_{\mathcal{C}},\, X}"] \\
                &F(X) &F(I_{\mathcal{C}} \otimes_{\mathcal{C}} X) \ar[l, "F(l^{\mathcal{C}}_X)"]
            \end{tikzcd}
        \end{center}

        \begin{center}
            \begin{tikzcd}[row sep=large, column sep=large]
                &F(X) \otimes_{\mathcal{D}}  I_{\mathcal{D}} \ar[d, "r^{\mathcal{D}}_{F(X)}"']\ar[r, "\mathrm{Id}_{F(X)} \otimes \varepsilon"] &F(X) \otimes_{\mathcal{D}} F(I_{\mathcal{C}})  \ar[d, "\mu_{X,\, I_{\mathcal{C}}}"] \\
                &F(X) &F(X \otimes_{\mathcal{C}} I_{\mathcal{C}}) \ar[l, "F(r^{\mathcal{C}}_X)"]
            \end{tikzcd}
        \end{center}
        
    \end{description}
    \tcblower
    \begin{itemize}
        \item 弱いモノイダル関手 $F$ の $\varepsilon$ と $\mu_{X,\, Y}$ が全て同型射ならば,$F$ は\textbf{強いモノイダル関手} (strong monoidal functor) と呼ばれる.
        \item 弱いモノイダル関手 $F$ の $\varepsilon$ と $\mu_{X,\, Y}$ が全て恒等射ならば,$F$ は\textbf{厳密なモノイダル関手} (strict monoidal functor) と呼ばれる.
    \end{itemize}
\end{mydef}

\begin{mydef}[label=def:monoidal-nat]{モノイダル自然変換}
    2つの\hyperref[redef:monoidal-category]{モノイダル圏} $\Cat{C},\, \Cat{D}$ の間の2つの\hyperref[redef:monidal-functor]{弱いモノイダル関手}
    $\Bigl(F_i\colon \Cat{C} \lto \Cat{D},\, \varepsilon_i \colon I_{\mathcal{D}} \lto F(I_{\mathcal{C}}),\, \Familyset[\big]{\mu_i{}_{X,\, Y} \colon F_i(X) \otimes F_i(Y) \lto F_i(X \otimes Y)}{X,\, Y \in \Obj{\Cat{C}}}\Bigr)\WHERE i = 1,\, 2$
    の間の\hyperref[def:nat]{自然変換}
    \begin{center}
        \begin{tikzcd}[row sep=large, column sep=large]
            \Cat{C} \ar[bend left=50,r, "F_1"{name=U, above}] \ar[bend right=50,r, "F_2"{name=D, below}] &\Cat{D}
            \ar[Rightarrow, from=U, to=D, "\bm{\tau}"]
        \end{tikzcd}
    \end{center}
    が\textbf{モノイダル自然変換} (monidal natural transformation) であるとは,$\forall X,\, Y \in \Cat{C}$ に対して以下の図式が可換になること:
    \begin{description}
        \item[\textbf{(テンソル積の保存)}] 
        
        \begin{center}
            \begin{tikzcd}[row sep=large, column sep=large]
                 &F_1(X) \otimes_{\mathcal{D}} F_1(Y) \ar[d, "\mu_1{}_{X,Y}"']\ar[r, "\bm{\tau}_X \otimes_{\mathcal{D}} \bm{\tau}_Y"] &F_2(X) \otimes_{\mathcal{D}} F_2(Y) \ar[d, "\mu_2{}_{X,Y}"] \\
                 &F_1(X \otimes_{\mathcal{C}} Y) \ar[r, "\bm{\tau}_{X \otimes_{\mathcal{C}} Y}"'] &F_2(X \otimes_{\mathcal{C}} Y)
            \end{tikzcd}
        \end{center}
        
        \item[\textbf{(単位対象の保存)}] 
        
        \begin{center}
            \begin{tikzcd}[row sep=large, column sep=large]
                 & &I_{\mathcal{D}} \ar[dl, "\varepsilon_1"'] \ar[dr, "\varepsilon_2"] & \\
                 &F_1(I_{\mathcal{C}})\ar[rr, "\bm{\tau}_{I_{\mathcal{C}}}"'] & &F_2(I_{\mathcal{C}})
            \end{tikzcd}
        \end{center}
        
    \end{description}
    
\end{mydef}

\section{フュージョン圏}

\begin{mydef}[label=def:fusion-cat]{フュージョン圏}
    圏 $\Cat{C}$ が\textbf{フュージョン圏} (fusion category) であるとは,
    \begin{itemize}
        \item $\Cat{C}$ は\hyperref[def:semisimple-cat]{半単純}な $\mathbb{K}$-上の\hyperref[def:additive-cat]{アーベル圏}
        \item $\Cat{C}$ は\hyperref[redef:rigid]{rigid}な\hyperref[redef:monoidal-category]{モノイダル圏}
        \item $\Cat{C}$ の\hyperref[def:semisimple-cat]{単純対象}の同型類が有限個
        \item 単位対象 $I \in \Obj{\Cat{C}}$ について,$\Hom{\Cat{C}}(I,\, I) = \mathbb{K}$
    \end{itemize}
    が成り立つこと.

    \tcblower

    フュージョン圏 $\Cat{C}$ が\textbf{組紐付きフュージョン圏} (braided fusion category) であるとは,$\Cat{C}$ が\hyperref[redef:braided-monoidal]{組紐付きモノイダル圏}でもあることを言う.
\end{mydef}

\section{2-群}

~\cite{baez2004higherdimensionalalgebrav2groups}に倣い\textbf{2-群} (2-group) を導入する.

\begin{mydef}[label=def:weak-inverse]{弱い逆対象}
    \hyperref[redef:monoidal-category]{モノイダル圏} $\Cat{C}$ の対象 $x \in \Obj{\Cat{C}}$ を1つとる.
    
    \begin{itemize}
        \item 対象 $y \in \Obj{\Cat{C}}$ が $X$ の\textbf{弱い逆対象} (weak inverse) であるとは,\hyperref[def:iso]{対象の同型}の意味で\footnote{\hyperref[nat]{自然同型}ではない} $x \otimes y \cong 1 \AND y \otimes x \cong 1$ が成り立つことを言う.
        \item $x$ が\textbf{弱可逆} (weakly invertible) であるとは,$x$ が弱い逆対象を持つことを言う.
    \end{itemize}
    
\end{mydef}

\begin{mydef}[label=def:W2G-C2G]{弱い2-群・コヒーレントな2-群}
    \begin{itemize}
        \item \textbf{弱い2-群} (weak 2-group) とは,\hyperref[redef:monoidal-category]{モノイダル圏} $\Cat{G}$ であって,任意の対象が\hyperref[def:weak-inverse]{弱可逆}でかつ任意の射が\hyperref[def:iso]{同型射}であるもののこと.
        \item \textbf{コヒーレントな2-群} (coherent 2-group) とは,\hyperref[redef:monoidal-category]{モノイダル圏} $\Cat{G}$ であって,任意の対象 $x \in \Obj{\Cat{G}}$ が\hyperref[def:iso]{可逆}な\hyperref[redef:dual]{unit,counit} $(x,\, \bar{x},\, i_x,\, e_x)$ を持ち,かつ任意の射が\hyperref[def:iso]{同型射}であるもののこと.
    \end{itemize}
\end{mydef}

\begin{mytheo}[label=thm:W2G-C2G]{弱い2-群はコヒーレントな2-群}
    任意の\hyperref[def:W2G-C2G]{弱い2-群}は\hyperref[def:W2G-C2G]{コヒーレントな2-群}にすることができる.
\end{mytheo}

\begin{proof}
    勝手な\hyperref[def:W2G-C2G]{弱い2-群} $\Cat{G}$ を1つ固定する.このとき $\forall x \in \Obj{\Cat{G}}$ に対してある $\bar{x} \in \Obj{\Cat{G}}$ および\hyperref[def:iso]{同型射} $i'_x \colon 1 \lto x \otimes \bar{x},\; e'_x \colon \bar{x} \otimes x \lto 1$ が存在する.

    ここで $e_x \coloneqq e'_x$ とおき,
    \begin{align}
        i_x \coloneqq (l_x \otimes \mathrm{Id}_{\bar{x}}) \circ a^{-1}_{1,x,\bar{x}} \circ (i'_x{}^{-1} \otimes \mathrm{Id}_{x \otimes \bar{x}}) \circ a^{-1}_{x,\bar{x},x\otimes \bar{x}} \circ (\mathrm{Id}_x \otimes a_{\bar{x},x,\bar{x}}) \circ (\mathrm{Id}_x \otimes e'_x{}^{-1} \otimes \mathrm{Id}_{\bar{x}}) \circ (\mathrm{Id}_x \otimes l^{-1}_{\bar{x}}) \circ i'_x
    \end{align}
    とおくと組 $(x,\, \bar{x},\, i_x,\, e_x)$ が\hyperref[redef:dual]{zig-zag equation}を充たすことを示す.
    実際,$i_x$ の定義をストリング図式で書くと
    \begin{align}
        i_x =
    \end{align}
    となるから,
\end{proof}

\begin{marker}
    定理\ref{thm:W2G-C2G}を踏まえ,以下では\hyperref[def:W2G-C2G]{コヒーレントな2-群}のことを単に\textbf{2-群} (2-group) と呼ぶ.
\end{marker}

\begin{mydef}[label=def:hom2G]{2-群の準同型}
    \hyperref[def:W2G-C2G]{2-群} $\mathcal{G},\, \mathcal{G}'$ の間の\textbf{準同型}とは,\hyperref[redef:monidal-functor]{モノイダル関手} $f \colon \mathcal{G} \lto \mathcal{G}'$ のこと.
\end{mydef}



\end{document}